% {{{ LaTeX prelude
\documentclass[11pt]{article}
\usepackage{fontspec}
\setmainfont[Ligatures={Common,TeX}]{Palatino}
\usepackage{url}
\usepackage{parskip}
\usepackage{natbib}
\bibpunct{(}{)}{;}{a}{,}{,}
\usepackage{titles}
% }}}

% {{{ LaTeX document
\begin{document}

% {{{ Title page
\begin{titlepage}
\title{Notes on Plato's \book{Laches}}
\author{Peter Aronoff}
\date{June 2013}
\maketitle
\end{titlepage}
% }}}

% {{{ Characters and setting
\section{Characters and setting}

% {{{ Characters
\subsection{Characters}

The dialogue has a relatively large number of speakers.  In addition to Socrates, there are two fathers, Lysimachus and Melesias and two generals, Laches and Nicias.

The fathers first. Lysimachus and Melesias both had famous politicians for fathers, Aristides the Just and Thucydides respectively. But neither Lysimachus or Melesias has amounted to much.  They blame their fathers for this, believing that their fathers did not raise them properly, and they would like to do better for their sons.  (Oddly, \book{Meno} (94a and 94c) indicates that both men received an excellent education.  So it's not entirely clear what their complaint is or whether it's a valid one.)  Lysimachus belongs to the same \foreign{deme} as Socrates, and it turns out that he was good friends with Socrates' father as well.  According to Thucydides the historian (VIII.86.9), Melesias was a member of the oligarchy put in charge of Athens in 411.

The generals next.  Nicias was an important general and politican in the time before and during the Peloponessian War.  His policies during the Sicilian expedition, where he lost his life in 413, helped to cause the Athenian's terrible defeat there.  Laches was also an important general during the Peloponessian war.  He died in 418 at the battle of Manitinea.  Laches and Socrates fought side by side during the retreat at Delium.  When Alcibiades describes Socrates behavior at Delium (\book{Symposium} 221a--b), he specifies that Socrates was ``more prudent" (ἔμφρων) than Laches, but I'm not sure how much that evidence is worth.
% }}} Characters

% {{{ Setting
\subsection{Setting}

Lysis and Melesias have invited Laches and Nicias to view a display by a teacher of hoplite fighting.  (The teacher's name is Stesilaus.  He doesn't speak, and we know of him only from this dialogue.)  The dialogue doesn't provide details about exactly where the men are, but clearly it is a \foreign{gymnasium} of some kind.  Socrates just happens to be around, so we might guess that the display takes place at one of his favorite haunts.  But obviously this is speculative---and probably not important since Plato didn't make a point about it.
% }}} Setting
% }}} Characters and setting

% {{{ Introduction (178a--190d6)
\section{Introduction (178a--190d6)}

content
% }}} Introduction (178a--190d6)

% {{{ Definitions of bravery (190d7--199e12)
\section{Definitions of bravery (190d7--199e12)}

content
% }}} Definitions of bravery (190d7--199e12)

% {{{ Conclusion (199e13--201c5)
\section{Conclusion (199e13--201c5)}

content
% }}} Conclusion (199e13--201c5)


\newpage
\bibliographystyle{apa}
\bibliography{plato}

\end{document}
% }}}
