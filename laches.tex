% {{{ LaTeX prelude
\documentclass[11pt]{article}
\usepackage{fontspec}
\setmainfont[Ligatures={Common,TeX}]{Palatino}
\usepackage{url}
\usepackage{parskip}
\usepackage{natbib}
\bibpunct{(}{)}{;}{a}{,}{,}
\usepackage{titles}
% }}}

% {{{ LaTeX document
\begin{document}

% {{{ Title page
\begin{titlepage}
\title{Notes on Plato's \book{Laches}}
\author{Peter Aronoff}
\date{June 2013}
\maketitle
\end{titlepage}
% }}}

% {{{ Characters and setting
\section{Characters and setting}

% {{{ Characters
\subsection{Characters}

The dialogue has a relatively large number of speakers.  In addition to Socrates, there are two fathers, Lysimachus and Melesias and two generals, Laches and Nicias.

The fathers first.  Lysimachus and Melesias both had famous politicians for fathers, Aristides the Just and Thucydides respectively.  But neither Lysimachus or Melesias has amounted to much.  They blame their fathers for this, believing that their fathers did not raise them properly, and they would like to do better for their sons.  (Oddly, \book{Meno} (94a and 94c) indicates that both men received an excellent education.  So it's not entirely clear what their complaint is or whether it's a valid one.)  Lysimachus belongs to the same \foreign{deme} as Socrates, and it turns out that he was good friends with Socrates' father as well.  According to Thucydides the historian (VIII.86.9), Melesias was a member of the oligarchy put in charge of Athens in 411.

The generals next.  Nicias was an important general and politican in the time before and during the Peloponessian War.  His policies during the Sicilian expedition, where he lost his life in 413, helped to cause the Athenian's terrible defeat there.  Laches was also an important general during the Peloponessian war.  He died in 418 at the battle of Manitinea.  Laches and Socrates fought side by side during the retreat at Delium.  When Alcibiades describes Socrates behavior at Delium (\book{Symposium} 221a--b), he specifies that Socrates was ``more prudent" (ἔμφρων) than Laches, but I'm not sure how much that evidence is worth.

% }}} Characters

% {{{ Setting
\subsection{Setting}

Lysis and Melesias have invited Laches and Nicias to view a display by a teacher of hoplite fighting.  (The teacher's name is Stesilaus.  He doesn't speak, and we know of him only from this dialogue.)  The dialogue doesn't provide details about exactly where the men are, but clearly it is a \foreign{gymnasium} of some kind.  Socrates just happens to be around, so we might guess that the display takes place at one of his favorite haunts.  But obviously this is speculative---and probably not important since Plato didn't make a point about it.
% }}} Setting
% }}} Characters and setting

% {{{ Introduction (178a--190d6)
\section{Introduction (178a--190d6)}

The introductory section of \book{Laches} is relatively long and involved.  Before we get to anything explicitly Socratic, we have the following.  Lysimachus makes a long speech inviting Nicias and Laches to help him and Melesias decide whether to train their sons in hoplite fighting.  Nicias and Laches agree, but Laches urges Lysimachus to involve Socrates as well.\footnote{The fathers invited Nicias and Laches.  Socrates luckily happens to be around.  His presence isn't very surprising though since he spent much of his life in and around training grounds.}  Lysimachus agrees, and he remembers that he and Socrates have an old family connection: Lysimachus was friendly with Socrates's father.  Nicias makes a speech in favor of training the boys in hoplite fighting, and Laches makes a speech against such training.  Lysimachus urges Socrates to break the tie, and only then do we reach something explicitly Socratic as Socrates argues that expertise and not numbers should decide the matter.  Socrates then moves the discussion in familiar ways until the generals are giving definitions of \word{bravery}.

Although much of this material is not explicitly Socratic, a number of familiar Socratic themes appear implicitly in the introduction.  First, Lysimachus and Melesias are very concerned for the proper education of their sons.  This is already Socratic.  But in addition they worry because their fathers were excellent men and well-renowned politicians, but they have not done so well.  In \book{Protagoras} Socrates argues that political excellence is not teachable, partly on the grounds that the most esteemed politicans were unable to pass on their success to their children (319e--320b).  Nicias also makes a claim that hints at Socratic intellectualism.  He argues that someone who has knowledge of hoplite fighting will fare better in battle (182b2--4).\footnote{Initially this may sound like simply a truism, but since the dialogue will eventually turn on whether or not bravery is a form of knowledge, I think it's an important bit of foreshadowing.}  Laches on the other hand argues that many trainers think and pretend that they have skill as fighters, but they they are shams.  This criticism implicitly recalls Socratic arguments showing that people think they know things that they actually do not.  Laches explicitly concludes by discussing people who ``think that they know" (184b4) something and the great ``slanders" (184b7) and ``envy" (184c1) that they face as a result.

Socrates becomes more actively involved at 184d5, and he guides things in a familiar direction.  Lysimachus, as I mentioned above, wants Socrates to break the tie between Nicias and Laches.  Socrates, however, objects on principle to making the decision by counting votes.  He immediately gets Melesias to agree that they must follow knowledge and not number (184d8-9).  Additionally he suggests that they don't even know what subject they need an expert in yet.  This suprises Melesias and Nicias, but Socrates makes a familiar argument:

\begin{enumerate}
    \item When someone seeks advice about a drug for their eyes, the decision is really about their eyes, not the drug.  And when someone seeks advice about a bit for a horse, the decision is about the horse not the bit.  Nicias agrees to both of these (185c5--d4).
    \item Therefore, more generally, when someone seeks advice about X for Y, the decision concerns Y and not truly X.  Nicias agrees to this too (185d5--8).\footnote{Unwisely, I think.}
    \item Therefore it is necessary to consider whether a fellow decision maker is an expert in the real goal of the discussion, not any proximate goal.  Again, Nicias agrees (185d9--12).
\end{enumerate}

Socrates then proposes that their consideration of training in hoplite fighting is ``for the sake of the soul of the young men" (185e1--2).  Therefore they must find an expert in training young people's souls not hoplite fighting.

Socrates assumes that if you an expert, then you should be able to prove it.  His basic demands for proof are either (1) the name of your teacher and proof of his \foreign{bona fides} or (2) evidence that you have taught others well, if you have not had a teacher but discovered things on your own.  Initially, it appears that Socrates will question Nicias and Laches about their status as experts, but the dialogue finds a clever way to avoid this.  Once they reach the point where it would be natural for Socrates to test them on these matters, Socrates suddenly thinks of an even better way to investigate: If they are really experts, then they should be able to give an account of the thing they are experts in.  So instead of asking them questions about their teachers or students, Socrates will simply ask them about the subject in question.

The subject in question is virtue (ἀρετή), and they immediately whittle that down to bravery.  Socrates worries that all of virtue would be too large a topic, so he proposes that they discuss a part of virtue (190c8-10).  It's important to notice that he is assuming here that virtue has parts.  This assumption will play an important role later in the dialogue.  Bravery is the obvious choice to pick, Socrates says, since they began all this by talking about hoplite fighting.

% }}} Introduction (178a--190d6)

% {{{ Definitions of bravery (190d7--199e12)
\section{Definitions of bravery (190d7--199e12)}

content
% }}} Definitions of bravery (190d7--199e12)

% {{{ Conclusion (199e13--201c5)
\section{Conclusion (199e13--201c5)}

content
% }}} Conclusion (199e13--201c5)


\newpage
\bibliographystyle{apa}
\bibliography{plato}

\end{document}
% }}}
