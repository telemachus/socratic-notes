% {{{ LaTeX prelude
\documentclass[11pt]{article}
\usepackage{fontspec}
\setmainfont[Ligatures={Common,TeX}]{Palatino}
\usepackage{url}
\usepackage{parskip}
\usepackage{natbib}
\bibpunct{(}{)}{;}{a}{,}{,}
\usepackage{titles}
% }}}

% {{{ LaTeX document
\begin{document}

% {{{ Title page
\begin{titlepage}
\title{Notes on Plato's \book{Laches}}
\author{Peter Aronoff}
\date{June 2013}
\maketitle
\end{titlepage}
% }}}

% {{{ Characters and setting
\section{Characters and setting}

% {{{ Characters
\subsection{Characters}

The dialogue has a relatively large number of speakers.  In addition to Socrates, there are two fathers, Lysimachus and Melesias and two generals, Laches and Nicias.

The fathers first.  Lysimachus and Melesias both had famous politicians for fathers, Aristides the Just and Thucydides respectively.  But neither Lysimachus or Melesias has amounted to much.  They blame their fathers for this, believing that their fathers did not raise them properly, and they would like to do better for their sons.  (Oddly, \book{Meno} (94a and 94c) indicates that both men received an excellent education.  So it's not entirely clear what their complaint is or whether it's a valid one.)  Lysimachus belongs to the same \foreign{deme} as Socrates, and it turns out that he was good friends with Socrates' father as well.  According to Thucydides the historian (VIII.86.9), Melesias was a member of the oligarchy put in charge of Athens in 411.

The generals next.  Nicias was an important general and politican in the time before and during the Peloponessian War.  His policies during the Sicilian expedition, where he lost his life in 413, helped to cause the Athenian's terrible defeat there.  Laches was also an important general during the Peloponessian war.  He died in 418 at the battle of Manitinea.  Laches and Socrates fought side by side during the retreat at Delium.  When Alcibiades describes Socrates behavior at Delium (\book{Symposium} 221a--b), he specifies that Socrates was ``more prudent" (ἔμφρων) than Laches, but I'm not sure how much that evidence is worth.

% }}} Characters

% {{{ Setting
\subsection{Setting}

Lysis and Melesias have invited Laches and Nicias to view a display by a teacher of hoplite fighting.  (The teacher's name is Stesilaus.  He doesn't speak, and we know of him only from this dialogue.)  The dialogue doesn't provide details about exactly where the men are, but clearly it is a \foreign{gymnasium} of some kind.  Socrates just happens to be around, so we might guess that the display takes place at one of his favorite haunts.  But obviously this is speculative---and probably not important since Plato didn't make a point about it.
% }}} Setting
% }}} Characters and setting

% {{{ Introduction (178a--190d6)
\section{Introduction (178a--190d6)}

The introductory section of \book{Laches} is relatively long and involved.  Before we get to anything explicitly Socratic, we have the following.  Lysimachus makes a long speech inviting Nicias and Laches to help him and Melesias decide whether to train their sons in hoplite fighting.  Nicias and Laches agree, but Laches urges Lysimachus to involve Socrates as well.\footnote{The fathers invited Nicias and Laches.  Socrates luckily happens to be around.  His presence isn't very surprising though since he spent much of his life in and around training grounds.}  Lysimachus agrees, and he remembers that he and Socrates have an old family connection: Lysimachus was friendly with Socrates's father.  Nicias makes a speech in favor of training the boys in hoplite fighting, and Laches makes a speech against such training.  Lysimachus urges Socrates to break the tie, and only then do we reach something explicitly Socratic as Socrates argues that expertise and not numbers should decide the matter.  Socrates then moves the discussion in familiar ways until the generals are giving definitions of \word{bravery}.

Although much of this material is not explicitly Socratic, a number of familiar Socratic themes appear implicitly in the introduction.  First, Lysimachus and Melesias are very concerned for the proper education of their sons.  This is already Socratic.  But in addition they worry because their fathers were excellent men and well-renowned politicians, but they have not done so well.  In \book{Protagoras} Socrates argues that political excellence is not teachable, partly on the grounds that the most esteemed politicans were unable to pass on their success to their children (319e--320b).  Nicias also makes a claim that hints at Socratic intellectualism.  He argues that someone who has knowledge of hoplite fighting will fare better in battle (182b2--4).\footnote{Initially this may sound like simply a truism, but since the dialogue will eventually turn on whether or not bravery is a form of knowledge, I think it's an important bit of foreshadowing.}  Laches on the other hand argues that many trainers think and pretend that they have skill as fighters, but they they are shams.  This criticism implicitly recalls Socratic arguments showing that people think they know things that they actually do not.  Laches explicitly concludes by discussing people who ``think that they know" (184b4) something and the great ``slanders" (184b7) and ``envy" (184c1) that they face as a result.

Socrates becomes more actively involved at 184d5, and he guides things in a familiar direction.  Lysimachus, as I mentioned above, wants Socrates to break the tie between Nicias and Laches.  Socrates, however, objects on principle to making the decision by counting votes.  He immediately gets Melesias to agree that they must follow knowledge and not number (184d8-9).  Additionally he suggests that they don't even know what subject they need an expert in yet.  This suprises Melesias and Nicias, but Socrates makes a familiar argument:

\begin{enumerate}
    \item When someone seeks advice about a drug for their eyes, the decision is really about their eyes, not the drug.  And when someone seeks advice about a bit for a horse, the decision is about the horse not the bit.  Nicias agrees to both of these (185c5--d4).
    \item Therefore, more generally, when someone seeks advice about X for Y, the decision concerns Y and not truly X.  Nicias agrees to this too (185d5--8).\footnote{Unwisely, I think.}
    \item Therefore it is necessary to consider whether a fellow decision maker is an expert in the real goal of the discussion, not any proximate goal.  Again, Nicias agrees (185d9--12).
\end{enumerate}

Socrates then proposes that their consideration of training in hoplite fighting is ``for the sake of the soul of the young men" (185e1--2).  Therefore they must find an expert in training young people's souls not hoplite fighting.

Socrates assumes that if you an expert, then you should be able to prove it.  His basic demands for proof are either (1) the name of your teacher and proof of his \foreign{bona fides} or (2) evidence that you have taught others well, if you have not had a teacher but discovered things on your own.  Initially, it appears that Socrates will question Nicias and Laches about their status as experts, but the dialogue finds a clever way to avoid this.  Once they reach the point where it would be natural for Socrates to test them on these matters, Socrates suddenly thinks of an even better way to investigate: If they are really experts, then they should be able to give an account of the thing they are experts in.  So instead of asking them questions about their teachers or students, Socrates will simply ask them about the subject in question.

The subject in question is virtue (ἀρετή), and they immediately whittle that down to bravery.  Socrates worries that all of virtue would be too large a topic, so he proposes that they discuss a part of virtue (190c8-10).  It's important to notice that he is assuming here that virtue has parts.  This assumption will play an important role later in the dialogue.  Bravery is the obvious choice to pick, Socrates says, since they began all this by talking about hoplite fighting.

% }}} Introduction (178a--190d6)

% {{{ Definitions of bravery (190d7--199e12)
\section{Definitions of bravery (190d7--199e12)}

% {{{ First definition of bravery (190d7--192b8)
\subsection{First definition of bravery (190d7--192b8)}

The first definition is a famous example of an interlocutor not giving Socrates the right type of answer.  Laches defines bravery as ``being willing to fight off the enemy while remaining in formation and not fleeing" (190e5--6).   Importantly, Socrates doesn't deny that acting as Laches describes is brave.  In fact, he explicitly agrees that it is (191a1--3).\footnote{See \citet{nehamas1975} for why this passage is \emph{not} an example of someone giving a particular when Socrates wants a universal.  The answer Laches gives is not a particular, though it is too specific.}  Nevertheless, Socrates is not satisfied with this answer.

Socrates objects that Laches' definition does not cover \emph{all} brave actions.

\begin{enumerate}
    \item The Scythians and some Homeric heroes flee while fighting, but they are brave (191a8--10)
    \item The Spartans at Plataea fled, but they were brave (191b8--c5)
    \item Bravery doesn't concern only one type of fighting.  It isn't even limited to fighting.  People also exhibit bravery while sailing and in the face of poverty, illness and politics.  Even more, people can exhibit bravery in the face of good things too: desires or pleasures (191c8--e2).\footnote{Laches agrees even to these last examples.  I'm not sure that he should have though.  Yes, someone can make good decisions concerning pleasure and desire, but are such cases really examples of bravery?}
\end{enumerate}

Laches agrees to try again, but he still isn't 100\% clear about what Socrates wants in an example, so Socrates gives an example definition of his own for speed.  Socrates first argues that speed, like bravery, occurs in a variety of cases: running, playing an instrument, speaking, learning and many others.  He then explains that his definition must fit all of these cases and any others where speed occurs.  His definition is ``speed is a power accomplishing many things in little time" (192a10--b2).  Laches approves of the answer, and Socrates invites him to give a similar answer for bravery.  This brings us to our second definition.

% }}} First definition of bravery (190d7--192b8)

% {{{ Second definition of bravery (192b9--194c6)
\subsection{Second definition of bravery (192b9--194c6)}

Laches shows that he paid attention to Socrates' criticism in his second definition.  He defines bravery as ``a kind of endurance of the soul" (καρτερία τις τῆς ψυχῆς, 192b8).  He is clearly trying to maintain the spirit of his first definition while at the same time to satisfy Socrates' demand for generality.  It would seem at first glance that \word{endurance} is a broad enough idea to cover military cases, which are foremost in Laches' mind, as well as other examples of bravery, such as those that Socrates mentioned.  In addition, the addition ``of soul" may hint at the later explorations of more intellectualist definitions by Nicias.

Socrates criticizes this definition for being too inclusive.  This is the opposite of the first definition's problem, and so these two definitions thus serve as an object lesson in Socratic definition: He wants all but only F for his answer to ``What is F?".

Socrates makes his case that endurance is too broad in familiar ways.  First, Socrates introduces cases where the proposed \foreign{definiens} would not be fine (καλόν) or good (ἀγαθόν) as all virtues should be. Second, Socrates brings forward cases where the proposed \foreign{definiens} is insufficiently grand to count as a virtue.  In cases like this, Socrates relies on his interlocutors' shame at certain associations.\footnote{I should get examples of other cases like this, right?  In \book{Gorgias}, I know he uses a shameful case against either Callicles or Polus---the shit-bird.}

\begin{enumerate}
    \item First, Laches agrees that bravery is fine (192c5--6).
    \item Then Laches agrees that thoughtful endurance is fine and good.
    \item But Laches also agrees that thoughtless endurance is harmful and evil-doing (192d1--3).
    \item Since this kind of endurance is not fine and bravery is fine, Laches concedes that this kind of endurance is not brave.  Only wise endurance can be brave (192d4--12).
    \item But Socrates restricts wise endurance as well.  Socrates asks Laches whether wise endurance applied to spending money would count as bravery?  What about a doctor who wisely endures the pleading of a sick patient for something harmful?  In both cases, Laches denies that these are examples of bravery (192e--193a2).\footnote{I wonder whether Laches would have been smarter to deny that these are examples of \word{endurance}?}
\end{enumerate}

The first part of this argument has intuitive appeal to me.  I would redescribe it this way.  For any action type you name, some examples may be good and some examples may be bad.  To use another Socratic example (from \book{Republic} i), giving someone back the axe that he lent me is not necessarily a good idea.  If the lender is drunk and enraged when he asks for it back, perhaps I should hold onto it instead.  I would like to put this point as follows: Whatever action we name should be done \emph{with reasonable discrimination}.  The discrimination of circumstances is how I would understand Socrates' demand for intelligence and thoughtfulness here.

Socrates returns to military examples, but now forces a different objection on Laches.  What about cases where a person endures wisely and is willing to fight, but he is aware that he has a significant advantage over his opponent(s)?  Intuitively Laches agrees that the opponent who is at a disadvantage seems \emph{braver} than the one with the great advantages.  But that seems to run against the argument that wisdom and bravery go together since Socrates claims (and Laches agrees) that the person who is at a disadvange displays ``more foolish endurance" (193b2--3).

This argument troubles me.  Socrates seems to assume that wisdom tracks or guarantees success, but this strikes me as wrong.  To begin with, it assumes (or implies or relies on or flirts with?) the Socratic notion that the good man cannot be harmed.  To the degree that we find that doubtful, we might also doubt that failure implies a lack of intelligence or virtue.  Secondly, I wonder if Socrates takes too thin a view of the example he considers.  Perhaps the people who are outnumbered and certain to die are brave but not thoughtless.  They know that they are unlikely to win, but this does not turn them away.  As an example, consider the Greeks at Thermopylae.

However, Socrates immediately gives two other examples that may serve his argument better.  First he asks whether someone who is experienced in cavalry fighting is less brave than a novice if both enter a fight on horseback; he makes a similar case for slingers or archers (193b5--10).  Then he considers divers.  If someone inexperienced in deep-well diving is willing to dive to the bottom of the well, is that person braver than an experienced diver (193c2--8)?  Laches is inclined to say the less experienced person in each case is braver.

All of these examples allow Socrates to draw out a contradiction from Laches' agreements so far:

\begin{enumerate}
    \item Unwise daring and endurance was agreed to be shameful and harmful (193d1--3).
    \item But bravery was agreed to be fine (193d4--5).
    \item But now the foolish endurance is said to be bravery (193d6--8).
\end{enumerate}

Socrates does not explicitly demonstrate the contradiction, but Laches readily agrees that he doesn't like where the argument has led them.  He expresses frustration and says that he finds himself unable to say what he thinks he understands well.  Socrates is sympathetic, but says that they must ``be brave" and ``endure", and he suggests that they invite Nicias to help.  This will bring us to our third definition.

% }}} Second definition of bravery (192b9--194c6)

% {{{ Third definition of bravery (194c7--199e12)
\subsection{Third definition of bravery (194c7--199e12)}

Nicias offers the third and final definition of bravery: bravery is a kind of wisdom (194d8--9).  In response to questions by Socrates, Nicias clarifies that wisdom is ``knowledge of fearful and bold things" (194e11--195a1).  It's important that Nicias claims to have heard these ideas from Socrates.  Presumably then, we should pay special attention to them.  Socrates acknowledges that he has said such things, but he never says explicitly whether he favors Nicias' theory himself.  In addition, he ultimately disproves it---at least in Nicias' opinion---and so we should be careful not to say that what we see in \book{Laches} is Socrates' theory.

% }}} Third definition of bravery (194c7--199e12)

% {{{ Objections from Laches (195a2--196c6)
\subsubsection{Objections from Laches (195a2--196c6)}

Laches strenuously and rudely objects to Laches.  In fact, Laches is genuinely unpleasant in the remainder of the dialogue.  Socrates asks him more than once to try to take Nicias' ideas seriously and not simply insult the man, and Nicias himself says (fairly) that Laches seems more interested in Nicias being wrong than in discovering the truth.  My sense is that all of this helps to show how infuriating Socratic ideas and arguments could be to ordinary people, even when they didn't come from Socrates.

Laches attempts to use examples of people who know the right things but are not brave.  His first example is doctors: They know what is fearful, but (as Nicias agrees) they are not brave (195b3--6).  His next example is farmers, and Laches quickly generalizes to say that all skilled craftspeople know what is fearful or brave in their respective areas, but nevertheless none of them is brave (195b7--c1).

Nicias responds by arguing that doctors know about health or illness, but they don't know about what is (really?) fearful.  He unpacks this by saying that they don't know whether it is fearful for someone to be healthy or to be sick.  As he goes on to argue (and Laches agrees), some people are better off dead.  A doctor cannot know which people these are: his knowledge is limited to health and illness, but he cannot say whether you are the sort of person who should live and be healthy (195c7--d9).

Laches objection to this is that Nicias sets the bar too high: his brave people must be prophets (195e1--4).  Nicias however denies that even a prophet has the requisite knowledge.  Again Nicias argues that a prophet might know what is going to happen but not whether what is going to happen is or is not good for someone (195d8--196a2).

Laches becomes so frustrated at this point that he turns the argument over to Socrates.  He continues to snipe at Nicias as Socrates speaks to him, but he doesn't contribute positively to the debate after this point.  He is frustrated because Nicias seems to him to have made it impossible for anyone but perhaps ``some god" to meet the standards for bravery (196a6).  In addition, he feels that Nicias is arguing eristically and changing his stance and position merely to avoid being caught out.

Laches simply cannot take Nicias' proposals seriously, and I think that this is no accident nor simply an indictment of Laches' temper in debate.  The point, I think, is that this is how people react to the wildly counter-intuitive proposals of Socrates.  (Again, remember that Nicias has taken his definition from Socrates.)  The anger, disbelief and frustration result from the violent clash of common sense and Socratic argument.  What Nicias proposes flies in the face of much that Laches (and many other Athenians) took for granted.  It requires a significant amount of patience and openness even to consider it seriously.

% }}} Objections from Laches (195a2--196c6)

% {{{ Objections from Socrates (196c7--e12)
\subsubsection{Objections from Socrates (196c7--e12)}

Socrates objects first that according to Nicias' definition, animals would not be brave (196e2--7).  This runs counter to the Greek commonplace that a soldier might be ``brave as a lion" (cf. \book{Iliad} \foreign{passim}).  Socrates tries to make the problem even more pointed by saying that lion and stag and bull and monkey would all be equally brave (192e7--9).  (Presumably, one would object that a lion is obviously braver than a stag and so forth.)

Nicias, however, accepts this consequence without blinking.  He agrees that animals cannot be brave, and he adds that there is no bravery when there is fearlessness as a result of ignorance.  As Nicias puts it ``lack of fear and bravery are not the same thing" (197b2).  Nicias also explicitly acknowledges here that his use of the words \word{bravery} and \word{brave} run counter to what ``many people" (197b4) say.  This too does not bother him.

Socrates next objection, and the argument that ends the dialogue ultimately, is that Nicias contradicts himself in the following way.

\begin{enumerate}
    \item Nicias agrees that bravery is a part of virtue; other parts of virtue are temperance, justice and more (197e10--198b1).
    \item Nicias agrees that fear and courage have to do with the future.  One feels fear over something bad that one believes is going to happen.  One feels courage in cases where one believes that nothing bad is going to happen.  Nicias also agrees that according to him bravery is the knowledge of these matters (198b2--198c8).
    \item Changing tack, Socrates gets Nicias to agree that domains of knowledge, the knowledge is not split between times.  That is, if I know about medicine, my knowledge covers health as it was, as it is, and as it will be.  We don't find three different areas of knowledge here: just one across all three possible times (198c9--199a9).
    \item But since bravery is a kind of knowledge, it too must be about not merely future events, but current and past ones as well. Therefore, they must emend their definition of bravery to include present and past (199a10--c2).
    \item If we say this, then bravery will include knowledge of all good and evil, not merely that which is going to happen (199c3--199d3).
    \item Therefore, bravery will be all of virtue, not merely a part of virtue (199d4--199e5).
    \item But they already agreed bravery was only a part of virtue.  Therefore, they haven't found bravery (199e6--12).
\end{enumerate}

Now as many scholars\footnote{Get references here, please.} have said, Nicias agrees to the final step but he need not have.  He could have affirmed some version of the unity of the virtues in an effort to save the argument.  Arguably \book{Laches} is one of those Socratic dialogues that only apparently ends in \foreign{aporia}, but that has a strongly-implied solution.


% }}} Objections from Socrates (196c7--e12)

% }}} Definitions of bravery (190d7--199e12)

% {{{ Conclusion (199e13--201c5)
\section{Conclusion (199e13--201c5)}

After Laches and Nicias bicker further, Laches repeats his earlier advice to Lysimachus and Melesias: Don't let go of Socrates.  Nicias agrees, and he too encourages the fathers to seek Socrates' advice and help.  Lysimachus asks Socrates, but Socrates begs off, saying that it makes no sense to seek his advice since he was as clueless as the other two men.

As a counter-proposal, Socrates suggests that all of the adults, himself included, seek a teacher for themselves.  Once they find a good teacher, that person can also work with the young men.  But in Socrates' view, they all need help, even though they are older.  He invokes Homer to support the view that if a person is in need, shame is no help.  That is, they should not be ashamed to admit that they need help and don't know what virtue or bravery are.  Lysimachus agrees, and they agree to begin their search tomorrow.  The dialogue ends with Socrates promising to visit Lysimachus the next day for this purpose.

% }}} Conclusion (199e13--201c5)


\newpage
\bibliographystyle{apa}
\bibliography{plato}

\end{document}
% }}}
