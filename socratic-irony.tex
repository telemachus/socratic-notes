% [[- LaTeX prelude
\documentclass[12pt,letterpaper]{article}

\usepackage[no-math]{fontspec}
\setmainfont{Baskerville}

\usepackage[nolocalmarks]{polyglossia}
\setdefaultlanguage{english}
\setotherlanguage[variant=classic]{latin}
\setotherlanguage[variant=ancient]{greek}
\newfontfamily\greekfont[Script=Greek,Scale=MatchLowercase]{GFS Neohellenic}

\usepackage{fnpct}

\usepackage{titlesec}
\titleformat*{\section}{\large\bfseries}
\titleformat*{\subsection}{\bfseries}
\titleformat*{\subsubsection}{\bfseries}

\usepackage{parskip}
\usepackage{csquotes}
\DeclareAutoPunct{.,}
\renewcommand{\mkcitation}[1]{\footnote{#1}}
\renewcommand{\mktextquote}[6]{#1#2#4#5#3#6}

\usepackage[style=windycity,citetracker=context,backend=biber]{biblatex}
\addbibresource{plato.bib}

\usepackage{enumitem}
\setlist{noitemsep}
\usepackage[super]{nth}

\begin{hyphenrules}{latin}
    \hyphenation{}
\end{hyphenrules}

\begin{hyphenrules}{greek}
    \hyphenation{}
\end{hyphenrules}

\usepackage{fancyhdr}
\fancypagestyle{notes}{%
    \fancyhf{}
    \renewcommand{\headrulewidth}{0pt}
    \lhead{}
    \chead{\MakeUppercase{Notes on Socratic Irony}}
    \rhead{}
    \lfoot{}
    \cfoot{\thepage}
    \rfoot{}
}
\fancypagestyle{references}{%
    \fancyhf{}
    \renewcommand{\headrulewidth}{0pt}
    \lhead{}
    \chead{\MakeUppercase{References}}
    \rhead{}
    \lfoot{}
    \cfoot{\thepage}
    \rfoot{}
}

\newcommand{\MONTH}{%
  \ifcase\the\month
  \or January% 1
  \or February% 2
  \or March% 3
  \or April% 4
  \or May% 5
  \or June% 6
  \or July% 7
  \or August% 8
  \or September% 9
  \or October% 10
  \or November% 11
  \or December% 12
  \fi}
% -]] Latex prelude

% [[- LaTeX document
\begin{document}

% [[- Title page
% \begin{titlepage}
% \title{Notes on Socratic Irony}
% \author{Peter Aronoff}
% \date{February 2020--\MONTH\ \the\year}
% \maketitle
% \thispagestyle{empty}
% \end{titlepage}
% -]]

\pagestyle{notes}

% [[- Gregory Vlastos on Socratic Irony
\section*{Gregory Vlastos on Socratic Irony}

\begin{quote}

    Here we see a new form of irony, unprecedented in Greek literature to my knowledge, which is peculiarly Socratic.
For want of a better name, I shall call it ``complex irony'' to contrast it with the simple ironies I have been dealing with in this chapter heretofore.
In ``simple'' irony what is said just isn't what is meant: taken in its ordinary, commonly understood, sense the statement is simply false.
In ``complex'' irony what is said both is and isn't what is meant: its surface content is meant to be true in one sense, false in another.
Thus when Socrates says [in Xenophon's \textit{Memorabilia}] he is a ``procurer'' he does not, and yet does, mean what he says.
He obviously does not in the common, vulgar, sense of the word.
But nonetheless he does in another sense he gives the word \textit{ad hoc}, making it mean someone ``who makes the procured attractive to those whose company he is to keep'' (4.57) (31).

\end{quote}

Vlastos primarily considers four examples of complex Socratic irony (see 237 for the first three and Chapter 1 for the fourth):

\begin{enumerate}
    \item Socrates disavows, yet avows, knowledge.
    \item Socrates disavows, yet avows, the art of teaching virtue.
    \item Socrates disavows, yet avows, engaging in politics (\textgreek{πράττειν τὰ πολιτικά}).
    \item Socrates loves, yet does not love, young attractive boys.
\end{enumerate}

The fourth example is very different from the first three.
Perhaps for this reason, Vlastos does not discuss it in his long note at the end of the book.

Vlastos distinguishes simple from complex irony.
If a speaker uses simple irony, they communicate (and intend to communicate) what they mean by words that say (more or less) the opposite of what they mean.
When Socrates uses complex irony, he both means and does not mean what he says.
He means what he says if certain key words are taken in a specific way.
He does not mean what he says if those same key words are taken in another way.
Socrates means what he says when we take his words in his special ways.
If we understand the words in everyday senses, then he does not mean what he says.


As an example, let's consider what Vlastos says about Socrates and knowledge.
According to Vlastos, Socrates sincerely disavows knowledge\textsubscript{c}, but he also sincerely avows knowledge\textsubscript{e}.
Knowledge\textsubscript{c} is certain knowledge, and Socrates believes that he can only have such knowledge if he possesses definitions.
Since he does not have definitions for virtue terms, he does not think that he has knowledge\textsubscript{c}.
Knowledge\textsubscript{e} is elenctic knowledge: that is, it is knowledge produced by means of elenctic testing.
Socrates has this, but he is aware that it is only provisional.

Vlastos also argues that complex irony is a test of sorts for interlocutors.

\begin{quote}

    If you are young Alcibiades courted by Socrates you are left to your own devices to decide what to make of his riddling ironies.
If you go wrong and he sees you have gone wrong, he may not lift a finger to dispel your error, far less feel the obligation to know it out of your head.
If this were happening over trivia no great harm would be done.
But what if it concerned the most important matters—whether or not he loves you?
He says he does in that riddling way which leaves you free to take it one way though you are meant to take it in another, and when he sees you have gone wrong he lets it go.
What would you say?
Not, surely, that he does not care that you should know the truth, but that he cares more for something else: that if you are to come to the truth, it must be by yourself for yourself (44).

\end{quote}

What do I like about this account?
First, I agree that Socrates sometimes means what he says, in one sense, while not meaning what he says, in another sense.
Second, I agree that Socrates pushes interlocutors to discover things for themselves rather than simply telling them things.
(Nevertheless, I wonder if Vlastos underestimates how much help Socrates gives.
I think we can argue that he does various things to tip his hand and push the interlocutors towards insight, even if he never completely gives away the game.)
Third, I think that I agree broadly with Vlastos about the attitude Socrates takes towards teaching.
Finally, I may also agree broadly with Vlastos about Socrates and politics.

What do I not like about this account?
First, I still see this as deceptive speech.
Vlastos seems to admit as much when he says that Alcibiades must figure things out on his own.
Socrates will not and does not give him answers.
Second, I do not agree with the account Vlastos gives of Socratic irony about knowledge.
To begin with, I do not believe that the elenchus can produce anything Socrates would call knowledge.
In addition, it is not clear to me that Socrates has core principles that he has derived from the elenchus itself.
(He has core principles, but these seem to me to be \textit{a priori} for him—or at least untested.)
Third, I do not see why it helps to call this ``irony'' at all.
That said, I am beginning to think that I want to limit the term ``irony'' too much.
Clearly, the word has spread.
I should maybe accept that.
% -]] Gregory Vlastos on Socratic Irony

% [[- Criticisms of Vlastos on Socratic Irony
\section*{Criticisms of Vlastos on Socratic Irony}

% [[- Paula Gottlieb
\subsection*{Paula Gottlieb}

Paula Gottlieb argues against Vlastos that ``irony'' did not change meaning from ``deception'' to an urbane form of wit.
Instead, according to Gottlieb, ``Socrates has two audiences, the `in-crowd'…and the `outsiders'.''
\footcite[][278]{gottlieb-complexity-socratic-irony-1992} The in-crowd know what Socrates means, but Socrates deceives the outsiders.
Gottlieb continues, ``The outsiders are justified in feeling cheated.
They engage in discussion in good faith, while Socrates and his friends share a private joke at their expense.''
\footcite[][278]{gottlieb-complexity-socratic-irony-1992}
Finally, Gottlieb adds that Plato adds dramatic irony to the base layer of Socratic irony.
Plato writes for an audience of in-crowd, or aspiring in-crowd, people.
They will appreciate what Socrates is up to, they will find him funny and clever, but 
``Like Socrates, Plato makes no special effort to aid those will will not understand, and who will identify with Socrates' accusers and feel angry and betrayed, for example, I. F. Stone.''
\footcite[][278]{gottlieb-complexity-socratic-irony-1992}

In sum, says Gottlieb, 
``Irony has not changed.
It is just that one's view of irony depends on whether or not one is its butt.
In short, Socratic irony, like irony in general, is even more complex than Professor Vlastos allows.''
\footcite[][279]{gottlieb-complexity-socratic-irony-1992}

What do I like about this?
Gottlieb does a good job of expressing the anger and frustration of the outsiders who meet Socrates and read Plato.
I do not think that theirs is the only valid viewpoint, but I think that it is an important and valid viewpoint.

What do I not like about this?
Vlastos agrees with Gottlieb that Socrates offers no aid to people who are stuck, but she does not discuss the ethical or practical implications of this reticence.
Is Socratic teaching productive?
Is it kind?
Is it moral?
I wish Gottlieb had more seriously considered how we should evaluate Socrates's refusal to help the interlocutors more—especially since, as she says, Plato appears to do the same thing.
% -]] Paula Gottlieb

% [[- Jill Gordon
\subsection*{Jill Gordon}

According to Jill Gordon, Vlastos cannot explain many examples of (what appears to be) clear irony.
\footcite{gordon-against-vlastos-1996}
Vlastos limits his account to verbal irony, and he ignores the dramatic element in Plato's dialogues.
As an example, she discusses the appearance of Anytus in \textit{Meno}.
Some of the irony in this scene depends on dramatic context.
Socrates introduces Anytus by talking about how wonderful his father was right when they start arguing that many excellent fathers failed to raise excellent sons.
In addition, much of the irony depends on an even larger context.
The audience knows that Anytus will prosecute Socrates, and that awareness colors the way readers interpret the quarrel between the two men in this dialogue.

What do I think about this?
I agree that we should discuss cases of dramatic irony, such as the ones that Gordon raises.
But I am not sure whether we should consider them continuous with, much less the same thing as, the kinds of irony that Vlastos considers.
In other words, Vlastos might have been wise to separate dramatic from verbal irony.
(I have no idea if he did this intentionally.
I don't recall him saying anything one way or the other about dramatic irony.)
% -]] Jill Gordon

% [[- Terence Irwin
\subsection*{Terence Irwin}

Terence Irwin assimilates irony to deception. He writes,

\begin{quote}

    Vlastos argues that irony is a pervasive feature of Socrates' expression of some of his major doctrines.
We might take this to imply that Vlastos thinks Socrates is habitually insincere or deceptive; but Vlastos emphatically denies any such implication.
\footcite[][242]{irwin-socratic-puzzles-1992}
\end{quote}

I don't think that irony implies deception.
In many cases, speakers use irony to convey what they mean, and, in many cases, they succeed.
They do not intend to deceive, and they do not deceive anyone unintentionally.
We can call this ``successful irony'' or ``communicative irony.''
Successful irony occurs frequently in face-to-face contexts.
In addition to words, speakers convey their intention by tone of voice, facial expressions, and physical gestures.
Listeners can pick up on irony using all of these cues plus contextual clues.
For example, imagine two people exit a building together.
Both know that they must go on a long walk, and both see that it is pouring rain.
Neither has an umbrella or protective clothing.
One turns to the other and says, while rolling their eyes, ``Great weather for a walk, right?''
They both smile, and head out into the rain.
The speaker does not intend to deceive, and the audience is not deceived.

However, Nathan provides an argument that partially supports Irwin's view.
He suggests that speakers open themselves up to the possibility of deception whenever they use irony.
A listener may miss the cues that give away irony, and they may end up deceived.
Writing increases the possibility that the audience misses the irony since writing does not normally convey physical gesture and tone of voice.
Someone may reply that all communication runs the risk of misunderstanding and thus deception.
Let's assume this is right.
Nevertheless, Nathan believes that irony increases the risk, all things considered.
This seems fair, and, to that extent, Nathan supports Irwin.
Nevertheless, I am not sure that there is a \textit{significant} risk of deception in normal cases of irony.
However, perhaps Socratic irony is not a normal case?

Vlastos distinguishes simple irony from complex irony, and he thinks that complex irony is especially important for Socrates.
Simple irony functions like my example above: a person says one thing while meaning another, usually the opposite of what is said.
(In fact, I am now reminded, that example comes from Vlastos!
\footcite[][41, note 7]{vlastos-disavowal-1985}) In cases of complex irony, however, the speaker both means and does not mean what is said.
Complex irony takes one word and uses it in multiple meanings in a highly compressed fashion.
So, for example, Vlastos argues that Socrates uses ``know'' in two very different ways.
When Socrates disavows knowledge, he means that he lacks certain knowledge based on self-illuminating principles.
When Socrates says that he does know something, he means that he has true beliefs that are well-justified by the \textit{elenchus}.
Are such complex ironies more likely to deceive people?
Irwin thinks so.

Irwin uses Vlastos against himself in order to challenge complex irony.
Vlastos firmly believes that Socrates does not cheat.
Irwin therefore demands the following:
``if we are to show that Socrates' use of a term in different senses involves complex irony rather than illegitimate equivocation, we must suppose that Socrates is aware of the different senses and intends his hearer to be aware of them.''
\footcite[][246]{irwin-socratic-puzzles-1992}
However, Irwin grants that speakers must ``take the trouble (as many do not) to understand'' Socrates.
\footcite[][245]{irwin-socratic-puzzles-1992} Irwin examines two important examples of complex irony: knowledge and happiness.
If Irwin is right, Vlastos fails to show that Socrates uses these complex ironies in a way that satisfies Irwin's criteria.
We are left, therefore, to wonder whether Socrates himself is clear about the distinctions Vlastos draws and whether he expresses those distinctions clearly enough to his interlocutors.

I have two questions about Irwin's case.
First, I worry that Irwin demands something of Socrates that Socrates would not agree to.
Irwin requires that (i) Socrates understand his own multiple senses and (ii) that he intends the hearer to be aware of them.
If Socrates is deliberately testing his interlocutors, then he may be comfortable with a great deal more ambiguity and difficulty than Irwin allows for.
In this way, Socrates may not agree to Irwin's second demand.
Second, how can we decide whether to blame Socrates or his interlocutors?
When an interlocutor fails to understand a complex irony, should we say that the interlocutor didn't work hard enough or that Socrates didn't give clear enough clues?
I think this will prove very difficult to decide.

Let's look at complex irony about knowledge first.
In this example, Vlastos distinguishes certain knowledge (know\textsubscript{c}) from elenctic knowledge (know\textsubscript{e}), and Vlastos says that Socrates can consistently deny knowledge\textsubscript{c} even though he claims knowledge\textsubscript{e}.
Finally, Vlastos assimilates the distinction between know\textsubscript{c} and know\textsubscript{e} to what Socrates says in \textit{Apology} about divine knowledge versus human knowledge.

Irwin argues that Vlastos cannot explain what Socrates says about divine versus human knowledge with the distinction of know\textsubscript{c} and know\textsubscript{e}.
First, Irwin worries about the scope of know\textsubscript{e} compared with the scope of human knowledge in \textit{Apology}.
In \textit{Apology}, Socrates appears to restrict human knowledge to the awareness of one's own ignorance.
Vlastos, on the other hand, thinks that knowledge\textsubscript{e} contains more than awareness of one's own ignorance.
Second, Irwin argues that what Socrates says about the oracle does not make sense if Vlastos is right.
Socrates says that he shows that he is wiser than other people and, therefore, that the oracle is right.
The wisdom that Socrates has cannot be knowledge\textsubscript{c} because ``his awareness of his own ignorance does not constitute knowledge\textsubscript{c}'' (249).
But if Socrates is wiser ``in a broad sense that includes knowledge\textsubscript{c} and knowledge\textsubscript{e},'' then Socrates is right that he is wiser than other people but wrong to restrict his knowledge to awareness of his own ignorance (249).
(Note that Irwin's second argument depends on his interpretation of \textit{Apology} 20d-e, 21d, and 23b.
See also Irwin's ``Socratic Inquiry and Politics,'' 408-408 for more on this.)

Irwin also argues that Vlastos cannot explain how Socrates proves that others lack knowledge by means of the \textit{elenchus}.
Irwin appears to restrict the elenchus to definitions.
This is perhaps unfair, but let's grant it.
According to Irwin, Vlastos faces a problem here:
``we could still know\textsubscript{c} and know\textsubscript{e} lots of things about brave people and actions, without being able to give a definition of bravery.
[Socrates's] claim to have exposed his own and his interlocutor's ignorance of 'anything fine and good' seems to be grossly exaggerated, if Vlastos is right'' (249).
I am not sure why Irwin believes this.
It seems reasonable for Vlastos to reply as follows:

\begin{enumerate}
    \item  A person knows\textsubscript{c} \textit{x} iff they can give a definition of \textit{x}.
        Thus, nobody who is unable to define bravery can know\textsubscript{c} bravery.
        Thus, Laches, who is unable to define bravery does not know\textsubscript{c} bravery.
    \item Socrates restricts ``knowing anything fine and good'' to knowledge\textsubscript{c}.
        Thus, if Socrates lacks knowledge\textsubscript{c}, he does not know anything fine and good.
        Equally, if Laches cannot demonstrate knowledge\textsubscript{c}, then, to that extent, he appears not to know anything fine and good.
\end{enumerate}

Nathan suggests that I am not entitled to (1) above, but why not?
I suspect the answer is that (1) is not entailed by the account Vlastos gives of knowledge\textsubscript{c}.
(Possibly (1) is even inconsistent with the account Vlastos gives of knowledge\textsubscript{c}?
I am not sure, but I doubt this.)
According to Irwin, Vlastos gives the following account of knowledge\textsubscript{c}:

\begin{quote}
    A knows-c that \textit{p} if and only if (1) \textit{p} is true, (2) A is certain and infallible about the truth of \textit{p}, and (3) the evidence for \textit{p} is logically inconsistent with the falsity of \textit{p} (248).\footnotemark
    \footnotetext{On 248, in footnote 8, Irwin claims to ``derive'' his version of the two senses of know from Vlastos, ``Socrates' Disavowal of Knowledge,'' \textit{Philosophical Quarterly}, 35 (1985), 1-31, on page 18.
    This essays is also collected in \textit{Socratic Studies}, and the reference for that version is to pages 56-57.
    I am not sure whether I think Irwin fairly reconstructs the two senses that Vlastos gives ``know.''}
\end{quote}

Vlastos adds something that Irwin does not discuss, but it may help Irwin's case.
Vlastos adds something that Irwin does not discuss, but it may help Irwin's case.
According to Vlastos, Plato and Aristotle share the conviction that knowledge\textsubscript{c} is based on first principles that are, as Aristotle says, ``known through themselves''
(\textit{Prior Analytics} 64b34-36, quoted on Vlastos 52).
If this is right, then Irwin can argue that such an account leaves no room for knowledge\textsubscript{c} via definition.
These initial truths are self illuminating.
(Is this right?
I am confused.)

I see that I am very wrong about this all.
According to Vlastos, when Socrates performs an \textit{elenchus} on someone, he is testing that person for knowledge\textsubscript{e} rather than knowledge\textsubscript{c} (56-58).
Of course, this makes sense in one way: how could you test someone for knowledge\textsubscript{c} using the \textit{elenchus}.
But, in another sense, I am more confused than ever.
If Socrates demands definitions of other people (to show that they have knowledge\textsubscript{e}), and Socrates himself lacks definitions, then how does Vlastos think Socrates can maintain that he has knowledge\textsubscript{e}?
% -]] Terence Irwin

% -]] Criticisms of Vlastos on Socratic Irony

% [[- Iakovos Vasiliou on Conditional Irony
\section*{Iakovos Vasiliou on Conditional Irony}

\begin{itemize}
    \item I don't think that Vasiliou takes the ``slippery slope'' argument seriously on 457.
    \item Vasiliou considers irony at a student
        ``as an insult, albeit a lighter one than if [the teacher] said, `You are performing horribly today. We cannot continue'.'' (460)
        It is not clear to me that the irony *is* lighter.
        A lot would depend on further factors: how the people know each other, the details of the case, the tone of the teacher's voice, etc.
        But *prima facie* I think a teacher who delivers a bad report via irony adds mockery to bad news, and I don't see how that is lighter.
        (This is not to say that someone could not be equally or more cruel without irony, just that irony by itself is not in any way a diminishment of the insult.)
\end{itemize}
% -]] Iakovos Vasiliou on Conditional Irony

% [[- Bibliography
\newpage
\pagestyle{references}
\defbibfilter{sources}{%
    ( keyword=edition or keyword=translation or keyword=commentary )
}
\defbibfilter{secondary}{%
    keyword=secondary and ( keyword=irony or keyword=all )
}
\printbibliography[filter=sources,title={Ancient Sources: Editions, Translations, Commentaries}]
\printbibliography[filter=secondary,title=Secondary Literature]
% -]] Bibliography

\end{document}
% -]]
