% [[- LaTeX prelude
\documentclass[12pt,letterpaper]{article}

\usepackage[no-math]{fontspec}
\setmainfont{Baskerville}

\usepackage[base]{babel} % Ugh: https://tex.stackexchange.com/a/400994/29387
\usepackage[nolocalmarks]{polyglossia}
\setdefaultlanguage{english}
\setotherlanguage[variant=classic]{latin}
\setotherlanguage[variant=ancient]{greek}
\newfontfamily\greekfont[Script=Greek,Scale=MatchLowercase]{GFS Neohellenic}

\usepackage{titlesec}
\titleformat*{\section}{\large\bfseries}
\titleformat*{\subsection}{\bfseries}
\titleformat*{\subsubsection}{\bfseries}

\usepackage{parskip}
\usepackage{csquotes}
\usepackage[style=windycity,citetracker=context,backend=biber]{biblatex}
\addbibresource{plato.bib}

\usepackage{enumitem}
\setlist{noitemsep}
\usepackage[super]{nth}

\begin{hyphenrules}{latin}
    \hyphenation{}
\end{hyphenrules}

\begin{hyphenrules}{greek}
    \hyphenation{}
\end{hyphenrules}

\usepackage{fancyhdr}
\fancypagestyle{notes}{%
    \fancyhf{}
    \renewcommand{\headrulewidth}{0pt}
    \lhead{}
    \chead{\MakeUppercase{Aristotle on Irony}}
    \rhead{}
    \lfoot{}
    \cfoot{\thepage}
    \rfoot{}
}
\fancypagestyle{references}{%
    \fancyhf{}
    \renewcommand{\headrulewidth}{0pt}
    \lhead{}
    \chead{\MakeUppercase{References}}
    \rhead{}
    \lfoot{}
    \cfoot{\thepage}
    \rfoot{}
}
% -]] Latex prelude

% [[- LaTeX document
\begin{document}

% [[- Title page
% \begin{titlepage}
% \title{Arisotle on Irony}
% \author{Peter Aronoff}
% \maketitle
% \thispagestyle{empty}
% \end{titlepage}
% -]]

\pagestyle{notes}

% [[- Introduction
\section*{Introduction}

Although Aristotle does not seem to have considered irony enormously important, we do find scattered remarks of interest in several of his works and the works of his school.
Aristotle briefly mentions irony twice in his biological works.
But he discusses \textgreek{εἰρωνεία} at greater length in his \textit{Nicomachean Ethics}, \textit{Eudemian Ethics}, and \textit{Rhetoric}.
In addition, the school of Aristotle also discusses \textgreek{εἰρωνεία}.
Both \textit{Magna Moralia} and Theophrastus, in his \textit{Characters}, consider the \textgreek{εἴρων} and \textgreek{εἰρωνεία}.

Several scholars believe that Aristotle fundamentally changed the way that people view \textgreek{εἰρωνεία}.
\footcites[See, for example, ][]{ribbeck-begriff-eiron-1876}{gooch-socratic-irony-and-arisotles-eiron-1987}{lane-reconsidering-socratic-irony-2011}
According to these scholars, \textgreek{εἰρωνεία} refers to deception and mockery before Aristotle.
In Aristophanes, for example, the word is clearly negative and even abusive.
Eventually, however, \textgreek{εἰρωνεία} becomes more forgiving: the \textgreek{εἴρων} becomes an amusing or even elegant self-deprecator not a deceiver.
If these scholars are right, Aristotle instigated or accelerated this change.
% -]] Introduction

% [[- Aristotle
\section*{Aristotle}

% [[- Biological works
\subsection*{Biological works}

Aristotle refers to \textgreek{εῤωνεία} or the \textgreek{εἴρων} twice in his biological writings.
In \textit{Historia Animalium} 1.9, Aristotle says that eyebrows that curve towards the temples are a sign of a \textgreek{μωκοῦ καὶ εἴρωνος} (491b18--19).
In \textit{Physiognomics} 3, Aristotle says that the \textgreek{εἴρων} ``is fat around the face, with wrinkles round his eyes, and he wears a drowsy expression'' (808b28--29).

What can we make of these passages?
Zoja Pavlovskis says that these descriptions are ``unemotional'' and that Aristotle ``witholds judgment'' and is ``objective'' here.%
\footcite[][23]{pavlovskis-aristotle-horace-ironic-man-1968}
Perhaps this is true, but I wonder about judgmental implications, especially in the \textit{Physiognomics} passage.
That said, I agree with Pavlovskis's larger point: we probably cannot do very much with these two short mentions of the \textgreek{εἴρων}.
% -]] Biological works

% [[- Rhetorical works
\subsection*{Rhetorical works}

Aristotle mentions \textgreek{εἰρωνεία} several times in \textit{Rhetoric}.
In Book 2, while discussing characteristic objects of anger, Aristotle says that people who employ \textgreek{εἰρωνεία} anger serious people: ``for irony is contempt'' (\textgreek{καταφρονητικὸν γὰρ ἡ εἰρωνεία}, 1379b31).
Also in Book 2, while discussing proper objects of fear, Aristotle says that ``the mild and \textgreek{εἴρωνες} and clever'' are more dangerous than outspoken and quick-tempered people (1382b20).
(Why?
Because we cannot tell what these people will do: they are opaque.)
In Book 3, Aristotle says that certain kinds of elevated diction suit poetry, but he adds that they can also be used \textgreek{μετ᾿ εἰρωνείας, ὅπερ Γοργίας ἐποίει καὶ τὰ ἐν τῷ Φαίδρῳ} (1408b20).
Also in Book 3, Aristotle discusses \textgreek{εἰρωνεία} as a form of comedy.
In that context, he says that \textgreek{εἰρωνεία} is more frank, noble, fit for a free man (\textgreek{ἐλευθέριος}) than coarse or ribald humor (1409b7).
\footnote{Aristotle says something weird about the \textgreek{εἴρων} versus the ``blue comedian'': \textgreek{ὁ μὲν γὰρ αὑτοῦ ἕνεκα ποιεῖ τὸ γελοῖον, ὁ δὲ βωμολόχος ἑτέρου}. %
I do not understand what this means.}
Finally, again in Book 3, Aristotle briefly mentions \textgreek{εἰρωνεία} as one way of concluding a speech (1420a1).

What do texts tell us about Aristotle's attitude towards \textgreek{εἰρωνεία}?
Aristotle recognizes two sides of \textgreek{εἰρωνεία}.
On the one hand, Aristotle acknowledges the negative view: \textgreek{εἰρωνεία} makes people angry, and \textgreek{εἴρωνες} worry us because we cannot read them.
On the other hand, \textgreek{εἰρωνεία} seems more elegant than crude humor or straightforward deception.
It is elevated and can be very effective when used well.
In sum, Aristotle highlights the dangers and problems with \textgreek{εἰρνωεία}, but he also recognizes its less threatening or positive possibilities.
% -]] Rhetorical works

% [[- Ethical works
\subsection*{Ethical works}

If we want to think about Socrates, we should focus on what Aristotle says about \textgreek{εἰρωνεία} in his ethical works.
% -]] Ethical works

% Historia Animalium: 491b18--19
% Physiognomica: 808b28--29
% Rhetoric: 1379b31, 1382b20, 1408b20, 1409b7, 1420a1
% Eudemian Ethics: 1221a6, 1221a25-26, 1233b38-1234a3
% Nicomachean Ethics: 1127a13ff.
% Magna Moralia: 1186a24-27, 1193a28-36

% -]] Aristotle

% [[- Bibliography
\newpage\
\pagestyle{references}
\defbibfilter{sources}{%
    ( keyword=edition or keyword=translation or keyword=commentary )
}
\defbibfilter{secondary}{%
    keyword=secondary
}
\printbibliography[filter=sources,title={Ancient Sources: Editions, Translations, Commentaries}]
\printbibliography[filter=secondary,title=Secondary Literature]
% -]] Bibliography

\end{document}
% -]]
