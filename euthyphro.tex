\documentclass[11pt]{article}
\usepackage{fontspec}
\setmainfont[Ligatures={Common,TeX}]{Palatino}
\usepackage{url}
\usepackage{parskip}
\usepackage{natbib}
\bibpunct{(}{)}{;}{a}{,}{,}
\usepackage{titles}

\begin{document}

\begin{titlepage}
\title{Notes on Plato's \book{Euthyphro}}
\author{Peter Aronoff}
\date{April 2013}
\maketitle
\end{titlepage}

\section{Characters and setting}

\subsection{Characters}

The dialogue is a conversation between Socrates and Euthyphro.  They run
into each other outside the office of the (so--called) king archon (ἡ τοῦ
βασιλέως στοά).  Both men are there because they have court cases before
the king archon.

Euthyphro appears to be the same person mentioned in \book{Charmides}
396a2--b1 and 396d2--e3.\footnote{\citet{burnet1924} 85 and
\citet{bailly2003} 16.}  He knows Socrates reasonably well, and he is very
surprised to see Socrates at court at all.  When Socrates explains that he's
being charged with corruption of the youth and religious crimes, Euthyphro
assumes that Socrates' \foreign{daimonion}\footnote{For a complete list of
references to the \foreign{daimonion}, see \citet{bailly2003} 32.}
is the reason.  Perhaps partly because of the \foreign{daimonion}, he sees
Socrates as a kindred spirit: Euthyphro views himself as a prophet and an
expert on religious matters.  Other Athenians, according to Euthyphro, view
both Socrates and Euthyphro as unusual because of their religious beliefs.
Other than the two Platonic dialogues, we have no other evidence about
Euthyphro.\footnote{\citet{bailly2003} 16.}

\subsection{Setting}

Because the king archon was responsible for various state religious
functions, he also oversaw certain trials~--- namely those that had to do
with religion.\footnote{\citet{burnet1924} 82--83.}  That's why both
Socrates and Euthyphro need to see him in particular.  It seems that
Euthyphro has already finished with the king archon since at the end of the
dialogue he runs off.\footnote{\citet{burnet1924} 82.}

The prosecutor\footnote{This word is somewhat misleading.  In ancient
Athens, a prosecutor was a private citizen, not someone who worked for the
state.  Only individuals could start a court case, whether civil or
criminal.} and the defendant met the king archon twice before the trial.
The first meeting was very preliminary.  The prosecutor gave the defendent
a written version of the charge and paid a fee.\footnote
{\citet{brickhouse2004} 7--8.}  The second meeting with the king archon was
the \foreign{anakrisis}.  The \foreign{anakrisis} was more elaborate. The
defendant entered a formal plea, and both sides answered questions about the
case from the king archon.  At the end of this meeting the king archon
decided whether or not to send the case forward to trial.\footnote
{\citet{burnet1924} 82--83 and \citet{brickhouse2004} 8.}

\section{Preliminaries (2a--4a3)}

\subsection{Socrates' case (2a--3e7)}

Euthyphro is shocked to see Socrates at court and asks who is prosecuting
him and what the trial is about.  (He assumes that Socrates is the
defendant and not the prosecutor.)  Socrates says that Meletus is the
prosecutor, and they spend a few minutes not knowing who he is.  Then
Socrates tells Euthyphro the charges: harming the young men via religious
impiety.  Socrates first discusses harming the youth.  He claims to find
Meletus admirable because he is so concerned for the youth of the city, as
a politician should be.  However, Socrates is openly confused by the charge
that he is not irreligious.  He describes that part of the charges as
strange (ἄτοπα 3b1), but he does clarify that the specifics of that part of
the charge are (1) that he introduces new gods and (2) that he does not
believe in the gods of Athens.

Euthyphro immediately assocates the first part of the irreligion charge
with Socrates' \foreign{daimonion}.  He expresses sympathy with Socrates,
and says that people always laugh at and mock him too, when he offers
prophecy in public meetings.  Socrates notes that there's a big difference
between being laughed at and a trial for your life.

\subsection{Euthyphro's case (3e8-4e3)}

Moving on to Euthyphro's case, Euthyphro seems somewhere between coy and
proud of his case.\footnote{Geach \citeyearpar{geach1966} 369 takes
Euthyphro to be embarrassed and reluctant to admit who he is prosecuting.
Bailly on the other hand \citeyearpar{bailly2003} 37 sees him as
``cavalier and cocky".} However that may be, Euthyphro is prosecuting his
father for the murder of a hired hand. The details are such that (1) there
is room to doubt whether the crime was really murder or simply manslaughter
with callous indifference and (2) many Athenians would feel that Euthyphro
was wrong to pursue the case no matter what.  For such people, Euthyphro is
completely out of order (1) to prosecute his father no matter what and (2)
to prosecute in a murky case like this when the deceased was himself
a murderer and a hired hand.  The disapproval from such people leads
Euthyphro to make the assertion that opens him up to Socratic elenchus:

\begin{quote}
    ἀνόσιον γὰρ εἶναι τὸ ὑιὸν πατρὶ φόνου ἐπεξιέναι· κακῶς εἰδότες,
    ὦ Σώκρατες, τὸ θεῖον ὡς ἔχει τοῦ ὁσίου τε πέρι καὶ τοῦ ἀνοσίου
    (4d9-4e2).

    For [they think] that it is impious for a son to prosecute his father
    for murder. But, Socrates, they understand poorly how the divine stands
    concerning piety and impiety.
\end{quote}

Euthyphro strongly implies here that he believes that \emph{he} does
understand piety and impiety, and this is what Socrates will latch on to
and question.

\section{Definitions (4e3--15c10)}

\section{Conclusion (15c11--16a4)}

\bibliographystyle{apa}
\bibliography{plato}

\end{document}
