\documentclass[10pt]{article}
\usepackage{fontspec}
\setmainfont[Ligatures={Common,TeX}]{DejaVu Serif}
\usepackage{url}
\usepackage{parskip}
\usepackage{natbib}
\bibpunct{(}{)}{;}{a}{,}{,}
\usepackage{titles}

\begin{document}

\begin{titlepage}
\title{Notes on Plato's \book{Euthyphro}}
\author{Peter Aronoff}
\date{April 2013}
\maketitle
\end{titlepage}

\section{Characters and setting}

\subsection{Characters}

The dialogue is a conversation between two people, Socrates and Euthyphro.
They run into each other outside the office of the (so--called) king archon
(ἡ τοῦ βασιλέως στοά).  Both men are there because they have court cases
before the king archon.

Euthyphro appears to be the same person mentioned in \book{Charmides}
\citep[85]{burnet1924}.  He knows Socrates reasonably well, and he is very
surprised to see Socrates at court at all.  When Socrates explains that
he's being charged with corruption of the youth and religious crimes,
Euthyphro assumes that Socrates' \foreign{daimonion}\footnote{For
a complete list of references to the \foreign{daimonion}, see \citealp[page
32]{bailly2003}.} is the reason.  Perhaps partly because of the
\foreign{daimonion}, he sees Socrates as a kindred spirit: Euthyphro views
himself as a prophet and an expert on religious matters.  Other Athenians,
according to Euthyprho, view both Socrates and Euthyprho as unusual because
of their religious beliefs.  Other than the two Platonic dialogues, we have
no other evidence about Euthyphro \cite[16]{bailly2003}.

\subsection{Setting}

Because the king archon was responsible for various state religious
functions, he also oversaw trials concerned with matters of state religion
\citep[82--83]{burnet1924}.  That's why both Socrates and Euthyphro need to
see him in particular.  It seems that Euthyprho has already finished with
the king archon since at the end of the dialogue he runs off
\citep[82]{burnet1924}.

You see the king archon twice before the trial. The first meeting is very
preliminary. The prosecutor gives the defendent a written version of the
charge and pays a fee \citep[7--8]{brickhouse2004}. The second meeting with
the king archon is the \foreign{anakrisis}. The \foreign{anakrisis} is more
elaborate. The defendent enters a formal plea, and both sides answer
questions about the case from the king archon.  At the end of this meeting
the king archon decides whether or not to send the case forward to a formal
trial (\citealp[82--83]{burnet1924}; \citealp[8]{brickhouse2004}).

\bibliographystyle{plainnat}
\bibliography{plato}

\end{document}
