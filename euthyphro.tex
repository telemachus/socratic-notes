% {{{ LaTeX prelude
\documentclass[11pt]{article}
\usepackage{fontspec}
\setmainfont[Ligatures={Common,TeX}]{Palatino}
\usepackage{url}
\usepackage{parskip}
\usepackage{natbib}
\bibpunct{(}{)}{;}{a}{,}{,}
\usepackage{titles}
% }}}

% {{{ LaTeX document
\begin{document}

% {{{ Title page
\begin{titlepage}
\title{Notes on Plato's \book{Euthyphro}}
\author{Peter Aronoff}
\date{April 2013}
\maketitle
\end{titlepage}
% }}}

% {{{ Characters and setting
\section{Characters and setting}

\subsection{Characters}

The dialogue is a conversation between Socrates and Euthyphro.  They run
into each other outside the office of the (so--called) king archon (ἡ τοῦ
βασιλέως στοά).  Both men are there because they have court cases before
the king archon.

Euthyphro appears to be the same person mentioned in \book{Charmides}
396a2--b1 and 396d2--e3.\footnote{\citet{burnet1924} 85 and
\citet{bailly2003} 16.}  He knows Socrates reasonably well, and he is very
surprised to see Socrates at court at all.  When Socrates explains that
he's being charged with corruption of the youth and religious crimes,
Euthyphro assumes that Socrates' \foreign{daimonion}\footnote{For
a complete list of references to the \foreign{daimonion}, see
\citet{bailly2003} 32.} is the reason.  Perhaps partly because of the
\foreign{daimonion}, he sees Socrates as a kindred spirit: Euthyphro views
himself as a prophet and an expert on religious matters.  Other Athenians,
according to Euthyphro, view both Socrates and Euthyphro as unusual because
of their religious beliefs.  Other than the two Platonic dialogues, we have
no other evidence about Euthyphro.\footnote{\citet{bailly2003} 16.}

\subsection{Setting}

Because the king archon was responsible for various state religious
functions, he also oversaw certain trials~--- namely those that had to do
with religion.\footnote{\citet{burnet1924} 82--83.}  That's why both
Socrates and Euthyphro need to see him in particular.  It seems that
Euthyphro has already finished with the king archon since at the end of the
dialogue he runs off.\footnote{\citet{burnet1924} 82.}

The prosecutor\footnote{This word is somewhat misleading.  In ancient
Athens, a prosecutor was a private citizen, not someone who worked for the
state.  Only individuals could start a court case, whether civil or
criminal.} and the defendant met the king archon twice before the trial.
The first meeting was very preliminary.  The prosecutor gave the defendent
a written version of the charge and paid a fee.\footnote
{\citet{brickhouse2004} 7--8.}  The second meeting with the king archon was
the \foreign{anakrisis}.  The \foreign{anakrisis} was more elaborate.  The
defendant entered a formal plea, and both sides answered questions about
the case from the king archon.  At the end of this meeting the king archon
decided whether or not to send the case forward to trial.\footnote
{\citet{burnet1924} 82--83 and \citet{brickhouse2004} 8.}
% }}}

% {{{ Introduction (2--5d7)
\section{Introduction (2--5d7)}

% {{{ Socrates' case (2--3e6)
\subsection{Socrates' case (2--3e6)}

Euthyphro is shocked to see Socrates at court and asks who is prosecuting
him and what the trial is about.  (He assumes that Socrates is the
defendant and not the prosecutor.)  Socrates says that Meletus is the
prosecutor, and they spend a few minutes not knowing who he is.  Then
Socrates tells Euthyphro the charges: harming the young men via religious
impiety.  Socrates first discusses harming the youth.  He claims to find
Meletus admirable because he is so concerned for the youth of the city, as
a politician should be.  However, Socrates is openly confused by the charge
that he is not irreligious.  He describes that part of the charges as
strange (ἄτοπα 3b1), but he does clarify that the specifics of that part of
the charge are (1) that he introduces new gods and (2) that he does not
believe in the gods of Athens.

Euthyphro immediately assocates the first part of the irreligion charge
with Socrates' \foreign{daimonion}.  He expresses sympathy with Socrates,
and says that people always laugh at and mock him too, when he offers
prophecy in public meetings.  Socrates notes that there's a big difference
between being laughed at and a trial for your life.
% }}}

% {{{ Euthyphro's case (3e7--4e3)
\subsection{Euthyphro's case (3e7-4e3)}

Moving on to Euthyphro's case, Euthyphro seems somewhere between coy and
proud of his case.\footnote{\citet{geach1966} 369 takes Euthyphro to be
embarrassed and reluctant to admit who he is prosecuting.  Bailly on the
other hand \citeyearpar{bailly2003} 37 sees him as ``cavalier and cocky".
\citet{nehamas1975} 290 describes Euthyphro best as ``sensitive, touchy,
and, in a slightly perverse way, proud."}  However that may be, Euthyphro
is prosecuting his father for the murder of a hired hand.  The details are
such that (1) there is room to doubt whether the crime was really murder or
simply manslaughter with callous indifference and (2) many Athenians would
feel that Euthyphro was wrong to pursue the case no matter what.  For such
people, Euthyphro is completely out of order (1) to prosecute his father no
matter what and (2) to prosecute in a murky case like this when the
deceased was himself a murderer and a hired hand.  The disapproval from
such people leads Euthyphro to make the assertion that opens him up to
Socratic elenchus:

\begin{quote}
    ἀνόσιον γὰρ εἶναι τὸ ὑιὸν πατρὶ φόνου ἐπεξιέναι· κακῶς εἰδότες,
    ὦ Σώκρατες, τὸ θεῖον ὡς ἔχει τοῦ ὁσίου τε πέρι καὶ τοῦ ἀνοσίου
    (4d9-4e2).

    For [they think] that it is impious for a son to prosecute his father
    for murder.  But, Socrates, they understand poorly how the divine stands
    concerning piety and impiety.
\end{quote}

Euthyphro strongly implies here that he believes that \emph{he} does
understand piety and impiety, and this is what Socrates will latch on to
and question.
% }}}

% {{{ Pre-definition methodology (4e4--5d7)
\subsection{Pre-definition methodology 4e4--5d7}

Socrates says that he should become Euthyphro's student in order to answer
the charges against him.  There is certainly \emph{some} irony here, though
not everyone would agree on how to interpret it.

Without getting too far into the irony wars, I'll just say this.  My take
is that the primary irony is a form of dramatic irony, and the source of
the irony is not so much Socrates as Plato.  That is, the author Plato is
ironically suggesting to the reader that Euthyphro the character is likely
to become yet another victim of Socrates: he begins with confident
assertion, but we expect him to end in \foreign{aporia}.  By focusing on
Plato and his readers, rather than on Socrates the character or Socrates
the person, I think that we can avoid the nightmare of trying to decide
whether or not Socrates is sincere.

In any case, Socrates lays out the formal question he wants answered at
5c8ff.

\begin{quote}
    νῦν οὖν πρὸς Διὸς λέγε μοι ὃ νυνδὴ σαφῶς εἰδέναι διισχθρίζου, ποῖόν τι
    τὸ εὐσεβὲς φῂς εἶναι καὶ τὸ ἀσεβὲς καὶ περὶ φόνου καὶ περὶ τῶν ἄλλων;
    ἢ οὐ ταὐτόν ἐστιν ἐν πάσῃ πράξει τὸ ὅσιον αὐτὸ αὑτῷ, καὶ τὸ ἀνόσιον αὖ
    τοῦ μὲν ὁσίου παντὸς ἐναντίον, αὐτὸ δὲ αὑτῷ ὅμοιον καὶ ἔχον μίαν τινὰ
    ἰδέαν κατὰ τὴν ἀνοσιότητα πᾶν ὅτιπερ ἂν μέλλῃ ἀνόσιον εἶναι;

    Therefore now, by Zeus, tell me what you were just now confidently
    asserting that you knew clearly: What sort of thing do you say is the
    holy and the unholy, both concerning murder and concerning everything
    else? Or isn't the pious the same thing in every activity, itself to
    itself and the impious in turn opposite of every pious, but itself the
    same as itself and having some single form insofar as impiety,
    everything which will be impious?
\end{quote}

My translation is neither beautiful nor especially clear, but as far as
that goes, it closely resembles the Greek.  In this passage Socrates lays
out a number of conditions that he wants Euthyphro's definition to meet.

\begin{itemize}
    \item The definition should be universal.  However Euthyphro defines
        pious, that definition should cover both murder and everything else
        (5d1).  Piety should be ``the same in every activity" (5d1).
    \item The two terms \word{piety} and \word{impiety} should be
        opposites (5d2).  This seems to suggest that they should
        \emph{exclude} one another, perhaps?
    \item Piety and impiety should ``have some single kind"\footnote{Many
        people get hot and bothered because Plato uses the word ἰδέα
        (\foreign{idea}), and this is one of his terms for \word{Form}
        in its technical sense.  Most scholars assume that this dialogue
        precedes the full theory of forms, so this use causes some anguish.
        See \citet{burnet1924} 111.} (5d4) inasmuch as they are what they
        are.\footnote{I'm unsure whether this third condition is really a
        restatement of the first or a new point.}
\end{itemize}

Note that the universality that Socrates demands here is \emph{not}
primarily a matter of words.  He's not looking to legislate or to discover
how we talk about things.  We can be sure of that because in this very
passage, he happily uses two different sets of opposed terms for what he
wants defined: \foreign{eusebes} versus \foreign{asebes} and
\foreign{hosion} versus \foreign{anhosion}.

Although the confidence of Euthyphro's response may surprise us, it is
characteristic of similar moments in other dialogues.  Euthyphro immediately
answers ``Absolutely" (Πάντως δήπου, 5d6).  We might have expected him to
ask ``Huh?" or at least to request clarification from Socrates.\footnote
{\citet{nehamas1975} 303.}  However, Euthyphro's answer is unsurprisingly
in the light of other Platonic dialogues.  As John Burnet says:

\begin{quote}
    Euthyphro appears to be quite familiar with the terminology used by
    Socrates, and accepts it without demur.  That becomes all the more
    striking when we find him boggling later at much more elementary
    things.  Plato always represents the matter in this way.  No one ever
    hesitates for a moment when Socrates talks of ἰδέαι and εἴδη, and
    Socrates never finds it necessary to explain the terms.\footnote
    {\citet{burnet1924} 112.}
\end{quote}

Similarly, Alexander Nehamas writes:

\begin{quote}
    [I]t is notorious that Socrates' interlocutors (even Meno) are quite
    willing to answer affirmatively...On this view, the existence of
    universals is taken for granted in the Socratic dialogues.\footnote
    {\citet{nehamas1975} 303--304.}
\end{quote}

Scholars disagree about many things here:

\begin{itemize}
    \item Why do the interlocutors agree so easily?
    \item What do they think they're agreeing to?
    \item If they understand what Socrates wants, why do they do so poorly
        at satisfying his demands?
\end{itemize}
% }}}
% }}}

% {{{ Definitions (5d8--15c10)
\section{Definitions (5d8--15c10)}

% {{{ First definition of piety (5d8--6a5)
\subsection{First definition of piety (5d8--6a5)}

Here is Euthyphro's first definition:

\begin{quote}
    τὸ μὲν ὅσιόν ἐστιν ὅπερ ἐγὼ νῦν ποιῶ, τῷ ἀδικοῦντι ἢ περὶ φόνους ἢ περὶ
    ἱερῶν κλοπὰς ἤ τι ἄλλο τῶν τοιούτων ἐξαμαρτάνοντι ἐπεξιέναι, ἐάντε
    πατὴρ ὢν τυγχάνῃ ἐάντε μήτηρ ἐάντε ἄλλος ὁστισοῦν, τὸ δὲ μὴ ἐπεξιέναι
    ἀνόσιον (5d8--5e1).

    Piety is exactly what I am doing now, to prosecute someone who commits
    a crime, whether murder or theft of temples or anything at all of this
    kind, whether the criminal happens to be father or mother or anyone at
    all; and not to prosecute is impiety.
\end{quote}

The idea seems to be that piety requires you to pursue \emph{anyone} who
commits religious crimes.

Unfortunately Euthyphro derails the conversation unintentionally right
away.  He says that people give him a hard time for prosecuting his father,
but they contradict themselves because they're perfectly happy to hear
stories about Zeus and Chronos physically assaulting \emph{their} fathers.
Socrates wonders if maybe this sort of thing is why he's in trouble: When
he hears such stories, he protests.  This line of argument will come back
to haunt Euthyphro when Socrates attacks his second definition.
% }}}

% {{{ Socrates rejects the first definition (6a6--6d10)
\subsection{Socrates rejects the first definition (6a6--6d10)}

Socrates puts aside, for the moment, his dislike of such stories in order
to get to his more basic complaint: prosecuting religious criminals may be
pious, but according to Euthyphro many other things are also pious.  What
Socrates wanted was the single pious: some one thing that we can use as
a yardstick whenever we need to judge whether anything else is pious or
impious.  Euthyphro's first definition, by his own admission, cannot serve
this purpose, so Socrates rejects it.
% }}}

% {{{ Second definition of piety (6e11--7a1)
\subsection{Second definition of piety (6e11--7a1)}

Euthyphro's second definition is simpler and more general:

\begin{quote}
    τὸ μὲν τοῖς θεοῖς προσφιλὲς ὅσιον, τὸ δὲ μὴ προσφιλὲς ἀνόσιον
    (6e11--7a1).

    That which is pleasing to the gods is pious, and that which is not
    pleasing to the gods is impious.
\end{quote}
% }}}

% {{{ Socrates refutes the second definition (7a2--8b6)
\subsection{Socrates refutes the second definition (7a2--8b6)}

Socrates praises the form of Euthyphro's second definition, but he attacks
its content.  He argues as follows:

\begin{enumerate}
    \item The gods war against each other, they disagree and there is
        enmity between them.
    \item Among humans, only \emph{some} disagreements lead to this kind of
        fighting and enmity.  For example, if people disagree about
        a number or measurement, that doesn't lead to such a fight.  Those
        kinds of problems are solved easily enough by counting, measuring
        or weighing.  But disagreements about justice and injustice, beauty
        and ugliness, good and bad~--- those disagreements lead to fighting
        and enmity.  They do so because in those cases people ``are unable
        to come to a sufficient agreement" (7d3--4).
    \item The gods quarrels and enmity results from these same sorts of
        disagreements.  Hence, the gods disagree about moral matters.
    \item What each god believes to be beautiful, good and just, that god
        loves.  They hate the opposites of such things.
    \item Therefore, \emph{the same things} are both hated and loved by the
        gods.
    \item Therefore, \emph{the same things} are both pious and impious.
\end{enumerate}

Socrates uses Euthyphro's earlier contention that the gods fight among
themselves against him.  In particular, Socrates comes back to the case of
Zeus, Chronos and Olympus and their relationship to Euthyphro's prosecution
of his father.  Socrates points out that Zeus may approve of what Euthyphro
is doing, as a son who had to punish his father, but that Chronos and
Olympus would not approve, as fathers who were punished by their sons.
Thus, we are unable to judge whether or not Euthyphro's controversial
action is pious or not solely by this yardstick.  Using divine approval as
a test gives contradictory results in this case.  Anything that gives
contradictory results isn't what Socrates asked for.  So, Euthyphro's
second definition fails as well.
% }}}

% {{{ Digression about the second definition (8b7--9b11)
\subsection{Digression about the second definition (8b7--9b11)}

Euthyphro tries at first to save part of his definition.  He argues that
all the gods agree about one thing at least: that whoever kills someone
unjustly should be punished.  Without agreeing or disagreeing, Socrates
picks up Euthyphro's claim and argues as follows:

\begin{enumerate}
    \item Every person agrees that wrongdoers deserve punishment, but they
        disagree about who the wrongdoers are.  In particular, they deny
        that they are wrongdoers.
    \item If the gods fight as Euthyphro says that they do, then they must
        be in the same situation.  They all agree that wrongdoers deserve
        punishment, but they fight over who the wrongdoers are.
    \item Therefore, if Euthyphro wants to maintain his argument on behalf
        of his prosecution of his father, he would need to show that
        \emph{every} god agrees that the day-laborer who died as a result
        of his father's neglect died unjustly (ἀκίκως 9a3), i.e. culpably.
\end{enumerate}
% }}}

% {{{ Second definition of piety reformulated (9c1--9e9)
\subsection{Second definition of piety reformulated (9c1--9e9)}

Socrates realizes that he doesn't actually care about Euthyphro's specific
case against his father, so he relieves him of that burden.  As he says,
even if Euthyphro convinces him that all the gods agree about this one
case, Socrates would only have learned about \emph{one case} not about
piety in general, as he wants.

At this point, Socrates takes over slightly, and he feeds Euthyphro his
next definition:

\begin{quote}
    ὃ μὲν ἂν πάντες οἱ θεοὶ μισῶσιν ἀνόσιόν ἐστιν, ὃ δ᾽ἂν φιλῶσιν, ὅσιον,
    ὃ δ᾽ἂν οἱ μὲν φιλῶσιν οἱ δὲ μισῶσιν, οὐδέτερα ἢ ἀμφότερα (9d2--3).

    Whatever all the gods hate is impious, and whatever they (all) love
    (is) pious; but whatever some love and some hate, this is neither or
    both.
\end{quote}

Socrates makes sure that Euthyphro accepts this new definition, and at
9e1--3 Euthyphro restates it himself.  Thereafter, it becomes Euthyphro's
argument once again.\footnote{Is there something worth saying here about
``Say what you believe"?}
% }}}

% {{{ Socrates refutes all forms of the second definition (10a1--11b5)
\subsection{Socrates refutes all forms of the second definition (10a1--11b5)}

This next argument is probably the dialogue's most important legacy.  In
this refutation, Socrates presents what is commonly known as ``The
Euthyphro dilemma".  First, I'll lay out the dilemma.  Then I'll briefly
discuss the difficulties in interpreting Plato's text here.

Here is what Socrates says initially:

\begin{quote}
    ἆρα τὸ ὅσιον, ὅτι ὅσιόν ἐστι, φιλεῖται ὑπὸ τῶν θεῶν, ἢ ὅτι φιλεῖται,
    ὅσιόν ἐστιν; (12a2--3)

    Is the pious loved by the gods because it is pious, or is it pious
    because it it loved (by the gods)?
\end{quote}

This suggests the following dilemma:

\begin{enumerate}
    \item Either the gods love what is pious because it is pious or it is
        pious because they love it.
    \item If it is pious because they love it, then that seems arbitrary.
        What if they loved murder?  Would murder then become pious?
    \item If they love it because it is pious, then we will probably need
        to care more about \emph{why} they love it than simply \emph{that}
        they love it.
    \item So either the gods are arbitrary overlords or they are themselves
        merely loving what is independently pious.  Either way, the gods
        cannot answer our ethical questions.
\end{enumerate}

This dilemma has had a great deal of historical importance inside religious
considerations of ethics and outside of religion.

Unfortunately, Socrates doesn't stop there.  Euthyphro doesn't understand
his initial formulation of the question.  In order to clarify things,
Socrates goes on a tangent about how we describe various things.  He
\emph{seems} to be contrasting either active versus passive or states
versus processes.  Either way, the language and the logic of his
(so-called) clarification\footnote{As \cite{cohen1971} 2 says, ``Socrates
agrees to `speak more plainly' (σαφέστερον φράσαι) and then produces the
most baffling part of the argument."} has only made readers scratch their
heads.  I'm going to ignore that whole can of worms.  For an excellent and
detailed breakdown of what Socrates says and what it means, see
\citet{cohen1971}.
% }}}

% {{{ Aporetic interlude: Socrates' effect on beliefs (11b6--11e1)
\subsection{Aporetic interlude: Socrates' effect on beliefs (11b6--11e1)}
After Socrates refutes Euthyphro's adjusted second definition, he again
asks that Euthyphro say ``zealously" (προθύμως) what piety and impiety are
(11b4--5).  Euthyphro reacts this time with clear frustration and
confusion:

\begin{quote}
    Ἀλλ᾽, ὦ Σώκρατες, οὐκ ἔχω ἔγωγε ὅπως σοι εἴπω ὃ νοῶ· περιέρχεται γάρ
    πως ἡμῖν ἀεὶ ὃ ἂν προθώμεθα καὶ οὐκ ἐθέλει μένειν ὅπου ἂν ἱδρυσώμεθα
    αὐτό (11b6--8).

    Well, Socrates, I don't have any way to say to you what I know. For
    whatever I put forth moves around somehow despite me, and it refuses
    to stay where we set it.
\end{quote}

The general tenor of this complaint is familiar from other dialogues.
Socrates picks up on the idea of movement and says that if the moving
arguments were his, Euthyphro could joke that he was like his ancestor
Daedalus.  This recalls \book{Meno} 97d where a similar reference to
Daedalus can be found.  The specific idea is that Daedalus was able to
create moving statues, and Socrates can make arguments move~--- even those
of other people.  Under his influence, Euthyphro's arguments don't ``stay
where he places them" and he's unable to make a stable definition of piety.
From Euthyphro's point of view, it's \emph{Socrates} who is to blame for
the failure of all the previous definitions.

This aporetic interlude is important in another way: Some scholars argue
that Plato frequently puts arguments that he favors \emph{after} such
interludes.\footnote{\citet{mcpherran1992} 221.  See also his references
in footnote 14 (235--6).}  This would give readers extra reason to pay
attention to the arguments after the interlude.
% }}}

% {{{ Third definition of piety (11e1--12e9)
\section{Third definition of piety (11e1--12e9)}

Socrates accuses Euthyphro of being lazy at this point, but he also
offers to help and this may be a tacit acknowledgement of Euthyphro's
frustration or limits.

Socrates suggests that piety is a part (or kind?) of justice. Initially,
his wording confuses Euthyphro:

\begin{quote}
    ΣΩ. ἰδὲ γὰρ εἰ οὐκ ἀναγκαῖόν σοι δοκεῖ δίκαιον εἶναι πᾶν τὸ ὅσιον.

    ΕΥΘ. Ἔμοιγε.

    ΣΩ. Ἆρ᾽οὖν καὶ πᾶν τὸ δίκαιον ὅσιον; ἢ τὸ μὲν ὅσιον πᾶν δίκαιον, τὸ δὲ
    δίκαιον οὐ πᾶν ὅσιον, ἀλλὰ τὸ μὲν αὐτοῦ ὅσιον, τὸ δὲ τι καὶ ἄλλο

    ΕΥΘ. Οὐχ ἕπομαι, ὦ Σώκρατες, τοῖς λεγομένοις (11e4--12a3).

    SO. Consider whether it doesn't seem necessary to you for everything
    pious to be just.

    EUTH. It seems so to me.

    SO. Therefore is everything just also pious? Or is the pious all just,
    but the just is not all pious, but some of what is just is pious and
    some is something else.

    EUTH. I don't follow your argument, Socrates.
\end{quote}

After going through a few examples, Socrates gets his point across to
Euthyphro.\footnote{\citet{bailly2003} 90 writes the following about
Socrates' approach at this point:

\begin{quote}
    The method of defining a general area into which an object falls, then
    more and more precisely delineating the object is called ``division,"
    and may be introduced into philosophy for the first time in the
    \book{Euthyphro} (the \book{Sophist} has extended instances and
    discussion of it).
\end{quote}

I'm not sure I buy this, but it's an interesting comparison.}  Socrates
then asks Euthyphro to specify what part of justice piety is, and Euthyphro
answers:

\begin{quote}
    Τοῦτο τοίνυν ἔμοιγε, ὦ Σώκρατες, τὸ μέρος τοῦ δικαίου εἶναι εὐσεβές τε
    καὶ ὅσιον, τὸ περὶ τὴν τῶν θεῶν θεραπείαν, τὸ δὲ περὶ τὴν τῶν ἀνθρώπων
    τὸ λοιπὸν εἶναι τοῦ δικαίου μέρος (12e6--9).

    Therefore, Socrates, holiness and piety appear to me to be this part of
    justice: the part concerned with the service towards the gods, and the
    part concerned with service towards people is the remaining part of
    justice.
\end{quote}
% }}}

% {{{ Socrates refutes the third definition (12e10--14a10)
\subsection{Socrates refutes the third definition (12e10--14a10)}

Socrates praises the form of Euthyphro's latest answer, but he says that he
needs ``some small" clarification (13a1).\footnote{\citet{bailly2003} 96
compares \book{Protagoras} 329b.  See also Columbo's ``Just one more
thing."}  The small thing concerns the word θεραπεία.\footnote{Reasonable
translations include \defn{service}, \defn{care}, \defn{caretaking}.
However, it's not the easiest idea to convey directly in English with one
or two words.  θεραπεία is a verbal noun derived from θεραπεύω.  The verb
means \defn{attend to}, \defn{serve}, \defn{care for}, \defn{provide for},
\defn{treat medically}.  The verb or noun might be used in a variety of
situations and across a variety of relationships.  So, for example, humans
\defn{attend to} gods, but gods \defn{provide for} humans.  Doctors
\defn{care for} or \defn{see to} patients, trainers \defn{care for} animals,
and farmers \defn{tend to} their crops~--- all these ideas can be expressed
by this one word.}  Socrates expresses shock that Euthyphro uses this word,
and he assumes that Euthyphro can't mean the normal thing by it.  Socrates
then takes a characteristic turn.  He quickly lists three commonplace
examples of θεραπεία: care for horses, dogs and cattle.  He then asks
Euthyphro if θεραπεία for the gods is like these cases.  Euthyphro says
that it is the same kind of thing.

Socrates continues by specifying how he understands θεραπεία in the normal
cases.  In all such cases, the person performing the θεραπεία aims ``at
some good and benefit for the one being cared for (τοῦ θεραπευομένου)"
(13b7--10).  Euthyphro agrees, after which Socrates asks him to explain
what benefit men do for gods when they exhibit piety.  Euthyphro naturally
shrinks from this implication, and the trap is sprung.

Euthyphro clarifies that the θεραπεία is not one where people
\emph{improve} the gods; it's more like slavery.  Socrates accepts this,
using the term ὑπηρετική.\footnote{This term also refers to \defn{service},
but the connotation is clearly servile or from lesser to better.
Presumably, Euthyphro thinks this will save his definition from seeming to
be impious itself.}  Socrates then effectively returns to his earlier line
of argument, though now without the embarrassing connotations.  He says
that in the case of medicine or shipbuilding there is a clear end product
that the service aims at, and Euthyphro agrees.  Socrates asks Euthyphro to
tell him the parallel goal of piety.

Euthyphro fumbles around, and Socrates does not hide his frustration.
First, Euthyphro returns to vague answers, telling Socrates that the
product of piety are ``many fine things" (13e14).  Socrates agrees, but he
explains generals and farmers \emph{also} produce many fine things, and yet
we can still say in summary that they aim at victory and the production of
crops.  What, Socrates asks again, is the summary end product of piety?
This will lead us to the fourth and final definition of piety.
% }}}

% {{{ Fourth definition of piety (14a11--14d4)
\subsection{Fourth definition of piety (14a11-14d4)}

At this point, Euthyphro repeats his earlier disclaimer that it is
a difficult thing to be specific, but he will say that if you know how to
say and do things pleasing to the gods, that is piety.  He adds that such
piety saves both individuals and cities.  Socrates objects to the length of
Euthyphro's response for the second time, and he also says that Euthyphro
was just on the cusp of teaching him what piety is, but that he ``turned
away" (14c1--2).\footnote{\citet{mcpherran1992} leans heavily on this
passage when he argues that the dialogue suggests a definition of piety
that Socrates himself approves of.}  Nevertheless, although Socrates is
clearly not thrilled, he explains that he must follow where the respondent
leads (14c3-5).

After a series of short questions and answers, Socrates describes
Euthyphro's latest definition makes piety a kind of commerce with the
gods.\footnote{The Greek term he uses is ἐμπορική.  Euthyphro is not
thrilled with the connotations, but he accepts that word.}  Socrates' point
is that according to Euthyphro's latest explanation, piety is a matter of
giving the gods what they want in exchange for what we want.  Euthyphro
probably takes himself to be agreeing to a series of commonplaces. After
all, what he describes is a standard \foreign{do ut des} view of
religion.\footnote{\citet{bailly2003} 103--104.}
% }}}

% {{{ Socrates refutes the fourth definition (14d5--15c10)
% }}}

% {{{ Conclusion (15c11--16a4)
\section{Conclusion (15c11--16a4)}
% }}}

\newpage
\bibliographystyle{apa}
\bibliography{plato}

\end{document}
% }}}
