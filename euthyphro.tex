% [[- LaTeX prelude
\documentclass[11pt]{article}
\usepackage{fontspec}
\setmainfont[Ligatures={Common,TeX}]{Baskerville}
\newfontfamily\g[Ligatures={Common,TeX}]{Times New Roman}
\usepackage{url}
\usepackage{parskip}
\usepackage{natbib}
\bibpunct{(}{)}{;}{a}{,}{,}
\usepackage{titles}
\usepackage{enumitem}
\setlist{nosep}
% -]]

% [[- LaTeX document
\begin{document}

% [[- Title page
\begin{titlepage}
\title{Notes on Plato's \book{Euthyphro}}
\author{Peter Aronoff}
\date{April 2013--August 2017}
\maketitle
\thispagestyle{empty}
\end{titlepage}
% -]]

% [[- Characters and setting
\section{Characters and setting}

\subsection{Characters}

The dialogue is a conversation between Socrates and Euthyphro.  They run into each other outside the office of the (so--called) king archon ({\g ἡ τοῦ βασιλέως στοά}).  Both men are there because they have court cases before the king archon.

Euthyphro appears to be the same person mentioned in \book{Cratylus} 396a2--b1 and 396d2--e3.\footnote{\citet{burnet1924} 85, \citet{bailly2003} 16, and \citet{nails2002} 152--153.}  Nails claims that Euthyphro should be roughly twenty in \book{Cratylus} and mid-forties in \book{Euthyphro}.  Euthyphro knows Socrates reasonably well, and he is very surprised to see Socrates at court at all.  When Socrates explains that he's being charged with corruption of the youth and religious crimes, Euthyphro assumes that Socrates's \foreign{daimonion}\footnote{For a list of references to the \foreign{daimonion}, see \citet{bailly2003} 32.} is the reason.  Perhaps partly because of the \foreign{daimonion}, he sees Socrates as a kindred spirit: Euthyphro views himself as a prophet and an expert on religious matters.  Other Athenians, according to Euthyphro, view both Socrates and Euthyphro as unusual because of their religious beliefs.  Diogenes Laertius claims (Book II 29) that Socrates convinced Euthyphro not to prosecute his father, but we can't know if this is true or not.

\subsection{Setting}

Because the king archon was responsible for various state religious functions, he also oversaw trials that had to do with religion.\footnote{\citet{burnet1924} 82--83.}  That's why both Socrates and Euthyphro need to see him in particular.  It seems that Euthyphro has already finished with the king archon since at the end of the dialogue he runs off.\footnote{\citet{burnet1924} 82.}

The prosecutor\footnote{This word is somewhat misleading.  In ancient Athens, a prosecutor was a private citizen, not someone who worked for the state.  Only individuals could start a court case, whether civil or criminal.} and the defendant met the king archon twice before the trial.  The first meeting was very preliminary.  The prosecutor gave the defendent a written version of the charge and paid a fee.\footnote {\citet{brickhouse2004} 7--8.}  The second meeting with the king archon was the \foreign{anakrisis}.  The \foreign{anakrisis} was more elaborate.  The defendant entered a formal plea, and both sides answered questions about the case from the king archon.  At the end of this meeting the king archon decided whether or not to send the case forward to trial.\footnote {\citet{burnet1924} 82--83 and \citet{brickhouse2004} 8.}

Brickhouse and Smith offer several arguments that Socrates is at the stoa for the first pre-trial meeting.  They argue as follows (\citet{brickhouse2004} 8--9):
\begin{enumerate}
    \item If it was the day of the \foreign{anakrisis}, then Euthyphro would already know about the case since charges were posted publically before the \foreign{anakrisis}.
    \item Socrates doesn't know much about Meletus, and he seems unclear about the exact nature of the charges and what exactly Meletus means by them.
    \item Socrates imagines becoming Euthyphro's student and then convincing Meletus to give up the prosecution.  Brickhouse and Smith argue that this only makes sense \textit{before} the \foreign{anakrisis}.  They say ``unless Socrates somehow thinks that all of his lessons might be completed while they wait in line at the king-archon's office, such an outcome would be impossible---on this very day, if it is the day of the \foreign{anakrisis}, Socrates's case will be bound over to trial, and it will be too late to persuade Meletus to desist from the prosecution'' (9).
\end{enumerate}

I agree with their conclusion, but the second and third arguments are not very good.  First, there's no reason to think that Socrates would know significantly more about Meletus after the (brief) first meeting than he would after the summons.  He knows his name, his deme, his age, and what he looks like.  This strikes me as consistent with either pre-trial meeting.  Second, I don't see why Meletus could not break off the prosecution \textit{after} the \foreign{anakrisis}.  Is this somehow impossible? If so, Brickhouse and Smith don't say why or cite any sources to that effect.  Even if it is impossible, however, it is far too literal to worry about Socrates completing his lessons with Euthyphro ``while they wait in line at the king-archon's office.'' Given that (as they acknowledge) Socrates only ironically wishes to become Euthyphro's student, there's no particular reason to take such implications very seriously.  All of this said, their first argument makes sense to me, so I accept their overall conclusion.

% -]]

% [[- Introduction (2--5d7)
\section{Introduction (2--5d7)}

% [[- Socrates's case (2--3e6)
\subsection{Socrates's case (2--3e6)}

Euthyphro is shocked to see Socrates at court and asks who is prosecuting him and what the trial is about.  (He assumes that Socrates is the defendant and not the prosecutor.)  Socrates says that Meletus is the prosecutor, and they spend a few minutes not knowing who he is.  Then Socrates tells Euthyphro the charges: harming the young men via religious impiety.  Socrates first discusses harming the youth.  He claims to find Meletus admirable because he is so concerned for the youth of the city, as a politician should be.  However, Socrates is openly confused by the charge that he is not irreligious.  He describes that part of the charges as strange ({\g ἄτοπα} 3b1), but he does clarify that the specifics of that part of the charge are (1) that he introduces new gods and (2) that he does not believe in\footnote{\citet{burnet1924} 15 (on 3b3) argues that {\g νομίζειν θεούς} does \emph{not} mean \phrase{believe in} but rather \phrase{worship} and that ``it refers primarily to religious `practice' ({\g τὰ νομιζόμενα}).''  See also his note on \book{Apology} 24c1.  I need to think more about this.} the gods of Athens.

Euthyphro immediately assocates the first part of the irreligion charge with Socrates's \foreign{daimonion}.  He expresses sympathy with Socrates, and says that people always laugh at and mock him too, when he offers prophecy in public meetings.  Socrates notes that there's a big difference between being laughed at and a trial for your life.

As a sidenote, when Socrates introduces the charges against him, he reveals something important about background assumptions and how he attributes assertions to people.  Euthyphro asks Socrates ``What charge has he written up against you?'' and Socrates responds {\g Ἥντινα; οὐκ ἀγεννῆ} (2c2).  Socrates goes on to say that Meletus knows ({\g οἶδε}) (1) how young people are being corrupted and (2) who is corrupting them.  Even more strongly, Socrates adds ``as he says'' ({\g ὥς φησιν}).  It is theoretically possible that Meletus actually asserted such knowledge, but I think it's far more likely that Socrates draws characteristic, and ironic, inferences.  Meletus accuses him of corrupting the youths, and Socrates infers that only someone who knew (1) and (2) above could make such a charge in good conscience.  Compare the way that Socrates will attribute beliefs to interlocutors, even when those interlocutors explicitly deny such beliefs.  (E.g., Polus and Callicles in \book{Gorgias}.) All of this irony sets up the continuation where Socrates asserts that Meletus practices politics well.  He cares for the young first, and presumably he will protect the old later.

On the other hand, we might argue that Socrates means {\g ὥς φησιν} literally in this passage (2c4, 3b4).  In that case, we need to imagine that Socrates pressed Meletus at some point.  Socrates could have asked Meletus questions when Meletus initially summoned him to court, or he could have done so at the first pre-trial meeting with the archon, if you believe that Socrates is at the stoa for the \foreign{anakrisis}.  This is certainly possible, but I find it less likely.

% -]]

% [[- Euthyphro's case (3e7--5a2)
\subsection{Euthyphro's case (3e7-5a2)}

Euthyphro seems somewhere between coy and proud of his case.\footnote{\citet{geach1966} 369 takes Euthyphro to be embarrassed and reluctant to admit who he is prosecuting.  Bailly on the other hand \citeyearpar{bailly2003} 37 sees him as ``cavalier and cocky.'' \citet{nehamas1975} 290 describes Euthyphro best as ``sensitive, touchy, and, in a slightly perverse way, proud.''}  He is prosecuting his father for the murder of a hired hand.  The details are such that (1) there is room to doubt whether the crime was really murder or simply manslaughter with callous indifference\footnote{This terminology is mine and not necessarily historically or legally accurate.} and (2) many Athenians would feel that Euthyphro was wrong to pursue the case no matter what.  For such people, a son should not prosecute his father.  This kind of disapproval leads Euthyphro to make the assertion that opens him up to Socratic elenchus:

\begin{quote}

    {\g
    ἀνόσιον γὰρ εἶναι τὸ ὑιὸν πατρὶ φόνου ἐπεξιέναι· κακῶς εἰδότες, ὦ Σώκρατες, τὸ θεῖον ὡς ἔχει τοῦ ὁσίου τε πέρι καὶ τοῦ ἀνοσίου
    } (4d9-4e2).

    For [they think] that it is impious for a son to prosecute his father for murder.  But, Socrates, they understand poorly how the divine stands concerning piety and impiety.

\end{quote}

Euthyphro strongly implies here that \emph{he} does understand piety and impiety.  This is exactly the kind of confidence that Socrates looks for and challenges. Thus, Socrates asks Euthyphro about just this:

\begin{quote}

    {\g
    οὑτωσὶ ἀκριβῶς οἴει ἐπίστασθαι περὶ τῶν θείων ὅπῃ ἔχει, καὶ τῶν ὁσίων τε καὶ ἀνοσίων, ὥστε τούτων οὕτω πραχθέντων ὡς σὺ λέγεις, οὐ φοβῇ δικαζόμενος τῷ πατρὶ ὅπως μὴ αὐ σὺ ἀνόσιον πρᾶγμα τυγχάνῃς πράττων;
    } (4e4--8).
    
\end{quote}

Note that this anticipates issues from later in the dialogue.  In particular, in controversial cases it is exceedingly difficult to decide what is the virtuous path of action. To decide requires knowledge and precision.  Socrates believes that if Euthyphro is as confident as he seems then he must believe that he possesses such knowledge.  Euthyphro will initially agree without realizing how difficult it is to satisfy the demands of Socratic knowledge.

% -]]

% [[- Pre-definition methodology (5a3--5d7)
\subsection{Pre-definition methodology 5a3--5d7}

Socrates says that he should become Euthyphro's student in order to answer the charges against him.  There is certainly \emph{some} irony here, though not everyone would agree on how to interpret it.  See my endnote about irony.

Socrates lays out the formal question he wants answered at 5c8ff.

\begin{quote}

    {\g
    νῦν οὖν πρὸς Διὸς λέγε μοι ὃ νυνδὴ σαφῶς εἰδέναι διισχθρίζου, ποῖόν τι τὸ εὐσεβὲς φῂς εἶναι καὶ τὸ ἀσεβὲς καὶ περὶ φόνου καὶ περὶ τῶν ἄλλων; ἢ οὐ ταὐτόν ἐστιν ἐν πάσῃ πράξει τὸ ὅσιον αὐτὸ αὑτῷ, καὶ τὸ ἀνόσιον αὖ τοῦ μὲν ὁσίου παντὸς ἐναντίον, αὐτὸ δὲ αὑτῷ ὅμοιον καὶ ἔχον μίαν τινὰ ἰδέαν κατὰ τὴν ἀνοσιότητα πᾶν ὅτιπερ ἂν μέλλῃ ἀνόσιον εἶναι;
    }

    Therefore now, by Zeus, tell me what you were just now confidently asserting that you knew clearly: What sort of thing do you say is the holy and the unholy, both concerning murder and concerning everything else? Or isn't the pious the same thing in every activity, itself to itself and the impious in turn opposite of every pious, but itself the same as itself and having some single form in respect to its impiety, everything which will be impious?

\end{quote}

My translation is neither beautiful nor especially clear, but as far as that goes, it closely resembles the Greek.  In this passage Socrates lays out a number of conditions that he wants Euthyphro's definition to meet.

\begin{itemize}

    \item The definition should be universal.  However Euthyphro defines pious, that definition should cover both murder and everything else (5d1).  Piety should be ``the same in every activity'' (5d1).
    \item The two terms \word{piety} and \word{impiety} should be opposites (5d2).  I assume Socrates thinks of them as contrary rather than contradictory.
    \item Piety and impiety should ``have some single kind''\footnote{Plato uses the word {\g ἰδέα} (\foreign{idea}), and this is one of his terms for \word{Form} in its technical sense.  Most scholars assume that this dialogue precedes the full theory of forms, so this use causes some anguish.  See \citet{burnet1924} 111.} (5d4) inasmuch as they are what they are.\footnote{I'm unsure whether this third condition is really a restatement of the first or a new point.}

\end{itemize}

Note that the universality that Socrates demands here is \emph{not} primarily a matter of words.  He's not looking to legislate or to discover how we talk about things.  We can be sure of that because in this very passage, he happily uses two different sets of opposed terms for what he wants defined: \foreign{eusebes} versus \foreign{asebes} and \foreign{hosion} versus \foreign{anhosion}.

Although the confidence of Euthyphro's response may surprise us, it is characteristic of similar moments in other dialogues.  Euthyphro immediately answers ``Absolutely'' ({\g Πάντως δήπου}, 5d6).  We might have expected him to ask ``Huh?'' or at least to request clarification from Socrates.\footnote{\citet{nehamas1975} 303.}  However, Euthyphro's answer is unsurprising in the light of other Platonic dialogues.  As John Burnet says:

\begin{quote}

    Euthyphro appears to be quite familiar with the terminology used by Socrates, and accepts it without complaint, confusion or shock.  That becomes all the more striking when we find him boggling later at much more elementary things.  Plato always represents the matter in this way.  No one ever hesitates for a moment when Socrates talks of {\g ἰδέαι} and {\g εἴδη}, and Socrates never finds it necessary to explain the terms.\footnote{\citet{burnet1924} 112.}

\end{quote}

Similarly, Alexander Nehamas writes:

\begin{quote}

    [I]t is notorious that Socrates' interlocutors (even Meno) are quite willing to answer affirmatively{\ldots}On this view, the existence of universals is taken for granted in the Socratic dialogues.\footnote{\citet{nehamas1975} 303--304.}

\end{quote}

Scholars disagree about many things here:

\begin{itemize}

    \item Why do the interlocutors agree so easily?
    \item What do they think they're agreeing to?
    \item If they understand what Socrates wants, why do they do so poorly at satisfying his demands?

\end{itemize}
% -]]

% -]]

% [[- Definitions (5d8--15c10)
\section{Definitions (5d8--15c10)}

% [[- First definition of piety (5d8--6a5)
\subsection{First definition of piety (5d8--6a5)}

Here is Euthyphro's first definition:

\begin{quote}

    {\g
    τὸ μὲν ὅσιόν ἐστιν ὅπερ ἐγὼ νῦν ποιῶ, τῷ ἀδικοῦντι ἢ περὶ φόνους ἢ περὶ ἱερῶν κλοπὰς ἤ τι ἄλλο τῶν τοιούτων ἐξαμαρτάνοντι ἐπεξιέναι, ἐάντε πατὴρ ὢν τυγχάνῃ ἐάντε μήτηρ ἐάντε ἄλλος ὁστισοῦν, τὸ δὲ μὴ ἐπεξιέναι ἀνόσιον
    } (5d8--5e1).

    Piety is exactly what I am doing now, to prosecute someone who commits a crime, whether murder or theft of temples or anything at all of this kind, whether the criminal happens to be father or mother or anyone at all; and not to prosecute is impiety.

\end{quote}

The idea seems to be that piety requires you to pursue \emph{anyone} who commits religious crimes.

Unfortunately, Euthyphro unintentionally derails the conversation right away.  He says that people give him a hard time for prosecuting his father, but they contradict themselves because they're perfectly happy to hear stories about Zeus and Chronos physically assaulting \emph{their} fathers.  Socrates wonders if maybe this sort of thing is why he's in trouble: When he hears such stories, he protests.  Socrates briefly hints at his desire for a more rational and purified theology, and he also suggests his displeasure with poets and other writers who spread stories about conflicts among the gods.  In terms of his attitude towards state religion, note that Socrates seems upset that the \foreign{peplos} given every year to Athena during the Panathenaia was ``filled with such imagery'' ({\g μεστὸς τῶν τοιούτων ποικιλμάτων} 6c3).  These issues will come back to haunt Euthyphro when Socrates attacks his second definition.

% -]]

% [[- Socrates rejects the first definition (6a6--6d10)
\subsection{Socrates rejects the first definition (6a6--6d10)}

Socrates puts aside, for the moment, his dislike of such stories in order to get to his more basic complaint: prosecuting religious criminals may be pious, but according to Euthyphro many other things are also pious.  What Socrates wanted was the single thing that made all pious things pious: something that he can use as a yardstick whenever he needs to judge whether anything is pious or impious.  Euthyphro's first definition, by his own admission, cannot serve this purpose, so Socrates rejects it.

% -]]

% [[- Second definition of piety (6e11--7a1)
\subsection{Second definition of piety (6e11--7a1)}

Euthyphro's second definition is simpler and more general:

\begin{quote}

    {\g
    τὸ μὲν τοῖς θεοῖς προσφιλὲς ὅσιον, τὸ δὲ μὴ προσφιλὲς ἀνόσιον
    } (6e11--7a1).

    That which is pleasing to the gods is pious, and that which is not
    pleasing to the gods is impious.

\end{quote}
% -]]

% [[- Socrates refutes the second definition (7a2--8b6)
\subsection{Socrates refutes the second definition (7a2--8b6)}

Socrates praises the form of Euthyphro's second definition, but he attacks
its content.  He argues as follows:

\begin{enumerate}

    \item The gods war against each other, they disagree and there is enmity between them (7b2--4).

    \item Among humans, only \emph{some} disagreements lead to this kind of fighting and enmity.  For example, if people disagree about a number or measurement, that doesn't lead to such a fight.  Those kinds of problems are solved easily enough by counting, measuring or weighing.  But disagreements about justice and injustice, beauty and ugliness, good and bad--- those disagreements lead to fighting and enmity.  They do so because in those cases people ``are unable to come to a sufficient agreement'' (7d3--4).

    \item The gods quarrels and enmity results from these same sorts of disagreements (7d9--10).  Hence, the gods disagree about moral matters (7e1--4).

    \item What each god believes to be beautiful, good and just, that god loves.  They hate the opposites of such things (7e6--8).

    \item Therefore, \emph{the same things} are both hated and loved by the gods (7e10--8a5).

    \item Therefore, \emph{the same things} are both pious and impious (8a7--8).

\end{enumerate}


Socrates uses Euthyphro's earlier contention that the gods fight among themselves against him.  In particular, Socrates comes back to the case of Zeus, Chronos and Olympus and their relationship to Euthyphro's prosecution of his father.  Socrates points out that Zeus may approve of what Euthyphro is doing, as a son who had to punish his father, but that Chronos and Olympus would not approve, as fathers who were punished by their sons.  Thus, we are unable to judge solely by this yardstick whether or not Euthyphro's controversial prosecution is pious or not.  Using divine approval as a test gives contradictory results in this case.  Anything that can give contradictory results isn't what Socrates asked for.  So, Euthyphro's second definition fails as well.

% -]]

% [[- Digression about the second definition (8b7--9b11)
\subsection{Digression about the second definition (8b7--9b11)}

Euthyphro tries at first to save part of his definition.  He argues that all the gods agree about one thing at least: that whoever kills someone unjustly should be punished.  Without agreeing or disagreeing, Socrates picks up Euthyphro's claim and argues as follows:

\begin{enumerate}

    \item Every person agrees that wrongdoers deserve punishment, but they disagree about who the wrongdoers are.  In particular, they deny that they are wrongdoers (8b10--d6).

    \item If the gods fight as Euthyphro says that they do, then they must be in the same situation.  They all agree that wrongdoers deserve punishment, but they fight over who the wrongdoers are (8d8--e2 and 8e5--9).

    \item Therefore, if Euthyphro wants to maintain his argument on behalf of his prosecution of his father, he would need to show that \emph{every} god agrees that the day-laborer who died as a result of his father's neglect died unjustly ({\g ἀκίκως} 9a3), i.e. culpably (9a1--b4).

\end{enumerate}

% -]]

% [[- Second definition of piety reformulated (9c1--9e9)
\subsection{Second definition of piety reformulated (9c1--9e9)}

Socrates doesn't especially care about Euthyphro's specific case against his father, so he relieves him of that burden.  As he says, even if Euthyphro convinces him that all the gods agree about this one case, Socrates would only have learned about \emph{one case} not about piety in general, as he wants.

At this point, Socrates takes over slightly, and he feeds Euthyphro his next definition:

\begin{quote}

    {\g
    ὃ μὲν ἂν πάντες οἱ θεοὶ μισῶσιν ἀνόσιόν ἐστιν, ὃ δ᾽ἂν φιλῶσιν, ὅσιον, ὃ δ᾽ἂν οἱ μὲν φιλῶσιν οἱ δὲ μισῶσιν, οὐδέτερα ἢ ἀμφότερα
    } (9d2--3).

    Whatever all the gods hate is impious, and whatever they $<$all$>$ love $<$is$>$ pious; but whatever some love and some hate is neither $<$impious nor pious$>$ or both $<$impious and pious$>$.

\end{quote}

Socrates makes sure that Euthyphro accepts this new definition, and at 9e1--3 Euthyphro restates it himself.  Thereafter, it becomes Euthyphro's argument once again.\footnote{Is there something worth saying here about ``Say what you believe''?}

% -]]

% [[- Socrates refutes all forms of the second definition (10a1--11b5)
\subsection{Socrates refutes all forms of the second definition (10a1--11b5)}

This next argument is probably the dialogue's most important legacy.  In this refutation, Socrates presents what is commonly known as ``The Euthyphro dilemma''.  First, I'll lay out the dilemma.  Then I'll briefly discuss the difficulties in interpreting Plato's text here.

Here is what Socrates says initially:

\begin{quote}

    {\g
    ἆρα τὸ ὅσιον, ὅτι ὅσιόν ἐστι, φιλεῖται ὑπὸ τῶν θεῶν, ἢ ὅτι φιλεῖται, ὅσιόν ἐστιν;
    } (10a2--3)

    Is the pious loved by the gods because it is pious, or is it pious because it it loved (by the gods)?

\end{quote}

This suggests the following dilemma:

\begin{enumerate}

    \item Either the gods love what is pious because it is pious or it is pious because they love it.

    \item If it is pious because they love it, then that seems arbitrary.  What if they loved murder?  Would murder then be pious?

    \item If they love it because it is pious, then we will probably need to care more about \emph{why} they love it than simply \emph{that} they love it.

    \item So, either the gods are arbitrary overlords or their judgments merely track what is independently pious.  Either way, we can't answer our deepest ethical questions simply by referring to what the gods love.

\end{enumerate}

This dilemma has had a great deal of historical importance in debates about the interrelations of ethics and religion.

Unfortunately, Socrates doesn't stop there.  Euthyphro doesn't understand his initial formulation of the question.  In order to clarify things, Socrates goes on a tangent about how we describe various things.  He \emph{seems} to be contrasting either active versus passive or states versus processes.  Either way, the language and the logic of his (so-called) clarification has only made readers scratch their heads.  As \cite{cohen1971} 2 says, ``Socrates agrees to `speak more plainly' ({\g σαφέστερον φράσαι}) and then produces the most baffling part of the argument.''  Without going too far into detail, I'll say that I largely agree with \citet{cohen1971} in two key respects.
\begin{enumerate}
    \item Socrates makes a valid and interesting argument here.
    \item Socrates argues for a conclusion that reinforces the intuitive thought behind what is usually now known as the ``Euthyphro dilemma'' in ethics. Namely that either (i) the gods are authoritarian tyrants and their opinions have no particular moral weight or (ii) the gods favor things that are \textit{independently} good, and so we can (largely) ignore the gods when we consider ethics.
\end{enumerate}

% -]]

% [[- Aporetic interlude: Socrates' effect on beliefs (11b6--11e1)
\subsection{Aporetic interlude: Socrates' effect on beliefs (11b6--11e1)}

After Socrates refutes Euthyphro's adjusted second definition, he again asks that Euthyphro say ``zealously'' ({\g προθύμως}) what piety and impiety are (11b4--5).  Euthyphro reacts this time with clear frustration and confusion:

\begin{quote}

    {\g
    Ἀλλ᾽, ὦ Σώκρατες, οὐκ ἔχω ἔγωγε ὅπως σοι εἴπω ὃ νοῶ· περιέρχεται γάρ πως ἡμῖν ἀεὶ ὃ ἂν προθώμεθα καὶ οὐκ ἐθέλει μένειν ὅπου ἂν ἱδρυσώμεθα αὐτό
    } (11b6--8).

    Well, Socrates, I don't have any way to say to you what I know.  For whatever we put forth moves around somehow despite us, and it refuses to stay where we set it.

\end{quote}

The general tenor of this complaint is familiar from other dialogues.  Socrates picks up on the idea of movement and says that if the moving arguments were his, Euthyphro could joke that he was like his ancestor Daedalus.  This recalls \book{Meno} 97d where a similar reference to Daedalus can be found.  The specific idea is that Daedalus was able to create moving statues, and Socrates can make arguments move---even those of other people.  Under his influence, Euthyphro's arguments don't ``stay where he places them'' and he's unable to make a stable definition of piety.  From Euthyphro's point of view, it's \emph{Socrates} who is to blame for the failure of all the previous definitions.

This aporetic interlude is important in another way: Some scholars argue that Plato frequently puts arguments that he favors \emph{after} such interludes.\footnote{\citet{mcpherran1992} 221.  See also his references in footnote 14 (235--6).}  This gives readers extra reason to pay attention to the arguments after the interlude.

% -]]

% [[- Third definition of piety (11e1--12e9)
\section{Third definition of piety (11e1--12e9)}

Socrates accuses Euthyphro of being lazy at this point, but he also offers to help and this may be a tacit acknowledgement of Euthyphro's frustration or limits.

Socrates suggests that piety is a part (or kind?) of justice.  Initially, his wording confuses Euthyphro:

\begin{quote}

    {\g
    ΣΩ. ἰδὲ γὰρ εἰ οὐκ ἀναγκαῖόν σοι δοκεῖ δίκαιον εἶναι πᾶν τὸ ὅσιον.

    ΕΥΘ. Ἔμοιγε.

    ΣΩ. Ἆρ᾽οὖν καὶ πᾶν τὸ δίκαιον ὅσιον; ἢ τὸ μὲν ὅσιον πᾶν δίκαιον, τὸ δὲ δίκαιον οὐ πᾶν ὅσιον, ἀλλὰ τὸ μὲν αὐτοῦ ὅσιον, τὸ δὲ τι καὶ ἄλλο

    ΕΥΘ. Οὐχ ἕπομαι, ὦ Σώκρατες, τοῖς λεγομένοις
    } (11e4--12a3).

    SO. Consider whether it doesn't seem necessary to you for everything pious to be just.

    EUTH. [It seems so] to me.

    SO. Therefore is everything just also pious? Or is the pious all just, but the just is not all pious, but some of what is just is pious and some is something else.

    EUTH. I don't follow your argument, Socrates.

\end{quote}

After going through a few examples, Socrates gets his point across to Euthyphro.  \citet{bailly2003} 90 interprets this as an example of collection and division.  He refers to Plato's \book{Sophist}, and claims that this may be the first appearance of such a method in all of philosophy.  I'm not convinced, however, that isn't something far simpler than true Platonic division as it appears in later dialogues.  That said, the comparison is still throught provoking.

Socrates then asks Euthyphro to specify what part of justice piety is, and
Euthyphro answers:

\begin{quote}

    {\g
    Τοῦτο τοίνυν ἔμοιγε, ὦ Σώκρατες, τὸ μέρος τοῦ δικαίου εἶναι εὐσεβές τε καὶ ὅσιον, τὸ περὶ τὴν τῶν θεῶν θεραπείαν, τὸ δὲ περὶ τὴν τῶν ἀνθρώπων τὸ λοιπὸν εἶναι τοῦ δικαίου μέρος
    } (12e6--9).

    Therefore, Socrates, holiness and piety appear to me to be this part of justice: the part concerned with the service towards the gods, and the part concerned with service towards people is the remaining part of justice.

\end{quote}

% -]]

% [[- Socrates refutes the third definition (12e10--14a10)
\subsection{Socrates refutes the third definition (12e10--14a10)}

Socrates praises the form of Euthyphro's latest answer, but he says that he needs ``some small'' clarification (13a1).\footnote{\citet{bailly2003} 96 compares \book{Protagoras} 329b.  See also Columbo's ``Just one more thing.''} The small thing concerns the word {\g θεραπεία}.\footnote{ Reasonable translations include \defn{service}, \defn{care}, \defn{caretaking}.  However, it's not the easiest idea to convey directly in English with one or two words.  {\g θεραπεία} is a verbal noun derived from {\g θεραπεύω}.  The verb means \defn{attend to}, \defn{serve}, \defn{care for}, \defn{provide for}, \defn{treat medically}.  The verb or noun might be used in a variety of situations and across a variety of relationships.  So, for example, humans \defn{attend to} gods, but gods \defn{provide for} humans.  Doctors \defn{care for} or \defn{see to} patients, trainers \defn{care for} animals, and farmers \defn{tend to} their crops---all these ideas can be expressed by this one word.}  Socrates expresses shock that Euthyphro uses this word, and he assumes that Euthyphro can't mean the normal thing by it.  Socrates then takes a characteristic turn.  He quickly lists three commonplace examples of {\g θεραπεία}: care for horses, dogs, and cattle.  He then asks Euthyphro if θεραπεία for the gods is like these cases.  Euthyphro says that it is the same kind of thing.

Socrates continues by specifying how he understands {\g θεραπεία} in the normal cases.  In all such cases, the person performing the {\g θεραπεία} aims ``at some good and benefit for the one being cared for ({\g τοῦ θεραπευομένου})'' (13b7--10).  Euthyphro agrees, after which Socrates asks him to explain what benefit men do for gods when they exhibit piety.  Euthyphro naturally shrinks from this implication, and the trap is sprung.

Euthyphro clarifies that the {\g θεραπεία} is not one where people \emph{improve} the gods; it's more like a form of service than caretaking.  Socrates accepts this, using the term {\g ὑπηρετική}.\footnote{This term also refers to \defn{slavery} or to when a lesser helps someone of higher station.  Perhaps Euthyphro thinks that this will save his definition from seeming to be impious itself.}  Socrates then effectively returns to his earlier line of argument, though now without the embarrassing connotations.  He says that in the cases of medicine, shipbuilding, and construction, there are clear end products that these services aim at. Euthyphro agrees.  Socrates asks Euthyphro to tell him the parallel goal of piety.

How do a {\g θεραπεία} and {\g ὑπηρετική} differ?  If we take seriously the examples that Socrates uses, the difference seems to be that the first is action done to take care of and benefit something that already exists while the second produces some new thing.  I'm not sure that Socrates means precisely this, but the examples point that way.

All of that said, I'm also not sure that Socrates takes seriously enough the example that Euthyphro suggests: slaves and the work they do for their owners.  Would Socrates say that such work makes the slaveowners better?  It isn't necessarily the case that slaves always make new products.  But Socrates might fairly insist that there is still an aspect of {\g θεραπεία} in this example.  The slavery example assumes that the people that slaves serve have needs.  In this sense, the slaves certainly do make the lives of their owners better.  Presumably Euthyphro would not want to say the same about the gods and humans.  Most likely, he would reject both parts of this.  Note that all of this raises a genuine conflict between a rationalized theology of perfect gods with no needs and the traditional religion of Greece where humans \textit{do} provide benefit to the gods.

In any case, when trying to answer Socrates and explain the {\g ἔργου ἀπεργασία} of piety, Euthyphro fumbles around. In response, Socrates does not hide his frustration.  First, Euthyphro returns to vague answers, telling Socrates that the product of piety are ``many fine things'' (13e14).  Socrates agrees, but he explains generals and farmers \emph{also} produce many fine things, and yet we can still say in summary that they aim at victory and the production of crops.  What, Socrates asks again, is the summary end product of piety?  This will lead us to the fourth and final definition of piety.

% -]]

% [[- Fourth definition of piety (14a11--14d3)
\subsection{Fourth definition of piety (14a11-14d3)}

At this point, Euthyphro repeats his earlier disclaimer that it is a difficult thing to be specific, but he will say that if you know how to say and do things pleasing to the gods, that is piety.  He adds that such piety saves both individuals and cities.  Socrates objects to the length of Euthyphro's response for the second time, and he also says that Euthyphro was just on the cusp of teaching him what piety is, but that he ``turned away'' (14c1--2).\footnote{\citet{mcpherran1992} leans heavily on this passage when he argues that the dialogue suggests a definition of piety that Socrates himself approves of.}  Nevertheless, although Socrates is clearly not thrilled, he explains that he must follow where the respondent leads (14c3-5).

Through a series of short questions and answers, Socrates clarifies Euthyphro's definition as follows.  Sacrifice and prayer amount to asking the gods for things and giving them things.  So, on Euthyphro's latest account, piety is a matter of a service to the gods whereby we give them things (through sacrifice) and ask them for things (through prayer).  Euthyphro emphatically agrees with this clarification.

% -]]

% [[- Socrates refutes the fourth definition (14d4--15c10)
\subsection{Socrates refutes the fourth definition (14d4--15c10)}

Socrates argues as follows:

\begin{enumerate}

    \item Assume that piety is service to the gods in asking them for things and giving them things.  Asking \emph{correctly} means asking for what we need.  Giving \emph{correctly} means giving what they need.  Thus,\footnote{Socrates himself draws attention to an inference here, at 14e6 with {\g ἄρα}, though it isn't clear to me how he's entitled to such a `thus'.} piety turns out to be a kind of commercial interaction with the gods.\footnote{The Greek term he uses is {\g ἐμπορική}.  Euthyphro is not thrilled with the connotations, but he accepts that word.}  Socrates' point is that according to Euthyphro's latest explanation, piety is a matter of giving the gods what they want in exchange for what we want.  Euthyphro probably takes himself to be agreeing to a series of commonplaces.  After all, what Socrates describes is a standard \foreign{do ut des} view of religion.\footnote{\citet{bailly2003} 103--104.}

    \item But if this is so, then Euthyphro should be able to say how we benefit the gods through prayer.  Otherwise, our commerce with them seems to amount to this: We humans get everything good in our lives from the gods, but the gods get nothing from us in return.

\end{enumerate}

Euthyphro shrinks from the idea that we benefit the gods by praying to them.  In response, he returns to the idea that they gain honors, prizes and gratification.  The word for gratification is {\g χάρις}, and this leads Socrates back to the idea that piety is what is pleasing ({\g κεχαρισμένον}) to the gods.  Socrates asks, ``So is it then pleasing but not beneficial or beloved?'' (15b1--2).  Euthyphro answers that he certainly thinks piety \emph{is} beloved by the gods, and at that point Socrates has him caught again.

Socrates reminds Euthyphro that this was his earlier definition, and he presents him with a dilemma: Either Euthyphro is wrong now, or he was wrong earlier to abandon that definition when it first came up.  Euthyphro grudgingly concedes that he is caught in this dilemma, but he doesn't come down on either side of the matter.

% -]]

% -]]

% [[- Conclusion (15c11--16a4)
\section{Conclusion (15c11--16a4)}

At this point Socrates proposes that they start all over again.  He says that he remains determined to learn about piety from Euthyphro.  Euthyphro, however, begs off, saying that he has an appointment.  Socrates is exceedingly and verbally disappointed.  The dialogue ends on that note.

% -]]

% [[- Poscript
\section{Poscript}

% [[- What's the point?
\subsection{What's the point?}

We should wonder: What is the point of this dialogue?  The answer is far from obvious, at least to me.  One suggestion is that the dialogue offers a hidden answer to the question \phrase{What is piety?}.  I'm not very persuaded by this, but even if I were, I would then want to ask why Plato chooses to give an answer in such a roundabout way.  On the other hand, if there's no answer in the dialogue, then what is the point?

Are we supposed to learn, with Euthyphro and Socrates, that we don't know something?  Does Euthyphro really learn this?  Or does he just leave in frustration?  Are we supposed to learn something about Euthyphro and perhaps other people like him?  If so, what is the lesson?  It can't be simply ``some people think they know things, but really they don't.''  Or at least, if it's only that, then this dialogue (and the many like it) seem to be killing a fly with a tank.  \emph{That} lesson could have been stated much more easily and in far fewer pages.  Or might Plato disagree?  Does he maybe think that we are that stubborn or blind to the ignorance of ourselves and the people around us?

A related question is whether we should read this dialogue alone or together with other early dialogues.  Someone could argue, I think, that this dialogue by itself simply frustrates, but that larger meanings come forth if we make connections with other early Socratic dialogues.  On the other hand, this again should make us wonder why Plato works as he does.

I'm drawn to the idea that Plato deliberately makes this more difficult for his readers.  I don't yet have any very detailed explanation of why or to what (ultimate) end, but I'm persuaded by this argument:

\begin{enumerate}

    \item The dialogues are difficult to interpret.  And not just a little bit difficult, but a lot difficult.

    \item Plato is not a poor writer or a weak thinker.

    \item Therefore, the difficulties we meet must be deliberate and not the result of bad writing, poor planning or stupidity on Plato's part.

\end{enumerate}

I grant that this still leaves most everything to be worked out.

% -]] What's the point?

% [[- Irony
\subsection{Irony}

Let's at least try to state \emph{why} it's so hard to give simple answers about Socratic irony.  As an example statement for analysis, I'll use what Socrates says to Euthyphro early in the dialogue:

\begin{quote}

    {\g
    ἆρ᾽οὖν μοι, ὦ θαυμάσιε Εὐθύφρων, κράτιστόν ἐστι μαθητῇ σῷ γενέσθαι
    } (5a3--4)

    Therefore, wondrous Euthyphro, is it best for me to become your student?

\end{quote}

Recall the context: Euthyphro has just said that he knows about divine matters with great precision (4e4--5a2).  So Socrates proposes that he become Euthyphro's student.  How are we to take this?

\begin{enumerate}

    \item Completely straight: Socrates simply states his desire to learn from Euthyprho.  No matter what we may think about Euthyphro's knowledge, Socrates is being sincere.  He wants to know more about divine matters in order (1) to be a better person and (2) to win his case against Meletus.  Therefore, he hopes to learn from Euthyphro.\footnote{For an attempt to read the passage this way, see \citet{wolfsdorf2007}, 176--177.}

    \item Completely ironic: Socrates thinks Euthyphro is an idiot, and he is mocking him, somewhat openly in fact.  It is only \emph{because} Euthyphro is such a dolt that he does not see what is clear to every reader of the dialogue: Socrates is about to cross-question him into \foreign{aporia}.

    \item Some mixture of irony and seriousness: In theory, we should strive to be nuanced.  Let's admit some of the palpable irony without entirely dismissing the serious purpose behind Socrates' words.  However, I'm unsure how to do this without fudging or hand-waving.  When Socrates speaks that sentence above, how are we to parse it so that it can be, all at the same time, both ironic and straight?  I confess that I would like to find a ``middle ground,'' but I'm not sure that such a place would be stable and solid.\footnote{The conditional irony that \citet{vasiliou1998} describes might be such a stable middle ground.  In a nutshell, Vasiliou argues that there is a truth in what Socrates says, but that truth relies on a conditional antecedent that Socrates does not accept.  Therefore, the message as a whole is ironic in context.}

\end{enumerate}

Each interpretation above has problems.  The first is hard to square with basic facts.  Socrates never finds anyone who has wisdom, and he says so repeatedly.  Also, characters inside the dialogues sometimes accuse him of being insincere.  So it seems willful to insist that he's serious.  The second makes better sense of \emph{those} facts, but has it's own problems.  In particular, why do so few characters see how ironic Socrates is?  Also, it might seem to make Socrates a real jerk.  (This can be a pro or a con, depending on your view of Socrates.)  Finally, as I said above, the third option seems unstable.

Over and above the question of \emph{Socratic} irony, there is the question of Platonic irony.  Many readers have noted that Plato often includes dramatic or situational irony in his work, particularly about the future fate of his characters.  So, for example, Callicles' implied threats in \book{Gorgias} of course come true for Socrates, and we later readers already know this.  In addition, however, Plato may be using other types of irony in order to push his readers one way or another.

% -]] Irony

% -]] Poscript

\newpage
\bibliographystyle{apa}
\bibliography{plato}

\end{document}
% -]]
