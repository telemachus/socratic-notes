% [[- LaTeX prelude
\documentclass[12pt,letterpaper]{article}

\usepackage[no-math]{fontspec}
\setmainfont{Baskerville}

\usepackage[nolocalmarks]{polyglossia}
\setdefaultlanguage{english}
\setotherlanguage[variant=ancient]{latin}
\setotherlanguage[variant=ancient]{greek}
\newfontfamily\greekfont[Script=Greek]{Times New Roman}

\usepackage{csquotes}
\usepackage[style=authoryear,backend=biber,dashed=false,%	
	abbreviate=false,maxnames=4]{biblatex}
\addbibresource{plato.bib}
% See the following resources:
%   https://orthostat.wordpress.com/2013/02/16/moving-to-biblatex-1/
%%
% Remove pp. in citations.
\DeclareFieldFormat{postnote}{#1}
% Remove pp. from bibliography.
\DeclareFieldFormat%
    [article,inbook,incollection,inproceedings,thesis,patent,unpublished]%
    {pages}{{\nopp#1}}
% Use single quotes around titles of chapters and articles, not double.
% \DeclareFieldFormat%
%     [article,inbook,incollection,inproceedings,thesis,patent,unpublished]%
%     {title}{`#1'}
% Use commas rather than periods to separate units in the bibliography.
% \renewcommand*{\newunitpunct}{\addcomma\space}
% Place a space rather than a comma between name and year in citations.
% \renewcommand{\nameyeardelim}{~}
% Place a colon+narrow space before page numbers in citations.
\renewcommand{\postnotedelim}{:\,}
% No "in:" for articles at all. For all else, remove the colon.
\renewbibmacro{in:}{%
	\ifentrytype{article}{}{\printtext{\bibstring{in}}}
}
% Remove parentheses around date in bibliography.
% See http://tex.stackexchange.com/a/40710/29387
\usepackage{xpatch}
\xpatchbibmacro{date+extrayear}{%
	\printtext[parens]%
}{%
	\setunit{\addperiod\space}%
	\printtext%
}{}{}

\usepackage{enumitem}
\setlist{noitemsep}
\usepackage[super]{nth}


\begin{hyphenrules}{latin}
    \hyphenation{}
\end{hyphenrules}

\begin{hyphenrules}{greek}
    \hyphenation{ἐκ-μανθάν-ειν}
\end{hyphenrules}

%\pagestyle{plain}
% -]] Latex prelude

% [[- LaTeX document
\begin{document}

% [[- Title page
\begin{titlepage}
\title{Notes on Plato's \textit{Phaedo}}
\author{Peter Aronoff}
\date{May 2017}
\maketitle
\thispagestyle{empty}
\end{titlepage}
% -]]

% [[- Characters and setting
\section{Characters and setting}

% [[- Characters
\subsection{Characters}

\subsubsection{Frame}

The frame narrative contains two speakers, Echecrates and Phaedo. Echecrates is from Phlius in the Pelopponesus; he is also a Pythagorean although the dialogue never explicitly mentions this. We don't learn much about him, but he is clearly interested in Socrates even if he isn't a formal follower of his. Phaedo is a Socratic. He is from Elis, and apparently was a prisoner of war at Athens during the Pelopponesian war, but he was freed with the help of Socrates.\footcite[7]{rowe1993} \textcite[231]{nails2002} argues that Phaedo should be about 20 at the dramatic time of the dialogue. He appears to be making his way home to Elis from Athens when he meets Echecrates in Phlius.

\subsubsection{Death scene}

We briefly see Xanthippe, Socrates's wife, and Crito, his good friend, but the main characters are Socrates, Simmias and Cebes. Simmias and Cebes, both originally from Thebes, are somehow associated with (\textgreek{συγγίγνεσθαι} 61d) Philolaus, a Pythogorean, but Rowe argues that the two men themselves are not necessarily Pythagoreans \parencite[7]{rowe1993}. He says ``associate with'' is vague, and he also argues that Simmias advances a theory that conflicts with Pythagoreanism, namely the harmony theory of the soul. However, Rowe also admits that Echecrates, who certainly was a Pythagorean, is all in favor of the theory. So I'm not sure what support of that theory shows. (Also note that \textcite[138]{nails2002} considers Echecrates as also a ``student of Philolaus.'') Both Simmias and Cebes are relatively young (\textgreek{νεανίσκος} 89a) at the time of Socrates's death, but otherwise, we don't know anything about when they lived.

% -]] Characters

% [[- Setting
\subsection{Setting}

\subsubsection{Frame}

There's no precise place, but the speakers are in Phlius, and Phaedo appears to be heading from Athens to Elis after the death of Socrates.

\subsubsection{Death scene}

It is the day of Socrates's death in 399, and the speakers meet at the jail where Socrates was held after his trial.

% -]] Setting

% -]] Characters and setting

% [[- Outer frame, part 1 (57--59c7)
\section{Outer frame, part 1 (57--59c7)}

Echecrates begins the dialogue by asking whether Phaedo was there when Socrates died or whether he heard the story from someone else. This is similar to other dialogues with frames. Listeners often hope to get their stories as directly as possible, and if the storyteller wasn't personally present, they will often give a more or less detailed explanation of how they got their information. In this way, the dialogues enact a kind of submission of \textit{bona fides}.

In addition, such introductions often allow Plato to slip in other material that helps to guide us as readers. In this case, Phaedo says that he was there himself, but he also tells us (1) why there was such a long delay between the trial and execution and (2) how strange an experience the day of Socrates's death was.

An annual religious duty caused the delay. Every year the Athenians honor Apollo by sending a ship and offering to Delos; they view this as carrying out an eternal obligation to Apollo for saving the life of Theseus when he traveled to Crete. During the time that the ship is away, the city is in a state of heightened purification, and so they cannot perform executions until the ship returns. Some years this takes a while since the ship can be stuck at Delos waiting for the winds to be right, and that is what happens in this case. This is the first way in which Apollo features in the dialogue.

Phaedo experiences an emotional state that he finds strange (\textgreek{θαυμάσια} 58e1) on the day of Socrates's execution. On the one hand, he doesn't feel pity, as might have been expected at the death of a close friend, because Socrates himself appeared to be so happy. But by the same token, he didn't enjoy the philosophical conversations of the day, as he normally would have, since the imminent death of Socrates overshadowed everything. As a result, Phaedo (and many of the others present) experienced an ``entirely unusual emotion'' (\textgreek{ἀτεχνῶς ἄτοπόν τι\dots πάθος (59a5)}, a blending of of pleasure and pain.\footnote{This seems to anticipate when Socrates says that pleasure and pain have a kind of connected relationship at the start of the narrative (60b ff.), but the exact connection between the ideas is unclear. Phaedo and Socrates seem to disagree, though commentators have been quick to say that there's no formal contradiction.}
% -]] Outer frame, part 1 (57--59c7)

% [[- Two attitudes towards death (59c8--64a3)
\section{Two attitudes towards death (59c8--64a3)}

\subsection{Pleasure and pain (60b1--c7)}

After a rapid narration of the morning meeting of Socrates's friends outside the jail and the quick dismissal of Xanthippe, Phaedo focuses on Socrates immediately after he has been unchained (in preparation for his execution). Phaedo pays particular attention here to how Socrates is sitting and a strange short observation Socrates makes about pleasure and pain. I've never been able to get a handle on what the sitting means,\footnote{Though note that there's more attention to how Socrates sits in the near sequel to this passage at 61c10--d2. I don't know what that means either.} but I want to look at least briefly at the pleasure and pain bit.

After the jailers have released Socrates from his shackles, he rubs at his legs, and he remarks that ``what people call pleasure'' is strange (\textgreek{ἄτοπον} 60b3). In particular, this alleged pleasure is quite odd to its apparent opposite, pain. The two of these things refuse to exist in one person at the same time, and yet if a person experiences one, then the other is bound to follow. If Aesop had noticed this phenomenon, he could have told a story about how a god wished to end their battling, but settled for joining them together by the heads.
% -]] Two attitudes towards death (59c8--64a3)

\newpage
\printbibliography

\end{document}
% -]]
