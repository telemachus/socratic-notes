% [[- LaTeX prelude
\documentclass[12pt,letterpaper]{article}

\usepackage[no-math]{fontspec}
\setmainfont{Baskerville}

% Ugh: https://tex.stackexchange.com/a/400994/29387
% This no longer seems necessary, but I’ll leave it for a bit.
% \usepackage[base]{babel}
\usepackage[nolocalmarks]{polyglossia}
\setdefaultlanguage{english}
\setotherlanguage[variant=classic]{latin}
\setotherlanguage[variant=ancient]{greek}
\newfontfamily\greekfont[Script=Greek,Scale=MatchLowercase]{GFS Neohellenic}

\usepackage{titlesec}
\titleformat*{\section}{\large\bfseries}
\titleformat*{\subsection}{\bfseries}
\titleformat*{\subsubsection}{\bfseries}

\usepackage{parskip}
\usepackage{csquotes}
\usepackage[style=windycity,citetracker=context,backend=biber]{biblatex}
\addbibresource{plato.bib}

\usepackage{enumitem}
\setlist{noitemsep}
\usepackage[super]{nth}

\begin{hyphenrules}{latin}
    \hyphenation{}
\end{hyphenrules}

\begin{hyphenrules}{greek}
    \hyphenation{}
\end{hyphenrules}

\usepackage{fancyhdr}
\fancypagestyle{notes}{%
    \fancyhf{}
    \renewcommand{\headrulewidth}{0pt}
    \lhead{}
    \chead{\MakeUppercase{Aristotle on Irony}}
    \rhead{}
    \lfoot{}
    \cfoot{\thepage}
    \rfoot{}
}
\fancypagestyle{references}{%
    \fancyhf{}
    \renewcommand{\headrulewidth}{0pt}
    \lhead{}
    \chead{\MakeUppercase{References}}
    \rhead{}
    \lfoot{}
    \cfoot{\thepage}
    \rfoot{}
}
% -]] Latex prelude

% [[- LaTeX document
\begin{document}

% [[- Title page
% \begin{titlepage}
% \title{Arisotle on Irony}
% \author{Peter Aronoff}
% \maketitle
% \thispagestyle{empty}
% \end{titlepage}
% -]]

\pagestyle{notes}

% [[- Characters, Setting, and Date
\section{Characters, Setting, and Date}

Unlike some dialogues, \textit{Hippias Minor} does not include much in the way of background or scene-setting.
However, Plato does provide some information that we should take note of, and we can say a little about the characters and possible dramatic date of the dialogue.

\subsection{Characters}

Hippias of Elis, the son of Diopeithes, was a famous sophist.
Nails puts his date of birth at around 470 BCE, and she thinks he dies after 399 BCE.%
\footcite[][168]{nails2002-people-of-plato}
Plato consistently portrays Hippias as full of himself.
He was apparently a self-professed polymath, and he also made his own clothing and jewelry.
In addition to his skill as a speaker, he also performed feats of memorization.

Eudicus is mentioned in \textit{Hippias Major} (at 286c), and he is a speaking character in this dialogue.
We learn here that his father was Apemantus, and he appears to enjoy sophistic demonstrations and debate.
However, ancient source mentions Eudicus other than these two dialogues.

\subsection{Setting and Date}

\textit{Hippias Minor} contains little information about the setting or situation of the dialogue.
Hippias says that he has been to the Olympic festival more than once, and the dialogue portrays him as well known.
In \textit{Hippias Major}, Hippias tells Socrates about a speech Eudicus asked Hippias to perform.
In the speech, Nestor gives Neoptolemus advice about how he should behave (\textgreek{πάμπολλα νόμιμα καὶ πάγκαλα}, 286b3--4).
Eudicus begins \textit{Hippias Minor} by asking Socrates what he thought of a speech that Hippias just gave, and we can infer that the speech involved Homer from what Socrates says to Hippias during the dialogue.
Thus, Waterfield infers that the two dialogues depict one time that Hippias visited Athens.
Waterfield places the visit around 420 BCE.%
\footcite[][213]{waterfield-hippias-minor-1987}
% -]] Characters, Setting, and Date

% [[- Bibliography
\newpage\
\pagestyle{references}
\defbibfilter{sources}{%
    ( keyword=edition or keyword=translation or keyword=commentary )
}
\defbibfilter{secondary}{%
    keyword=secondary
}
\printbibliography[filter=sources,title={Ancient Sources: Editions, Translations, Commentaries}]
\printbibliography[filter=secondary,title=Secondary Literature]
% -]] Bibliography

\end{document}
% -]]
