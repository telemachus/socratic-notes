% {{{ LaTeX prelude
\documentclass[11pt]{article}
\usepackage{fontspec}
\setmainfont[Ligatures={Common,TeX}]{Baskerville}
\newfontfamily\g[Ligatures={Common,TeX}]{Times New Roman}
\usepackage{url}
\usepackage{parskip}
\usepackage{natbib}
\bibpunct{(}{)}{;}{a}{,}{,}
\usepackage{titles}
\usepackage{enumitem}
\setlist{noitemsep}
\hyphenation{ἐκ-μανθάν-ειν}
% }}}

% {{{ LaTeX document
\begin{document}

% {{{ Title page
\begin{titlepage}
\title{Notes on Plato's \book{Ion}}
\author{Peter Aronoff}
\date{August 2013}
\maketitle
\thispagestyle{empty}
\end{titlepage}
% }}}

% {{{ Characters and setting
\section{Characters and setting}

% {{{ Characters
\subsection{Characters}

We know nothing about Ion outside of what \book{Ion} tells us.  He is from
Ephesus, and he has just been at Epidaurus.  He won a rhapsodic competition
there at a festival for Asclepius (530a).  He may also be something of
a fancy dresser (530b6, 535d2).

Rhapsodes were professional reciters of poetry, especially Homer's epics.
They traveled around the Greek world and competed at religious festivals.
In some ways, they must have struck Plato as similiar to the sophists: the
traveling, the fame, the reputation for wisdom among ordinary people, the
vanity.  \citet{murray1996} describes Ion's performance style as
``histrionic" (97), and she believes that rhapsodes would have not only
recited the epics but also commented on them.  But I'm inclined to think
that in this context \word{interpretation} means demonstrating the meaning
of the text through appropriate performance. Certainly Ion is ready to
comment on Homer in discussion, but this is not his normal or primary
activity.

% }}} Characters

% {{{ Setting
\subsection{Setting}

The dialogue does nearly no scene-setting.  We're obviously in Athens, but
nothing either man says indicates where.  In addition, there doesn't seem
to be any other audience that might help us place the dialogue.

\citet{murray1996} suggests a dramatic date ``sometime before 412" (96) on
the basis of 541c3--4.  In that passage, Ion refers to Athenian control of
Ephesus.  As a member of the Delian league, Ephesus was under Athenian
control, but in 412 there was a larger Ionian revolt that ended this
control.  So presumably the dialogue takes place sometime before
that.\footnote{See \citet{moore1974}.}

% }}} Setting

% }}} Characters and setting

% {{{ Introductory material (530a--530d8)
\section{Introductory material (530a--530d8)}

The two men meet and briefly exchange small talk before Socrates begins to
grill Ion.  Ion tells Socrates that he has just come from a competition at
Epidaurus and that he won first prize there.  Socrates congratulates him
and says that he has often envied rhapsodes:

\begin{quote}
    {\g
    Καὶ μὴν πολλάκις γε ἐζήλωσα ὑμᾶς τοὺς ῥαψῳδούς, ὦ Ἴων, τῆς τέχνης· τὸ
    γὰρ ἅμα μὲν τὸ σῶμα κεκοσμῆσθαι ἀεὶ πρέπον ὑμῶν εἶναι τῇ τέχνῃ καὶ ὡς
    καλλίστοις φαίνεσθαι, ἅμα δὲ ἀναγκαῖον εἶναι ἔν τε ἄλλοις ποιηταῖς
    διατρίβειν πολλοῖς καὶ ἀγαθοῖς καὶ δὴ καὶ μάλιστα ἐν Ὁμήρῳ, τῷ ἀρίστῳ
    καὶ θειοτάτῳ τῶν ποιητῶν, καὶ τὴν τούτου διάνοιαν ἐκμανθάνειν, μὴ μόνον
    τὰ ἔπη, ζηλωτόν ἐστιν. οὐ γὰρ ἂν γένοιτό ποτε ἀγαθὸς ῥαψῳδός, εἰ μὴ
    συνείη τὰ λεγόμενα ὑπὸ τοῦ ποιητοῦ. τὸν γὰρ ῥαψῳδὸν ἑρμηνέα δεῖ τοῦ
    ποιητοῦ τῆς διανοίας γίγνεσθαι τοῖς ἀκούουσι· τοῦτο δὲ καλῶς ποιεῖν μὴ
    γιγνώσκοντα ὅτι λέγει ὁ ποιητὴς ἀδύνατον. ταῦτα οὖν πάντα ἄξια
    ζηλοῦσθαι
    } (530b5--c6).

    Indeed, I often envied you rhapsodes, Ion, for your skill. For it
    deserves envy that your skill makes it appropriate for you to always be
    dressed up and appear as beautiful as possible and at the same time
    that it is necessary to spend time among the many excellent poets, and
    in particular with Homer, the best and most divine of the poets and to
    thoroughly learn his thought, not only his words. For someone could not
    become a good rhapsode unless he understood the things said by the
    poet. For a rhapsode must be an interpreter of the poet's thought for
    his audience. And it is impossible to do this well unless you know what
    the poet means. All of these things, therefore, deserve to be envied.
\end{quote}

I quoted this at length because it is a classic Socratic set up.  How are
we to read Socrates' professed envy here?  The obvious answer is that he's
being ironic.  But what would \word{ironic} mean here?  I submit that it
cannot mean the most common contemporary thing, namely something like
\phrase{using words to convey the opposite of their literal meaning}.  It
can't mean that because in the normal contemporary case, the source of such
irony \emph{wants} the target to pick up on the intended opposite meaning.
Socrates is very unlikely to want Ion to pick up on any such opposed
meaning.  That would only antagonize Ion, but Socrates wants to engage him
in conversation.

I think we can make the best sense of this by distinguishing two levels of
speaker and audience.  Internally, the speaker is Socrates and the audience
is Ion.  Externally, the author is Plato and the reader is the audience.
In order to make the terms uniform, I'll speak of \word{source} and
\word{target}.  Socrates and Plato are both sources; Ion and the readers of
the dialogue are targets.

With this distinction in place, we can understand this passage---and others
like it---better and more easily.  Socrates is being ironic in the ancient
Greek sense of {\g εἴρων}: he is falsely humble in praise of another.  But
he is trying to deceive Ion, as I said above.  So Socrates is not ironic in
the contemporary sense.  If he were, his irony would lack a target, and he
would be speaking to himself.  Plato, on the other hand, is being ironic in
more like our sense: he very much wants his readers to see that it is
Socrates and not Ion who is more likely to have the better of the following
discussion.

\citet{ferrari2008} offers a very different view of all of this.  He
understands all irony as primarily pretence, and he is not troubled that
Socratic irony often lacks a target.  He describes this as ``solipsistic
irony", and compares a mother being ironic to a baby or someone being
ironic with a pet.  Neither the baby nor the pet can understand the irony,
but it is there nonetheless.  I'm not sure what I think of this yet, to be
honest.

% }}} Introductory material (530a--530d8)

% {{{ Why does Ion understand only Homer? (530d9--533c3)
\section{Why does Ion understand only Homer? (530d9--533c3)}

Socrates asks Ion whether he is a gifted speaker about only Homer or also
about Hesiod and Archilochus.  This question may seem odd or unexpected at
first, but it quickly becomes clear why it matters to Socrates.  Socrates
believes that if you possess a skill ({\g τέχνη}), then your ability is general
not specific to one case.  So what he is leading up to here is a challenge
to Ion's belief that he excels in some skill concerning Homer.

However that may be, Ion readily admits that he is only gifted concerning
Homer (531a3--4).  In fact, as he says later, when any other poet comes up,
Ion ``nods off to sleep" (533c2).

Socrates presses Ion by reminding him that Homer and other poets sometimes
talk about the same subjects.  When this happens, they sometimes agree and
they sometimes disagree.  According to Socrates, it would take a specific
expert in any given case to determine who speaks well and who doesn't,
whether they agree or disagree.  He uses the example of prophecy, and Ion
agrees that a prophet would be better able than him to determine the value
of what Homer and Hesiod say about prophecy (531b1--10).

Socrates also gets Ion to agree that all the poets, including Homer speak
about the same general variety of subjects.  As examples, he mentions war,
the gatherings of men and gods, the heavens and Hades and the origins of
gods and heroes.  Ion insists that Homer speaks \emph{better} about these
things, but he grants that they all cover the same subjects (531c1--d11).

Socrates returns to his understanding of skill, this time with more
extended examples.  He argues as follows:

\begin{enumerate}

    \item When many people are talking about math and one person speaks the
        best, someone will be able to recognize the good speaker.  Ion
        agrees.(531d13--e1)

    \item The same person who can recognize a good speaker will also be
        able to recognize a bad speaker.  Again, Ion agrees. (531e1--3)

    \item The person who can recognize good and bad speakers about math is
        someone who possesses mathematical skill ({\g τέχνη}).  Ion agrees
        again. (531e3--4)

    \item Similarly when people talk about healthy foods, the same person
        can recognize good and bad speakers and that person is a doctor.
        Ion agrees. (531e4--9)

    \item Therefore, says Socrates, they can draw the general conclusion
        that the same person will recognize good and bad speakers about one
        subject.  The same person is gifted concerning both people who
        speak well and people who speak poorly.  Ion agrees. (531e9--532a4)

    \item But Ion says that the poets talk about the same things sometimes,
        that he recognizes that Homer speaks better than the others.
        Therefore, by parity of reasoning with the other cases, Ion should
        be able to recognize those who speak worse as well as the one who
        speaks well.  Ion agrees that this seems right, presumably based on
        the argument this far, but he balks at the final inference.
        (532a4--c4)

\end{enumerate}

Ion's position is simple and not uncommon in the Socratic dialogues.  He
agrees with each step in the argument, and he agrees with the logical
connection of steps.  However, he is convinced that the conclusion is
untrue.  He seems to recognize that he should grant the conclusion since he
agrees to all the individual steps and to the logic of the argument.  But
he resists because of his certainty that the conclusion doesn't follow.

In this case, unlike many dialogues, Ion does not become angry or
reluctant.  He simply asks socrates, sincerely I think, how it's possible
that the he's in the state he's in, given their previous agreements.  In
response, Socrates explains that Ion doesn't speak about Homer by means of
skill (τέχνη) and knowledge (ἐπιστήμη) (532c5--9).  Socrates insists that
this is ``not difficult to infer" and that the answer is ``clear to
everyone" (532c5--6).  Given the way Socrates understands skill and
knowledge, Socrates absolutely cannot attribute what Ion does to either of
them.

To hammer the point home, Socrates quickly runs through three more
examples.  He covers painting, sculpture and music.  And in each case, he
argues---and Ion agrees---that you can't know the work of one great artist
in the field but not all the others.  In each case, Ion agrees readily.

% }}} Why does Ion understand only Homer? (530d9--533c3)

% {{{ Conclusion (--542b4)
\section{Conclusion (--542b4)}

% }}} Conclusion (--542b4)

\newpage
\bibliographystyle{apa}
\bibliography{plato}

\end{document}
% }}}
