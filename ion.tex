% {{{ LaTeX prelude
\documentclass[11pt]{article}
\usepackage{fontspec}
\setmainfont[Ligatures={Common,TeX}]{Baskerville}
\newfontfamily\g[Ligatures={Common,TeX}]{Times New Roman}
\usepackage{url}
\usepackage{parskip}
\usepackage{natbib}
\bibpunct{(}{)}{;}{a}{,}{,}
\usepackage{titles}
\usepackage{enumitem}
\setlist{noitemsep}
\hyphenation{ἐκ-μανθάν-ειν}
% }}}

% {{{ LaTeX document
\begin{document}

% {{{ Title page
\begin{titlepage}
\title{Notes on Plato's \book{Ion}}
\author{Peter Aronoff}
\date{August 2013}
\maketitle
\thispagestyle{empty}
\end{titlepage}
% }}}

% {{{ Characters and setting
\section{Characters and setting}

% {{{ Characters
\subsection{Characters}

We know nothing about Ion outside of what \book{Ion} tells us.  He is from
Ephesus, and he has just been at Epidaurus.  He won a rhapsodic competition
there at a festival for Asclepius (530a).  He may also be something of
a fancy dresser (530b6, 535d2).

Rhapsodes were professional reciters of poetry, especially Homer's epics.
They traveled around the Greek world and competed at religious festivals.
In some ways, they must have struck Plato as similiar to the sophists: the
traveling, the fame, the reputation for wisdom among ordinary people, the
vanity.  \citet{murray1996} describes Ion's performance style as
``histrionic" (97), and she believes that rhapsodes would have not only
recited the epics but also commented on them.  But I'm inclined to think
that in this context \word{interpretation} means demonstrating the meaning
of the text through appropriate performance. Certainly Ion is ready to
comment on Homer in discussion, but this is not his normal or primary
activity.

% }}} Characters

% {{{ Setting
\subsection{Setting}

The dialogue does nearly no scene-setting.  We're obviously in Athens, but
nothing either man says indicates where.  In addition, there doesn't seem
to be any other audience that might help us place the dialogue.

\citet{murray1996} suggests a dramatic date ``sometime before 412" (96) on
the basis of 541c3--4.  In that passage, Ion refers to Athenian control of
Ephesus.  As a member of the Delian league, Ephesus was under Athenian
control, but in 412 there was a larger Ionian revolt that ended this
control.  So presumably the dialogue takes place sometime before
that.\footnote{See \citet{moore1974}.}

% }}} Setting

% }}} Characters and setting

% {{{ Introductory material (530a--530d8)
\section{Introductory material (530a--530d8)}

The two men meet and briefly exchange small talk before Socrates begins to
grill Ion.  Ion tells Socrates that he has just come from a competition at
Epidaurus and that he won first prize there.  Socrates congratulates him
and says that he has often envied rhapsodes:

\begin{quote}
    {\g
    Καὶ μὴν πολλάκις γε ἐζήλωσα ὑμᾶς τοὺς ῥαψῳδούς, ὦ Ἴων, τῆς τέχνης· τὸ
    γὰρ ἅμα μὲν τὸ σῶμα κεκοσμῆσθαι ἀεὶ πρέπον ὑμῶν εἶναι τῇ τέχνῃ καὶ ὡς
    καλλίστοις φαίνεσθαι, ἅμα δὲ ἀναγκαῖον εἶναι ἔν τε ἄλλοις ποιηταῖς
    διατρίβειν πολλοῖς καὶ ἀγαθοῖς καὶ δὴ καὶ μάλιστα ἐν Ὁμήρῳ, τῷ ἀρίστῳ
    καὶ θειοτάτῳ τῶν ποιητῶν, καὶ τὴν τούτου διάνοιαν ἐκμανθάνειν, μὴ μόνον
    τὰ ἔπη, ζηλωτόν ἐστιν. οὐ γὰρ ἂν γένοιτό ποτε ἀγαθὸς ῥαψῳδός, εἰ μὴ
    συνείη τὰ λεγόμενα ὑπὸ τοῦ ποιητοῦ. τὸν γὰρ ῥαψῳδὸν ἑρμηνέα δεῖ τοῦ
    ποιητοῦ τῆς διανοίας γίγνεσθαι τοῖς ἀκούουσι· τοῦτο δὲ καλῶς ποιεῖν μὴ
    γιγνώσκοντα ὅτι λέγει ὁ ποιητὴς ἀδύνατον. ταῦτα οὖν πάντα ἄξια
    ζηλοῦσθαι
    } (530b5--c6).

    Indeed, I often envied you rhapsodes, Ion, for your skill. For it
    deserves envy that your skill makes it appropriate for you to always be
    dressed up and appear as beautiful as possible and at the same time
    that it is necessary to spend time among the many excellent poets, and
    in particular with Homer, the best and most divine of the poets and to
    thoroughly learn his thought, not only his words. For someone could not
    become a good rhapsode unless he understood the things said by the
    poet. For a rhapsode must be an interpreter of the poet's thought for
    his audience. And it is impossible to do this well unless you know what
    the poet means. All of these things, therefore, deserve to be envied.
\end{quote}

I quoted this at length because it is a classic Socratic set up.  How are
we to read Socrates' professed envy here?  The obvious answer is that he's
being ironic.  But what would \word{ironic} mean here?  I submit that it
cannot mean the most common contemporary thing, namely something like
\phrase{using words to convey the opposite of their literal meaning}.  It
can't mean that because in the normal contemporary case, the source of such
irony \emph{wants} the target to pick up on the intended opposite meaning.
Socrates is very unlikely to want Ion to pick up on any such opposed
meaning.  That would only antagonize Ion, but Socrates wants to engage him
in conversation.

I think we can make the best sense of this by distinguishing two levels of
speaker and audience.  Internally, the speaker is Socrates and the audience
is Ion.  Externally, the author is Plato and the reader is the audience.
In order to make the terms uniform, I'll speak of \word{source} and
\word{target}.  Socrates and Plato are both sources; Ion and the readers of
the dialogue are targets.

With this distinction in place, we can understand this passage---and others
like it---better and more easily.  Socrates is being ironic in the ancient
Greek sense of an εἴρων: he is falsely humble in praise of another.  But he
is trying to deceive Ion.  Plato, on the other hand, is being ironic in
more like our sense: he very much wants his readers to see that it is
Socrates and not Ion who is more likely to have the better of the following
discussion.

% }}} Introductory material (530a--530d8)

% {{{ Conclusion (--542b4)
\section{Conclusion (--542b4)}

% }}} Conclusion (--542b4)

\newpage
\bibliographystyle{apa}
\bibliography{plato}

\end{document}
% }}}
