% {{{ LaTeX prelude
\documentclass[11pt]{article}
\usepackage{fontspec}
\setmainfont[Ligatures={Common,TeX}]{Palatino}
\usepackage{url}
\usepackage{parskip}
\usepackage{natbib}
\bibpunct{(}{)}{;}{a}{,}{,}
\usepackage{titles}
% }}}

% {{{ LaTeX document
\begin{document}

% {{{ Title page
\begin{titlepage}
\title{Notes on Plato's \book{Crito}}
\author{Peter Aronoff}
\date{June 2013}
\maketitle
\end{titlepage}
% }}}

% {{{ Characters and setting
\section{Characters and setting}

% {{{ Characters
\subsection{Characters}

Crito was a close friend of Socrates.  They were similar in age and belonged to the same deme.  In \book{Phaedo} Crito leads away Socrates' wife and children early in the dialogue (60a7--8), he alone accompanies Socrates for his final bath (116a2--3) and Socrates gives his final instructions to him (118a7--8).  All of these actions suggest a very close relationship.  Crito makes clear early in \book{Crito} that he visited Socrates often while he was in jail (48a7--8).  The familiar and intimate tone of their conversation in this dialogue also indicates how well they knew one another.

We have evidence that Crito was reasonably wealthy.  In \book{Apology}, he is one of the men who guarantees the large fine that Socrates finally proposes as a counter-penalty (38b8), and in \book{Crito} says that he has sufficient money to help Socrates escape and live abroad comfortably (45b1).  Xenophon mentions a farm (\book{Memorabilia} 2.9), but otherwise I'm not aware of any particular source for his money.

Although Crito was a constant companion of Socrates, he doesn't appear especially philosophical.  In \book{Crito} he argues from exceedingly un-Socratic and commonplace premises.  Although he says that he agrees with core Socratic principles,\footnote{See for example 47a--b.} his arguments run against those ideals and it's unclear how well he understood what he was agreeing to.  \citet{brickhouse2004} (page?) also note that Crito takes no part in the philosophical conversation in \book{Phaedo}, though they argue that he is philosophicaly more adept than I'm claiming.
% }}} Characters

% {{{ Setting
\subsection{Setting}

The conversation between Socrates and Crito takes place in Socrates' jail cell.  \citet{brickhouse2004} (page?) argue that the jail would have been close to the court where the trial was held.  They place it somewhere near the agora on the east (?) side.  \book{Crito} itself doesn't do much with the scenery or location, except perhaps hint that Crito bribed a guard.  (See below for more on this.)
% }}} Setting
% }}} Characters and setting

% {{{ Introduction (43a1--44a)
\section{Introduction (43a1--44a)}

content
% }}} Introduction (43a1--44a)

% {{{ Crito urges Socrates to escape (44b1--46a9)
\section{Crito urges Socrates to escape (44b1--46a9)}

content
% }}} Crito urges Socrates to escape (44b1--46a9)

% {{{ Socrates initial response (46b1--c6)
\section{Socrates initial response (46b1--c6)}

content
% }}} Socrates initial response (46b1--c6)

% {{{ Socrates on method (46c7--47d7)
\section{Socrates on method (46c7--47d7)}

content
% }}} Socrates on method (46c7--47d7)

% {{{ Whose opinions matter? (46d7--47d7)
\section{Whose opinions matter? (46d7--47d7)}

content
% }}} Whose opinions matter? (46d7--47d7)

% {{{ Living versus living well (47d8--49a3)
\section{Living versus living well (47d8--49a3)}

content
% }}} Living versus living well (47d8--49a3)

% {{{ Basic moral principles (49a4--e2)
\section{Basic moral principles (49a4--e2)}

content
% }}} Basic moral principles (49a4--e2)

% {{{ The principle of just agreements (49e2--50a5)
\section{The principle of just agreements (49e2--50a5)}

content
% }}} The principle of just agreements (49e2--50a5)

% {{{ The Laws speak (50a6--51c5)
\section{The Laws speak (50a6--51c5)}

content
% }}} The Laws speak (50a6--51c5)

% {{{ Obey or persuade (50e1-51c4)
\section{Obey or persuade (50e1-51c4)}

content
% }}} Obey or persuade (50e1-51c4)

% {{{ The argument from just agreements (51c6--53a)
\section{The argument from just agreements (51c6--53a)}

content
% }}} The argument from just agreements (51c6--53a)

% {{{ Will Socrates benefit anyone by escaping? (53a9--54b2)
\section{Will Socrates benefit anyone by escaping? (53a9--54b2)}

content
% }}} Will Socrates benefit anyone by escaping? (53a9--54b2)

% {{{ The Laws of Hades (54b3--d2)
\section{The Laws of Hades (54b3--d2)}

content
% }}} The Laws of Hades (54b3--d2)

% {{{ Conclusion 54d3--e2
\section{Conclusion 54d3--e2}

content
% }}} Conclusion 54d3--e2

\newpage
\bibliographystyle{apa}
\bibliography{plato}

\end{document}
% }}}
