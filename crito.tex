% {{{ LaTeX prelude
\documentclass[11pt]{article}
\usepackage{fontspec}
\setmainfont[Ligatures={Common,TeX}]{Palatino}
\usepackage{url}
\usepackage{parskip}
\usepackage{natbib}
\bibpunct{(}{)}{;}{a}{,}{,}
\usepackage{titles}
% }}}

% {{{ LaTeX document
\begin{document}

% {{{ Title page
\begin{titlepage}
\title{Notes on Plato's \book{Crito}}
\author{Peter Aronoff}
\date{June 2013}
\maketitle
\end{titlepage}
% }}}

% {{{ Characters and setting
\section{Characters and setting}

% {{{ Characters
\subsection{Characters}

Crito was a close friend of Socrates.  They were similar in age and belonged to the same deme.  In \book{Phaedo} Crito leads away Socrates' wife and children early in the dialogue (60a7--8), he alone accompanies Socrates for his final bath (116a2--3) and Socrates gives his final instructions to him (118a7--8).  All of these actions suggest a very close relationship.  Crito makes clear early in \book{Crito} that he visited Socrates often while he was in jail (48a7--8).  The familiar and intimate tone of their conversation in this dialogue also indicates how well they knew one another.

We have evidence that Crito was reasonably wealthy.  In \book{Apology}, he is one of the men who guarantees the large fine that Socrates finally proposes as a counter-penalty (38b8), and in \book{Crito} says that he has sufficient money to help Socrates escape and live abroad comfortably (45b1).  Xenophon mentions a farm (\book{Memorabilia} 2.9), but otherwise I'm not aware of any particular source for his money.

Although Crito was a constant companion of Socrates, he doesn't appear especially philosophical.  In \book{Crito} he argues from exceedingly un-Socratic and commonplace premises.  Although he says that he agrees with core Socratic principles,\footnote{See for example 47a--b.} his arguments run against those ideals and it's unclear how well he understood what he was agreeing to.  \citet{brickhouse2004} 196 also note that Crito takes no part in the philosophical conversation in \book{Phaedo}, though they argue that he is philosophicaly more adept than I'm claiming.
% }}} Characters

% {{{ Setting
\subsection{Setting}

The conversation between Socrates and Crito takes place in Socrates' jail cell.  \citet{brickhouse2004} 197 argue that the jail would have been close to the court where the trial was held.  They place it somewhere near the agora on the southern side.  \book{Crito} itself doesn't do much with the scenery or location, except perhaps hint that Crito bribed a guard.  (See below for more on this.)
% }}} Setting
% }}} Characters and setting

% {{{ Introduction (43a1--44b5)
\section{Introduction (43a1--44b5)}

The dialogue begins with Socrates waking up in his cell and finding Crito already there.  Socrates is surprised to find Crito there so early and that the guard let Crito in at all.  They talk, and Crito tells Socrates that he will be put to death tomorrow.  Socrates' execution was delayed by a religious ceremony, and the return of a boat from Delos will mark the end of that ceremony.  Socrates says that he has had a dream hinting at the same thing, but that he thinks it will be two days before he is executed.

The opening conversation between Crito and Socrates may seem like small talk, but as is often the case in Socratic dialogues, it plants two important seeds in the readers minds.

\begin{enumerate}
    \item There is a strong hint in 43a8 that Crito has bribed (or \word{tipped}, if you insist) a guard.\footnote{\citet{burnet1924} 255 and \citet{brickhouse2004} 249 deny the clear implication of the Greek.  But Burnet's only argument seems to be that the implication is ``vulgar", which is no proof that it isn't true.  (\citet{brickhouse2004} simply cite Burnet; they provide no independent proof.)  Both \citet{adam1988} 22-23 and \citet{rose1983} believe that Crito gave the guard money.}  This suggests immediately that Crito is someone who solves problems with money.
    \item Socrates says that it would be foolish for him to be upset about dying, given how old he is.  Crito replies that many people do it just the same when they fall into bad luck.  Socrates lets this pass, but it foreshadows Crito's use of common opinion later as well as Socrates lack of interest in what most people believe.
\end{enumerate}
% }}} Introduction (43a1--44b5)

% {{{ Crito urges Socrates to escape (44b5--46a9)
\section{Crito urges Socrates to escape (44b5--46a9)}

Crito encourages Socrates to flee.  His opening language suggests that he has been encouraging Socrates to do so for some time, but that Socrates won't listen.  Initially, he makes two arguments:

\begin{enumerate}
    \item If Socrates dies, Crito will lose an irreplaceable friend (44b8--9).
    \item If Socrates dies, many people who don't know Socrates and Crito will think that Crito was cheap and let Socrates die when he might have saved him (44b9--c5).
\end{enumerate}

Socrates ignores the first point and dismisses the second quickly.  He says that they should not care what οἱ πολλοί think. Only the opinion of the most reasonable people matters, and they will not think Crito was cheap (44b6--9).

Crito counters by arguing that the many have great power:  Socrates' situation proves that they can do very great harm to someone.  To this Socrates offers a surprising reply: He wishes that they \emph{could} do great harm.  The background thought here appears to be the following:

\begin{enumerate}
    \item The greatest good for someone is an improvement of their soul, and the greatest harm is the degradation of their soul.
    \item Someone who can harm someone's soul could also improve it.  (This relies on the thought that knowledge of such a matter is holistic: a doctor, in Socrates' opinion, can produce sickness as much as health.\footnote{\citet{adam1988} 31 refers to \book{Hippias Minor} 366bff, and \citet{brickhouse2004} 200 cite \book{Republic} 333e3ff.  \citet{burnet1924} 259--260 also compares \book{Antigone} 334ff.})
    \item Socrates therefore wishes that they were capable of harming someone greatly since that would imply that they were capable of benefitting someone equally greatly.
\end{enumerate}

Following this initial skirmish, Crito again asks if Socrates is worried about what may happen to his friends if they help him escape.  Socrates says that is is concerned about this---though he also hints that he has ``many other" concerns (45a1--2).  Crito focuses only on the first part, however, and he launches into an impassioned speech.  In this second speech, he makes a number of arguments.

\begin{enumerate}
    \item It won't even cost very much to free Socrates (45a6--8).
    \item Socrates shouldn't worry about someone bringing his friends to trial for helping him escape.  They can easily (and cheaply) bribe any potential prosecutors (45a8--b1).
    \item Socrates has Crito's money to use as he wishes.  If Socrates doesn't want to ask his Athenian friends for money, there are others who can provide the money.  He mentions Simmias and Cebes by name (45b1--6).
    \item Socrates shouldn't worry about what he said at trial, namely that he has nowhere to go.  There are many places he might go.  Crito mentions Thessaly in particular (45b6--c5).
    \item What Socrates is doing is unjust.  He is betraying himself when it's possible for him to live.  He is hurrying to do to himself what his enemies want to do to him (45c6--9).
    \item Socrates is also betraying his children.  As a parent, Socrates has an obligation to raise and educate his children (45c10--d6).
    \item Socrates is behaving like a coward.  This is especially glaring since he has spent his life sayin that he cares so much about ἀρετή (45d6--e4).
    \item Socrates makes his friends look bad because it appears that they did not do the little it would require to save him (45e5--46a3).
\end{enumerate}

These arguments come tumbling out of Crito one after the other.  He doesn't spend especially long on any of them.  Since he is so emotional, I wouldn't necessarily draw any conclusions about Crito's dialectical skills from this passage.  The overall impression is that he terribly wants Socrates to agree, he half knows that Socrates won't and so he dumps every possible argument in an effort to somehow move Socrates.
% }}} Crito urges Socrates to escape (44b5--46a9)

% {{{ Socrates initial response (46b1--c6)
\section{Socrates initial response (46b1--c6)}

Socrates response is measured. He says that Crito's enthusiasm does him credit --- provided that Crito is enthusiastic about something right. Otherwise, the vehemence of Crito's encouragment only makes things worse.  Socrates then states a very basic principle:

\begin{quote}
    ἐγὼ οὐ νῦν πρῶτον ἀλλὰ καὶ ἀεὶ τοιοῦτος οἷος τῶν ἐμῶν μηδενὶ ἄλλῳ πείθεσθαι ἢ τῷ λόγῳ ὃς ἄν μοι λογιζομένῳ βέλτιστος φαίνηται (46b4--6).

    I am not now for the first time but always [have been] the type to obey nothing of mine other than the argument which seems best to me as I reason.
\end{quote}

Socrates continues by saying that he can't dump his previous arguments just because he has suffered this bad luck --- provided that they still seem correct to him.

The focus throughout this section is on reason.  Socrates emphatically states that he won't obey a anything other than whatever argument seems best to him. Nor will he throw an argument overboard if it has a bad outcome for him or if he has had back luck and really wants a different answer.  Reason is in the driver's seat.

The phrase τῶν ἐμῶν μηδενὶ ἄλλῳ πείθεσθαι ἢ τῷ λόγῳ is ambiguous.  On the one hand, we're likely to take τῶν ἐμῶν to be people: Socrates won't be persuaded by any of his friends or relatives over the dicates of reason.  On the other hand, the phrase could also be used if Socrates meant parts of himself, so to speak. That is, ``I obey nothing in me other than reason."  If we read it that way, he is distancing himself from emotion and desire.

Arguments can be made for either reading of τῶν ἐμῶν. Both \citet{burnet1924} 268 and \citet{rose1983} 27 take the phrase to be neuter and indicate parts of Socrates himself.  Burnet cites 47c5 and 47e8, both of which are excellent parallels for Socrates uses a plural neuter to talk about ``parts" of a person.\footnote{In those passages, Socrates has in mind a distinction between body and soul, so there is no danger of importing a Platonic division of the soul into this dialogue.}  A second reason to take τῶν ἐμῶν as neuter and referring to emotions is that at 46c3ff. Socrates specifically says that he will not yield to fear no matter what the many threaten him with.  On the other hand, \citet{adam1988} 40 takes τῶν ἐμῶν as Socrates' reply to Crito's πείθου μοι at 46a.  As Adam says, ``τῶν ἐμῶν includes Socrates' friends as well as everything else that could be called his" (40).\footnote{The argument is not really over whether τῶν ἐμῶν is neuter or masculine.  I think that everyone takes it to be neuter.  The question is how much the phrase encompasses.}
% }}} Socrates initial response (46b1--c6)

% {{{ Socrates on method (46c7--47d7)
\section{Socrates on method (46c7--47d7)}

content
% }}} Socrates on method (46c7--47d7)

% {{{ Whose opinions matter? (46d7--47d7)
\section{Whose opinions matter? (46d7--47d7)}

content
% }}} Whose opinions matter? (46d7--47d7)

% {{{ Living versus living well (47d8--49a3)
\section{Living versus living well (47d8--49a3)}

content
% }}} Living versus living well (47d8--49a3)

% {{{ Basic moral principles (49a4--e2)
\section{Basic moral principles (49a4--e2)}

content
% }}} Basic moral principles (49a4--e2)

% {{{ The principle of just agreements (49e2--50a5)
\section{The principle of just agreements (49e2--50a5)}

content
% }}} The principle of just agreements (49e2--50a5)

% {{{ The Laws speak (50a6--51c5)
\section{The Laws speak (50a6--51c5)}

content
% }}} The Laws speak (50a6--51c5)

% {{{ Obey or persuade (50e1-51c4)
\section{Obey or persuade (50e1-51c4)}

content
% }}} Obey or persuade (50e1-51c4)

% {{{ The argument from just agreements (51c6--53a)
\section{The argument from just agreements (51c6--53a)}

content
% }}} The argument from just agreements (51c6--53a)

% {{{ Will Socrates benefit anyone by escaping? (53a9--54b2)
\section{Will Socrates benefit anyone by escaping? (53a9--54b2)}

content
% }}} Will Socrates benefit anyone by escaping? (53a9--54b2)

% {{{ The Laws of Hades (54b3--d2)
\section{The Laws of Hades (54b3--d2)}

content
% }}} The Laws of Hades (54b3--d2)

% {{{ Conclusion 54d3--e2
\section{Conclusion 54d3--e2}

content
% }}} Conclusion 54d3--e2

\newpage
\bibliographystyle{apa}
\bibliography{plato}

\end{document}
% }}}
