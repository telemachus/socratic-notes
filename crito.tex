% [[- LaTeX prelude
\documentclass[12pt,letterpaper]{article}

\usepackage[no-math]{fontspec}
\setmainfont{Baskerville}

\usepackage[nolocalmarks]{polyglossia}
\setdefaultlanguage{english}
\setotherlanguage[variant=ancient]{latin}
\setotherlanguage[variant=ancient]{greek}
\newfontfamily\greekfont[Script=Greek,Scale=MatchLowercase]{GFS Neohellenic}

\usepackage{titlesec}
\titleformat*{\section}{\Large\bfseries}
\titleformat*{\subsection}{\large\bfseries}
\titleformat*{\subsubsection}{\large\bfseries}

\usepackage{parskip}
\usepackage{csquotes}
\usepackage[style=windycity,citetracker=context,backend=biber]{biblatex}
\addbibresource{plato.bib}

\usepackage{enumitem}
\setlist{noitemsep}
\usepackage[super]{nth}

\begin{hyphenrules}{latin}
    \hyphenation{}
\end{hyphenrules}

\begin{hyphenrules}{greek}
    \hyphenation{}
\end{hyphenrules}

\usepackage{fancyhdr}
\fancypagestyle{notes}{%
    \fancyhf{}
    \renewcommand{\headrulewidth}{0pt}
    \lhead{}
    \chead{Notes on Plato's \textit{Crito}}
    \rhead{}
    \lfoot{}
    \cfoot{\thepage}
    \rfoot{}
}
\fancypagestyle{references}{%
    \fancyhf{}
    \renewcommand{\headrulewidth}{0pt}
    \lhead{}
    \chead{References}
    \rhead{}
    \lfoot{}
    \cfoot{\thepage}
    \rfoot{}
}

\newcommand{\MONTH}{%
  \ifcase\the\month
  \or January% 1
  \or February% 2
  \or March% 3
  \or April% 4
  \or May% 5
  \or June% 6
  \or July% 7
  \or August% 8
  \or September% 9
  \or October% 10
  \or November% 11
  \or December% 12
  \fi}
% -]] Latex prelude

% [[- LaTeX document
\begin{document}

% [[- Title page
% \begin{titlepage}
% \title{Notes on Plato's \textit{Crito}}
% \author{Peter Aronoff}
% \date{June 2013--\MONTH\ \the\year}
% \maketitle
% \thispagestyle{empty}
% \end{titlepage}
% -]]

% [[- Characters and setting
\pagestyle{notes}

\section*{Characters and setting}

% [[- Characters
\subsection*{Characters}

The dialogue has two characters: Socrates and Crito: they are similar in age and belong to the same deme. In \textit{Phaedo}, we can see how close the two men are. Early in the dialogue, Crito leads away Socrates's wife and children (60a7--8), and later only Crito accompanies Socrates for his final bath (116a2--3). At the very end of the dialogue, Socrates gives his final instructions to Crito (118a7--8). All of these moments suggest a very close relationship. In \textit{Crito}, Crito makes clear that he visited Socrates often while he was in jail (48a7--8). The familiar and intimate tone of their conversation in this dialogue also indicates how well they knew one another.

Unlike Socrates, however, Crito is wealthy. In \textit{Apology}, he is one of the men who guarantees a large fine for Socrates (38b8), and in \textit{Crito}, he says that he has sufficient money to help Socrates escape and live abroad comfortably (45b1). Debrah Nails notes that ``Crito's considerable affluence was derived from agriculture.''\footcite[][115. She cites \textit{Euthydemus} 291e, Xenophon's \textit{Memorabilia} 2.9.4, and Diogenes Laertius 2.31.]{nails2002-people-of-plato} Nails argues that Crito's farm was ``probably in Alopece'' because Alopece was near the protective walls of the city, and this protection would explain why Crito didn't lose more money during the Peloponnesian war. In fact, as she says, Crito remained wealthy enough that his sons were ``among the richest men in Athens.''\footcite[][115]{nails2002-people-of-plato}

Although Crito was a constant companion of Socrates, he doesn't appear especially philosophical. In \textit{Crito}, he argues from exceedingly un-Socratic and commonplace premises. Although he says that he agrees with core Socratic principles,\footnote{See for example 47a--b.} his arguments run against those ideals, and it's unclear how well he understood what he was agreeing to. Thomas Brickhouse and Nicholas Smith also note that Crito takes no part in the philosophical conversation in \textit{Phaedo}, though they argue that he is philosophicaly more adept than I think he is.\footcite[][196]{brickhouse-smith2004-plato-trial-of-socrates}
% -]] Characters

% [[- Setting
\subsection*{Setting}

The conversation between Socrates and Crito takes place in Socrates' jail cell. Brickhouse and Smith argue that the jail was close to the court where the trial was held.\footcite[][197]{brickhouse-smith2004-plato-trial-of-socrates} They place it somewhere near the agora on the southern side. This sounds reasonable, but also like a guess. Plato doesn't use the setting much, but Socrates and Crito talk about the jailer who lets Crito in early. (See below for more on this.)
% -]] Setting

% -]] Characters and setting

% [[- Introductory material (43a1--50a5)
\section*{Introductory material (43a1--50a5)}

% [[- Introduction (43a1--44b5)
\subsection*{Introduction (43a1--44b5)}

As the dialogue opens, Socrates wakes up in his cell and finds Crito already there. Socrates is surprised that Crito is there so early and that the guard let Crito in so early. They talk, and Crito explains why he came during the night: he has learned that Socrates will be executed soon. Normally, Athenian courts carried out death sentences quickly after a trial. However, the Athenians halted all executions during certain religious festivals, and Socrates was convicted right as such a festival began. The return of a boat from Delos marks the end of the festival. Socrates has been in jail waiting for ``a long time'' (\textit{Phaedo} 58c4) because the boat was delayed.\footnote{Xenophon is more precise: he says that there was a thirty-day gap between the trial and the execution (\textit{Memorabilia} 4.8.2).} However, Crito heard that the boat will arrive today, and so Socrates will die the next day. Socrates replies that he has had a dream hinting at the same thing, but he thinks it will be two days before he is executed.

The opening conversation between Crito and Socrates may seem like small talk, but it plants important seeds in the readers' minds. (I think that narrative frames and opening conversations in Plato often serve this purpose.)

\begin{enumerate}

    \item There is a strong hint in 43a8 that Crito has bribed a guard. John Burnet denies that Crito gave the guard money. He claims that the phrase \textgreek{καί τι καὶ εὐεργέτηται ὑπ᾽ ἐμοῦ} ``characterizes the kindly Crito at once. The man is under an obligation to him, which should not be vulgarized into a `tip' with some editors.''\footcite[][on 43a8]{burnet1924-euthyphro-apology-crito} I don't find this argument very compelling. First, whether or not the suggestion is vulgar seems very subjective; second, something can be vulgar and yet true. Brickhouse and Smith cite Burnet approvingly, and they add, ``there is no reason to think that Crito has done anything more than been of assistance to the guard in some undisclosed way in the past. There is no reason to think that this is a reference to bribery.''\footcite[][249]{brickhouse-smith2004-plato-trial-of-socrates}  I believe that these scholars ignore the implication of the Greek, though I grant that implications are also subjective. On the other side, Louis Dyer, James Adam, and Gil Rose believe that Crito gave the guard money.\footcites[][on 43a9]{dyer-apology-crito-2007}[][on 43a9]{adam1988-crito}[][on 43a8]{rose1983} Rose offers \textit{Symposium} 184b2 as a parallel. In that speech, Pausanias distinguishes heavenly from vulgar Eros. And in this specific passage, he demands that young men show contempt when offered money or political power (\textgreek{εὐεργετούμενος εἰς χρήματα ἢ εἰς διαπράξεις πολιτικὰς}).\footnote{A skeptical reader of Rose might say that \textit{εἰς χρήματα} makes all the difference.} If these scholars are right about \textgreek{εὐεργέτηται}, then Plato suggests early in the dialogue that Crito is the sort of person who solves problems with money.

    \item Brickhouse and Smith believe that we can draw important conclusions from the wakefulness of Crito and the peaceful sleep of Socrates. In particular, if Socrates can sleep so well in his situation, then it ``was not bravado'' when he says in \textit{Apology} that no harm comes to a good person\footcite[][197]{brickhouse-smith2004-plato-trial-of-socrates}.

    \item Socrates says that it would be foolish for him to be upset about dying, given how old he is. Crito replies that many other old people complain when they fall into bad luck. Socrates agrees that other people complain despite their age. Later, however, the two men will disagree about a deeper issue: Crito relies on common opinion in his arguments, and Socrates thinks that we should not care what most people believe or do.

    \item After Crito tells Socrates that the boat is near, Socrates reveals a dream he had shortly before waking up. In the dream, a woman appears to Socrates, calls him by name, and recites to him a lightly adapted version of \textit{Iliad} 9.363. In the original, Achilles tells the Embassy that he will not return to the fighting; instead, he will leave for home. If Poseidon gives him a good trip, says Achilles, ``I would arrive at rich-soiled Phthia on the third day'' (\textit{Iliad} 9.363). In Socrates's dream, the woman tells him, ``You would arrive at rich-soiled Phthia on the third day.'' As Burnet says, Socrates understands the dream to mean that he will return home---i.e., what Phthia is for Achilles---in three days.\footcite[][on 44b2]{burnet1924-euthyphro-apology-crito} This suggests an otherworldly view of life and death: when people die, they go (or return) home, and death rather than life is their (true?) home. Crito finds the dream strange (\textgreek{ἄτοπον} 43b4). At the very end of \textit{Phaedo}, we see again that Crito does not accept, and perhaps cannot understand, how Socrates views life and death.
\end{enumerate}

% -]] Introduction (43a1--44b5)

% [[- Crito urges Socrates to escape (44b5--46a9)
\subsection*{Crito urges Socrates to escape (44b5--46a9)}

Crito encourages Socrates to flee. His opening language suggests that he has been encouraging Socrates to do so for some time, but that Socrates won't listen. Initially, he makes two arguments:

\begin{enumerate}

    \item If Socrates dies, Crito will lose an irreplaceable friend
        (44b8--9).

    \item If Socrates dies, many people who don't know Socrates and Crito will think that Crito was cheap and let Socrates die when he might have saved him (44b9--c5).

\end{enumerate}

Socrates ignores the first point and dismisses the second quickly. He says that they should not care what \textgreek{οἱ πολλοί} think. Only the opinion of the most reasonable people matters, and they will not think Crito was cheap (44b6--9).

Crito counters by arguing that the many have great power: Socrates' situation proves that they can do very great harm to someone. To this Socrates offers a surprising reply: He wishes that they \emph{could} do great harm. Socrates is cryptic here, but he appears to have the following in mind:

\begin{enumerate}

    \item If you do someone the greatest good, you make that person wise (\textgreek{φρόνιμον} 44d9); and if you do someone the greatest harm, you make that person foolish (\textgreek{ἄφρονα} 44d9).

    \item Anyone who can make people foolish can also make people wise. Such an ability is holistic: doctors, in Socrates' opinion, can make people sick just as they can make people healthy. (For parallel arguments, Adam refers to \textit{Hippias Minor} 366b and 369b; and Brickhouse and Smith refer to \textit{Republic} 333e3.\footcites[][on 44d]{adam1988-crito}[][200]{brickhouse-smith2004-plato-trial-of-socrates})

    \item Socrates therefore wishes that \textgreek{οἱ πολλοί} were capable of the greatest harm since that would imply that they were equally capable of the greatest good.

\end{enumerate}

Following this initial skirmish, Crito again asks whether Socrates is worried about what may happen to his friends if they help him escape. Socrates answers yes---though he also says that he has ``many other" concerns (45a1--2). I suspect that the other concerns are more important to Socrates, but Crito does not ask Socrates what his other concerns are. Instead, Crito launches into an impassioned speech with several brief arguments or considerations why Socrates shouldn't be worried for his friends and why he should flee.

\begin{enumerate}

    \item It won't cost much to free Socrates (45a6--8).

    \item Socrates shouldn't worry about someone bringing his friends to trial for helping him escape. They can easily (and cheaply) bribe any potential prosecutors (45a8--b1).

    \item Socrates can get all the money he needs from Crito. If Socrates doesn't want to ask his Athenian friends for money, there are non-Athenian friends who have plenty to give. Crito mentions Simmias and Cebes by name (45b1--6).

    \item Socrates shouldn't worry about what he said at trial, namely that he has nowhere to go. There are many places he can go. Crito mentions Thessaly in particular (45b6--c5).

    \item What Socrates is doing is unjust. He betrays himself when it's possible for him to live. He rushes to do the same thing to himself that his enemies want to do to him (45c6--9).

    \item Socrates also betrays his children. As a parent, Socrates has an obligation to raise and educate his children (45c10--d6).

    \item Socrates takes the easy way out if he allows the Athenians to kill him. Crito criticizes this especially pointedly, on the grounds that Socrates has spent his life saying that he cares about \textgreek{ἀρετή} (45d6--e4).

    \item Socrates makes his friends and himself look bad because it appears that they were too cowardly to save Socrates even though it would have been very easy to do so (45e5--46a3).

\end{enumerate}

These arguments come tumbling out of Crito one after the other. He doesn't spend long on any of them. Since he is so emotional, I don't draw any conclusions about Crito's dialectical skills from this passage. My overall impression is that he terribly wants Socrates to agree, he half knows that Socrates won't, and so he dumps every possible argument in an effort to convince Socrates.

% -]] Crito urges Socrates to escape (44b5--46a9)

% [[- Socrates initial response (46b1--c6)
\subsection*{Socrates initial response (46b1--c6)}

Socrates response is measured. He says that Crito's enthusiasm does him credit---provided that Crito is enthusiastic about something right. Otherwise, the vehemence of Crito's encouragment only makes things worse. Socrates then states a very basic principle:

\begin{quote}
    \textgreek{ἐγὼ οὐ νῦν πρῶτον ἀλλὰ καὶ ἀεὶ τοιοῦτος οἷος τῶν ἐμῶν μηδενὶ ἄλλῳ πείθεσθαι ἢ τῷ λόγῳ ὃς ἄν μοι λογιζομένῳ βέλτιστος φαίνηται} (46b4--6).

    I am not now for the first time but always [have been] the type to obey nothing of mine other than the argument which seems best to me as I reason.
\end{quote}

Socrates continues by saying that he can't dump his previous arguments---provided that they still seem correct to him---just because he has bad luck.

The focus throughout this section is on reason. Socrates emphatically states that he only obeys the argument that seems best to him. He will not throw an argument overboard simply because its conclusion is bad for him. Reason is in the driver's seat.

The phrase \textgreek{τῶν ἐμῶν μηδενὶ ἄλλῳ πείθεσθαι ἢ τῷ λόγῳ} is ambiguous. On the one hand, \textgreek{τῶν ἐμῶν} may refer to people: Socrates won't be persuaded by any of his friends or relatives over the dicates of reason. On the other hand, Socrates may mean parts of himself, so to speak. That is, ``I obey nothing in me other than reason." If we read it that way, Socrates distances himself from emotion and desire. 

Scholars have argued primarily for the second reading of \textgreek{τῶν ἐμῶν} or for an interpretation that includes both readings. Both Burnet and Rose interpret \textgreek{τῶν ἐμῶν} as parts of Socrates himself.\footcites[][on 46b5]{burnet1924-euthyphro-apology-crito}[][on 46b4--5]{rose1983} Burnet cites \textit{Crito} 47c5 and 47e8, both of which are excellent parallels for Socrates uses a plural neuter to talk about ``parts" of a person. (In those passages, Socrates has in mind a distinction between body and soul, so there is no danger of importing a Platonic division of the soul into this dialogue.) In addition, at 46c3ff. Socrates says that he will not give in to fear no matter what threats he faces. This too suggests that \textgreek{τῶν ἐμῶν} refer to emotions. On the other hand, Dyer and Adam take \textgreek{τῶν ἐμῶν} to refer to people as well as features of himself.\footcites[][on 46b5]{dyer-apology-crito-2007}[][on 46b6]{adam1988-crito}. Dyer writes ``\textgreek{τὰ ἐμά} includes all the faculties and functions both of body and of mind, but very likely \textit{friends}, as well. Among these \textgreek{λόγος} is included as his wisest counselor.'' Adam views \textgreek{τῶν ἐμῶν} as Socrates' reply to Crito's \textgreek{πείθου μοι} at 46a. He argues ``\textgreek{τῶν ἐμῶν} includes Socrates' friends as well as everything else that could be called his" (40). This interpretation fits well with what Socrates argues next, namely that we should carefully consider \textit{whose} opinions we care about.
% -]] Socrates initial response (46b1--c6)

% [[- Whose opinions matter? (46c7--48a10)
\subsection*{Whose opinions matter? (46c7--48a10)}

Socrates begins by responding to Crito's appeals to the many. In this section, Socrates does not care so much \emph{what} Crito argued but \emph{how} he argued for it. By appealing to the many, Crito has violated an important principle of Socratic method. As Socrates explains, they have often said that some opinions are worth listening to and others are not. Socrates then walks Crito through a short elenchus as follows.

\begin{enumerate}

    \item Some opinions are worth listening to and others are not (46c8--47a6).

    \item The opinions worth listening to are good and the others bad. Good opinions belong to relevantly wise people, and bad ones belong to relevantly unintelligent people (47a7--11).

    \item If anyone takes advice about exercise from the many rather than from a doctor or a trainer, they are likely to suffer physical harm (47a12--47c8).

    \item Likewise, if anyone follows the advice of the many in regards to justice and injustice, they are likely to suffer harm in whatever part of a person justice and injustice benefit and harm (47c8--d7).

    \item If the body is terribly damaged, then life is intolerable (47d8--47e6).

    \item The part of a person that justice and injustice benefit and harm is more important than the body. Thus, people should avoid harming that part of themselves even more than they avoid harming their bodies. Thus, people should not heed the opinions of the many but only the opinions of those with relevant knowledge (47e7--48a7).

    \item Thus, Crito was wrong to begin by appealing to the opinion of the many (47a7--48a10).

\end{enumerate}

Socrates begins this argument with a pointed dramatic irony. Socrates implores Crito for his opinion: \textgreek{σὺ γὰρ ὅσα γε τἀνθρώπεια ἐκτὸς εἶ τοῦ μέλλειν ἀποθνῄσκειν αὔριον, καὶ οὐκ ἄν σε παρακρούοι ἡ παροῦσα συμφορά} (46e3--47a2). Of course, ``the current misfortune'' rattles Crito far more than Socrates. The reader already knows this from the start of the dialogue. Socrates sleeps like a baby, but sadness and fear keep Crito awake.
% -]] Whose opinions matter? (46c7--48a10)

% [[- An objection and summary on method (48a10--49e3)
\subsection*{The power of the many (48a10--48b9)}

Socrates anticipates an objection: ``But the many are able to kill us." Crito emphatically agrees with the objection, and, in response, Socrates gets his smug on. Yes, the many can kill us, but we should care about living well, not simply living. In more detail, Socrates argues as follows:

\begin{enumerate}
    \item We should care most of all about living well not about (merely) living (48b4--5).
    \item Living well and finely and justly are the same thing (48b7).
    \item Therefore, Socrates and Crito should investigate whether it would be just or injust for Socrates to sneak out of jail and avoid his execution (48b10--c1).
\end{enumerate}

Crito himself agreed to steps (1) and (2) in the past. When Socrates asks Crito if he stands by these agreements, Crito says yes. In this way, Socrates cuts off the only objection to the previous points that he considers worth mentioning.

Socrates follows up by pointedly saying that they should ignore considerations of money, the opinion of other people, Socrates' children and the rest of Crito's arguments. They will focus \emph{only} on whether the proposed actions are just or not.

This is not fair to Crito or to his earlier arguments. Socrates can certainly argue that, above all, he cares about being just, but Crito can respond that it's unjust to abandon your children when you don't have to. Also, what if Socrates will do something unjust no matter which course of action he chooses? What does Socrates recommend in cases where we face multiple unjust choices, but we must choose? Does he believe that this never happens? We don't find out here since Socrates never considers such questions.

Socrates also foreshadows ``persuade or obey.'' First, Socrates says that Crito keeps urging him to leave Athens \textgreek{ἀκόντων Ἀθηναίων}, and the willingness or unwillingness of the Athenians will matter more later. Second, Socrates tells Crito that he very much wants to do whatever he does ``after persuading you\dots, but not against your will'' (\textgreek{πείσας σε}\dots\textgreek{ἀλλὰ μὴ ἄκοντος} 48e4). Socrates wants to convince Crito rather than to simply ignore him or compel him. This echoes what Socrates says later about how he should treat the Laws.
% -]] An objection and summary on method (48a10--49e3)

% [[- Basic moral principles (49a4--e4)
\subsection*{Basic moral principles (49a4--e4)}

Socrates reminds Crito that they have often said that they should never commit injustice willingly. He adds that it is neither good nor fine to do so (ever). Crito agrees to this rule once again, and Socrates continues with a number of variations in order to make sure that he and Crito agree on these basic principles. Here are the principles they agree to:

\begin{enumerate}
    \item One must never do injustice (49a4--b8). Not even in return for injustice should one do injustice (49b9--c1).
    \item One should not harm (49c2--3). Not even in return for wrong should one do wrong (49c4--6). To do someone harm is no different than to do injustice (49c7--9).
\end{enumerate}

Socrates goes out of his way to stress the following. First, Crito must explicitly agree to these principles. To that end, Socrates repeats them again and and again, each time asking if Crito agrees. After even all that, Socrates restates the principles again, and he explicitly says that Crito must decide if he really agrees to these things (49c10--e3). Second, as \cite{kraut1984} explains, (25--27), Socrates also makes clear to Crito (and Plato makes clear to his readers) that he will use a number of terms interchangeably. For his purposes the following are the same: \textgreek{ ἀδικεῖν}, \textgreek{κακουργεῖν}, \textgreek{κακῶς ποιεῖν τινα}. Not everyone would have treated these as identical, but Socrates will at present.

\cite{kraut1984} makes two additional good points about this stretch of
the dialogue. First, Plato hints here at the direction that the argument
will take. Socrates will likely concede that the decision to put him to
death was unjust, but nevertheless argue that he must not answer injustice
with injustice. Second, this passage only lays the groundwork for Socrates
later arguments. It is not itself yet a direct argument that Socrates
should not escape. That argument will come later, though it will rely on
the principles that Crito agrees to here.

% -]] Basic moral principles (49a4--e4)

% [[- The principle of just agreements (49e5--8)
\subsection*{The principle of just agreements (49e5--8)}

Before moving on, Socrates briefly extracts another principle from Crito: just agreements must be kept. As \cite{kraut1984} and \cite{brickhouse-smith2004-plato-trial-of-socrates} note, the restriction \emph{just} is essential. Socrates does not say simply that one must do whatever one agrees to. He seems aware that someone might accidently agree to do something unjust, and in that case, the basic principle that one should never do injustice presumably takes precedence.

\cite{kraut1984} 32 argues that we should read this section together with things that the Laws say later. The Laws say that Socrates made his agreements with them in a just manner. A \emph{just manner} meaning that he was not compelled, he was not lied to, he was not rushed and he understood what he was agreeing to. Thus, agreements can be just in two ways: they cover just actions and they are justly made. Agreements to unjust action are never binding, since injustice must never be done. Agreements to just things made unjustly are not necessarily binding.
% -]] The principle of just agreements (49e5--8)

% [[- Crito can't quite see the upshot (49e9--50a5)
\subsection*{Crito can't quite see the upshot (49e9--50a5)}

Drawing out the implications of what they have agreed to so far, Socrates
asks Crito whether (1) it would be just for him to escape without
persuading the city and (2) whether he would be violating a just agreement
if he did so. Crito is unable to answer since he does not follow. He says
bluntly ``I do not understand" (50a5).

I'm inclined to agree with \cite{brickhouse-smith2004-plato-trial-of-socrates} 211--212 that this does
not suggest that Crito is insincere or unintelligent. I also don't see any
reason to think that Crito has been inattentive to the argument up to this
point. Socrates hints at two points here that require further elaboration.
Either one of them would merit Crito holding things up. First, Socrates
says that they might do injustice to ``those who least deserve it". Who are
these who least deserve it, and why do they least deserve injustice?
Second, what agreements does Socrates have in mind?

Dramatically and logically, Crito's confusion makes perfect sense. It
provides Plato a reason for Socrates to introduce the Laws. We readers are
likely to be pretty much where Crito is, so the speech of the Laws will
help us understand things better. And in terms of the argument, many
people are likely to have agreed with Socrates up until now on basic
principles. But now Socrates must make more substantive arguments to show
how escape would harm the state, and why citizens owe the state particular
loyalty.

% -]] Crito can't quite see the upshot (49e9--50a5)

% -]] Introductory material (43a1--50a5)

% [[- The Laws (50a6--54d2)
\section*{The Laws (50a6--54d2)}

% [[- The Laws enter (50a6--50c1)
\subsection*{The Laws enter (50a6--50c1)}

Socrates imagines the Laws of Athens addressing him if he tries to flee. In their initial argument, the Laws claim that Socrates is trying to destroy them ``for his part". They justify this by saying that a state cannot exist in which individuals can ignore legal decisions at their whim.

\cite{kraut1984} 42 describes this as a very early generalization argument, perhaps the first in Western philosophy. He spells this out in a somewhat Kantian manner (though he doesn't mention Kant). If everyone should behave like Socrates and invalidate legal rulings that they don't like, then the state would not survive. This gives point to the Laws saying that Socrates is trying to destroy them ``for his part".
% -]] The Laws enter (50a6--50c1)

% [[- Why we owe the Laws so much (50c1--51c5)
\subsection*{Why we owe the Laws so much (50c1--51c5)}

The next section helps to explain why Socrates says that by running away he would be harming those ``whom he least ought to" (50a2). Socrates imagines that he might reply to the Laws first statement by saying that the city does him an injustice since his case was wrongly decided (50c1--2). Crito emphatically agrees, and this gives Socrates an opening to make an argument that he and the city are not on even terms.

The Laws explain that some things are allowed to them which are not allowed
to citizens. Socrates owes them nearly everything. They are responsible
for his birth, insofar as they joined his father and mother together in
marriage. They are responsible for his care and education, since they set
the standards for such matters. Thus Socrates is like a child or a slave
while the Laws are like parents or owners. They are not equals: Socrates
may not do everything that the Laws may do. The argument here relies on an
Athenian norm: a parent might strike or yell at a child, but a child would
have no right to return this in kind.

% [[- Kraut's reconstruction of this argument
\subsubsection*{Kraut's reconstruction of this argument}

Richard Kraut, in \cite{kraut1984}, reconstructs this argument in the following way:

\begin{enumerate}

    \item One must never treat any city or person unjustly. (49b8)

    \item By escaping, Socrates would be destroying Athens, for his part (50d1--51c1)

    \item Athens is responsible for the birth and education of Socrates. (50a8--b5)

    \item If some person or city X is responsible for the birth or education of Y, then Y treats X unjustly if Y uses violence against X. (51c2--3)

    \item Whoever destorys a city, for his part, uses violence against that city (supplied)

    \item If Socrates escapes, he will be treating Athens unjustly.

\end{enumerate}

There are two things to notice about Kraut's reconstruction of the argument. First, Kraut distinguishes this argument from the argument to ``persuade or obey" that is buried inside the section on cities as parents. Second, Kraut insists that Socrates does \emph{not} argue that political violence is always wrong. In particular, Kraut argues (cf. 47, 50) that without the analogy of cities and parents, Socrates' argument would be incomplete. The second step above in Kraut's reconstruction is not enough to draw the conclusion that Socrates does Athens injustice by trying to escape, on Kraut's view.

Kraut has to do a fair amount of work to massage the argument into this shape since the text is far from clear. \cite{brickhouse-smith2004-plato-trial-of-socrates} 212--220 understand this stretch of \textit{Crito} differently, for example.\footnote{I can't actually follow their reconstruction. They seem to just pull out strands they like without worrying too much about how the strands form an argument.}  Having said that, the Laws are especially scattershot in their presentation, so I don't necessarily hold that against Kraut (or anyone else) particularly.

% -]] Kraut's reconstruction of this argument

% [[- Obey or persuade
\subsubsection*{Obey or persuade}

The Laws present say that citizens have an obligation to ``obey or persuade", although this argument is somewhat buried within their larger claims for special status. The two alternatives first appear at 51b4:

\begin{quote}

    \textgreek{καὶ σέβεσθαι δεῖ καὶ μᾶλλον ὑπείκειν καὶ θωπεύειν πατρίδα χαλεπαίνουσαν ἢ πατέρα, καὶ ἢ πείθειν ἢ ποιεῖν ἃ ἂν κελεύῃ} (51b2--4).

    And it is necessary to revere and to yield to and to serve one's country, even when it gives one trouble, more than one's father; and [it is necessary] either to persuade or to do whatever it orders.

\end{quote}

The same idea is restated and slightly expanded immediately below at 51c1:

\begin{quote}

    \textgreek{ἀλλὰ καὶ ἐν πολέμῳ καὶ ἐν δικαστηρίῳ καὶ πανταχοῦ ποιητέον ἃ ἂν κελεύῃ ἡ πόλις καὶ ἡ πατρίς, ἢ πείθειν αὐτὴν ᾗ τὸ δίκαιον πέφυκε} (51b9--c1).

    But in war and in court and everywhere one must do whatever the city and fatherland order or persuade it what the nature of justice is.

\end{quote}

The expansion suggests that the options are (1) obey or (2) persuade the city that its orders are unjust. Socrates would be wrong to escape his sentence since he chooses neither available option.

% -]] Obey or persuade

% -]] Why we owe the Laws so much (50c1--51c5)

% [[- The argument from just agreements (51c6--53a)
\subsection*{The argument from just agreements (51c6--53a)}

The Laws next remind Socrates that he has made an agreement with them. Anyone who does not like how the Laws do things (i.e. how Athens is run?) may leave:

\begin{quote}

    \textgreek{προαγορεύομεν τῷ ἐξουσίαν πεποιηκέναι Ἀθηναίων τῷ βουλομένῳ, ἐπειδὰν δοκιμασθῇ καὶ ἴδῃ τὰ ἐν τῇ πόλει πράγματα καὶ ἡμᾶς τοὺς νόμους, ᾧ ἂν μὴ ἀρέσκωμεν ἡμεῖς ἐξεῖναι λαβόντα τὰ αὑτοῦ ἀπιέναι ὅποι ἂν βούληται} (51d2--5).

    We announce in advance by having made a possibility for anyone of the Athenians who wants when he comes of age and see the affairs of the city and us Laws, to whomever we are not pleasant, that it is possible to take his things and go wherever he wishes.

\end{quote}

Any Athenian may leave, but the Laws argue that those who stay have ``made an agreement in deed" (51e4) to \emph{obey or persuade}. The Laws also add (at 52e) that the agreements are not forced, deceptive or rushed. This brings out how the agreements are \emph{just}. Neither Socrates nor the Laws dwell on the point, but it's important. An agreement made under duress, deceptively or in a very brief time is intuitively not as binding as one made freely, with full information and after due deliberation.

% -]] The argument from just agreements (51c6--53a)

% [[- ad hominem regarding Socrates (52a3--52d8)
\subsubsection*{ad hominem regarding Socrates (52a3--52d8)}

The Laws briefly focus on Socrates and they claim that he especially has made such an implicit agreement. Their reasoning is that insofar as Socrates left Athens almost never (only for military service, apparently), he has especially made the \textit{de facto} agreement that he approves of Athens and her laws.

% -]] ad hominem regarding Socrates (52a3--52d8)

% [[- Will Socrates benefit anyone by escaping? (53a9--54b2)
\subsection*{Will Socrates benefit anyone by escaping? (53a9--54b2)}

The Laws next turn to a more pragmatic question: Will Socrates do himself or his friends and relatives any good if he escapes?  The Laws argue that he will not. His friends are likely to be tried, lose their money and go into exile themselves. And wherever Socrates goes, he will be viewed as an enemy of the state by anyone who cares about their state's wellbeing. Furthermore, if Socrates goes on as he did before---talking about virtue and justice---he will be ridiculous. The topics he discusses and his behavior will be completely out of sync. Alternatively, he could go somewhere where the living is easy and just party all the time. But then again he will look ridiculous, and his life will be out of sync with his earlier words. Finally, the Laws say that if he stays alive for the sake of his children, he should think about the consequences for them. On the one hand, he could make them exiles too, but then they would face the same problems abroad that he will. On the other hand, he could count on friends at home to care for them. But those friends would have done the same if Socrates had died. So that gives him no extra reason to stay alive.
% -]] Will Socrates benefit anyone by escaping? (53a9--54b2)

% [[- The Laws of Hades (54b3--d2)
\subsection*{The Laws of Hades (54b3--d2)}

The Laws conclude with a threat of future punishment. Even if he escapes them and does wrong in return for wrong, Socrates will still have to face the Laws in Hades. They will avenge their earthly fellow laws if Socrates runs off into exile. Thus, for prudential reasons too, Socrates should not break the law.
% -]] The Laws of Hades (54b3--d2)

% -]] The Laws (50a6--54d2)

% [[- Conclusion 54d3--e2
\section*{Conclusion 54d3--e2}

Socrates closes the dialogue as a whole in an odd way. He tells Crito that he is ready to hear any objection that Crito has, but at the same time he says that the arguments of the Laws ring in his ears just as ``Corybantes seem to hear the pipes" (54d4--5). The Corybantes were the priests of Cybele, and they were famous for their frenzy and irrationality (see Catullus 63, for example). So it's disconcerting for Socrates to say all at once ``Do you have any other argument?" and ``I'm carried away by the irrational love of the previous arguments." \cite{rose1983} 40 says, ``[W]e can at least say here that Plato wishes to suggest that \textgreek{λόγοι}, reasoned arguments, create for Socrates the ecstasy for which others turn to orgiastic and irrational rites."

Nevertheless it bothers me. Just as Socrates asks Crito for further argument, he seems to say that he's not open to any such persuasion.
% -]] Conclusion 54d3--e2

% [[- Bibliography
\newpage
\pagestyle{references}
\nocite{burnet1903}
\defbibfilter{sources}{%
    ( keyword=edition or keyword=translation or keyword=commentary )
    and
    keyword=crito
}
\defbibfilter{secondary}{%
    keyword=secondary and ( keyword=crito or keyword=all )
}
\printbibliography[filter=sources,title={Ancient Sources: Editions, Translations, Commentaries}]
\printbibliography[filter=secondary,title=Secondary Literature]
% -]] Bibliography

\end{document}
% -]]
