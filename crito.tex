% {{{ LaTeX prelude
\documentclass[11pt]{article}
\usepackage{fontspec}
\setmainfont[Ligatures={Common,TeX}]{Baskerville}
\newfontfamily\g[Ligatures={Common,TeX}]{Times New Roman}
\usepackage{url}
\usepackage{parskip}
\usepackage{natbib}
\bibpunct{(}{)}{;}{a}{,}{,}
\usepackage{titles}
\usepackage{enumitem}
\setlist{noitemsep}
% }}}

% {{{ LaTeX document
\begin{document}

% {{{ Title page
\begin{titlepage}
\title{Notes on Plato's \book{Crito}}
\author{Peter Aronoff}
\date{June 2013}
\maketitle
\thispagestyle{empty}
\end{titlepage}
% }}}

% {{{ Characters and setting
\section{Characters and setting}

% {{{ Characters
\subsection{Characters}

Crito was a close friend of Socrates.  They were similar in age and belonged to
the same deme.  In \book{Phaedo} Crito leads away Socrates' wife and children
early in the dialogue (60a7--8), he alone accompanies Socrates for his final
bath (116a2--3) and Socrates gives his final instructions to him (118a7--8).
All of these actions suggest a very close relationship.  Crito makes clear
early in \book{Crito} that he visited Socrates often while he was in jail
(48a7--8).  The familiar and intimate tone of their conversation in this
dialogue also indicates how well they knew one another.

We have evidence that Crito was reasonably wealthy.  In \book{Apology}, he is
one of the men who guarantees the large fine that Socrates finally proposes as
a counter-penalty (38b8), and in \book{Crito} says that he has sufficient money
to help Socrates escape and live abroad comfortably (45b1).  Xenophon mentions
a farm (\book{Memorabilia} 2.9), but otherwise I'm not aware of any particular
source for his money.

Although Crito was a constant companion of Socrates, he doesn't appear
especially philosophical.  In \book{Crito} he argues from exceedingly
un-Socratic and commonplace premises.  Although he says that he agrees with
core Socratic principles,\footnote{See for example 47a--b.} his arguments run
against those ideals and it's unclear how well he understood what he was
agreeing to.  \citet{brickhouse2004} 196 also note that Crito takes no part in
the philosophical conversation in \book{Phaedo}, though they argue that he is
philosophicaly more adept than I'm claiming.

% }}} Characters

% {{{ Setting
\subsection{Setting}

The conversation between Socrates and Crito takes place in Socrates' jail cell.
\citet{brickhouse2004} 197 argue that the jail would have been close to the
court where the trial was held.  They place it somewhere near the agora on the
southern side.  \book{Crito} itself doesn't do much with the scenery or
location, except perhaps hint that Crito bribed a guard.  (See below for more
on this.)

% }}} Setting

% }}} Characters and setting

% {{{ Introductory material (43a1--50a5)
\section{Introductory material (43a1--50a5)}

% {{{ Introduction (43a1--44b5)
\subsection{Introduction (43a1--44b5)}

The dialogue begins with Socrates waking up in his cell and finding Crito
already there.  Socrates is surprised to find Crito there so early and that the
guard let Crito in at all.  They talk, and Crito tells Socrates that he will be
put to death tomorrow.  Socrates' execution was delayed by a religious
ceremony, and the return of a boat from Delos will mark the end of that
ceremony.  Socrates says that he has had a dream hinting at the same thing, but
that he thinks it will be two days before he is executed.

The opening conversation between Crito and Socrates may seem like small talk,
but as is often the case in Socratic dialogues, it plants two important seeds
in the readers minds.

\begin{enumerate}

    \item There is a strong hint in 43a8 that Crito has bribed (or
        \word{tipped}, if you insist) a guard.\footnote{\citet{burnet1924} 255
        and \citet{brickhouse2004} 249 deny the clear implication of the Greek,
        namely that Crito gave the guard money.  Instead they believe that
        Crito simply did the guard a favor of some kind.  Burnet's only
        argument seems to be that the implication is ``vulgar", which is no
        proof that it isn't true.  \citet{brickhouse2004} simply cite Burnet;
        they provide no independent argument or evidence.  Both
        \citet{adam1988} 22-23 and \citet{rose1983} believe that Crito gave the
        guard money.}  This suggests immediately that Crito is someone who
        solves problems with money.

    \item Socrates says that it would be foolish for him to be upset about
        dying, given how old he is.  Crito replies that many people do it just
        the same when they fall into bad luck.  Socrates lets this pass, but it
        foreshadows Crito's use of common opinion later as well as Socrates
        lack of interest in what most people believe.

\end{enumerate}

% }}} Introduction (43a1--44b5)

% {{{ Crito urges Socrates to escape (44b5--46a9)
\subsection{Crito urges Socrates to escape (44b5--46a9)}

Crito encourages Socrates to flee.  His opening language suggests that he has
been encouraging Socrates to do so for some time, but that Socrates won't
listen.  Initially, he makes two arguments:

\begin{enumerate}

    \item If Socrates dies, Crito will lose an irreplaceable friend (44b8--9).

    \item If Socrates dies, many people who don't know Socrates and Crito will
        think that Crito was cheap and let Socrates die when he might have
        saved him (44b9--c5).

\end{enumerate}

Socrates ignores the first point and dismisses the second quickly.  He says
that they should not care what {\g οἱ πολλοί} think. Only the opinion of the
most reasonable people matters, and they will not think Crito was cheap
(44b6--9).

Crito counters by arguing that the many have great power:  Socrates' situation
proves that they can do very great harm to someone.  To this Socrates offers
a surprising reply: He wishes that they \emph{could} do great harm.  The
background thought here appears to be the following:

\begin{enumerate}

    \item The greatest good for someone is an improvement of their soul, and
        the greatest harm is the degradation of their soul.

    \item Someone who can harm someone's soul could also improve it.  (This
        relies on the thought that knowledge of such a matter is holistic:
        a doctor, in Socrates' opinion, can produce sickness as much as
        health.\footnote{\citet{adam1988} 31 refers to \book{Hippias Minor}
        366bff, and \citet{brickhouse2004} 200 cite \book{Republic} 333e3ff.
        \citet{burnet1924} 259--260 also compares \book{Antigone} 334ff.})

    \item Socrates therefore wishes that they were capable of harming someone
        greatly since that would imply that they were capable of benefitting
        someone equally greatly.

\end{enumerate}

Following this initial skirmish, Crito again asks if Socrates is worried about
what may happen to his friends if they help him escape.  Socrates says that is
is concerned about this---though he also hints that he has ``many other"
concerns (45a1--2).  Crito focuses only on the first part, however, and he
launches into an impassioned speech.  In this second speech, he makes a number
of arguments.

\begin{enumerate}

    \item It won't even cost very much to free Socrates (45a6--8).

    \item Socrates shouldn't worry about someone bringing his friends to trial
        for helping him escape.  They can easily (and cheaply) bribe any
        potential prosecutors (45a8--b1).

    \item Socrates has Crito's money to use as he wishes.  If Socrates doesn't
        want to ask his Athenian friends for money, there are others who can
        provide the money.  He mentions Simmias and Cebes by name (45b1--6).

    \item Socrates shouldn't worry about what he said at trial, namely that he
        has nowhere to go.  There are many places he might go.  Crito mentions
        Thessaly in particular (45b6--c5).

    \item What Socrates is doing is unjust.  He is betraying himself when it's
        possible for him to live.  He is hurrying to do to himself what his
        enemies want to do to him (45c6--9).

    \item Socrates is also betraying his children.  As a parent, Socrates has
        an obligation to raise and educate his children (45c10--d6).

    \item Socrates is behaving like a coward.  This is especially glaring since
        he has spent his life sayin that he cares so much about {\g ἀρετή}
        (45d6--e4).

    \item Socrates makes his friends look bad because it appears that they did
        not do the little it would require to save him (45e5--46a3).

\end{enumerate}

These arguments come tumbling out of Crito one after the other.  He doesn't
spend especially long on any of them.  Since he is so emotional, I wouldn't
necessarily draw any conclusions about Crito's dialectical skills from this
passage.  The overall impression is that he terribly wants Socrates to agree,
he half knows that Socrates won't and so he dumps every possible argument in an
effort to somehow move Socrates.

% }}} Crito urges Socrates to escape (44b5--46a9)

% {{{ Socrates initial response (46b1--c6)
\subsection{Socrates initial response (46b1--c6)}

Socrates response is measured. He says that Crito's enthusiasm does him
credit~---provided that Crito is enthusiastic about something right. Otherwise,
the vehemence of Crito's encouragment only makes things worse.  Socrates then
states a very basic principle:

\begin{quote}
    {\g ἐγὼ οὐ νῦν πρῶτον ἀλλὰ καὶ ἀεὶ τοιοῦτος οἷος τῶν ἐμῶν μηδενὶ ἄλλῳ
    πείθεσθαι ἢ τῷ λόγῳ ὃς ἄν μοι λογιζομένῳ βέλτιστος φαίνηται} (46b4--6).

    I am not now for the first time but always [have been] the type to obey
    nothing of mine other than the argument which seems best to me as I reason.
\end{quote}

Socrates continues by saying that he can't dump his previous arguments just
because he has suffered this bad luck --- provided that they still seem correct
to him.

The focus throughout this section is on reason.  Socrates emphatically states
that he won't obey a anything other than whatever argument seems best to him.
Nor will he throw an argument overboard if it has a bad outcome for him or if
he has had back luck and really wants a different answer.  Reason is in the
driver's seat.

The phrase {\g τῶν ἐμῶν μηδενὶ ἄλλῳ πείθεσθαι ἢ τῷ λόγῳ} is ambiguous.  On the
one hand, we're likely to take {\g τῶν ἐμῶν} to be people: Socrates won't be
persuaded by any of his friends or relatives over the dicates of reason.  On
the other hand, the phrase could also be used if Socrates meant parts of
himself, so to speak. That is, ``I obey nothing in me other than reason."  If
we read it that way, he is distancing himself from emotion and desire.

Arguments can be made for either reading of {\g τῶν ἐμῶν}. Both
\citet{burnet1924} 268 and \citet{rose1983} 27 take the phrase to be neuter and
indicate parts of Socrates himself.  Burnet cites 47c5 and 47e8, both of which
are excellent parallels for Socrates uses a plural neuter to talk about
``parts" of a person.\footnote{In those passages, Socrates has in mind
a distinction between body and soul, so there is no danger of importing
a Platonic division of the soul into this dialogue.}  A second reason to take
{\g τῶν ἐμῶν} as neuter and referring to emotions is that at 46c3ff. Socrates
specifically says that he will not yield to fear no matter what the many
threaten him with.  On the other hand, \citet{adam1988} 40 takes {\g τῶν ἐμῶν}
as Socrates' reply to Crito's {\g πείθου μοι} at 46a.  As Adam says, ``{\g τῶν
ἐμῶν} includes Socrates' friends as well as everything else that could be
called his" (40).\footnote{The argument is not really over whether {\g τῶν
ἐμῶν} is neuter or masculine.  I think that everyone takes it to be neuter.
The question is how much the phrase encompasses.}

% }}} Socrates initial response (46b1--c6)

% {{{ Whose opinions matter? (46c7--48a10)
\subsection{Whose opinions matter? (46c7--48a10)}

Socrates begins by responding to Crito's appeals to the many.  In this section,
Socrates does not care so much \emph{what} Crito argued but \emph{how} he
argued for it.  By appealing to the many, Crito has violated an important
principle of Socratic method.  As Socrates explains, they have often said that
some opinions are worth listening to and others are not.  Socrates then walks
Crito through a short elenchus as follows.

\begin{enumerate}
    \item Some opinions are worth listening to and others are not (46c8--47a6).
    \item The opinions worth listening to are good and the others bad.  Good
        opinions belong to wise people, and bad ones belong to unintelligent
        people (47a7--11).
    \item If someone takes the exercise advice of the many rather than a doctor
        or a trainer, he is likely to suffer physical harm (47a12--47c8).
    \item Likewise, if you follow the advice of the many in regards to justice
        and injustice, you are likely to suffer harm in whatever part of
        a person benefits from justice and suffers under injustice (47c8--d7).
    \item If the body is terribly damaged, then life is intolerable
        (47d8--47e6).
    \item The part of us benefitted and harmed by justice and injustice is more
        important than the body.  Thus all the more should we avoid harming it.
        Thus we should not care about the opinions of the many but only about
        the opinions of the one who knows what he is talking about
        (47e7--48a7).
    \item Thus Crito was wrong to begin by appealing to the opinion of the many
        (47a7--48a10).
\end{enumerate}

% }}} Whose opinions matter? (46c7--48a10)

% {{{ The power of the many (48a10--48b9)
\subsection{The power of the many (48a10--48b9)}

Socrates anticipates an objection: ``But the many are able to kill us." Crito
agrees emphatically, and now Socrates gets his smug on. Yes, the many can kill
us, but what we should care about is living well, not simply living.  This too
is a previous subject of conversation, and when Socrates reminds Crito of this
principle, Crito agrees to it as well.  This effectively cuts off the only
objection to the previous points that Socrates considers worth mentioning.

% }}} The power of the many (48a10--48b9)

% {{{ Summary on method (48b10--49a3)
\subsection{Summary on method (48b10--49a3)}

Relying on these agreements, Socrates restates their problem as a question
about justice:

\begin{quote}
    {\g πότερον δίκαιον ἐμὲ ἐνθένδε πειρᾶσθαι ἐξιέναι μὴ ἀφιέντων Ἀθηναίων ἢ οὐ
    δίκαιον· καὶ ἐὰν μὲν φαίνηται δίκαιον, πειρώμεθα, εἰ δὲ μή, ἐῶμεν}
    (48b11--c2).

    Is it just for me to try to go away from here without the Athenians
    allowing it or is it not just? And if it appears to be just, let's try, but
    if no, let's leave it.
\end{quote}

Socrates follows up by pointedly saying that they should ignore considerations
of money, the opinion of other people, Socrates' children and the rest of
Crito's arguments.  They will focus \emph{only} on whether the proposed actions
would be just or not.

I'm not sure that this is quite fair to Crito or his earlier arguments.
Socrates can certainly argue that he cares most of all about being just, but
can't Crito respond that it's unjust to abandon your children when you don't
have to?  Also, what about the possibility that Socrates will do something
unjust, no matter which course of action he chooses?  Would Socrates have some
notion then of choosing the lesser evil?

% }}} Summary on method (48b10--49a3)

% {{{ Basic moral principles (49a4--e4)
\subsection{Basic moral principles (49a4--e4)}

Socrates reminds Crito that they have often said that they should never commit injustice willingly.  He adds that it is neither good nor fine to do so (ever).  Crito agrees to this rule once again, and Socrates continues with a number of variations in order to make sure that he and Crito agree on these basic principles.  Here are the principles they agree to:

\begin{enumerate}
    \item One must never do injustice (49a4--b8).  Not even in return for
        injustice should one do injustice (49b9--c1).
    \item One should not harm (49c2--3).  Not even in return for wrong should
        one do wrong (49c4--6).  To do someone harm is no different than to do
        injustice (49c7--9).
\end{enumerate}

Socrates goes out of his way to stress the following.  First, Crito must
explicitly agree to these principles.  To that end, Socrates repeats them again
and and again, each time asking if Crito agrees.  After even all that, Socrates
restates the principles again, and he explicitly says that Crito must decide if
he really agrees to these things (49c10--e3).  Second, as \citet{kraut1984}
explains, (25--27), Socrates also makes clear to Crito (and Plato makes clear
to his readers) that he will use a number of terms interchangeably.  For his
purposes the following are the same: {\g ἀδικεῖν}, {\g κακουργεῖν}, {\g κακῶς
ποιεῖν τινα}.  Not everyone would have treated these as identical, but Socrates
will at present.

\citet{kraut1984} makes two additional good points about this stretch of the
dialogue.  First, Plato hints here at the direction that the argument will
take. Socrates will likely concede that the decision to put him to death was
unjust, but nevertheless argue that he must not answer injustice with
injustice.  Second, this passage only lays the groundwork for Socrates later
arguments.  It is not itself yet a direct argument that Socrates should not
escape.  That argument will come later, though it will rely on the principles
that Crito agrees to here.

% }}} Basic moral principles (49a4--e4)

% {{{ The principle of just agreements (49e5--8)
\subsection{The principle of just agreements (49e5--8)}

Before moving on, Socrates briefly extracts another principle from Crito: just
agreements must be kept.  As \citet{kraut1984} and \citet{brickhouse2004} note,
the restriction \emph{just} is essential.  Socrates does not say simply that
one must do whatever one agrees to.  He seems aware that someone might
accidently agree to do something unjust, and in that case, the basic principle
that one should never do injustice presumably takes precedence.

\citet{kraut1984} 32 argues that we should read this section together with
things that the Laws say later.  The Laws say that Socrates made his agreements
with them in a just manner.  A \phrase{just manner} meaning that he was not
compelled, he was not lied to, he was not rushed and he understood what he was
agreeing to.  Thus, agreements can be just in two ways: they cover just actions
and they are justly made.  Agreements to unjust action are never binding, since
injustice must never be done.  Agreements to just things made unjustly are not
necessarily binding.

% }}} The principle of just agreements (49e5--8)

% {{{ Crito can't quite see the upshot (49e9--50a5)
\subsection{Crito can't quite see the upshot (49e9--50a5)}

Drawing out the implications of what they have agreed to so far, Socrates asks
Crito whether (1) it would be just for him to escape without persuading the
city and (2) whether he would be violating a just agreement if he did so.
Crito is unable to answer since he does not follow.  He says bluntly ``I do not
understand" (50a5).

I'm inclined to agree with \citet{brickhouse2004} 211--212 that this does not
suggest that Crito is insincere or unintelligent.  I also don't see any reason
to think that Crito has been inattentive to the argument up to this point.
Socrates hints at two points here that require further elaboration.  Either one
of them would merit Crito holding things up.  First, Socrates says that they
might do injustice to ``those who least deserve it". Who are these who least
deserve it, and why do they least deserve injustice?  Second, what agreements
does Socrates have in mind?

Dramatically and logically, Crito's confusion makes perfect sense.  It provides
Plato a reason for Socrates to introduce the Laws.  We readers are likely to be
pretty much where Crito is, so the speech of the Laws will help us understand
things better.  And in terms of the argument, many people are likely to have
agreed with Socrates up until now on basic principles.  But now Socrates must
make more substantive arguments to show how escape would harm the state, and
why citizens owe the state particular loyalty.

% }}} Crito can't quite see the upshot (49e9--50a5)

% }}} Introductory material (43a1--50a5)

% {{{ The Laws (50a6--54d2)
\section{The Laws (50a6--54d2)}

% {{{ The Laws enter (50a6--50c1)
\subsection{The Laws enter (50a6--50c1)}

Socrates imagines the Laws of Athens addressing him if he tries to flee.  In
their initial argument, the Laws claim that Socrates is trying to destroy them
``for his part".  They justify this by saying that a state cannot exist in
which individuals can ignore legal decisions at their whim.

\citet{kraut1984} 42 describes this as a very early generalization argument,
perhaps the first in Western philosophy.  He spells this out in a somewhat
Kantian manner (though he doesn't mention Kant).  If everyone should behave
like Socrates and invalidate legal rulings that they don't like, then the state
would not survive.  This gives point to the Laws saying that Socrates is trying
to destroy them ``for his part".

% }}} The Laws enter (50a6--50c1)

% {{{ Why we owe the Laws so much (50c1--51c5)
\subsection{Why we owe the Laws so much (50c1--51c5)}

The next section helps to explain why Socrates says that by running away he
would be harming those ``whom he least ought to" (50a2).  Socrates imagines
that he might reply to the Laws first statement by saying that the city does
him an injustice since his case was wrongly decided (50c1--2).  Crito
emphatically agrees, and this gives Socrates an opening to make an argument
that he and the city are not on even terms.

The Laws explain that some things are allowed to them which are not allowed to
citizens.  Socrates owes them nearly everything.  They are responsible for his
birth, insofar as they joined his father and mother together in marriage.  They
are responsible for his care and education, since they set the standards for
such matters.  Thus Socrates is like a child or a slave while the Laws are like
parents or owners. They are not equals: Socrates may not do everything that the
Laws may do.  The argument here relies on an Athenian norm: a parent might
strike or yell at a child, but a child would have no right to return this in
kind.

% {{{ Kraut's reconstruction of this argument
\subsubsection{Kraut's reconstruction of this argument}

Richard Kraut, in \citet{kraut1984}, reconstructs this argument in the
following way:

\begin{enumerate}
    \item One must never treat any city or person unjustly. (49b8)
    \item By escaping, Socrates would be destroying Athens, for his part
        (50d1--51c1)
    \item Athens is responsible for the birth and education of Socrates.
        (50a8--b5)
    \item If some person or city X is responsible for the birth or education of
        Y, then Y treats X unjustly if Y uses violence against X. (51c2--3)
    \item Whoever destorys a city, for his part, uses violence against that
        city (supplied)
    \item If Socrates escapes, he will be treating Athens unjustly.
\end{enumerate}

There are two things to notice about Kraut's reconstruction of the argument.
First, Kraut distinguishes this argument from the argument to ``persuade or
obey" that is buried inside the section on cities as parents.  Second, Kraut
insists that Socrates does \emph{not} argue that political violence is always
wrong.  In particular, Kraut argues (cf. 47, 50) that without the analogy of
cities and parents, Socrates' argument would be incomplete.  The second step
above in Kraut's reconstruction is not enough to draw the conclusion that
Socrates does Athens injustice by trying to escape, on Kraut's view.

Kraut has to do a fair amount of work to massage the argument into this shape
since the text is far from clear.  \citet{brickhouse2004} 212--220 understand
this stretch of \book{Crito} differently, for example.\footnote{I can't
actually follow their reconstruction.  They seem to just pull out strands they
like without worrying too much about how the strands form an argument.}  Having
said that, the Laws are especially scattershot in their presentation, so
I don't necessarily hold that against Kraut (or anyone else) particularly.

% }}} Kraut's reconstruction of this argument

% {{{ Obey or persuade
\subsubsection{Obey or persuade}

The Laws present say that citizens have an obligation to ``obey or persuade",
although this argument is somewhat buried within their larger claims for
special status.  The two alternatives first appear at 51b4:

\begin{quote}
    {\g καὶ σέβεσθαι δεῖ καὶ μᾶλλον ὑπείκειν καὶ θωπεύειν πατρίδα χαλεπαίνουσαν
    ἢ πατέρα, καὶ ἢ πείθειν ἢ ποιεῖν ἃ ἂν κελεύῃ} (51b2--4).

    And it is necessary to revere and to yield to and to serve one's country,
    even when it gives one trouble, more than one's father; and [it is
    necessary] either to persuade or to do whatever it orders.
\end{quote}

The same idea is restated and slightly expanded immediately below at 51c1:

\begin{quote}
    {\g ἀλλὰ καὶ ἐν πολέμῳ καὶ ἐν δικαστηρίῳ καὶ πανταχοῦ ποιητέον ἃ ἂν κελεύῃ
    ἡ πόλις καὶ ἡ πατρίς, ἢ πείθειν αὐτὴν ᾗ τὸ δίκαιον πέφυκε} (51b9--c1).

    But in war and in court and everywhere one must do whatever the city and
    fatherland order or persuade it what the nature of justice is.
\end{quote}

The expansion suggests that the options are (1) obey or (2) persuade the city
that its orders are unjust. Socrates would be wrong to escape his sentence
since he chooses neither available option.

% }}} Obey or persuade

% }}} Why we owe the Laws so much (50c1--51c5)

% {{{ The argument from just agreements (51c6--53a)
\subsection{The argument from just agreements (51c6--53a)}

The Laws next remind Socrates that he has made an agreement with them.  Anyone
who does not like how the Laws do things (i.e. how Athens is run?) may leave:

\begin{quote}
    {\g προαγορεύομεν τῷ ἐξουσίαν πεποιηκέναι Ἀθηναίων τῷ βουλομένῳ, ἐπειδὰν
    δοκιμασθῇ καὶ ἴδῃ τὰ ἐν τῇ πόλει πράγματα καὶ ἡμᾶς τοὺς νόμους, ᾧ ἂν μὴ
    ἀρέσκωμεν ἡμεῖς ἐξεῖναι λαβόντα τὰ αὑτοῦ ἀπιέναι ὅποι ἂν βούληται}
    (51d2--5).

    We announce in advance by having made a possibility for anyone of the
    Athenians who wants when he comes of age and see the affairs of the city
    and us Laws, to whomever we are not pleasant, that it is possible to take
    his things and go wherever he wishes.
\end{quote}

Any Athenian may leave, but the Laws argue that those who stay have ``made an
agreement in deed" (51e4) to \phrase{obey or persuade}.  The Laws also add (at
52e) that the agreements are not forced, deceptive or rushed.  This brings out
how the agreements are \word{just}.  Neither Socrates nor the Laws dwell on the
point, but it's important.  An agreement made under duress, deceptively or in
a very brief time is intuitively not as binding as one made freely, with full
information and after due deliberation.

% }}} The argument from just agreements (51c6--53a)

% {{{ ad hominem regarding Socrates (52a3--52d8)
\subsubsection{ad hominem regarding Socrates (52a3--52d8)}

The Laws briefly focus on Socrates and they claim that he especially has made
such an implicit agreement.  Their reasoning is that insofar as Socrates left
Athens almost never (only for military service, apparently), he has especially
made the \foreign{de facto} agreement that he approves of Athens and her laws.

% }}} ad hominem regarding Socrates (52a3--52d8)

% {{{ Will Socrates benefit anyone by escaping? (53a9--54b2)
\subsection{Will Socrates benefit anyone by escaping? (53a9--54b2)}

The Laws next turn to a more pragmatic question: Will Socrates do himself or
his friends and relatives any good if he escapes?  The Laws argue that he will
not.  His friends are likely to be tried, lose their money and go into exile
themselves.  And wherever Socrates goes, he will be viewed as an enemy of the
state by anyone who cares about their state's wellbeing.  Furthermore, if
Socrates goes on as he did before---talking about virtue and justice---he will
be ridiculous.  The topics he discusses and his behavior will be completely out
of sync.  Alternatively, he could go somewhere where the living is easy and
just party all the time.  But then again he will look ridiculous, and his life
will be out of sync with his earlier words.  Finally, the Laws say that if he
stays alive for the sake of his children, he should think about the
consequences for them.  On the one hand, he could make them exiles too, but
then they would face the same problems abroad that he will.  On the other hand,
he could count on friends at home to care for them.  But those friends would
have done the same if Socrates had died.  So that gives him no extra reason to
stay alive.

% }}} Will Socrates benefit anyone by escaping? (53a9--54b2)

% {{{ The Laws of Hades (54b3--d2)
\subsection{The Laws of Hades (54b3--d2)}

The Laws conclude with a threat of future punishment.  Even if he escapes them
and does wrong in return for wrong, Socrates will still have to face the Laws
in Hades.  They will avenge their earthly fellow laws if Socrates runs off into
exile.  Thus, for prudential reasons too, Socrates should not break the law.

% }}} The Laws of Hades (54b3--d2)

% }}} The Laws (50a6--54d2)

% {{{ Conclusion 54d3--e2
\section{Conclusion 54d3--e2}

Socrates closes the dialogue as a whole in an odd way.  He tells Crito that he
is ready to hear any objection that Crito has, but at the same time he says
that the arguments of the Laws ring in his ears just as ``Corybantes seem to
hear the pipes" (54d4--5).  The Corybantes were the priests of Cybele, and they
were famous for their frenzy and irrationality (see Catullus 63, for example).
So it's disconcerting for Socrates to say all at once ``Do you have any other
argument?" and ``I'm carried away by the irrational love of the previous
arguments."  \citet{rose1983} 40 says, ``[W]e can at least say here that Plato
wishes to suggest that {\g λόγοι}, reasoned arguments, create for Socrates the
ecstasy for which others turn to orgiastic and irrational rites."

Nevertheless it bothers me.  Just as Socrates asks Crito for further argument,
he seems to say that he's not open to any such persuasion.

% }}} Conclusion 54d3--e2

\newpage
\bibliographystyle{apa}
\bibliography{plato}

\end{document}
% }}}
