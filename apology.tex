% {{{ LaTeX prelude
\documentclass[11pt]{article}
\usepackage{fontspec}
\setmainfont[Ligatures={Common,TeX}]{Palatino}
\usepackage{url}
\usepackage{parskip}
\usepackage{natbib}
\bibpunct{(}{)}{;}{a}{,}{,}
\usepackage{titles}
% }}}

% {{{ LaTeX document
\begin{document}

% {{{ Title page
\begin{titlepage}
\title{Notes on Plato's \book{Apology}}
\author{Peter Aronoff}
\date{May 2013}
\maketitle
\end{titlepage}
% }}}

% {{{ Defense speech (17a--35d)

% {{{ Introduction (17a--19a7)
\section{Proem (17a--18a6)}

Socrates starts by saying that the prosecutors spoke so powerfully than
even \emph{he} nearly forgot who Socrates was.  However, he continues, they
said nearly nothing true.   He also says that the most shameless lie of all
was that he was clever speaker, since his coming speech will immediately
disprove the prosecutors.\footnote{Of course, you could easily draw up
a long list of the rhetorical commonplaces in just this section of Socrates'
speech.  For example: eloquent but false versus plain but true; the simple
truth-speaker lost in the world of fancy speakers; truth needs no
adornments.}  This kicks off the tone and content of the rest of this
introduction.  Socrates promises to speak only the truth, but he begs
indulgence for his simple, unsophisticated speaking style.  He ends by
reminding the jurors that the ἀρετή of a juror is to judge whether someone
says just things or not, while the ἀρετή of a speaker is to speak the
truth.
% }}}

% {{{ Socrates' two sets of accusers (18a7--19a7)
Socrates also explains that he has two sets of accusers and that he fears
the earlier accusers more than the current ones.  The current accusers are
the ones you would expect: Anytus, Meletus and Lycon.  The earlier accusers
are the ones who say that

\begin{quote}
    ἔστιν τις Σωκράτης σοφὸς ἀνήρ, τά τε μετέωρα φροντιστὴς καὶ τὰ ὑπὸ γῆς
    ἅπαντα ἀνεζητηκὼς καὶ τὸν ἥττω λόγον κρείττω ποιῶν (18b7--c1).

    There is some wise man Socrates, a thinker about heavenly matters and
    one who has investigated all things under earth and someone who makes
    the weaker argument stronger.
\end{quote}

These are more or less generic descriptions of sophistic activities.  The
only standard sophist accusation missing is that Socrates takes money to
teach other people to do these same things.

Socrates believes that they earlier accusers are dangerous to him for
a number of reasons.  First, people tend to believe that men like those
generic sophists don't believe in the gods.  This would obviously prejudice
jurors in his current case.  Second, there have been many such accusers and
they have been at it for a long time.  Third, they started in on the jurors
(and presumably on all the people of Athens) when they were young.  As
Socrates says, young people are especially credible.  Fourth, they've been
spreading these claims with nobody pleading the opposite case.\footnote{Both
Plato and Xenophon write partly for this reason.  Xenophon is more explicit,
but the desire to improve the reputation of Socrates is clear in Plato as
well.}  Fifth, Socrates does not know their names~--- except for
Aristophanes, who for some reason Socrates \emph{doesn't} name as
such.\footnote{He calls him ``a certain comic poet" at 17d1.  Is there some
point to this?}  Since he does not know their names, he cannot call them to
the stand and question them.

For all these reasons, Socrates asks for indulgence.  His jurors should be
patient as he defends himself first against his first accusers and then
against his current ones.  Socrates also expresses doubt that he will be
able to answer this long-standing slander in the short time given to him,
but he says simply, ``ὅμως τοῦτο μὲν ἴτω ὅπῃ τῷ θεῷ φίλον, τῷ δὲ νόμῷ
πειστέον καὶ ἀπολογητέον" (19a6--7).\footnote{"Nevertheless, let this go
however is pleasing to the god; I must obey the law and give my defence."
Note how Socrates manages to be both pious and obedient to the law of his
city here.}
% }}}

% }}}

% {{{ Penalty speech (35e--38b)
% }}}

% {{{ Post-trial speech (38c--42a)
% }}}

\newpage
\bibliographystyle{apa}
\bibliography{plato}

\end{document}
% }}}
