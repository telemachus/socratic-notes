% {{{ LaTeX prelude
\documentclass[11pt]{article}
\usepackage{fontspec}
\setmainfont[Ligatures={Common,TeX}]{Palatino}
\usepackage{url}
\usepackage{parskip}
\usepackage{natbib}
\bibpunct{(}{)}{;}{a}{,}{,}
\usepackage{titles}
\usepackage{titlesec}
\setcounter{secnumdepth}{4}
\titleformat{\paragraph}
{\normalfont\normalsize\bfseries}{\theparagraph}{1em}{}
\titlespacing*{\paragraph}
{0pt}{3.25ex plus 1ex minus .2ex}{1.5ex plus .2ex}
% }}}

% {{{ LaTeX document
\begin{document}

% {{{ Title page
\begin{titlepage}
\title{Notes on Plato's \book{Apology}}
\author{Peter Aronoff}
\date{May--June 2013}
\maketitle
\end{titlepage}
% }}}

% {{{ On the trial and dialogue
\section{Background}

In 399 BCE, Socrates was brought to court by Meletus, Anytus and Lycon on charges of religious impiety and corrupting the youth.  He was convicted, sentenced to death and executed.  The trial was formally a \foreign{graphê} rather than a \foreign{dikê}.  In the case of a \foreign{graphê}, anyone could prosecute, but only a victim or close relative could initiate a \foreign{dikê}.  In addition, the specific charge that Socrates faced was a \foreign{agôn timêtos}.  This meant that it didn't come with a specific penalty.  The prosecutor would demand a penalty in the indictment, and if the defendant was convicted he or she had an opportunity to specify a counter-penalty.  As a religious crime, Socrates trial was held at the office of the king-archon, whose jurisdiction included religious crimes.\footnote{For all these legal matters and more historical background, see \citet{brickhouse2004}, especially 72--79.}

Plato's \book{Apology} is only one of our sources for Socrates' trial, and there are many questions about it.  We don't know exactly when Plato wrote it, we don't know how accurate it is, we don't know how best to compare it with the other surviving defense speech, Xenophon's \book{Apology}.  A very common view is that (1) it is a very early Platonic work and (2) it generally represents Socrates more accurately than Xenophon's speech does.  I worry that both assumptions rely too much on Plato's eloquence rather than on firm (or even specific) arguments.  Nevertheless, it's the majority view.  On the other hand, there's not much agreement as to how much of Plato's \book{Apology} itself is Plato and how much is Socrates.\footnote{ \citet{brickhouse2004} 80--83 do a good job of showing how vague and unhelpful this whole question is.}
% }}} On the trial and dialogue

% {{{ Defense speech (17a--35d)

\section{Defense speech (17a--35d)}

% {{{ Introduction (17a--19a7)
\subsection{Proem (17a--18a6)}

Socrates starts by saying that the prosecutors spoke so powerfully that even \emph{he} nearly forgot who Socrates was.  However, he continues, they said nearly nothing true.   He also says that the most shameless lie of all was that he was clever speaker, since his coming speech will immediately disprove the prosecutors.\footnote{Of course, you could easily draw up a long list of the rhetorical commonplaces in just this section of Socrates' speech.  For example: eloquent but false versus plain but true; the simple truth-speaker lost in the world of fancy speakers; truth needs no adornments.}  This kicks off the tone and content of the rest of this introduction.  Socrates promises to speak only the truth, but he begs indulgence for his simple, unsophisticated speaking style.  He ends by reminding the jurors that the ἀρετή of a juror is to judge whether someone says just things or not, while the ἀρετή of a speaker is to speak the truth.
% }}}

% {{{ Socrates' two sets of accusers (18a7--19a7)
\subsection{Socrates' two sets of accusers (18a7--19a7)}

Socrates also explains that he has two sets of accusers and that he fears the earlier accusers more than the current ones.  The current accusers are the ones you would expect: Anytus, Meletus and Lycon.  The earlier accusers are the ones who say that

\begin{quote}
    ἔστιν τις Σωκράτης σοφὸς ἀνήρ, τά τε μετέωρα φροντιστὴς καὶ τὰ ὑπὸ γῆς ἅπαντα ἀνεζητηκὼς καὶ τὸν ἥττω λόγον κρείττω ποιῶν (18b7--c1).

    There is some wise man Socrates, a thinker about heavenly matters and one who has investigated all things under earth and someone who makes the weaker argument stronger.
\end{quote}

These are more or less generic descriptions of sophistic activities.  The only standard sophist accusation missing is that Socrates takes money to teach other people to do these same things.

Socrates believes that they earlier accusers are dangerous to him for a number of reasons.  First, people tend to believe that men like those generic sophists don't believe in the gods.  This would obviously prejudice jurors in his current case.  Second, there have been many such accusers and they have been at it for a long time.  Third, they started in on the jurors (and presumably on all the people of Athens) when they were young.  As Socrates says, young people are especially credible.  Fourth, they've been spreading these claims with nobody pleading the opposite case.\footnote{Both Plato and Xenophon write partly for this reason.  Xenophon is more explicit, but the desire to improve the reputation of Socrates is clear in Plato as well.}  Fifth, Socrates does not know their names~--- except for Aristophanes, who for some reason Socrates \emph{doesn't} name as such.\footnote{He calls him ``a certain comic poet" at 17d1.  Is there some point to this?}  Since he does not know their names, he cannot call them to the stand and question them.

For all these reasons, Socrates asks for indulgence.  His jurors should be patient as he defends himself first against his first accusers and then against his current ones.  Socrates also expresses doubt that he will be able to answer this long-standing slander in the short time given to him, but he says simply, ``ὅμως τοῦτο μὲν ἴτω ὅπῃ τῷ θεῷ φίλον, τῷ δὲ νόμῳ πειστέον καὶ ἀπολογητέον" (19a6--7).\footnote{"Nevertheless, let this go however is pleasing to the god; I must obey the law and give my defence." Note how Socrates manages to be both pious and obedient to the law of his city here.}
% }}}

% {{{ Socrates against his first accusers (19a8--24c2)
\subsection{Socrates against his first accusers (19a8--24c2)}

% {{{ The first charges (19a8--20c3)
\subsubsection{The first charges (19a8--20c3)}

Socrates begins by imagining what the first charges against him would have been.

\begin{quote}
    Σωκράτης ἀδικεῖ καὶ περιεργάζεται ζητῶν τά τε ὑπὸ γῆς καὶ οὐράνια καὶ τὸν ἥττω λόγον κρείττω ποιῶν καὶ ἄλλους ταὐτὰ ταῦτα διδάσκων (19b4--c2).

    Socrates' crime: He is meddlesome, investigating things below the earth and the heavens and making the weaker argument stronger and teaching others these same things.
\end{quote}

As evidence that something like this is the earlier charge against him, Socrates refers to ``the comedy of Arisophanes" (19c3).  In response to this multi-part charge, Socrates says that although he respects people who have knowledge of these physical matters and who are able to teach others, he has no such knowledge or ability.  He says that those who have heard him talking will defend him on this score.
% }}}

% {{{ The source of anti-Socratic feeling (20c4--24b2)
\subsubsection{The source of anti-Socratic feeling (20c4--24b2)}

Socrates acknowledges that this only pushes the problem back: if he \emph{doesn't} do any of these things, then \emph{why} does he have this reputation?  He explains this by telling the story of his divine mission.

% {{{ Socrates' divine mission (20e6--23c1)
\paragraph{Socrates' divine mission (20e6--23c1)}

Once upon a time, Chaerephon asked the Delphic oracle if anyone was wiser than Socrates and the oracle said that no one was wiser.  When Socrates heard this, he was at a loss (ἠποροῦν, 20b7).  He knew that the god would not lie, but he didn't think that he was wise at all.  In order to understand better, Socrates set out to investigate and test the god's words.  He went to someone believed to be wise, and his intention was to show the god that this person was wiser than him.  However, Socrates discovered that although that person \emph{thought} he was wise, he actually was not.  Socrates showed this person that he wasn't actually wise, and unsurprisingly this made the person angry at Socrates.  A number of bystanders also grew angry at Socrates.  Socrates realized, however, that although neither of them knew much of anything, Socrates at least knew this while the other person thought he was wise. In this respect at least, Socrates had to admit that he was wiser than the person he spoke with.

Socrates repeated this process systematically, first with politicians, then poets and then craftspeople.  In each case, he had essentially the same result.  Socrates was wiser in virtue of knowing that he wasn't wise.  And in each case, people grew angry at Socrates.  Socrates attributes the ill will against him to these investigations and others like them.  Socrates also draws a lesson about what the god originally meant by his riddle: Socrates is wisest insofar as he knows that he doesn't know much of anything.  Socrates has continued to question people in service of the god.

It's worth saying that Socrates flirts with hubris at the start of this story.  Although he acknowledges that the god wouldn't lie, he nevertheless sets out to show the god someone wiser than himself.  Initially at least, he seems to assume that he will find such a person.

Also note that Socrates experiences a kind of elenchus here at the hands of the god that is not unlike what he dishes out to other people later. That is, he is confused and frustrated; he is at a loss; he begins thinking that he knows something,\footnote{Ironically enough, what he initially thinks he knows is that he doesn't know anything.} but he eventually realizes that he doesn't quite know what he thinks he does.  Unlike most Socratic interlocutors, however, Socrates really learns from his experience.  He doesn't dig in and become stuborn or refuse to face what he sees.
% }}} Socrates' divine mission (20e6--23c1)

% {{{ Socrates inspires bad habits in young Athenians (23c2-24b2)
\paragraph{Socrates inspires bad habits in young Athenians (23c2-24b2)}

In addition to the people Socrates offends, he also has to worry about his young imitators offending people.  Many wealthy young Athenians follow Socrates around, and they the imitate his procedure themselves. This means that they infuriate other people, and those people blame Socrates for having given the youth these habits.  Because these angry people have nothing real to say against Socrates, they lob generic ``evil philosopher" complaints against him: investigating heavens and earth, religious novelty, making weaker argument stronger.\footnote{Don't the angry people have the genuine complaint against Socrates that he corrupted the youth in the way he describes? Why does he not consider that a real complaint?}
% }}} Socrates inspires bad habits in young Athenians (23c2-24b2)

% }}} The source of anti-Socratic feeling (20c4--24b2)

% }}} Socrates against his first accusers (19a8--24c2)

% {{{ Socrates against his current accusers (24c3--28b2)
\subsection{Socrates against his current accusers (24c3--28b2)}

% {{{ Does Socrates harm the youth? (24c3--26b2)
\subsubsection{Does Socrates harm the youth? (24c3--26b2)}

Socrates says that Meletus ~--- not Socrates --- harms the youth by claiming to care about them when he does not. Socrates engages in a brief elenchus with Meletus.  Since Meletus has brought Socrates to trial for corrupting the youth, Socrates infers that Meletus must care a great deal about who harms and who benefits young people. But Socrates is able to show rather quickly that Meletus has not put very much thought into these questions.  Meletus ends up saying that the laws of Athens and \emph{everyone} except Socrates benefits young people and that Socrates alone harms them.  Socrates then employs a version of his characteristic τεχνή argument: In the case of horses and other animals, many people may harm them and only a small group of experts benefit them. Therefore, by analogy, isn't it odd to think that \emph{everyone} in Athens except Socrates benefits the youth?  Although this argument is rapid and sketchy, the passage as a whole does imply that Meletus hasn't given the matter much thought at all.

Next Socrates forces a dilemma on Meletus. Either he corrupted the youth involuntarily and so deserves advice not punishment, or he deliberately acted in a way likely to harm himself which nobody does.

\begin{enumerate}
    \item It is better to live among good citizens than bad ones because good fellow citizens benefit you while bad fellow citizens harm you (25c5--10).
    \item No one wishes to be harmed by their fellow citizens rather than being benefited by them (25d1--5). (At the end, this becomes simply ``Is there anyone who wishes to be harmed?" Strictly speaking, Meletus denies only the more limited version, though Socrates may be relying on the stronger version in what follows.)
    \item It is not believable that Socrates wouldn't know that if he makes his fellow citizens base, they are likely to do him some evil (25d9--25e5).
    \item Therefore, Socrates concludes that either (1) he didn't corrupt the youths or (2) he did so unwillingly (25e6--26a1).
\end{enumerate}

In one sense, this argument seems characteristically Socratic.  Socrates frequently understands \phrase{harming someone} as making them morally worse.  So it makes sense for him to interpret, corrupting (διαφθείρειν) the youth as a matter of making them worse people.  Nevertheless, that makes the argument awful, since we have no particular reason to believe that corrupted youth harm \emph{all} their fellow citizens indiscriminately.  It's easy to imagine, for example, that the young men that Socrates corrupts wouldn't harm \emph{him} because they hold him in such high esteem.

An additional problem with this argument is that it seems to prove too much.  If we accept it, then isn't the same line of defense open to everyone?  If so, then apparently nobody commits this particular crime in a way that merits punishment.  If anything, then, the argument is not so much a specific defense of Socrates as a reason to change the law.  Socrates, however, does not choose to explicitly present the argument in that light.
% }}} Does Socrates harm the youth? (24c3--26b2)

% {{{ Does Socrates commit religious crimes? (26b2--28a1)
\subsubsection{Does Socrates commit religious crimes? (26b2--28a1)}

Next Socrates considers the accusation that he commits religious crimes.  He begins by noting that according to Meletus irreligion is the specific means by which Socrates corrupts the youth.  So the two charges are very closely connected (26b2--6).

The argument quickly becomes odd. Socrates hones in only on part of the irreligion charge: that he doesn't believe in the gods of Athens.  Socrates, however, feigns (?) confusion and asks Meletus whether the charge is atheism (full stop) or merely not believing in the gods of Athens.\footnote{I say \word{feigns} because the charge itself seems entirely clear everywhere else.  Meletus formally charged Socrates with not believing in the city's gods.  I'm unsure where Socrates came up with the idea of asking about atheism, and I don't see why Meletus takes the bait.  Either Plato is very unfair to Meletus at this point, or something about the charge has been lost in transmission.}  Meletus goes for atheism, and at that point Socrates has him.

Socrates quickly shows that Meletus contradicted himself.  On the one hand, Meletus has just said that Socrates is an atheist.  On the other hand, in his original indictment, Meletus also accused Socrates of believing in ``other new gods" (24c1).  Socrates argues as follows:

\begin{enumerate}
    \item Nobody believes in things-related-to-X and yet doesn't believe in X.  As examples, Socrates asks whether anyone believes in ``human things" but not humans or ``equine things" but not horses.  Socrates also gets Meletus to concede that no one believes in δαιμόνια but not δαίμονας (27b13--27c3).
    \item Meletus has already stated that Socrates introduces new divinities (27c5--8).
    \item Therefore, by Meletus' own argument Socrates must believe in some divine beings, namely δαίμονες (27c8--10).
    \item Therefore, since δαίμονες are gods, then Socrates cannot be a complete atheist, and so what Meletus just said contradicts his earlier charges (27c10--d10).
\end{enumerate}

According to \citeauthor{brickhouse2004} Socrates is sincere here, and his argument is reasonably strong.  They believe that ``the charge means only what Meletus says that it means" (112), and so Socrates is within his rights to ask whether Meletus thinks that he is an atheist.  In addition, they argue that it wasn't really open to Meletus to say ``Yes, you believe in gods, but you don't believe in the official gods of Athens" (116).  Meletus cannot do this, they claim, since it would undermine the connection between these official charges and the earlier slanders against Socrates.  But Meletus and the other prosecutors are relying on such a connection.  Hence, they wouldn't want to admit that Socrates believes in any gods.
% }}} Does Socrates commit religious crimes? (26b2--28a1)

% }}} Socrates against his current accusers (24c3--28b2)

% {{{ Socrates defends the life of philosophy (28b3--30d6)
\subsection{Socrates defends the life of philosophy (28b3--30d6)}

Socrates has wrapped up his formal defense (see 28a2--28b2), and he shifts gears to a more general defense of his odd lifestyle and choices.

First he imagines someone who asks, ``Aren't you ashamed to live a life that might get you killed?" (28b3--5).  In response Socrates argues that it shows poor judgement to worry about life or death. Instead, he claims that the only thing a person should consider when acting is whether or not he does just things (28b6--c1).  Socrates also makes an analogy to the soldiers in Homer's \book{Iliad} who acted without consideration for death (28c1--d4).

This leads Socrates to compare his philosophical mission to his actions as a soldier.  When he was ordered to stand his ground in battle, he did so.  In the same way, the god has now ordered Socrates to philosophize, to investigate himself and others.  It would be impious for him to betray these orders (28d5--29a5).\footnote{Socrates even jokes that if he did so, then it would be appropriate to bring him to trial for impiety (29a2--5).}

Socrates argues that disobeying the orders of the god because he feared death would amount to thinking he was wise when he was not (29a4--5).  Specifically, fearing death is thinking one is wise when one is not (29a5--29b2).  Nobody, says Socrates, knows whether death is a benefit or a harm, but people fear it as if they knew.

Following this line of thought, Socrates makes one of his strongest claims to knowledge anywhere in the early dialogues:

\begin{quote}
    τὸ δὲ ἀδικεῖν καὶ ἀπειθεῖν τῷ βελτίονι καὶ θεῷ καὶ ἀνθρώπῳ, ὅτι κακὸν καὶ αἰσχρόν ἐστιν οἶδα (29b6--7).

    But to commit an injustice and to disobey one's better, whether divine or human, I know that is evil and shameful.
\end{quote}

We might argue that Socrates claims knowledge here only because these ethical claims are almost true by definition.  Nevertheless, it is worth noting the claim.

At this point, Socrates imagines the jury releasing him on the condition that he stop doing philosophy.  He rejects this imagined proposal politely but very firmly.  First, he says that however much he cherishes the men of his city, he will obey the god rather than them (29d2--4).  He also suggests that he is doing the city a favor insofar as he urges all the people he meets to be better people.  In particular, Socrates tries to convince everyone he meets to care more about ἀρετή and their ψυχαί rather than money or physical pleasures.

Socrates argues that they will hurt themselves more than him if they put him to death (30b6--d5).  He makes the famous claim that a good person can't be harmed by a worse one:

\begin{quote}
    εὖ γὰρ ἴστε, ἐάν με ἀποκτείνητε τοιοῦτον ὄντα οἷον ἐγὼ λέγω, οὐκ ἐμὲ μείζω βλάψετε ἢ ὑμᾶς αὐτούς· ἐμὲ μὲν γὰρ οὐδὲν ἂν βλάψειεν οὔτε Μέλητος οὔτε Ἄνυτος~--- οὐδὲ γὰρ ἂν δύναιτο~--- οὐ γὰρ οἴομαι θεμιτὸν εἶναι ἀμείνονι ἀνδρὶ ὑπὸ χείρονος βλάπτεσθαι (30c7--d2).

    Know well, if you kill me, being such as I say, you will not harm me more than yourselves. For neither Meletus nor Anytus could harm me at all~--- in fact, it would not be possible~--- for I don't think that it is proper\footnote{The word here is θεμιτός.  It is difficult to translate: it indicates religious correctness primarily, but can also imply possibility, in the sense of what is right or proper in the universe as decided or overseen by the gods.} for a better man to be harmed by a worse one.
\end{quote}

According to \citet{brickhouse2004} (134--7) the best overall interpretation of this sort of thing is that Socrates means only that the soul of a good man cannot be harmed by ordinary ``evils".  Socrates is aware that it is possible for a good person to suffer harm in all sorts of other ways, and he would even admit that enough of such harms can ruin a good person's life.  But in passages such as this, Socrates focuses on the invulnerability and self-sufficiency of a good person's psychic state.

This is pleasingly sane, but like a lot of what \citeauthor{brickhouse2004} say, it makes Socrates appear almost too sane.  They remove everything that makes Socrates extreme and interesting.  \citeauthor{brickhouse2004} save Socrates' arguments here (and often), but at the cost of making him tame and dull.  I doubt the trade is worth it overall.  I also very much doubt that it's true to what Socrates says.

Immediately after the passage I quoted above, Socrates continues:

\begin{quote}
    ἀποκτείνειε μεντἂν ἴσως ἢ ἐξελάσειεν ἢ ἀτιμώσειεν· ἀλλὰ ταῦτα οὗτος μὲν ἴσως οἴεται καὶ ἄλλος τίς που μεγάλα κακά, ἐγὼ δ᾽οὐκ οἴομαι, ἀλλὰ πολὺ μᾶλλον ποιεῖν ἃ οὗτος νυνὶ ποιεῖ, ἄνδρα ἀκίκως ἐπιχειρεῖν ἀποκτεινύναι (30d2--6).

    However, he might perhaps kill [me] or exile [me] or strip [me] of citizenship. But this man perhaps, and likely others, thinks that these things are great evils, but I do not think [so]. But [I think that it is evil] far more to do what this man is doing right now: attempting to kill a man unjustly.
\end{quote}

Socrates anticipates something like what \citeauthor{brickhouse2004} argue, only in order to reject it explicitly.  He specifically says that \emph{other people} may think that execution, exile and loss of rights are great harms, but that \emph{he} does not.  Socrates is so far from thinking these things are bad that he believes instead that the harm here is rather what the prosecutors are trying to do: kill someone unjustly.

I grant that \citeauthor{brickhouse2004} could use this text against me.  They might argue that Socrates only denies here that execution, exile and so on are \emph{great} harms, not that they are harms full stop.  And, in addition, when Socrates say that trying to kill a man is ``far more [harmful]", this implies that the other things are at least somewhat harmful.  Nevertheless, this argument is not very persuasive to me.  It runs counter to the spirit of everything Socrates says here. And it can't account for the emphatic ``οὐδέν" when Socrates say ``Neither Meletus nor Anytus could harm me at all (οὐδέν)" (30c9--10).

It's easy to see why \citeauthor{brickhouse2004} interpret Socrates as they do: they want to give him a stronger argument.  We can see the kind of trouble Socrates has if we interpret him in the traditional way, consider Bernard Williams:

\begin{quote}
    Socrates gave an account of [well-being] in terms of knowledge and the powers of discursive reason, and he could give this account because of the drastically dualistic terms in which he conceived of soul and body. Well-being was the desirable state of one's soul~--- and that meant of oneself as a soul, since an indestructible and immaterial sould was what one really was.\footnote{Williams cites \book{Phaedo} 115C--D here.  Socrates famously tells Crito that they should bury him however they like, provided they can catch him.}  Such a conception underlay Socrates' conception of our deepest interests and made it easier for him to believe that, in a famous phrase, \phrase{the good man cannot be harmed}, since the only thing that could touch him would be something that could touch the good state of his soul, and that was inviolable.  It is a problem for this view that, in describing ethical motivations, it takes a very spiritual view of one's own interests, but the subject matter of ethics requires it to give a less spritual view of other people's interests. If bodily harm is no real harm, why does virtue require us so strongly not to hurt other people's bodies? \citep[34]{williams1985}
\end{quote}

Although I think that Williams unhelpfully runs together Platonic and Socratic ideas here, nevertheless he exposes a very weak thread in Socrates' position.  Still, I think it's better to deal with what Socrates actually says, rather than to read it so charitably that he comes out saying something innocuous but more defensible.
% }}} Socrates defends the life of philosophy (28b3--30d6)

% {{{ Socrates the gadfly (30d6--31c3)
\subsection{Socrates the gadfly (30d6--31c3)}

Socrates tells the jury that he is not so much speaking on his own behalf at this point as he is on behalf of the Athenians themselves.  Socrates famously compares himself to a gadfly and Athens to a noble but sluggish horse (30e1--31a2).  If they kill him, they will lose the great gift that god has given them.  Without Socrates, the people of Athens are likely to go through the rest of their lives happy in their dogmatic slumber.  He also points to his poverty as a partial proof that he only lives as he does because of divine orders: he is not getting anything out of it.
% }}} Socrates the gadfly (30d6--31c3)

% {{{ Why no politics for Socrates? (31c4--32e1)
\subsection{Why no politics for Socrates? (31c4--32e1)}

While making the gadfly comparison, Socrates says that he goes around to people ``privately, like a father or older brother" (32b3--4).  This leads Socrates to his next point, which is to explain why ``he gives advice privately and is a busybody" (31c4--5), but he takes no part in the political life of the city.
% }}} Why no politics for Socrates? (31c4--32e1)

% {{{ Socrates' daimonion (31c7--32a3)
\subsubsection{Socrates' daimonion (31c7--32a3)}

The reason Socrates doesn't get involved in politics is that if he did, he would get killed.  And the reason he knew not to get involved in politics was his \foreign{daimonion}.  As Socrates explains, the \foreign{daimonion} opposed his getting involved in politics, and Socrates knew to listen to the \foreign{daimonion} (31d4--6).

This section is a major source for our understanding of the \foreign{daimonion} in Socrates' life.  We learn here (1) that the \foreign{daimonion} appeared to Socrates from a very early age and (2) that it only opposed actions without ever advising any.

Socrates states both bluntly and vehemently that if he had engaged in political life, he would have died long ago (31d6--31e1).  He adds, even more generally, that it is not possible for anyone to survive in politics whether at Athens or elsewhere\footnote{Socrates says that ``nobody could survive, if he opposes \emph{you or any other people}" (31e2--3, emphasis mine).  The word Socrates uses for \word{people} is πλῆθος, and this may suggest that Socrates only opposes democracies.  I don't think this is the best overall interpretation though.  See below for the examples chosen for the ``two great proofs".} if one opposes injustices and illegal actions.
% }}} Socrates' daimonion (31c7--32a3)

% {{{ The two great proofs (32a4--32e1)
\subsubsection{The two great proofs (32a4--32e1)}

Socrates now offers two ``great proofs" of the need to say out of politics if you are ``truly battling on behalf of justice" (32a1--2).  The examples strike me as very carefully chosen.\footnote{I have no particular opinion about whether Plato or Socrates chose these examples, but I do think they are not random.}  The two cases cover both democracy and oligarchy, and they involve both active resistance and passive disobedience.  \citet{brickhouse2004} (148ff.) view these as examples of Socrates refusal to commit injustice even at the risk of his life.  However, I think that Socrates offers these primarily as proofs that \emph{if} you engage yourself in politics \emph{and} you are honest, then you won't live long.  When Socrates finishes the two examples, he sums up by saying that if you are just and get involved in politics, you won't live long (32e2--33a1).

Socrates first example concerns his behavior during the trial of the generals after the battle of Arginousae in 406bce.  Although the historical details are not entirely clear,\footnote{For fuller details, see the articles \citet{andrewes1974}, \citet{lang1990} and \citet{lang1992}.} here are the facts that concern us.  The Athenians defeated the Spartans in a naval battle, but they did not retrieve the bodies of their dead, as was customary.  Back home in Athens, families of the dead were so angry that they tried and executed six of the generals together in a single trial.\footnote{You will often see references to the trial of ``the Ten generals", but two died and two did not return for the trial, choosing instead to flee into exile.  See \citet{burnet1924} 212 on 32b2 and \citet{lang1992}.}  Socrates viewed the group trial as illegal, and he publically opposed the decision even in the face of anger and threats.

Socrates next example is the arrest and execution of Leon of Salamis.  As Socrates says, this case took place during the oligarchy, when the Thirty Tyrants had taken over Athens.  The Thirty wished to implicate as many people as possible in their crimes, and so they ordered Socrates and four other men to go get Leon and bring him back to Athens to be executed.  The other four followed orders, but Socrates simply went home.  According to Socrates, this shows that he is unconcerned with death if it conflicts with justice and piety.  He adds that if the Thirty had not fallen soon after, he might have been killed himself.
% }}} The two great proofs (32a4--32e1)

% {{{ Socrates denies being a teacher (32e2--34b5)
\subsection{Socrates denies being a teacher (32e2--34b5)}

There is a somewhat subtle transition here.\footnote{Where by \word{subtle}, I mean that this section initially looks like a non-sequitur or a repetition of earlier points.}  Socrates begins this section by restating (1) that he wouldn't have lived so long if he had been in politics and (2) that he is completely consistent in private and public.  The connection between the preceding sections and this next section is in (2).  Socrates says that he does not ``yield to anyone contrary to justice, not to anyone else and not to any of those people whom those slandering me say are my students" (33a3--5).  He then denies emphatically that he teaches anyone and explains why people \emph{think} that he is a teacher.

The connection concerns the relationship between Socrates and the oligarchy of the Thirty Tyrants.  Among the Thirty were Critias and Charmides, and Socrates was believed to be their teacher.  In addition, Socrates was also often associated with another notorious betrayer of Athens: Alcibiades.  So in order to head off the argument either (1) that Socrates is responsible for these people's bad behavior or (2) that Socrates colluded with them and was in that way unjust, Socrates denies teaching anyone.

Socrates makes a familiar argument.  I am not a teacher.  I do not charge money, and I do not associate only with certain people.  I go around and talk to anyone who will talk to me, both rich and poor.  I am also not responsible for how the people I talk to turn out, since I never promised to teach anyone anything.  In addition, I have never kept ``secret doctrines" which only some people hear.  I say the same thing in public to everyone that I do to anyone in private. (33a5--b8).\footnote{The emphatic denial of secret teachings at the end of this paragraph should have put an end to Straussian bullshit before it ever got started.}

To support his case, Socrates then tries to explain why people follow him around.  The implication, I think, is ``If you're not a teacher, why do you trail people who \emph{feel} like students?"  He explains that his followers enjoy hearing him question people who think they're wise but who are not actually wise.  His point being ``Sure, I look like I have a school of followers, but in fact they're just kids who dig my rap."

Socrates rounds off this section with an argument that he does not harm young people.  If he did, then surely some of them would have come forward in this trial as witnesses.  That is, having grown up and realized what Socrates did to them, they would give testimony against him.  And if they still did not see the truth, surely one of their relatives would give testimony against Socrates.  If Meletus forgot to request such a witness, Socrates offers to give up some of his own time now.  Of course, says Socrates, nobody will witness against him in this way.  Just the opposite: they and their relatives are trying to help him.  Socrates admits again that it would make sense for the young men he corrupted not to give testimony against him, but he says that the only reason why the relatives would offer to support him is that they know Meletus is lying and he, Socrates, is telling the truth.
% }}} Socrates denies being a teacher (32e2--34b5)

% {{{ Socrates explains why he won't beg (34b6--35d9)
\subsection{Socrates explains why he won't beg (34b6--35d9)}

Now that Socrates has finished the substantive part of his defense, he concludes by talking about begging.  It was customary for defendents to present their families and friends and use these people in an appeal to pity.  Socrates won't do so, and he wants to say why.  Before he gives his reasons, however, he engages in a brief \foreign{praeteritio} where he manages to mention his family (wife and sons), even while saying that he won't bring them into things.

\begin{enumerate}
    \item Socrates has a reputation for being a certain kind of person.  Whether that reputation is deserved or not, he has it. If, therefore, Socrates pleads, he will bring shame to the Athenians because foreigners will think that those whom the Athenians consider outstanding are really foolish and cowardly (34d9--35b9).\footnote{Socrates adds a gratuitous bit of sexism, saying that people who behave this way ``are just like women" (35b3).}
    \item Socrates also believes that pleading is intrinsically unjust, in addition to being embarrassing for the pleader and his city.  A juror should judge based on the facts and laws.  He should not be swayed by such emotional appeals.
\end{enumerate}
% }}} Socrates explains why he won't beg (34b6--35d9)

% }}} Defense speech (17a--35d)

% {{{ Penalty speech (35e--38b)
\section{Penalty speech (35e-38b)}

Socrates begins with a little twist of the knife.  He says that he's not surprised with the verdict, but he is surprised at the count. If only thirty votes had gone the other way, he would have won.  Socrates interprets this to mean that he beat Meletus and that unless Anytus and Lycon had joined in, he would have been acquited and Meletus would owe a fine for bringing a frivolous lawsuit.

As a first counter penalty, Socrates proposes that he receive free food like an Olympic victor (36b3--37a2).  His argument is that he was a benefit to the city and the people of the city and that he is poor and in need.  Like Xenophanes, Socrates adds that he dserves this reward more than an Olympic victor.

Socrates reminds the jurors that he doesn't have enough time to make his case.  He believes firmly that nobody does wrong willingly, but he won't be able to persuade the jurors of this in the time he has.  Nevertheless, he is not willing to propose a penalty he thinks is bad instead of death, which as he's said earlier, may be good for all he knows.  He rejects exile or prison for this reason.  Prison would prevent his mission, though Socrates doesn't say so explicitly.  As for exile, if the Athenians wouldn't put up with Socrates~--- his own fellow citizens --- then obviously nobody else would.

Well, Socrates imagines someone asking, can't you just live a quiet life?  The answer is an emphatic no.

\begin{quote}
    ὁ δὲ ἀνεξέταστος βίος οὐ βιωτὸς ἀνθρώπῳ (38a5--6).

    The unexamined life is not [fit] for a person to live.
\end{quote}

Although Socrates again thinks he won't be able to persuade the jury, the life of philosophy is the best possible human life.  He will not give it up.

Socrates does agree to offer a fine.  Initially, he offers a very small amount since he's so poor.  He offers to pay a mina.  However, some of the wealthy friends of Socrates offer to stand as guarantee: they tell him to offer thirty minas, and he does.
% }}} Penalty speech (35e--38b)

% {{{ Post-trial speech (38c--42a)
\section{Post-trial speech (38c--42a)}

% {{{ To those who voted for the death penalty (38c--39d)
\subsection{To those who voted for the death penalty (38c--39d)}

Socrates begins this final speech on a bitter note.  He tells the jurors that they will forever have the reputation for having killed Socrates, but that in a brief time, he would have died anyhow of old age.  Socrates then clarifies that he is speaking only to those who voted for the death penalty.  To those jurors, he adds that he only lost because he had too much shame to say and do anything to gain pity.  He again makes the point that in battle and in life it is shameful to do \emph{anything} in order to survive.  He finishes this section with an elaborate joke about evil and death. Evil runs faster than death.  Socrates as an old man has been caught by death: the slower one for a slow-moving older person.  The prosecutors have been caught by evil: this suits younger men who run faster.

Socrates also makes a prediction to those who condemned him.  They believe that by killing him, they will be free of the troubles of philosophy.  But in fact, Socrates has been holding back many more agressive young men.  Once Socrates is dead, these young men will be let loose on Athens.  And then, the jurors will be sorry for what they've done.  Socrates notes that you should not try to stop people from criticizing you by killing them.  Instead, you should change the things about yourself that deserve criticism.
% }}} To those who voted for the death penalty (28c--39d)

% {{{ To those who voted against the death penalty (39e--42a5)
\subsection{To those who voted against the death penalty (39e--42a5)}

Socrates says that he is happy to speak with those who voted against the death penalty and that he has some time since the jailers are busy.\footnote{Both \citet{adam1914} 97 and \citet{burnet1924} 244 argue that the court officials would be handling normal post-trial business at this point and that the delay is not remarkable or unbelievable.  Some earlier readers argued that the third speech had to be a fiction on the grounds that the delay made no sense.  But as \citet{burnet1924} 242 points out, Xenophon also has Socrates speak after the trial.  Although he admits that Xenophon may have taken this last speach from Plato, he nevertheless argues that it shows that such a speech wasn't impossible.  For if it such a speech had been impossible, Xenophon would not have copied the idea from Plato.}  This way he says this is slightly confusing:

\begin{quote}
    \dots ἐν ᾧ οἱ ἄρχοντες ἀσχολίαν ἄγουσιν καὶ οὔπω ἔρχομαι οἷ ἐλθόντα με δεῖ τεθνάναι (39e2--3).

    \dots while the people in charge are taking a break and I'm not yet going where I will die upon arrival.
\end{quote}

If you read only \book{Apology}, you will probably think that Socrates was excecuted immediately after the trial.  It's only by reading \book{Crito} that you realize there was a delay.  I wonder whether Plato wrote it this way in order to depict the scene at the trial---when people thought Socrates would be killed that day---or because he took it for granted that people would know better.

% {{{ The daimonion did not prevent Socrates (40a1--40c4)
\subsubsection{The daimonion did not prevent Socrates (40a1--40c4)}

Socrates tells the jurors\footnote{He makes a point of saying that these men he can call \word{jurors} correctly.  This explains why all along earlier in the work he called the jurors as a whole \phrase{men of Athens} and not \word{jurors}.  Apparently, you're only a juror if you get things right.} that something remarkable has occurred: the \foreign{daimonion} has not appeared to him.  As he explains, the \foreign{daimonion} normally appeared to him very frequently if he was about to do anything wrong, even in trivial matters (40a4--7).  But he has not heard from it all day: not when he was leaving his house to go to the court and not while he was speaking.  What has happened to Socrates today, however, is something that most people would consider a very great harm.  Socrates infers from this that what has happened is good:

\begin{quote}
    κινδθνεύει γὰρ μοι τὸ συμβεβηκὸς τοῦτο ἀγαθὸν γεγονέναι, καὶ οὐκ ἔσθ᾽ὅπως ἡμεῖς ὀρθῶς ὑπολαμβάνομεν, ὅσοι οἰόμεθα κακὸν εἶναι τὸ τεθνάναι. μέγα μοι τεκμήριον τούτου γέγονεν· οὐ γὰρ ἔσθ᾽ὅπως οὐκ ἠναντιώθη ἄν μοι τὸ εἰωθὸς σημεῖον, εἰ μή τι ἔμελλον ἐγώ ἀγαθὸν πράξειν (40b7--c4).

    For it seems to me that this outcome has proven to be good and there is no way that we believe correctly, those of us who think that death is a harm. And a great proof of this, in my eyes, is the following: There is no way that my customary sign would not have opposed me unless I were about to do something good.
\end{quote}

This inference requires a very strong conception of the \foreign{daimonion}.
As \citet{reeve1989} 181 suggests, we must believe not only that if the
\foreign{daimonion} appears, Socrates is always doing something wrong; we must
believe that the \foreign{daimonion} always appears when Socrates is about to
do something wrong.  Such a belief gives Socrates an infallible way to avoid
doing wrong.  On the one hand, that seems unlikely and far more than Socrates appears to say elsewhere about the \foreign{daimonion}.  On the other hand, it would explain why Socrates seems so confident (to the point of being smug) about his own morality, even though he disclaims moral knowledge.\footnote{This is my bad idea not Reeve's.  I'm not entirely serious, but Socrates truly can be a smug bastard.}
% }}} The daimonion did not prevent Socrates (40a1--40c4)

% {{{ A constructive dilemma (40c4--41c7)
\subsubsection{A constructive dilemma (40c4--41c7)}

Socrates next offers a constructive dilemma suggesting that there is ``great hope" (πολλὴ ἐλπίς, 40c4) that death is good.

\begin{enumerate}
    \item Death is one of two things: complete destruction of the person who dies or a migration to the afterlife (40c5--9).
    \item If death is the complete destruction of the person who dies and there is no perception, then it is like a deep dreamless sleep.  Such a thing would be a gain (κέρδος, 40e2).
    \item If death is a migration to the afterlife, then this would be a great good.\footnote{Socrates then imagines himself visiting with the \emph{real} jurors there and doing philosophy with the legendary figures of myth and epic.}
    \item Thus, whether we choose (2) or (3), death is a good thing.
\end{enumerate}

Relying on the phrase ``great hope", we could distance Socrates from this argument.\footnote{\citet{reeve1989} 182.}  Since this dilemma appears weak on many fronts, I'm sympathetic to the goal of this reading.  However, I think it doesn't do justice to the text.  Socrates explicitly offers the dilemma as a continuation and further argument in support of the same conclusion as the previous argument about the \foreign{daimonion}.  I base this on his use of καὶ τῇδε at the start of this section:

\begin{quote}
    Ἐννοήσωμεν δὲ καὶ τῇδε ὡς πολλὴ ἐλπίς ἐστιν ἀγαθὸν αὐτὸ εἶναι (40c4--5).

    And let us consider \emph{also in this way} that there is much hope that it [sc. death] is good.
\end{quote}

By linking the two arguments in this way, Socrates makes it difficult to say that the first is serious but the second is only speculative.

I'm surprised that people have not been more bothered by Socrates on death as the complete destruction of the person who dies.  The argument is far worse (or odder) than people imagine.  I think I may try to write an article about this, but for the moment, here are my concerns:

\begin{enumerate}
    \item Socrates says that a night of dreamless sleep is better than most days of people's lives.  How can he possibly believe that?  Is lack of sensation better than a day spent philosophizing?  Setting aside what Socrates believes, why should anyone else believe this?  In order to avoid this issue, commentators regularly focus \emph{only} on the nights.  That is, they act as though Socrates says ``a night of dreamless sleep is better than most nights of our lives." But Socrates says ``nights and days" three times in the passage,\footnote{40d4, 40d6--7, 40e1--2.} so there's really no denying this part of his argument.
    \item Why does Socrates mention the Great King?  Yes, most people might view the King of Persia as a standard of human happiness, but surely Socrates does not.  So what is he doing in this argument?  Is his presence a sign that Socrates is only speaking to the vulgar here, i.e. that Socrates does not actually believe what he says?  If so, then it's still bizarre since the life that makes the Great King a standard of human happiness is precisely the kind of life that wouldn't find a dreamless sleep very good at all.  The Great King would presumably say to Socrates what the Cyrenaics said to Epicurus: Your good is the good of a corpse.
    \item It's not as easy as you might suppose to distance Socrates from this argument.  Consider how he ends this part of the dilemma:
        \begin{quote}
            εἰ οὖν τοιοῦτον ὁ θάνατός ἐστιν, κέρδος ἔγωγε λέγω (40e4--5).

            Therefore, if death is such a thing, I certainly call [it a] gain.
        \end{quote}

        Socrates leaves room for the possibility that death is not the complete end of the person who dies, but if death is complete annhilation, Socrates strongly affirms, in his own voice, that it is a good thing.  What is weakest about this argument is not the speculation that death might be the complete destruction of the person who dies.  The most vulnerable section is Socrates' attempt to show that death is a good thing because it's like a deep, dreamless sleep.  This is exactly the part of the argument that Socrates affirms here so strongly, so there's no saving him from the argument's problems.
\end{enumerate}
% }}} A constructive dilemma (40c5--41c7)

% {{{ The good person cannot be harmed and farewell (41c8-42)
\subsubsection{The good person cannot be harmed and farewell (41c8-42)}

Socrates concludes by revisiting the argument that a good person cannot be harmed.\footnote{See \S6 above.}  He repeats the claim that a good person cannot be harmed, adding here ``neither alive nor dead" (40d1--2).  Socrates also affirms that the gods take care for the affairs of a good person.  Finally, Socrates says that it is clear to him that it is better for him now to die and to be separated from his troubles.\footnote{This last thought suggests consolatory rhetoric to me.  If this final part of the speech does draw from consolatory traditions, that might explain some of the weird arguments in the constructive dilemma.}  Socrates asks that the Athenians ``punish" his sons in the same way he punished the Athenians: by reminding them to care most about virtue and not money or anything else and by criticizing them if they care about the wrong things or think that they are something that they are not.

Finally, Socrates must depart.  He leaves with the famous and characteristic goodbye:

\begin{quote}
    ἀλλὰ γὰρ ἤδη ὥρα ἀπιέναι, ἐμοὶ μὲν ἀποθανουμένῳ, ὑμῖν δὲ βιωσομένοις· ὁπότεροι δὲ ἡμῶν ἔρχονται ἐπὶ ἄμεινον πρᾶγμα, ἄδηλον παντὶ πλὴν ἢ τῷ θεῷ (42a2--5).

    But now it is time to depart: for me to die and for you to live. Which of the us goes towards a better thing is unclear to all save the god.
\end{quote}
% }}} The good person cannot be harmed and farewell (41c8-42)
% }}} To those who voted against the death penalty (39e--42a5)
% }}} Post-trial speech (38c--42a)

\newpage
\bibliographystyle{apa}
\bibliography{plato}

\end{document}
% }}}
