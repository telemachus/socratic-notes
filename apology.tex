% {{{ LaTeX prelude
\documentclass[11pt]{article}
\usepackage{fontspec}
\setmainfont[Ligatures={Common,TeX}]{Palatino}
\usepackage{url}
\usepackage{parskip}
\usepackage{natbib}
\bibpunct{(}{)}{;}{a}{,}{,}
\usepackage{titles}
% }}}

% {{{ LaTeX document
\begin{document}

% {{{ Title page
\begin{titlepage}
\title{Notes on Plato's \book{Apology}}
\author{Peter Aronoff}
\date{May 2013}
\maketitle
\end{titlepage}
% }}}

% {{{ On the trial and dialogue
\section{Background}

Here are some basic facts about the trial.  Socrates was brought to court
by Meletus, Anytus and Lycon.  The charges were impiety and corrupting the
youth.  In 399BCE, Socrates was convicted, sentenced to death and executed.
The trial was a \foreign{graphê} rather than a \foreign{dikê}.  In the case
of a \foreign{graphê}, anyone could prosecute, but only a victim or close
relative could prosecute a \foreign{dikê}.  In addition, the specific
charge that Socrates faced was a \foreign{agôn timêtos}.  This meant that
it didn't come with a specific penalty.  The prosecutor would demand
a penalty in the indictment, and if the defendant was convicted he or she
had an opportunity to specify a counter-penalty.  In this case, the
prosecution proposed the death penalty and Socrates proposed a penalty that
was no penalty at all.\footnote{See below for more on his proposed
penalty.}  As a religious crime, Socrates trial was held at the office of
the king-archon, whose jurisdiction included religious crimes.\footnote{For
all these legal matters and more historical background, see
\citet{brickhouse2004}, especially 72--79.}

Plato's \book{Apology} is only one of our sources for Socrates' trial, and
there are many questions about it.  We don't know exactly when Plato wrote
it, we don't know how accurate it is, we don't know how best to compare it
with the other surviving defense speech, Xenophon's \book{Apology}.  A very
common view is that (1) it is a very early Platonic work and (2) it
generally represents Socrates more accurately than Xenophon's speech does.
I worry that both assumptions rely too much on Plato's eloquence rather
than on firm (or even specific) arguments.  Nevertheless, it's the majority
view.  On the other hand, there's not much agreement as to how much of
Plato's \book{Apology} itself is Plato and how much is Socrates.\footnote{
\citet{brickhouse2004} 80--83 do a good job of showing how vague and
unhelpful this whole question is.}
% }}} On the trial and dialogue

% {{{ Defense speech (17a--35d)

% {{{ Introduction (17a--19a7)
\section{Proem (17a--18a6)}

Socrates starts by saying that the prosecutors spoke so powerfully than
even \emph{he} nearly forgot who Socrates was.  However, he continues, they
said nearly nothing true.   He also says that the most shameless lie of all
was that he was clever speaker, since his coming speech will immediately
disprove the prosecutors.\footnote{Of course, you could easily draw up
a long list of the rhetorical commonplaces in just this section of Socrates'
speech.  For example: eloquent but false versus plain but true; the simple
truth-speaker lost in the world of fancy speakers; truth needs no
adornments.}  This kicks off the tone and content of the rest of this
introduction.  Socrates promises to speak only the truth, but he begs
indulgence for his simple, unsophisticated speaking style.  He ends by
reminding the jurors that the ἀρετή of a juror is to judge whether someone
says just things or not, while the ἀρετή of a speaker is to speak the
truth.
% }}}

% {{{ Socrates' two sets of accusers (18a7--19a7)
\section{Socrates' two sets of accusers (18a7--19a7)}

Socrates also explains that he has two sets of accusers and that he fears
the earlier accusers more than the current ones.  The current accusers are
the ones you would expect: Anytus, Meletus and Lycon.  The earlier accusers
are the ones who say that

\begin{quote}
    ἔστιν τις Σωκράτης σοφὸς ἀνήρ, τά τε μετέωρα φροντιστὴς καὶ τὰ ὑπὸ γῆς
    ἅπαντα ἀνεζητηκὼς καὶ τὸν ἥττω λόγον κρείττω ποιῶν (18b7--c1).

    There is some wise man Socrates, a thinker about heavenly matters and
    one who has investigated all things under earth and someone who makes
    the weaker argument stronger.
\end{quote}

These are more or less generic descriptions of sophistic activities.  The
only standard sophist accusation missing is that Socrates takes money to
teach other people to do these same things.

Socrates believes that they earlier accusers are dangerous to him for
a number of reasons.  First, people tend to believe that men like those
generic sophists don't believe in the gods.  This would obviously prejudice
jurors in his current case.  Second, there have been many such accusers and
they have been at it for a long time.  Third, they started in on the jurors
(and presumably on all the people of Athens) when they were young.  As
Socrates says, young people are especially credible.  Fourth, they've been
spreading these claims with nobody pleading the opposite case.\footnote{Both
Plato and Xenophon write partly for this reason.  Xenophon is more explicit,
but the desire to improve the reputation of Socrates is clear in Plato as
well.}  Fifth, Socrates does not know their names~--- except for
Aristophanes, who for some reason Socrates \emph{doesn't} name as
such.\footnote{He calls him ``a certain comic poet" at 17d1.  Is there some
point to this?}  Since he does not know their names, he cannot call them to
the stand and question them.

For all these reasons, Socrates asks for indulgence.  His jurors should be
patient as he defends himself first against his first accusers and then
against his current ones.  Socrates also expresses doubt that he will be
able to answer this long-standing slander in the short time given to him,
but he says simply, ``ὅμως τοῦτο μὲν ἴτω ὅπῃ τῷ θεῷ φίλον, τῷ δὲ νόμῳ
πειστέον καὶ ἀπολογητέον" (19a6--7).\footnote{"Nevertheless, let this go
however is pleasing to the god; I must obey the law and give my defence."
Note how Socrates manages to be both pious and obedient to the law of his
city here.}
% }}}

% {{{ Socrates against his first accusers (19a8--24ca2)
\section{Socrates against his first accusers (19a8--24ca2)}

% {{{ The first charges (19a8--20c3)
\subsection{The first charges (19a8--20c3)}

Socrates begins by imagining what the first charges against him would have
been.

\begin{quote}
    Σωκράτης ἀδικεῖ καὶ περιεργάζεται ζητῶν τά τε ὑπὸ γῆς καὶ οὐράνια καὶ
    τὸν ἥττω λόγον κρείττω ποιῶν καὶ ἄλλους ταὐτὰ ταῦτα διδάσκων
    (19b4--c2).

    Socrates' crime: He is meddlesome, investigating things below the earth
    and the heavens and making the weaker argument stronger and teaching
    others these same things.
\end{quote}

As evidence that something like this is the earlier charge against him,
Socrates refers to ``the comedy of Arisophanes" (19c3).  In response to
this multi-part charge, Socrates says that although he respects people who
have knowledge of these physical matters and who are able to teach others,
he has no such knowledge or ability.  He says that those who have heard him
talking will defend him on this score.
% }}}

% {{{ The source of anti-Socratic feeling (20c4--23c1)
\subsection{The source of anti-Socratic feeling (20c4--23c1)}

Socrates acknowledges that this only pushes the problem back: if he
\emph{doesn't} do any of these things, then \emph{why} does he have this
reputation?  He explains this by telling the story of his divine mission.

% {{{ Socrates' divine mission (20e6--23c1)
\subsubsection{Socrates' divine mission (20e6--23c1)}

Once upon a time, Chaerephon asked the Delphic oracle if anyone was wiser
than Socrates and the oracle said that no one was wiser.  When Socrates
heard this, he was at a loss (ἠποροῦν, 20b7).  He knew that the god would
not lie, but he didn't think that he was wise at all.  In order to
understand better, Socrates set out to investigate and test the god's
words.  He went to someone believed to be wise, and his intention was to
show the god that this person was wiser than him.  However, Socrates
discovered that although that person \emph{thought} he was wise, he
actually was not.  Socrates showed this person that he wasn't actually
wise, and unsurprisingly this made the person angry at Socrates.  A number
of bystanders also grew angry at Socrates.  Socrates realized, however,
that although neither of them knew much of anything, Socrates at least knew
this while the other person thought he was wise. In this respect at least,
Socrates had to admit that he was wiser than the person he spoke with.

Socrates repeated this process systematically, first with politicians, then
poets and then craftspeople.  In each case, he had essentially the same
result.  Socrates was wiser in virtue of knowing that he wasn't wise.  And
in each case, people grew angry at Socrates.  Socrates attributes the ill
will against him to these investigations and others like them.  Socrates
also draws a lesson about what the god originally meant by his riddle:
Socrates is wisest insofar as he knows that he doesn't know much of
anything.  Socrates has continued to question people in service of the god.

It's worth saying that Socrates flirts with hubris at the start of this
story.  Although he acknowledges that the god wouldn't lie, he nevertheless
sets out to show the god someone wiser than himself.  Initially at least,
he seems to assume that he will find such a person.

Also note that Socrates experiences a kind of elenchus here at the hands of
the god that is not unlike what he dishes out to other people later. That
is, he is confused and frustrated; he is at a loss; he begins thinking that
he knows something,\footnote{Ironically enough, what he initially thinks he
knows is that he doesn't know anything.} but he eventually realizes that
he doesn't quite know what he thinks he does.  Unlike most Socratic
interlocutors, however, Socrates really learns from his experience.  He
doesn't dig in and become stuborn or refuse to face what he sees.
% }}} Socrates' divine mission (20e6--23c1)

% }}} The source of anti-Socratic feeling

% }}} Socrates against his first accusers

% }}} Defense speech

% {{{ Penalty speech (35e--38b)
% }}} Penalty speech

% {{{ Post-trial speech (38c--42a)
% }}} Post-trial speech

\newpage
\bibliographystyle{apa}
\bibliography{plato}

\end{document}
% }}}
