% {{{ LaTeX prelude
\documentclass[11pt]{article}
\usepackage{fontspec}
\setmainfont[Ligatures={Common,TeX}]{Palatino}
\usepackage{url}
\usepackage{parskip}
\usepackage{natbib}
\bibpunct{(}{)}{;}{a}{,}{,}
\usepackage{titles}
% }}}

% {{{ LaTeX document
\begin{document}

% {{{ Title page
\begin{titlepage}
\title{Notes on Plato's \book{Apology}}
\author{Peter Aronoff}
\date{May 2013}
\maketitle
\end{titlepage}
% }}}

% {{{ On the trial and dialogue
\section{Background}

Here are some basic facts about the trial.  Socrates was brought to court by Meletus, Anytus and Lycon.  The charges were impiety and corrupting the youth.  In 399BCE, Socrates was convicted, sentenced to death and executed.  The trial was a \foreign{graphê} rather than a \foreign{dikê}.  In the case of a \foreign{graphê}, anyone could prosecute, but only a victim or close relative could prosecute a \foreign{dikê}.  In addition, the specific charge that Socrates faced was a \foreign{agôn timêtos}.  This meant that it didn't come with a specific penalty.  The prosecutor would demand a penalty in the indictment, and if the defendant was convicted he or she had an opportunity to specify a counter-penalty.  In this case, the prosecution proposed the death penalty and Socrates proposed a penalty that was no penalty at all.\footnote{See below for more on his proposed penalty.}  As a religious crime, Socrates trial was held at the office of the king-archon, whose jurisdiction included religious crimes.\footnote{For all these legal matters and more historical background, see \citet{brickhouse2004}, especially 72--79.}

Plato's \book{Apology} is only one of our sources for Socrates' trial, and there are many questions about it.  We don't know exactly when Plato wrote it, we don't know how accurate it is, we don't know how best to compare it with the other surviving defense speech, Xenophon's \book{Apology}.  A very common view is that (1) it is a very early Platonic work and (2) it generally represents Socrates more accurately than Xenophon's speech does.  I worry that both assumptions rely too much on Plato's eloquence rather than on firm (or even specific) arguments.  Nevertheless, it's the majority view.  On the other hand, there's not much agreement as to how much of Plato's \book{Apology} itself is Plato and how much is Socrates.\footnote{ \citet{brickhouse2004} 80--83 do a good job of showing how vague and unhelpful this whole question is.}
% }}} On the trial and dialogue

% {{{ Defense speech (17a--35d)

% {{{ Introduction (17a--19a7)
\section{Proem (17a--18a6)}

Socrates starts by saying that the prosecutors spoke so powerfully that even \emph{he} nearly forgot who Socrates was.  However, he continues, they said nearly nothing true.   He also says that the most shameless lie of all was that he was clever speaker, since his coming speech will immediately disprove the prosecutors.\footnote{Of course, you could easily draw up a long list of the rhetorical commonplaces in just this section of Socrates' speech.  For example: eloquent but false versus plain but true; the simple truth-speaker lost in the world of fancy speakers; truth needs no adornments.}  This kicks off the tone and content of the rest of this introduction.  Socrates promises to speak only the truth, but he begs indulgence for his simple, unsophisticated speaking style.  He ends by reminding the jurors that the ἀρετή of a juror is to judge whether someone says just things or not, while the ἀρετή of a speaker is to speak the truth.
% }}}

% {{{ Socrates' two sets of accusers (18a7--19a7)
\section{Socrates' two sets of accusers (18a7--19a7)}

Socrates also explains that he has two sets of accusers and that he fears the earlier accusers more than the current ones.  The current accusers are the ones you would expect: Anytus, Meletus and Lycon.  The earlier accusers are the ones who say that

\begin{quote}
    ἔστιν τις Σωκράτης σοφὸς ἀνήρ, τά τε μετέωρα φροντιστὴς καὶ τὰ ὑπὸ γῆς ἅπαντα ἀνεζητηκὼς καὶ τὸν ἥττω λόγον κρείττω ποιῶν (18b7--c1).

    There is some wise man Socrates, a thinker about heavenly matters and one who has investigated all things under earth and someone who makes the weaker argument stronger.
\end{quote}

These are more or less generic descriptions of sophistic activities.  The only standard sophist accusation missing is that Socrates takes money to teach other people to do these same things.

Socrates believes that they earlier accusers are dangerous to him for a number of reasons.  First, people tend to believe that men like those generic sophists don't believe in the gods.  This would obviously prejudice jurors in his current case.  Second, there have been many such accusers and they have been at it for a long time.  Third, they started in on the jurors (and presumably on all the people of Athens) when they were young.  As Socrates says, young people are especially credible.  Fourth, they've been spreading these claims with nobody pleading the opposite case.\footnote{Both Plato and Xenophon write partly for this reason.  Xenophon is more explicit, but the desire to improve the reputation of Socrates is clear in Plato as well.}  Fifth, Socrates does not know their names~--- except for Aristophanes, who for some reason Socrates \emph{doesn't} name as such.\footnote{He calls him ``a certain comic poet" at 17d1.  Is there some point to this?}  Since he does not know their names, he cannot call them to the stand and question them.

For all these reasons, Socrates asks for indulgence.  His jurors should be patient as he defends himself first against his first accusers and then against his current ones.  Socrates also expresses doubt that he will be able to answer this long-standing slander in the short time given to him, but he says simply, ``ὅμως τοῦτο μὲν ἴτω ὅπῃ τῷ θεῷ φίλον, τῷ δὲ νόμῳ πειστέον καὶ ἀπολογητέον" (19a6--7).\footnote{"Nevertheless, let this go however is pleasing to the god; I must obey the law and give my defence." Note how Socrates manages to be both pious and obedient to the law of his city here.}
% }}}

% {{{ Socrates against his first accusers (19a8--24c2)
\section{Socrates against his first accusers (19a8--24c2)}

% {{{ The first charges (19a8--20c3)
\subsection{The first charges (19a8--20c3)}

Socrates begins by imagining what the first charges against him would have been.

\begin{quote}
    Σωκράτης ἀδικεῖ καὶ περιεργάζεται ζητῶν τά τε ὑπὸ γῆς καὶ οὐράνια καὶ τὸν ἥττω λόγον κρείττω ποιῶν καὶ ἄλλους ταὐτὰ ταῦτα διδάσκων (19b4--c2).

    Socrates' crime: He is meddlesome, investigating things below the earth and the heavens and making the weaker argument stronger and teaching others these same things.
\end{quote}

As evidence that something like this is the earlier charge against him, Socrates refers to ``the comedy of Arisophanes" (19c3).  In response to this multi-part charge, Socrates says that although he respects people who have knowledge of these physical matters and who are able to teach others, he has no such knowledge or ability.  He says that those who have heard him talking will defend him on this score.
% }}}

% {{{ The source of anti-Socratic feeling (20c4--24b2)
\subsection{The source of anti-Socratic feeling (20c4--24b2)}

Socrates acknowledges that this only pushes the problem back: if he \emph{doesn't} do any of these things, then \emph{why} does he have this reputation?  He explains this by telling the story of his divine mission.

% {{{ Socrates' divine mission (20e6--23c1)
\subsubsection{Socrates' divine mission (20e6--23c1)}

Once upon a time, Chaerephon asked the Delphic oracle if anyone was wiser than Socrates and the oracle said that no one was wiser.  When Socrates heard this, he was at a loss (ἠποροῦν, 20b7).  He knew that the god would not lie, but he didn't think that he was wise at all.  In order to understand better, Socrates set out to investigate and test the god's words.  He went to someone believed to be wise, and his intention was to show the god that this person was wiser than him.  However, Socrates discovered that although that person \emph{thought} he was wise, he actually was not.  Socrates showed this person that he wasn't actually wise, and unsurprisingly this made the person angry at Socrates.  A number of bystanders also grew angry at Socrates.  Socrates realized, however, that although neither of them knew much of anything, Socrates at least knew this while the other person thought he was wise. In this respect at least, Socrates had to admit that he was wiser than the person he spoke with.

Socrates repeated this process systematically, first with politicians, then poets and then craftspeople.  In each case, he had essentially the same result.  Socrates was wiser in virtue of knowing that he wasn't wise.  And in each case, people grew angry at Socrates.  Socrates attributes the ill will against him to these investigations and others like them.  Socrates also draws a lesson about what the god originally meant by his riddle: Socrates is wisest insofar as he knows that he doesn't know much of anything.  Socrates has continued to question people in service of the god.

It's worth saying that Socrates flirts with hubris at the start of this story.  Although he acknowledges that the god wouldn't lie, he nevertheless sets out to show the god someone wiser than himself.  Initially at least, he seems to assume that he will find such a person.

Also note that Socrates experiences a kind of elenchus here at the hands of the god that is not unlike what he dishes out to other people later. That is, he is confused and frustrated; he is at a loss; he begins thinking that he knows something,\footnote{Ironically enough, what he initially thinks he knows is that he doesn't know anything.} but he eventually realizes that he doesn't quite know what he thinks he does.  Unlike most Socratic interlocutors, however, Socrates really learns from his experience.  He doesn't dig in and become stuborn or refuse to face what he sees.
% }}} Socrates' divine mission (20e6--23c1)

% {{{ Socrates inspires bad habits in young Athenians (23c2-24b2)
\subsection{Socrates inspires bad habits in young Athenians (23c2-24b2)}

In addition to the people Socrates offends, he also has to worry about his young imitators offending people.  Many wealthy young Athenians follow Socrates around, and they the imitate his procedure themselves. This means that they infuriate other people, and those people blame Socrates for having given the youth these habits.  Because these angry people have nothing real to say against Socrates, they lob generic ``evil philosopher" complaints against him: investigating heavens and earth, religious novelty, making weaker argument stronger.\footnote{Don't the angry people have the genuine complaint against Socrates that he corrupted the youth in the way he describes? Why does he not consider that a real complaint?}
% }}} Socrates inspires bad habits in young Athenians (23c2-24b2)

% }}} The source of anti-Socratic feeling (20c4--24b2)

% }}} Socrates against his first accusers (19a8--24c2)

% {{{ Socrates against his current accusers (24c3--28b2)
\section{Socrates against his current accusers (24c3--28b2)}

% {{{ Does Socrates harm the youth? (24c3--26b2)
\subsection{Does Socrates harm the youth? (24c3--26b2)}

Socrates says that Meletus ~--- not Socrates --- harms the youth by claiming to care about them when he does not. Socrates engages in a brief elenchus with Meletus.  Since Meletus has brought Socrates to trial for corrupting the youth, Socrates infers that Meletus must care a great deal about who harms and who benefits young people. But Socrates is able to show rather quickly that Meletus has not put very much thought into these questions.  Meletus ends up saying that the laws of Athens and \emph{everyone} except Socrates benefits young people and that Socrates alone harms them.  Socrates then employs a version of his characteristic τεχνή argument: In the case of horses and other animals, many people may harm them and only a small group of experts benefit them. Therefore, by analogy, isn't it odd to think that \emph{everyone} in Athens except Socrates benefits the youth?  Although this argument is rapid and sketchy, the passage as a whole does imply that Meletus hasn't given the matter much thought at all.

Next Socrates forces a dilemma on Meletus. Either he corrupted the youth involuntarily and so deserves advice not punishment, or he deliberately acted in a way likely to harm himself which nobody does.

\begin{enumerate}
    \item It is better to live among good citizens than bad ones because good fellow citizens benefit you while bad fellow citizens harm you (25c5--10).
    \item No one wishes to be harmed by their fellow citizens rather than being benefited by them (25d1--5). (At the end, this becomes simply ``Is there anyone who wishes to be harmed?" Strictly speaking, Meletus denies only the more limited version, though Socrates may be relying on the stronger version in what follows.)
    \item It is not believable that Socrates wouldn't know that if he makes his fellow citizens base, they are likely to do him some evil (25d9--25e5).
    \item Therefore, Socrates concludes that either (1) he didn't corrupt the youths or (2) he did so unwillingly (25e6--26a1).
\end{enumerate}

In one sense, this argument seems characteristically Socratic.  Socrates frequently understands \phrase{harming someone} as making them morally worse.  So it makes sense for him to interpret, corrupting (διαφθείρειν) the youth as a matter of making them worse people.  Nevertheless, that makes the argument awful, since we have no particular reason to believe that corrupted youth harm \emph{all} their fellow citizens indiscriminately.  It's easy to imagine, for example, that the young men that Socrates corrupts wouldn't harm \emph{him} because they hold him in such high esteem.

An additional problem with this argument is that it seems to prove too much.  If we accept it, then isn't the same line of defense open to everyone?  If so, then apparently nobody commits this particular crime in a way that merits punishment.  If anything, then, the argument is not so much a specific defense of Socrates as a reason to change the law.  Socrates, however, does not choose to explicitly present the argument in that light.
% }}} Does Socrates harm the youth? (24c3--26b2)

% {{{ Does Socrates commit religious crimes? (26b2--28a1)
\subsection{Does Socrates commit religious crimes? (26b2--28a1)}

Next Socrates considers the accusation that he commits religious crimes.  He begins by noting that according to Meletus irreligion is the specific means by which Socrates corrupts the youth.  So the two charges are very closely connected (26b2--6).

The argument quickly becomes odd. Socrates hones in only on part of the irreligion charge: that he doesn't believe in the gods of Athens.  Socrates, however, feigns (?) confusion and asks Meletus whether the charge is atheism (full stop) or merely not believing in the gods of Athens.\footnote{I say \word{feigns} because the charge itself seems entirely clear everywhere else.  Meletus formally charged Socrates with not believing in the city's gods.  I'm unsure where Socrates came up with the idea of asking about atheism, and I don't see why Meletus takes the bait.  Either Plato is very unfair to Meletus at this point, or something about the charge has been lost in transmission.}  Meletus goes for atheism, and at that point Socrates has him.

Socrates quickly shows that Meletus contradicted himself.  On the one hand, Meletus has just said that Socrates is an atheist.  On the other hand, in his original indictment, Meletus also accused Socrates of believing in ``other new gods" (24c1).  Socrates argues as follows:

\begin{enumerate}
    \item Nobody believes in things-related-to-X and yet doesn't believe in X.  As examples, Socrates asks whether anyone believes in ``human things" but not humans or ``equine things" but not horses.  Socrates also gets Meletus to concede that no one believes in δαιμόνια but not δαίμονας (27b13--27c3).
    \item Meletus has already stated that Socrates introduces new divinities (27c5--8).
    \item Therefore, by Meletus' own argument Socrates must believe in some divine beings, namely δαίμονες (27c8--10).
    \item Therefore, since δαίμονες are gods, then Socrates cannot be a complete atheist, and so what Meletus just said contradicts his earlier charges (27c10--d10).
\end{enumerate}

According to \citeauthor{brickhouse2004} Socrates is sincere here, and his argument is reasonably strong.  They believe that ``the charge means only what Meletus says that it means" (112), and so Socrates is within his rights to ask whether Meletus thinks that he is an atheist.  In addition, they argue that it wasn't really open to Meletus to say ``Yes, you believe in gods, but you don't believe in the official gods of Athens" (116).  Meletus cannot do this, they claim, since it would undermine the connection between these official charges and the earlier slanders against Socrates.  But Meletus and the other prosecutors are relying on such a connection.  Hence, they wouldn't want to admit that Socrates believes in any gods.
% }}} Does Socrates commit religious crimes? (26b2--28a1)

% }}} Socrates against his current accusers (24c3--28b2)

% {{{ Socrates defends the life of philosophy (28b3--30d5)
\section{Socrates defends the life of philosophy (28b3--30d5)}

Socrates has wrapped up his formal defense (see 28a2--28b2), and he shifts gears to a more general defense of his odd lifestyle and choices.

First he imagines someone who asks, ``Aren't you ashamed to live a life that might get you killed?" (28b3--5).  In response Socrates argues that it shows poor judgement to worry about life or death. Instead, he claims that the only thing a person should consider when acting is whether or not he does just things (28b6--c1).  Socrates also makes an analogy to the soldiers in Homer's \book{Iliad} who acted without consideration for death (28c1--d4).

This leads Socrates to compare his philosophical mission to his actions as a soldier.  When he was ordered to stand his ground in battle, he did so.  In the same way, the god has now ordered Socrates to philosophize, to investigate himself and others.  It would be impious for him to betray these orders (28d5--29a5).\footnote{Socrates even jokes that if he did so, then it would be appropriate to bring him to trial for impiety (29a2--5).}

Socrates argues that disobeying the orders of the god because he feared death would amount to thinking he was wise when he was not (29a4--5).  Specifically, fearing death is thinking one is wise when one is not (29a5--29b2).  Nobody, says Socrates, knows whether death is a benefit or a harm, but people fear it as if they knew.

Following this line of thought, Socrates makes one of his strongest claims to knowledge anywhere in the early dialogues:

\begin{quote}
    τὸ δὲ ἀδικεῖν καὶ ἀπειθεῖν τῷ βελτίονι καὶ θεῷ καὶ ἀνθρώπῳ, ὅτι κακὸν καὶ αἰσχρόν ἐστιν οἶδα (29b6--7).

    But to commit an injustice and to disobey one's better, whether divine or human, I know that is evil and shameful.
\end{quote}

We might argue that Socrates claims knowledge here only because these ethical claims are almost true by definition.  Nevertheless, it is worth noting the claim.

At this point, Socrates imagines the jury releasing him on the condition that he stop doing philosophy.  He rejects this imagined proposal politely but very firmly.  First, he says that however much he cherishes the men of his city, he will obey the god rather than them (29d2--4).  He also suggests that he is doing the city a favor insofar as he urges all the people he meets to be better people.  In particular, Socrates tries to convince everyone he meets to care more about ἀρετή and their ψυχαί rather than money or physical pleasures.

As a final point, Socrates argues that they will hurt themselves more than him if they put him to death (30b6--d5).  He argues famously that a good person can't be harmed by a worse one:

\begin{quote}
    εὖ γὰρ ἴστε, ἐάν με ἀποκτείνητε τοιοῦτον ὄντα οἷον ἐγὼ λέγω, οὐκ ἐμὲ μείζω βλάψετε ἢ ὑμᾶς αὐτούς· ἐμὲ μὲν γὰρ οὐδὲν ἂν βλάψειεν οὔτε Μέλητος οὔτε Ἄνυτος~---οὐδὲ γὰρ ἂν δύναιτο~---οὐ γὰρ οἴομαι θεμιτὸν εἶναι ἀμείνονι ἀνδρὶ ὑπὸ χείρονος βλάπτεσθαι (30c6--10).

    Know well, if you kill me, being such as I say, you will not harm me more than yourselves. For neither Meletus nor Anytus could harm me at all~--- in fact, it would not be possible~--- for I don't think that it is proper\footnote{The word here is θεμιτός.  It is difficult to translate: it indicates religious correctness primarily, but can also imply possibility, in the sense of what is right or proper in the universe as decided or overseen by the gods.  I should check the big dictionary as well as translations.} for a better man to be harmed by a worse one.
\end{quote}
% }}} Socrates defends the life of philosophy (28b3--30d5)

% }}} Defense speech

% {{{ Penalty speech (35e--38b)
% }}} Penalty speech

% {{{ Post-trial speech (38c--42a)
% }}} Post-trial speech

\newpage
\bibliographystyle{apa}
\bibliography{plato}

\end{document}
% }}}
