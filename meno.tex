% [[- LaTeX prelude
\documentclass[12pt,letterpaper]{article}

\usepackage[no-math]{fontspec}
\setmainfont{Minion Pro}

\usepackage[nolocalmarks]{polyglossia}
\setdefaultlanguage{english}
\setotherlanguage[variant=ancient]{latin}
\setotherlanguage[variant=ancient]{greek}
\newfontfamily\greekfont[Script=Greek,Scale=MatchLowercase]{GFS Neohellenic}

\usepackage{titlesec}
\titleformat*{\section}{\Large\bfseries}
\titleformat*{\subsection}{\large\bfseries}
\titleformat*{\subsubsection}{\large\bfseries}

\usepackage{parskip}
\usepackage{csquotes}
\usepackage[notes,backend=biber]{biblatex-chicago}
\addbibresource{plato.bib}

\usepackage{enumitem}
\setlist{noitemsep}
\usepackage[super]{nth}


\begin{hyphenrules}{latin}
    \hyphenation{}
\end{hyphenrules}

\begin{hyphenrules}{greek}
    \hyphenation{ἐκ-μανθάν-ειν}
\end{hyphenrules}

\usepackage{fancyhdr}
\fancypagestyle{notes}{%
    \fancyhf{}
    \renewcommand{\headrulewidth}{0pt}
    \lhead{}
    \chead{Notes on Plato's \textit{Meno}}
    \rhead{}
    \lfoot{}
    \cfoot{\thepage}
    \rfoot{}
}
\fancypagestyle{references}{%
    \fancyhf{}
    \renewcommand{\headrulewidth}{0pt}
    \lhead{}
    \chead{References}
    \rhead{}
    \lfoot{}
    \cfoot{\thepage}
    \rfoot{}
}
% -]] Latex prelude

% [[- LaTeX document
\begin{document}

% [[- Title page
% \begin{titlepage}
% \title{Notes on Plato's \textit{Meno}}
% \author{Peter Aronoff}
% \date{August 2015}
% \maketitle
% \thispagestyle{empty}
% \end{titlepage}
% -]]

% [[- Characters and setting
\pagestyle{notes}
\section*{Characters and setting}

% [[- Characters
\subsection*{Characters}

In addition to Socrates, the dialogue has three other speaking characters: Meno, an unnamed slave of Meno's, and Anytus. Meno is a wealthy young Thessalian. \textcite{sedleylong2010} say that he died an ugly death in war at a relatively young age (xii). We learn nearly nothing about the slave: he speaks Greek and was born in Meno's house. Anytus is one of the future prosecutors of Socrates. So his appearance and clear hatred for Socrates is ominous and appropriate.

% -]] Characters

% [[- Setting
\subsection*{Setting}

\textcite[76]{morrison1942} suggests a date around 402, and both \textcite[120]{bluck1961} and \textcite[xii]{sedleylong2010} follow him. The dialogue does not make the place of the conversation entirely clear, but \textcite[120]{bluck1961} makes a good argument that it is outdoors, most likely a gymnasium.

% -]] Setting

% -]] Characters and setting

% [[- How is ἀρετή acquired? (70a--71d)
\section*{How is \textgreek{ἀρετή} acquired? (70a--71d)}

Meno begins abruptly, asking Socrates,

\begin{quote}
    \begin{greek}Ἔχεις μοι εἰπεῖν, ὦ Σώκρατες, ἆρα διδακτὸν ἡ ἀρετή; ἢ οὐ διδακτὸν ἀλλ᾽ἀσκητόν; ἢ οὔτε ἀσκητὸν οὔτε μαθητόν, ἀλλά φύσει παραγίγνεται τοῖς ἀνθρώποις ἢ ἄλλῳ τινὶ τρόπῳ;\end{greek}
    
   Can you tell me, Socrates, whether excellence can be taught? Or if it cannot be taught, can it instead be acquired through training? Or if it is neither acquired through training nor teaching, does it come to people naturally or in some other way? (70a1--4).
\end{quote}

As \textcite[4]{nehamas1985} notes, the question was familiar from Sophistical debates. And the answer mattered a great deal. Most Greeks of the age would desire \textgreek{ἀρετή}, and there were many people claiming to provide it.

Socrates replies that he cannot answer Meno's question. He offers heavy (and presumably ironic) praise of the Thessalians and the teaching that they've picked up from Gorgias, but he explains that things are very different in Athens, where there is a ``kind of drought of wisdom'' (70c4). If Meno asks an Athenian his question, he will be met with laughter and a ignorance. The Athenians cannot say how \textgreek{ἀρετή} is taught, and they do not even know at all (\textgreek{τὸ παράπαν}) what \textgreek{ἀρετή} is (71a). Socrates too cannot answer these questions: he claims to be as poor off as the rest of the Athenians (71b1--3). 

Socrates then states the priority of definition rule: \textgreek{ὃ δὲ μὴ οἶδα τί ἐστιν, πῶς ἂν ὁποῖον γέ τι εἰδείην;} (71b3--4). It is not possible to know what something is like (\textgreek{ποῖος}) if you do not know what it is (\textgreek{τί}). To help explain this rule, he gives an example, asking Meno: Does it seem possible that someone who has no idea who Meno is know whether Meno is handsome or wealthy or noble (71b4--8)?

Although he agrees with the priority of definition rule, and the example Socrates gives, it doesn't appear that Meno's agreement runs very deep. That is, Socrates briefly states the priority of definition and gives the example of knowing about Meno if you don't at all know Meno, and then he asks, ``Does [this] seem possible to you?'' Meno says no so briefly (\textgreek{Οὐκ ἔμοιγε.}) that I suspect we're meant to wonder if he has any idea what he just agreed with. This is borne out later in the dialogue when Meno reveals that he really doesn't understand the kind of definition Socrates seeks nor the criteria that a proper definition must meet.

In addition, Meno quickly turns away from questions about definition because something else that Socrates said has grabbed all his attention. Meno is shocked that Socrates disavows knowledge of \textgreek{ἀρετή}. I think his shock is twofold. First, by the standards of the time, it is shocking for Socrates to say that he does not know what \textgreek{ἀρετή} is. Second, Meno is surprised by the strength of Socrates' disavowal: Can Socrates really not know \textit{at all} what \textgreek{ἀρετή} is? Meno cannot imagine that there is any real question about what \textgreek{ἀρετή} is, only how best to acquire it.

Socrates assures Meno that he doesn't know and that, moreover, he's never met anyone who does. Meno says ``Not even Gorgias?'' In reply to this, Socrates says that he he is forgetful (\textgreek{Οὐ πάνυ εἰμὶ μνήμων} 71c8)---surely a joke given the later importance of recollection—but that perhaps Gorgias did know. Since Meno is very familiar with Gorgias' opinions perhaps he can speak for Gorgias and tell Socrates what \textgreek{ἀρετή} is? Socrates will be very fortunate if he turns out to be wrong and Gorgias and others do know the definition of \textgreek{ἀρετή}. In fact, since Meno must agree with Gorgias, they can leave Gorgias to one side and Meno himself can tell Socrates the definition. Meno agrees and this leads to the first elenchus.
% -]] How is ἀρετή acquired? (70a--71d)

% [[- Elenchus: What is ἀρετή (71e--80d4)
\section*{Elenchus with Meno (71e--80d4)}

% [[- First definition (71e1--72a5)
\subsection*{Meno's first definition (71e1--72a5)}

Meno replies confidently, but his answer shows that he doesn't fully understand what Socrates wants from an answer. (This isn't to his discredit since Socrates hasn't done anything to make this clear yet. As often in Socratic dialogues, the initial fumble seems partly aimed at Plato's readers. We too need to learn what sorts of questions and answers matter.) Meno offers a laundry list of commonplace examples of \textgreek{ἀρετή}, and he breaks these down by gender, age and status. A male citizen should be capable of handling the matters of his city, he should help his friends and harm his enemies. A woman should manage her household well. And so on, elliptically, for young and old, slave and free.
% -]] First definition (71e1--72a5)

% [[- Socrates clarifies: one over many (72a6--73c8)
\subsection*{Socrates clarifies his ``What is F?'' question (72a6--73c8)}

Socrates explain by analogy that he wants the ``one over the many''. Since Meno has given him a ``swarm'' (\textgreek{σμῆνος}) of virtues, Socrates brings up bees. He says that there are many different types of bees, but nevertheless, they share some one thing in virtue of which they are all bees. In respect to this thing, all bees are the same, no matter if they differ in size or any other quality. Meno agrees, and Socrates explains that \textit{this} is what he wants as an answer to his question about \textgreek{ἀρετή}.\footnote{Note that Socrates uses the terms \textgreek{οὐσία} (72b2) and \textgreek{εἶδος} (72c7) when explaining what he wants.}

Meno follows but not fully, and so Socrates gives further examples. Socrates asks whether health, size, and strength differs for a man and a woman or if health itself is the same thing whether it is a man or woman who is healthy. Meno agrees in these cases that there is a one over the many, but he still hesitates in the case of \textgreek{ἀρετή}.

Socrates tries to show Meno that all people possess \textgreek{ἀρετή} in the same way:

\begin{enumerate}
    \item A man should govern his city well and a woman should govern her house well.
    \item To govern well, whether a city or a house, a person must govern prudently and with justice.
    \item Both man and woman therefore require the same things—prudence and justice—in order to govern well.
    \item Likewise young or old people become good only if they are prudent and just.
    \item Therefore, since all people are good in the same way—namely through possession of prudence and justice---then the \textgreek{ἀρετή} they possess must be the same thing.
\end{enumerate}
% -]] Socrates clarifies: one over many (72a6--73c8)

% [[- Second definition (73c9--d10)
\subsection*{Second definition: \textgreek{ἀρετή} is ability to rule (73c9--d10)}

Meno is not thrilled with this, but he concedes and offers a new definition in light of Socrates' argument. He now defines \textgreek{ἀρετή} as the ability to rule people. (This was probably suggested to him by Socrates though I don't know that this was what Socrates intended by his previous argument.)

Socrates appears to accept the form of the answer this time, but he shoots down its content. He reminds Meno that his definition will not apply to children or slaves. Thus, the definition is too narrow: it cannot cover all cases of \textgreek{ἀρετή}. Socrates also asks whether ``ruling'' is a sufficient definition even for the cases it covers. He suggests that they need to add ``justly'' to the definition. Meno agrees, but immediately fumbles by saying that ``justice is \textgreek{ἀρετή}'' (73d9--10).
% -]] Second definition (73c9--d10)

% [[- Clarification: a virtue or virtue (full stop)? (73e--74b1)
\subsection*{Clarification: a virtue or virtue (full stop)? (73e--74b1)}

Socrates asks whether justice is \textgreek{ἀρετή} or \textgreek{ἀρετή τις}, and this leads to further confusion.\footnote{Compare \textit{Hippias Major} 287d9.} Meno initially does not understand what Socrates means by the distinction. After Socrates explains, using the example of round and shape, Meno gets it. In answer to Socrates, Meno is then able to list several other virtues--—thus showing that he knows that justice is not (all of) \textgreek{ἀρετή}. However, Socrates complains now they they have once again reached an answer of several \textgreek{ἀρεταί} when he only wanted one. There is also a verbal echo of Meno's earlier position: compare \textgreek{ἄλλαι πάμπολλαι} (74a5-6) with \textgreek{ἄλλαι πάμπολλαι ἀρεταί εἰσιν} (72a1).\footnote{I don't think that the complaint is fair here. Meno was simply replying to Socrates' request for him to list other \textgreek{ἀρεταί} just as he could list other shapes than round. Meno was \textit{not} offering this list as an answer to the original question ``What is \textgreek{ἀρετή}?''} There is also an initial hint of \textgreek{οὐ μᾶλλον} here (74e1 and 74e4--5), but I'm not sure what—if anything—to make of it. In any case, Meno concedes that he is unable to answer in the way Socrates wants. He is frustrated and confused, but not yet in a state of complete \textgreek{ἀπορία}.
% -]] Clarification: a virtue or virtue (full stop)? (73e--74b1)

% [[- Socrates gives examples to help Meno (74b2--77b1)
\subsection*{Socrates helps Meno with examples (74b2--77b1)}

Socrates returns to shapes---and colors though he focuses primarily on shapes. He tells Meno the kind of account he wants at reasonable length, using shape as an example. He then invites Meno to give a ``one over the many'' account of shape in order to practice for \textgreek{ἀρετή}. Meno asks Socrates to do it himself, and Socrates agrees. He defines shape as ``that which, alone of all things, always accompanies color'', and Socrates adds that he would be happy with a definition from Meno along those lines (75b8-c1).

Meno, however, finds this definition foolish. His reason is somewhat odd. He says that the definition would not help someone who didn't understand what color is. This is true as far as it goes, but it's not clear why it's a problem. Nevertheless, Socrates is willing to give Meno's complaint some time. He distinguishes between eristic (and agonistic) discussion and discussion between friends. In the first case, it would be the interlocutor's business to disprove the definition. But in the second case---a friendly dialogue---then Socrates agrees that they should reply in a gentler and more dialectic manner. Socrates explands the idea of ``more dialectic'' with this rule:

\begin{quote}
    A more dialectic answer is not only true but also proceeds through terms that respondents agree they know.
\end{quote}

Therefore Socrates is willing to refine his definition of shape. He says that shape is the ``limit of a solid'' (76a7). Meno then immediately asks Socrates to define color. Socrates gives him some grief, saying that Meno is demanding multiple answers from an old man while refusing to give one answer in return. Nevertheless, Socrates complies. He offers an elaborate (mocking?) definition of color as ``the effluence of shape that is fit for and perceptible to sight'' (76d4--5).
% -]] Socrates gives examples to help Meno (74b2--77b1)

% [[- Third definition (77b2--79e4)
\subsection*{Third definition: \textgreek{ἀρετή} is acquiring goods (77b2--79e4)}

Meno's next definition takes a bit of time to gel. Initially he says that \textgreek{ἀρετή} is, ``just as the poet says'',\footnote{The poet in question is unknown. \textcite[257]{bluck1961} says that Thompson suggested Simonides, though this appears to be just a guess.} to enjoy good things and be able (sc. to get them). He immediately restates this as ``desiring good things and being able to acquire them'' (77b4--5). It's not clear to me why he puts this definition these two ways, but it won't matter very much since Socrates will fixate on the idea that some (i.e. only some) people desire good things.

Socrates objects to the idea that only some people desire good things. He first clarifies that this is what Meno means. Socrates presses Meno on several points:

\begin{itemize}
    \item Does everyone desire (only?) good things, or do some people also desire bad things.
    \item If some people desire bad things, do they also know that those things are bad when they desire them?
    \item Does Meno agree that ``to desire'' is the same as ``to want something to happen to (or belong to) oneself''?
    \item Do people who desire bad things know that bad things harm the person who has them or do they think that bad things benefit their possessor?
    \item Does Meno think that people who believe that bad things are beneficial know that they are thinking of bad things?
\end{itemize}

I'm not sure whether Socrates is trying to clarify the argument or shame Meno into submission. For the most part, Meno sticks to his guns. He backs off only on the claim that people believe bad things are beneficial in full awareness that those things are bad. He won't go that far. However, he does still believe that some people knowingly desire bad things, even though they believe such things are harmful.

Socrates turns Meno around by arguing as follows:

\begin{enumerate}
    \item People who desire bad things, knowing that they are bad and knowing that bad things harm the person they befall, know that they will be harmed by these bad things.
    \item They also know that someone harmed is wretched to the degree that they are harmed and that the wretched are ill-fortuned.
    \item But nobody wants to be wretched and ill-fortuned.
    \item Therefore nobody wants bad things.
\end{enumerate}

Meno agrees that this seems right, and he backs away from his claim that some people want bad things knowingly. Once he does so, Socrates argues that the only part of Meno's latest definition that matters is that \textgreek{ἀρετή} is the power of getting good things. Meno agrees, and now Socrates turns to properly consider this definition.

Socrates defeats Meno's latest suggestion with two arguments, though he runs them somewhat together. That is, I'm not sure Socrates sees these as distinct arguments. I think he does, but I'm not sure.

\begin{enumerate}
    \item Acquiring goods can be done justly or injustly.
    \item If it is done justly, it is \textgreek{ἀρετή}. If it is done injustly, it is \textgreek{κακία}.
    \item In fact, \textit{not acquiring goods} can be \textgreek{ἀρετή} under the right circumstances, namely when it would be injust to acquire those goods.
    \item Therefore, acquiring goods is, by itself, no more \textgreek{ἀρετή} than not acquiring goods. Instead, whatever is done justly is \textgreek{ἀρετή}, and whatever is done injustly is \textgreek{κακία}. Whether or not one acquires goods appears to drop out as unimportant. (78c4--79a2)
\end{enumerate}

Meno agrees to this first argument. (Or should we call it the first stage of the overall argument?) Socrates continues as follows.

\begin{enumerate}
    \item Justice, prudence and all such things are parts of \textgreek{ἀρετή}.
    \item \textgreek{ἀρετή} is acquiring good things with justice, etc.
    \item So \textgreek{ἀρετή} is whatever is done with a part of \textgreek{ἀρετή}.
    \item This account relies on understanding what justice itself is, but that is impossible unless you already know what \textgreek{ἀρετή} is. For, Socrates argues, you cannot know a part of \textgreek{ἀρετή} unless you already know what \textgreek{ἀρετή} itself is. (Socrates refers here to the principle that definitions should not rely on terms that the interlocutor does not know.)
    \item Socrates presses the point, saying that if you define \textgreek{ἀρετή}, or anything else, using its parts, then the same initial question will emerge for the parts. That is, if you say \textgreek{ἀρετή} is action done with justice, we will immediately need to ask, ``What is justice?'' Hence, the problem of definition is not resolved, only pushed back. (79a3--e3)
\end{enumerate}

\textcite[59]{fine2014} points out that Socrates may not be entitled to the claim that we cannot know a part unless we know its whole. She gives the example of porridge, oats, and water. We may know what oats and water are without knowing what porridge is or that it \textit{is} oats and water. However, she also notes that Plato might have a further argument: namely that oats and water may be independent of porridge in a way that justice is not independent of \textgreek{ἀρετή}. In either case, this argument is tricky and Socrates blows through it way too fast.  
% -]] Third definition (77b2--79e4)

% [[- Meno's ἀρπορία (79e7--80d4)
\section*{Meno's \textgreek{ἀπορία} (79e7--80d4)}

At this point, Meno reaches \textgreek{ἀπορία}. He exhibits many of the signs of \textgreek{ἀπορία} as seen in other dialogues. Meno says that Socrates is famous for being confused\footnote{He uses the verb \textgreek{ἀπορέω}.} and for bewitching and confusing\footnote{He uses the noun \textgreek{ἀπορία}.} other people. He likens Socrates to a torpedo fish that numbs anyone who approaches. In addition, Meno admits that he is now completely unable to say what \textgreek{ἀρετή} is,\footnote{Meno's phrasing echoes what Socrates says earlier. At the start of the dialogue, Socrates says, \textgreek{οὐδὲ αὐτὸ ὅτι ποτ᾽ἐστὶ τὸ παράπαν ἀρετὴ τυγχάνω εἰδώς} (71a6--7). Meno says here \textgreek{οὐδ᾽ὅτι ἐστὶν τὸ παράπαν ἔχω εἰπεῖν} (80b4).} although he has spoken many fine speeches about this subject in the past.\footnote{Meno also says that Socrates is smart not to travel more. If he behaved like this in a foreign land, he would anger people as if he were a mage. This is probably intended as a light joke, not a threat. But when we read Anytus' threat later in the dialogue and remember how Socrates dies, the joke may appear more ominous.}

Socrates accepts the analogy of himself to a torpedo fish, but only under a specific interpretation. Socrates insists that he is only like such a fish if we imagine a numbing fish that is itself numb. That is, Socrates reiterates here, as he often does, that he is not doing this to others because or when he knows the answers. He confuses others perhaps, but he is himself confused. In addition, Socrates insists that he is more confused than anyone else (80c9).\footnote{\textcite[71--72]{fine2014} has good comments about this. She notes that although it may seem paradoxical, Socrates is \textit{more} confused because he has investigated \textit{more}. She likens this to a physicist or mathematician. Such a person knows much more than a layperson about their subjects, but they may also be far more confused than the layperson. They know enough to know how much they do not know. They can see areas of ignorance and confusion that a layperson doesn't even know exists.}

Socrates also describes their current situation in a way that invites Meno's famous paradox. He reiterates that he does not know what \textgreek{ἀρετή} is (80d1). He then says that perhaps Meno did know what it was before he came into contact with Socrates, but that now he seems not to know (80d1--3). \textit{Nevertheless} Socrates concludes that he still wants to investigate and inquire with Meno what \textgreek{ἀρετή} is (80d3--4). The use here of `nevertheless' shows that Socrates is aware that his continued eagerness is surprising or counterintuitive. Socrates himself is already aware that their combined lack of knowledge would keep many people from going forward. Thus, he tees up Meno's paradox.
% -]] Meno's ἀρπορία (79e7--80d4)

% -]] Elenchus: What is ἀρετή? (71e--80d4)

% [[- Meno's paradox (80d5--86c3)
\section*{Meno's paradox (80d5--86c3)}

% [[- Meno's version
\subsection*{Meno's version (80d5-8)}

In his frustration, Meno asks Socrates how he proposes to carry out \textit{any} investigation. His complain is reasonable given the strict and constricting demands Socrates places on definition and inquiry. Although Dominic Scott considers Meno lazy and coercive as an interlocutor, he still believes that in this case Meno helps us to see a problem with Socratic inquiry \parencite[69--74]{scott2006}. More precisely, Scott believes that the character Meno is being lazy and evasive, but that ``Whatever his motives, Meno has put his finger on something genuinely important'' \parencite[71--72]{scott2006}. Scott cites \textit{Euthydemus}, \textit{Phaedo} (89c11--90d7), and \textit{Republic} (537d7--539d6) as three other places where Plato seriously considers the differences between eristic and elenchus and addresses fears that the elenchus may have negative or harmful effects.

\begin{quote}
\begin{greek}Καὶ τίνα τρόπον ζητήσεις, ὦ Σώκρατες, τοῦτο ὃ μὴ οἶσθα τὸ παράπαν ὅτι ἐστίν; ποῖον γὰρ ὧν οὐκ οἶσθα προθέμενος ζητήσεις; ἢ εἰ καὶ ὅτι μάλιστα ἐντύχοις αὐτῷ, πῶς εἴσῃ ὅτι τοῦτό ἐστιν ὃ σὺ οὐκ ᾔδησθα; \textenglish{(80d5-8)}\end{greek}
\end{quote}

Meno poses two problems here. First, how can inquiry even begin if the inquirer doesn't at all know what they are looking for? Second, even if the inquirer gets lucky and stumbles onto what they are looking for, how can they know that this is what they didn't know earlier?
% -]] Meno's version

% [[- Socrates's restatement (80e1--5)
\subsection*{Socrates's restatement (80e1--5)}

Socrates restates Meno's paradox, though with significant differences. Instead of worrying about (i) how to begin inquiry and (ii) how to end inquiry, Socrates worries about the possibility of inquiry (i) if someone has knowledge or (ii) if someone does not have knowledge. In theory, what Socrates says under (ii) might cover both (i) and (ii) in Meno's version. In that case, Socrates doesn't ignore anything that Meno says; he only adds something new. However, I'm not so sure that this is the best way to take Socrates's restatement.

\begin{quote}
    \begin{greek}Μανθάνω οἷον βούλει λέγειν, ὦ Μένων. ὁρᾷς τοῦτον ὡς ἐριστικὸν λόγον κατάγεις, ὡς οὐκ ἄρα ἔστιν ζητεῖν ἀνθρώπῳ οὔτε ὃ οἶδεν οὔτε ὃ μὴ οἶδε; οὔτε γὰρ ἂν ὅ γε οἶδεν ζητοῖ—οἶδεν γάρ, καὶ οὐδὲν δεῖ τῷ γε τοιούτῳ ζητήσεως—οὔτε ὃ μὴ οἶδεν—οὐδὲ γὰρ οἶδεν ὅτι ζητήσει. \textenglish{(80e1--5)}\end{greek}
\end{quote}
% -]] Socrates's restatement (80e1--5)

% -]] Meno's paradox (80d5--86c3)

% [[- Bibliography
\newpage
\pagestyle{references}
\nocite{burnet1903}
\defbibfilter{sources}{%
    ( keyword=edition or keyword=translation or keyword=commentary )
    and
    keyword=meno
}
\defbibfilter{secondary}{%
    keyword=secondary and keyword=meno
}
\printbibliography[filter=sources,title={Ancient Sources: Editions, Translations, Commentaries}]
\printbibliography[filter=secondary,title=Secondary Literature]

% -]] Bibliography

\end{document}
% -]]
