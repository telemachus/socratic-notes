% {{{ LaTeX prelude
\documentclass[11pt]{article}
\usepackage{fontspec}
\setmainfont[Ligatures={Common,TeX}]{Palatino}
\usepackage{url}
\usepackage{parskip}
\usepackage{natbib}
\bibpunct{(}{)}{;}{a}{,}{,}
\usepackage{titles}
% }}}

% {{{ LaTeX document
\begin{document}

% {{{ Title page
\begin{titlepage}
\title{Notes on Plato's \book{Lysis}}
\author{Peter Aronoff}
\date{June 2013}
\maketitle
\end{titlepage}
% }}}

% {{{ Characters and setting
\section{Characters and setting}

% {{{ Characters
\subsection{Characters}

Socrates speaks with a number of young men in this dialogue.\footnote{The information below is taken from \citet{race1983} and \citet{watt1987}.}

\begin{enumerate}
    \item Hippothales is a teenager and the lover of Lysis.  At the start of the dialogue, he spots Socrates and calls him over.  So, they clearly have some relationship.  He does not appear elsewhere in Plato, but Diogenes Laertius says that he was a follower of Socrates (III.46).
    \item Ctesippus is around Hippothales' age and a friend of his.  He mocks Hippothales for being so madly in love with Lysis.  He is Menexenus cousin, and both of them were present at the death of Socrates in \book{Phaedo} (59b).
    \item Menexenus is the same age as Lysis, and both of them are younger than Hippothales or Ctessipus.  He is Ctessipus cousin and a follower of Socrates.  He is probably the same Menexenus as appears in the dialogue of that name.
    \item Lysis is a close friend of Menexenus and the beloved of Ctesippus.  We don't know of him other than what is said in this dialogue.
\end{enumerate}
% }}} Characters

% {{{ Setting
\subsection{Setting}

The conversation takes place near and inside a newly opened gymnasium run by Miccus.  Hippothales says that Miccus is an admirer and companion of Socrates, but we don't know anything about him other than what's said in this dialogue.  The gymnasium is somewhere between the Academy and the Lyceum since Socrates says that he was coming from the former and heading for the latter.  It's near a ``little gate by the spring of Panops" (203a2--3).  \citet{race1983} places this ``near the Lyceum" and ``just outside the eastern city wall."
% }}} Setting
% }}} Characters and setting

% {{{ Introduction (203a--207d4)
\section{Introduction (203a--207d4)}

Socrates offers advice on love to one Hippothales who is desperately in lust with Lysis.  After Ctesippus, Hippothales and friends catch sight of Socrates, Ctesippus reveals to Socrates that Hippothales is crazy about Lysis.  What happens next is a little surprising: Socrates begins to offer advice.  He tells Hippothales that he's a fool if he praises someone before the person is caught.  (Praise just makes them arrogant, don't you know.)  Hippothales asks Socrates for help, and Socrates agrees to give it.  They come up with a sitcom-worthy plan for Socrates to talk to Lysis within earshot of Hippothales.

After starting a conversation with Lysis and Menexenus, Socrates turns the conversation towards friendship (φιλία).  This will be the real topic of the dialogue.  As \citet{watt1987} points out, there is an implicit contrast in the start of the dialogue between ἔρως and φιλία, two types of ``love" in the opening pages (119--120).  Hippothales lusts after Lysis, but Lysis and Menexenus have deep friendship for one another.
% }}} Introduction (203a--207d4)

\newpage
\bibliographystyle{apa}
\bibliography{plato}

\end{document}
% }}}
