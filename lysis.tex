% {{{ LaTeX prelude
\documentclass[11pt]{article}
\usepackage{fontspec}
\setmainfont[Ligatures={Common,TeX}]{Palatino}
\usepackage{url}
\usepackage{parskip}
\usepackage{natbib}
\bibpunct{(}{)}{;}{a}{,}{,}
\usepackage{titles}
% }}}

% {{{ LaTeX document
\begin{document}

% {{{ Title page
\begin{titlepage}
\title{Notes on Plato's \book{Lysis}}
\author{Peter Aronoff}
\date{June 2013}
\maketitle
\end{titlepage}
% }}}

% {{{ Characters and setting
\section{Characters and setting}

% {{{ Characters
\subsection{Characters}

Socrates speaks with a number of young men in this dialogue.\footnote{The information below is taken from \citet{race1983} and \citet{watt1987}.}

\begin{enumerate}
    \item Hippothales is a teenager and the lover of Lysis.  At the start of the dialogue, he spots Socrates and calls him over.  So, they clearly have some relationship.  He does not appear elsewhere in Plato, but Diogenes Laertius says that he was a follower of Socrates (III.46).
    \item Ctesippus is around Hippothales' age and a friend of his.  He mocks Hippothales for being so madly in love with Lysis.  He is Menexenus cousin, and both of them were present at the death of Socrates in \book{Phaedo} (59b).
    \item Menexenus is the same age as Lysis, and both of them are younger than Hippothales or Ctessipus.  He is Ctessipus cousin and a follower of Socrates.  He is probably the same Menexenus as appears in the dialogue of that name.
    \item Lysis is a close friend of Menexenus and the beloved of Ctesippus.  We don't know of him other than what is said in this dialogue.
\end{enumerate}
% }}} Characters

% {{{ Setting
\subsection{Setting}

The conversation takes place near and inside a newly opened gymnasium run by Miccus.  Hippothales says that Miccus is an admirer and companion of Socrates, but we don't know anything about him other than what's said in this dialogue.  The gymnasium is somewhere between the Academy and the Lyceum since Socrates says that he was coming from the former and heading for the latter.  It's near a ``little gate by the spring of Panops" (203a2--3).  \citet{race1983} places this ``near the Lyceum" and ``just outside the eastern city wall."
% }}} Setting
% }}} Characters and setting

% {{{ Introduction (203a--207d4)
\section{Introduction (203a--207d4)}

Socrates offers advice on love to one Hippothales who is desperately in lust with Lysis.  After Ctesippus, Hippothales and friends catch sight of Socrates, Ctesippus reveals to Socrates that Hippothales is crazy about Lysis.  What happens next is a little surprising: Socrates begins to offer advice.  He tells Hippothales that he's a fool if he praises someone before the person is caught.  (Praise just makes them arrogant, don't you know.)  Hippothales asks Socrates for help, and Socrates agrees to give it.  They come up with a sitcom-worthy plan for Socrates to talk to Lysis within earshot of Hippothales.

After starting a conversation with Lysis and Menexenus, Socrates turns the conversation towards friendship (φιλία).  This will be the real topic of the dialogue.  As \citet{watt1987} points out, there is an implicit contrast at the start of the dialogue between ἔρως and φιλία, two types of ``love" (119--120).  Hippothales lusts after Lysis, but Lysis and Menexenus have deep friendship for one another.
% }}} Introduction (203a--207d4)

% {{{ The importance of wisdom (207d5--210d8)
\section{The importance of wisdom (207d5--210d8)}

Socrates leads Lysis through a complex question and answer session about wisdom and friendship.  The upshot of the discussion seems to be that in areas where you are wise, people will trust you and you will be free to do as you please.  Indeed they will even turn their affairs over to you.  But where you are not wise, you are a slave to others.  I wouldn't really call this section an argument since so much of it relies on unlikely assertions that receive no support.  It's also not an elenchus since Lysis doesn't assert anything initially.

It's difficult to know how seriously to take the passage.  On the one hand, many of the ideas seem Socratic.  On the other hand, at the end of the discussion Socrates indicates that the whole conversation was an exercise in humbling and demeaning an ἐρόμενος in an effort to woo him (210e2--5).\footnote{Yes, that's right.  Socrates sounds like a modern pick up artist game-player here.}  If we take that seriously, then I'm not sure any of the preceding conversation should be taken very seriously.

Nevertheless here's how the conversation goes:

\begin{enumerate}
    \item Socrates gets Lysis to agree that his parents love him very much and that therefore they want him to be as εὐδαίμων as possible (207d5--8).
    \item Next Socrates suggests that a person is unhappy when a slave an unable to do what he or she wishes (207e1--3).
    \item Therefore, Socrates continues, if his parents love Lysis and want him to be as happy as possible, they must allow him to do as he wishes, not rebuke him and not prevent him from doing what he wants (207e3--7).
    \item But, says Lysis, they \emph{do} prevent him from very many things.  Socrates is shocked, just shocked (207e8-208a1).
    \item Socrates next asks about a number of specific things that Lysis might want to do.  In each case, Lysis is not allowed to do what he wants, but a paid servant or slave is allowed to do it.  For example, Lysis cannot take out one of his father's chariots whenever he wants and drive it as he likes.  In fact, Lysis cannot even fully control himself: He has a slave who takes him to school and the like.  Socrates dwells for a bit on the idea that a free person might in these ways be bettered by a slave even.  Lysis' father has assigned many people to rule over the boy, and his mother too does not let him do what he wants when it comes to her domain (280a1--209a4).
        Lysis says that they don't let him do these things because he is too young, but Socrates isn't having it.  In fact, his parents let Lysis do certain things regardless of his age.  They let him read and write for example.  And they let him handle instruments however he likes (209a4--209c1).
        These examples finally lead Lysis to draw a more Socratic conclusion: they don't let him act as he likes where he lacks knowledge; where he has knowledge, he can do as he wants (209c2).
        Socrates expands this beyond the family of Lysis.  A neighbor, the people of Athens, even the Great King would entrust things to Lysis, if they believed that he was knolwedgeable about those matters (209c6--210a8).\footnote{At this point I'm really unsure what's happening.  Up until now, things were borderline plausible.  But I don't see why we (or Lysis) should swallow such huge assertions without argument.  Also, Socrates seems to wink in this passage: the Great King will entrust his meat-boiling to Lysis.}
    \item Socrates sums up as follows.  In domains where we are knowledgeable, people will turn matters over to us.  In those cases, we will be free to do as we want, and we will own those things, insofar as we will profit from them.  But where we are without knowledge, we will not be in control, we will be at the command of other people, things will not belong to us insofar as we will not profit from them (210a9--c5).
    \item From all of this, Socrates draws some shocking conclusions.  First we are not friends with someone nor is anyone friends with us unless there is benefit.  Second, and drawing on the first, parents do not love their children if the children are of no benefit to them.  Therefore, if you want friends and you want your parents to love you, you must become wise (210c5--210d4).
    \item As a final weird point, Socrates gets Lysis to agree that you can't be arrogant about areas where you're not wise (210c4--210c8).
\end{enumerate}
% }}} The importance of wisdom (207d5--210d8)

\newpage
\bibliographystyle{apa}
\bibliography{plato}

\end{document}
% }}}
