% {{{ LaTeX prelude
\documentclass[11pt]{article}
\usepackage{fontspec}
\setmainfont[Ligatures={Common,TeX}]{Palatino}
\usepackage{url}
\usepackage{parskip}
\usepackage{natbib}
\bibpunct{(}{)}{;}{a}{,}{,}
\usepackage{titles}
% }}}

% {{{ LaTeX document
\begin{document}

% {{{ Title page
\begin{titlepage}
\title{Notes on Plato's \book{Lysis}}
\author{Peter Aronoff}
\date{June 2013}
\maketitle
\end{titlepage}
% }}}

% {{{ Characters and setting
\section{Characters and setting}

% {{{ Characters
\subsection{Characters}

Socrates speaks with a number of young men in this dialogue.\footnote{The information below is taken from \citet{race1983} and \citet{watt1987}.}

\begin{enumerate}
    \item Hippothales is a teenager and the lover of Lysis.  At the start of the dialogue, he spots Socrates and calls him over.  So, they clearly have some relationship.  He does not appear elsewhere in Plato, but Diogenes Laertius says that he was a follower of Socrates (III.46).
    \item Ctesippus is around Hippothales' age and a friend of his.  He mocks Hippothales for being so madly in love with Lysis.  He is Menexenus cousin, and both of them were present at the death of Socrates in \book{Phaedo} (59b).
    \item Menexenus is the same age as Lysis, and both of them are younger than Hippothales or Ctessipus.  He is Ctessipus cousin and a follower of Socrates.  He is probably the same Menexenus as appears in the dialogue of that name.
    \item Lysis is a close friend of Menexenus and the beloved of Ctesippus.  We don't know of him other than what is said in this dialogue.
\end{enumerate}
% }}} Characters

% {{{ Setting
\subsection{Setting}

The conversation takes place near and inside a newly opened gymnasium run by Miccus.  Hippothales says that Miccus is an admirer and companion of Socrates, but we don't know anything about him other than what's said in this dialogue.  The gymnasium is somewhere between the Academy and the Lyceum since Socrates says that he was coming from the former and heading for the latter.  It's near a ``little gate by the spring of Panops" (203a2--3).  \citet{race1983} places this ``near the Lyceum" and ``just outside the eastern city wall."
% }}} Setting
% }}} Characters and setting

% {{{ Introduction (203a--207d4)
\section{Introduction (203a--207d4)}

Socrates offers advice on love to one Hippothales who is desperately in lust with Lysis.  After Ctesippus, Hippothales and friends catch sight of Socrates, Ctesippus reveals to Socrates that Hippothales is crazy about Lysis.  What happens next is a little surprising: Socrates begins to offer advice.  He tells Hippothales that he's a fool if he praises someone before the person is caught.  (Praise just makes them arrogant, don't you know.)  Hippothales asks Socrates for help, and Socrates agrees to give it.  They come up with a sitcom-worthy plan for Socrates to talk to Lysis within earshot of Hippothales.

After starting a conversation with Lysis and Menexenus, Socrates turns the conversation towards friendship (φιλία).  This will be the real topic of the dialogue.  As \citet{watt1987} points out, there is an implicit contrast at the start of the dialogue between ἔρως and φιλία, two types of ``love" (119--120).  Hippothales lusts after Lysis, but Lysis and Menexenus have deep friendship for one another.
% }}} Introduction (203a--207d4)

% {{{ The importance of wisdom (207d5--210d8)
\section{The importance of wisdom (207d5--210d8)}

Socrates leads Lysis through a complex question and answer session about wisdom and friendship.  The upshot of the discussion seems to be that in areas where you are wise, people will trust you and you will be free to do as you please.  Indeed they will even turn their affairs over to you.  But where you are not wise, you are a slave to others.  I wouldn't really call this section an argument since so much of it relies on unlikely assertions that receive no support.  It's also not an elenchus since Lysis doesn't assert anything initially.

It's difficult to know how seriously to take the passage.  On the one hand, many of the ideas seem Socratic.  On the other hand, at the end of the discussion Socrates indicates that the whole conversation was an exercise in humbling and demeaning an ἐρόμενος in an effort to woo him (210e2--5).\footnote{Yes, that's right.  Socrates sounds like a modern pick up artist game-player here.}  If we take that seriously, then I'm not sure any of the preceding conversation should be taken very seriously.

Nevertheless here's how the conversation goes:

\begin{enumerate}
    \item Socrates gets Lysis to agree that his parents love him very much and that therefore they want him to be as εὐδαίμων as possible (207d5--8).
    \item Next Socrates suggests that a person is unhappy when a slave an unable to do what he or she wishes (207e1--3).
    \item Therefore, Socrates continues, if his parents love Lysis and want him to be as happy as possible, they must allow him to do as he wishes, not rebuke him and not prevent him from doing what he wants (207e3--7).
    \item But, says Lysis, they \emph{do} prevent him from very many things.  Socrates is shocked, just shocked (207e8-208a1).
    \item Socrates next asks about a number of specific things that Lysis might want to do.  In each case, Lysis is not allowed to do what he wants, but a paid servant or slave is allowed to do it.  For example, Lysis cannot take out one of his father's chariots whenever he wants and drive it as he likes.  In fact, Lysis cannot even fully control himself: He has a slave who takes him to school and the like.  Socrates dwells for a bit on the idea that a free person might in these ways be bettered by a slave even.  Lysis' father has assigned many people to rule over the boy, and his mother too does not let him do what he wants when it comes to her domain (280a1--209a4).
        Lysis says that they don't let him do these things because he is too young, but Socrates isn't having it.  In fact, his parents let Lysis do certain things regardless of his age.  They let him read and write for example.  And they let him handle instruments however he likes (209a4--209c1).
        These examples finally lead Lysis to draw a more Socratic conclusion: they don't let him act as he likes where he lacks knowledge; where he has knowledge, he can do as he wants (209c2).
        Socrates expands this beyond the family of Lysis.  A neighbor, the people of Athens, even the Great King would entrust things to Lysis, if they believed that he was knolwedgeable about those matters (209c6--210a8).\footnote{At this point I'm really unsure what's happening.  Up until now, things were borderline plausible.  But I don't see why we (or Lysis) should swallow such huge assertions without argument.  Also, Socrates seems to wink in this passage: the Great King will entrust his meat-boiling to Lysis.}
    \item Socrates sums up as follows.  In domains where we are knowledgeable, people will turn matters over to us.  In those cases, we will be free to do as we want, and we will own those things, insofar as we will profit from them.  But where we are without knowledge, we will not be in control, we will be at the command of other people, things will not belong to us insofar as we will not profit from them (210a9--c5).
    \item From all of this, Socrates draws some shocking conclusions.  First we are not friends with someone nor is anyone friends with us unless there is benefit.  Second, and drawing on the first, parents do not love their children if the children are of no benefit to them.  Therefore, if you want friends and you want your parents to love you, you must become wise (210c5--210d4).
    \item As a final weird point, Socrates gets Lysis to agree that you can't be arrogant about areas where you're not wise (210c4--210c8).
\end{enumerate}
% }}} The importance of wisdom (207d5--210d8)

% {{{ Transition to Menexenus (210e1--211d5)
\section{Transition to Menexenus (210e1--211d5)}

As this first conversation between Socrates and Lysis winds down, the focus turns to Menexenus.  He returns from helping with the sacrifice, and Lysis is very eager for Socrates to speak with him as he spoke with Lysis himself.  It seems that Lysis wants to humble Menexenus as well.\footnote{See esepcially 211c3 where Lysis says that he wants Socrates to speak with Menexenus "in order to punish him."}

Ctesippus also gets involved, and there's a funny sort of multi-way flirting going on.  Ctesippus doesn't want Socrates and Lysis conversing all to themselves, Hippothales is listening to all of this carefully, and Lysis wants Socrates to beat up on Menexenus for him.  There's also a lot of the kind of playful ``Do this", ``Well if you ask, then I must" sort of thing that you see in flirting.
% }}} Transition to Menexenus (210e1--211d5)

% {{{ Socrates questions Menexenus about friends (211d6--213d5)
\section{Socrates questions Menexenus about friends (211d6--213d5)}

This section is a more traditional Socratic elenchus.  Socrates presents Menexenus as a kind of expert, though Menexenus does not make such a claim for himself.  Nevertheless, Socrates says that he has always prized friendship very highly, and that he marvels at Lysis and Menexenus. For they are so young, but have already acquired each other --- such an excellent friend each has.  Initially Socrates says that what is doesn't know is ``How does someone become a friend of someone's" (212a5--6), but this isn't actually the question he pursues with Menexenus.

The question that Socrates actually asks Menexenus is odder: When one person is friendly to another, who becomes whose friend? (212a8--b1).  As Socrates goes on to explain, he wonders when one person befriends another which of the following is the case (212a8--b2):

\begin{enumerate}
    \item The (active) befriender [becomes friend of] the (passive) befriended.
    \item The (passive) befriended [becomes friend of] the (active) befriender.
    \item No difference.
\end{enumerate}

Menexenus chooses the ``no difference" option.  It's not immediately clear what ``no difference" means, but Socrates reply to Menexenus revels that ``no difference" amounts to ``both become friends of the other."  Socrates immediately sets out to cross-question Menexenus on his choice.  His argument is as follows.

\begin{enumerate}
    \item From Menexenus: If one person befriends another, then both become friends of each other (212b2--3).
    \item Sometimes someone befriends another person but is not befriended in turn by that person.  Menexenus agrees (212b5--6).
    \item Sometimes someone befriends another person, but that person hates the one befriending him.  Again, Menexenus agrees (212b5--c3).
    \item Therefore in cases like this, one person befriends and the other is befriended. [But it doesn't seem right to say that both become friends of the other in cases like this.] (212c3--4)
\end{enumerate}

Thus, Socrates asks again who is whose friend. And he offers three options, though now the options are slightly different:

\begin{enumerate}
    \item The befriending person is a friend of the befriended one, whether he is befriended in return or even hated (212c5--6).
    \item The befriended person is a friend of the befriending one (212c6--7).
    \item Neither is a friend of the other unless both befriend each other (212c7--8).
\end{enumerate}

Menexenus goes for the new option: neither is a friend of the other unless both are friends.  Socrates rephrases this as ``Nothing is a friend to something befriending it unless it befriends in return" (212d4--5).  He then proceeds to disprove this claim as follows:

\begin{enumerate}
    \item Assuming the given definition, nobody could be a lover of horses, a lover of quail, a lover of dogs, a lover of wine, of wrestling or of wisdom.
    \item For all such things cannot love in return.
\end{enumerate}

Menexenus rejects the conclusion, so they turn to the option that it is the beloved that is a friend to the one befriending, whether the beloved loves in return or not --- even if the beloved hates the one befriending.  Socrates offers an example of infant children.  They cannot love in return and even sometimes seem to hate, but nevertheless they are most beloved to their parents.

Although the first example seems fine, things get uglier when Socrates generalizes.  In the generalized version, Socrates says that the befriended is a friend, no matter the feelings of that person. And the hated is an enemy, again regardless of the feelings of that person.  But this means that many are befriended by their enemies and hated by their friends; and it means that many are friends to their enemies and enemies to their friends.  The absurdity of these results leads Socrates and Menexenus to reject this claim as well.

At this point, Socrates regroups.  He reminds Menexenus that they have rejected the following options\footnote{Did they actually disprove all these options?  I'm not sure that they did.  Certainly they didn't handle them in this order.}:

\begin{enumerate}
    \item The befriender is a friend [to whomever he befriends].
    \item The befriended is a friend [to whomever befriends him].
    \item Both befriender and befriended are friends [to each other].
\end{enumerate}

Socrates asks if Menexenus can think of any other people who can become friends, since none of these worked out.  Menexenus, unsurprisingly, has no ideas.  Socrates wonders whether they have not been going about the whole thing wrong, and at this point, Lysis gets involved again.  Lysis says that he does think they are going about things all wrong, and Socrates turns back to him.
% }}} Socrates questions Menexenus about friends (211d6--213d5)

% {{{ Like is friend to like (213d6--215c2)
\section{Like is friend to like (213d6--215c2)}

Having switched back to Lysis as interlocutor, Socrates proposes that they follow the wisdom of the poets who teach that ``like befriends like".  In particular, Socrates quotes a line of verse (poet unnamed) that says ``A god always leads like to like" (214a6).  (Socrates also notes that the natural philosophers approve this idea also.)

Socrates immediately picks at the thesis, saying it's at least half false.  The false half concerns bad people.  In the case of bad people, Socrates argues, they are not each other's friends.  He argues as follows\footnote{After this argument, Socrates adds a characteristic flourish: Bad people aren't really like anyone, not even themselves (214c7-d1).  Therefore, when the poets say like is to like, they are riddling and really only mean \word{good} like people.}:

\begin{enumerate}
    \item A bad person does wrong [to other people] (214c2).
    \item People doing wrong and the people they do wrong to cannot be friends (214c2--3).
\end{enumerate}

Following this argument to dismiss bad people, Socrates says that he still has some trouble with the argument.  He argues thus:

\begin{enumerate}
    \item If two people are like each other, then each cannot harm or benefit the other in any new way.  The one could already do it alone (214e3--215a1).
    \item But if this is true, then there's no cause for affection because affection follows benefit (215a1--2).
    \item Where there is no affection, there is no friendship (215a3).
    \item Therefore, like is not friend to like.
\end{enumerate}

Socrates then considers whether good people are friends not insofar as they are similar to each other but insofar as they are good.  However, he rejects this possibility with an argument much like the one above.  Good people are ``capable"\footnote{His word is ἱκανός.}  And insofar as someone is capable, that person does not need anything.  No need, no affection.  The rest as above, more or less (215a6--b8).  The one new idea is that a friendship between good people could never get started: When they are apart, they don't desire one another, and when they are together, they have no need of one another.  The importance of where the chain of desire \emph{begins} will return later in the dialogue.
% }}} Like is friend to like (213d6--215c2)

% {{{ Unlike is friend to unlike (215c3--216b9)
\section{Unlike is friend to unlike (215c3--216b9)}

Socrates regroups again, and again he turns to a poet.  He remembers that he once heard someone say that similar people and good people were natural enemies to each other.  That is, like quarrels with like, and a good person is an enemy to another good person.\footnote{I feel reasonably sure that the second part of this marks the following argument as potentially unSocratic.  But see below.}  This person used the Hesiodic wisdom that ``potter hates potter" (\book{Works and days 25}).  The idea is that similarity brings out rivalry and hatred while different people fit each other's needs.  So a weak person needs a strong one for protection, and a sick person needs a doctor.  Socrates also cites some common-sense notions that dry needs wet, cold needs hot and so on.  Menexenus agrees that this sounds good, and they agree to investigate the thesis that unlike is friend to unlike.

But Socrates immediately worries that eristic debaters will attack and make fools of them.  The argument he presents in their voice quickly squashes this new answer to the question.  These eristics, he imagines, will say that enmity is most unlike friendship.  They will then ask ``Is enemity friends with friendship, or friendship friendly with enmity?"  Similarly is justice friends with injustice or wisdom with foolishness or good with evil?  These are all absurd conclusions, and so Menexenus and Socrates abandon the claim that unlike is friend to unlike.
% }}} Unlike is friend to unlike (215c3--216b9)

% {{{ What is neither good nor bad (216c1-218c3)
\section{What is neither good nor bad (216c1-218c3)}

Socrates tries to break through this problem by proposing that what becomes a friend of the good is something that is ``neither good nor bad" (216c1--2).  Menexenus doesn't follow, and Socrates admits that he is himself somewhat loopy from all the different possibilities they've considered.  Nevertheless, I think he's serious here.

Socrates argues thus initially:

\begin{enumerate}
    \item There are three categories: good, bad and neither (216d5--7).
    \item As already shown, good is not friend to good, nor bad to bad nor good to bad (216d7--e1).
    \item The remaining options are (1) neither good nor bad is friend to the good or (2) to something else neither good nor bad.  (It is summarily ruled out that it might be friends to what is bad.) (216e1--4).
    \item But a previous argument just showed that like is not friend to like.  Therefore, what is neither good nor bad is friend to what is good (216e5--217a2).
\end{enumerate}

In order to see whether this argument holds up, Socrates considers the case of health and the body.  In and of itself, Socrates says, a body is neither good nor bad.  A healthy body does not need medical care, and no one who is healthy befriends a doctor because of his health.  However a sick person does befriend a doctor because of his illness.  So, Socrates generalizes, that which is neither good nor bad befriends what is good as a result of the presence of something bad.

Socrates makes the further point that the thing which is neither good nor bad can become bad --- or bad enough? --- that it no longer desires the good.  In order to make more sense of this, Socrates considers hair dye.\footnote{Yes, you read that right.}  Some things can be present in something without making that something truly as they are.  For example, your hair can have dye in it without truly becoming the color of the dye.  So in the case of something neither good nor bad.  It can have some bad in it without truly becoming bad.  But if it does become truly bad, then it will no longer desire or befriend the good.

Socrates describes philosophy using this same analysis.  The wise don't do philosophy, presumably because they don't need it.  Nor do people who are so ignorant that they have become bad.  The people who do philosophy are bad insofar as they are ignorant, but they still believe that they do not know the things that they do not know.  Parts of this sound very familiar.  In particular the people who believe that they don't know what they don't know sound very much like Socrates as he describes himself in \book{Apology}.  But I'm not sure how to understand the idea that ``good people don't philosophize" (218a2-4 and 218b3) as a serious Socratic proposal.
% }}} What is neither good nor bad (216c1-218c3)

% {{{ An infinite regress? (218c4--220e6)
\section{An infinite regress? (218c4--220e6)}

Socrates rejoices briefly at the success of the previous argument, but he immediately has a concern.  He worries that the argument has unfortunate implications.  As he explains:

\begin{enumerate}
    \item Whatever is a friend, is a friend (1) to something, (2) for the sake of something and (3) because of something.  The difference between (2) and (3) is initially confusing, but Socrates explains with the example of illness.  A person is a friend to a doctor (2) for the sake of health, but (3) because of illness.  The illness causes him to want health, and thus indirectly to want the doctor's help (218d6--219a4).
    \item Using the terminology from the previous discussion, Socrates uses this structure to describe friendship: ``that which is neither good nor bad is (1) a friend of the good (3) because of the bad (2) for the sake of the good and friendly" (219a3--4).  He then reformulates this further as ``The friend is a friend of the friend for the sake of the friend because of the enemy" (219b2--3).
    \item The wording of the last version suggests a number of problems.  First, is runs afoul of the earlier argument that like is not friend to like, but Socrates dismisses this fear for the moment.  Socrates worries more about the implications of the ``for the sake of" clause.  If the thing we are pursuing is also a friend, then \emph{it} must be sought in turn for the sake of something and an infinite regress threatens.  So Socrates wonders if there is something that can serve as a beginning of the chain without itself referring to anything else.  Socrates also says that all of the other things are ``friends" only by a convenient figure of speech.  The only thing that is truly a friend is the one that grounds all the others (219d2--5).
\end{enumerate}

Socrates expands on the final point with two examples.  First example: A father loves his son.  As a result, he may love many other things which he believes can help his son.  But he does not really love those other things \foreign{per se}. He loves them only insofar as they contribute to his son's wellbeing.  Second example: people love money (Socrates actually says ``gold and silver").  But really what people love are the things that money can get.

By applying this argument to friendship, Socrates solves one problem.  Since the objects other than the primary one are not really loved (or ``beloved", that is φίλος), then we can escape the fear that ``friend is friend of friend on behalf of friend".  This means that we are safe, I suppose, from the earlier arguments that ``like is not friend to like".

As a next issue, Socrates takes back up the good.  If the good is a friend (φίλον), is that because of the bad?  As Socrates argues, the answer is yes.  Only because of something bad is the good useful or necessary.  The good is like a drug: Without illness it would be useless.

Therefore, the friend at the start of the chain is unlike all the other things we call friends.  The first friend is only a friend because of bad and harm, but all the others are friends because of some other friend and benefit.  I believe that saves us both from the infinite regress problem and the problem that Socrates quickly passed over above.\footnote{I suspect Plato set things up that way deliberately.  Initially Socrates says ``Let's not worry about this part." and then later Socrates solves that problem.  Thus, he was justified in ignoring the problem \foreign{ex post facto}.}
% }}} An infinite regress? (218c4--220e6)

% {{{ Desire is the cause of friendship (220e6--221d6)
\section{Desire is the cause of friendship (220e6--221d6)}

Socrates immediately takes aim at the claim that κακόν (bad, harm) is the cause of friendship.  He argues as follows:

\begin{enumerate}
    \item Sometimes desires stem from harm, but not always.  Therefore even if harms disappear, some desires will remain (220e6--221b6).
    \item Whoever desires and loves something is friendly towards that something (221b7--8).
    \item Therefore, even if all harm disappears, there can still be some friends (221b8--c1).
    \item If harm were the cause of friendship, this would be impossible since something cannot occur if its cause disappears (221c2--5).
    \item Therefore, they were wrong to believe that harm is the cause of friendship, since friendship can survive the (imagined) disappearance of harm (221c5--d2).
    \item Desire is the true cause of friendship (221d2--6).
\end{enumerate}

This argument has at least two significant problems.  First, in the first step, Socrates seems to waffle between a desire being for something harmful and the desire itself being harmful.  I'm not sure if this is a culpable ambiguity or not since I can't entirely follow the first step. (I've paraphrased it above to make it more comprehensible.)  Second, as \citet{watt1987} points out, the final step is a non-sequitur.  I can't see why Socrates thinks that he's entitled to it.  My best guess is something like this:

\begin{enumerate}
    \item There must be a univocal cause of friendship (axiomatic?).
    \item The argument above proves that there is sometimes friendship where there is not harm and that in those cases desire is the cause of the friendship.
    \item Therefore, desire must always be the cause of friendship and not harm.
\end{enumerate}

It might be worthwhile to go back and try to reanalyze the cases from the previous argument, replacing \word{harm} with \word{desire}.  (For example, how would the case of the sick person look?  The sick person is a friend of a doctor for the sake of health because of his desire to be well?  In that case, the \phrase{for the sake of} and \phrase{because of} clauses seem to collapse into one another.)
% }}} Desire is the cause of friendship (220e6--221d6)


\newpage
\bibliographystyle{apa}
\bibliography{plato}

\end{document}
% }}}
