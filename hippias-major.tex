% {{{ LaTeX prelude
\documentclass[11pt]{article}
\usepackage{fontspec}
\setmainfont[Ligatures={Common,TeX}]{Palatino}
\usepackage{url}
\usepackage{parskip}
\usepackage{natbib}
\bibpunct{(}{)}{;}{a}{,}{,}
\usepackage{titles}
% }}}

% {{{ LaTeX document
\begin{document}

% {{{ Title page
\begin{titlepage}
\title{Notes on Plato's \book{Hippias Major}}
\author{Peter Aronoff}
\date{July 2013}
\maketitle
\end{titlepage}
% }}}

% {{{ Characters and setting
\section{Characters and setting}

% }}} Characters and setting

% {{{ Characters
\subsection{Characters}

Hippias was a famous sophist who probably lived circa 470--395 BCE. \footnote{\citet{waterfield1987a} 213.}  His dates aren't certain, but he was younger than Socrates and Gorgias and he outlived Socrates.  He was very much a polymath: David Sider describes him as ``lecturer, historian, poet, teacher, ambassador, potter, engraver, metalworker, weaver, and cobbler".\footnote{\citet{sider1986} iv.}  In addition to the two eponymous dialogues, Hippias appears in \book{Protagoras}.  He appears at 315c, sitting among a group of people asking him questions about astronomy.\footnote{Later in the dialogue, Protagoras makes fun of Hippias, among other sophists, for dragging his students through too many subjects that they don't need or want (318d5--e4).}  He seems to be putting on an ``ask me whatever you want" type of showcase.  In addition to that sort of thing and regular teaching for money, he also gave epideictic displays.  For example, at the start of \book{Hippias Major}, he tells Socrates about a piece he wrote imagining Nestor giving moral advice to Neoptolemus (286a3--c2).

Hippias was from Elis in the northwestern Peloponnese.  The people of Elis apparently held him in high regard: They sent him on diplomatic missions all over the Greek world.  But I'm not aware of any more significant political service by Hippias.

% }}} Characters

% {{{ Setting
\subsection{Setting}

There's nothing to place the dialogue physically, but we can say something about the date.  Gorgias has already visited, so that puts the date after 427.  It is also after the death of Pheidias a sculptor (died circa 420).  It also appears to be peacetime.  David Sider guesses sometime during the Peace of Nicias (421-416), and Robin Waterfield guesses 420.\footnote{\citet{sider1986} iv; \citet{waterfield1987a} 213.}

% }}} Setting

% {{{ Introduction (281a--286c2)
\section{Introduction (281a--286c2)}


% }}} Introduction (281a--286c2)

% {{{ Definitions of the fine (286c3--303d10)
\section{Definitions of the fine (286c3--303d10)}


% }}} Definitions of the fine (286c3--303d10)

% {{{ Conclusion (303e11--304e9)
\section{Conclusion (303e11--304e9)}


% }}} Conclusion (303e11--304e9)


\newpage
\bibliographystyle{apa}
\bibliography{plato}

\end{document}
% }}}
