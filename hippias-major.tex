% {{{ LaTeX prelude
\documentclass[11pt]{article}
\usepackage{fontspec}
\setmainfont[Ligatures={Common,TeX}]{Palatino}
\usepackage{url}
\usepackage{parskip}
\usepackage{natbib}
\bibpunct{(}{)}{;}{a}{,}{,}
\usepackage{titles}
% }}}

% {{{ LaTeX document
\begin{document}

% {{{ Title page
\begin{titlepage}
\title{Notes on Plato's \book{Hippias Major}}
\author{Peter Aronoff}
\date{July 2013}
\maketitle
\end{titlepage}
% }}}

% {{{ Characters and setting
\section{Characters and setting}

% }}} Characters and setting

% {{{ Characters
\subsection{Characters}

Hippias was a famous sophist who probably lived circa 470--395 BCE. \footnote{\citet{waterfield1987a} 213.}  His dates aren't certain, but he was younger than Socrates and Gorgias and he outlived Socrates.  He was very much a polymath: David Sider describes him as ``lecturer, historian, poet, teacher, ambassador, potter, engraver, metalworker, weaver, and cobbler".\footnote{\citet{sider1986} iv.}  In addition to the two eponymous dialogues, Hippias appears in \book{Protagoras}.  He appears at 315c, sitting among a group of people asking him questions about astronomy.\footnote{Later in the dialogue, Protagoras makes fun of Hippias, among other sophists, for dragging his students through too many subjects that they don't need or want (318d5--e4).}  He seems to be putting on an ``ask me whatever you want" type of showcase.  In addition to that sort of thing and regular teaching for money, he also gave epideictic displays.  For example, at the start of \book{Hippias Major}, he tells Socrates about a piece he wrote imagining Nestor giving moral advice to Neoptolemus (286a3--c2).

Hippias was from Elis in the northwestern Peloponnese.  The people of Elis apparently held him in high regard: They sent him on diplomatic missions all over the Greek world.  But I'm not aware of any more significant political service by Hippias.

% }}} Characters

% {{{ Setting
\subsection{Setting}

There's nothing to place the dialogue physically, but we can say something about the date.  Gorgias has already visited, so that puts the date after 427.  It is also after the death of Pheidias a sculptor (died circa 420).  It also appears to be peacetime.  David Sider guesses sometime during the Peace of Nicias (421-416), and Robin Waterfield guesses 420.\footnote{\citet{sider1986} iv; \citet{waterfield1987a} 213.}

% }}} Setting

% {{{ Introduction (281a--286c2)
\section{Introduction (281a--286c2)}

The dialogue begins with Socrates saying that it's been a long time since he's seen Hippias and Hippias telling Socrates what he's been up to.  He has been very busy acting as an ambassador for Elis, his native city---in particular, he's been busy at Sparta.  Socrates takes this opportunity to flatter Hippias: he is so capable, both as a private teacher for pay and as a citizen working for his city-state in politics.  But, Socrates wonders, why did the previous generations of wise men hold off from politics?  Hippias says that they were simply less capable than the current generation, and Socrates (ironically, one assumes) agrees that like all other craftspeople, sophists become better each generation.

They then banter a bit about money. Hippias brags about how much he earns compared with other sophists, and Socrates again draws an ironic (?) contrast between the current generation of wise men and previous ones.  The previous wise men didn't care at all about money, while the current generation look out for themselves first and they make sure to get paid.

The introduction concludes with an ironic and paradoxical argument that the Spartans, although most lawful, are lawbreakers.  It all begins when Socrates learns from Hippias that Hippias didn't earn any money in Sparta by teaching young men.  However, Hippias concedes to Socrates the following:

\begin{enumerate}
    \item Hippias improves the people who associate with him through his wisdom.  He can improves the sons of the Spartans in this same way (283c1--d2).
    \item The Spartans can afford to pay Hippias (283d2--3).
    \item The Spartans cannot educate their own sons better than Hippias (283d4--e1).
    \item The Spartans don't begrudge their children a good education.  That is, the Spartan fathers don't keep their children from Hippias as a result of resentment (283e2--8).
    \item Sparta is a city of excellent laws and customs and Hippias knows how to pass along this kind of virtue to children very well.  Normally, if a city is renowned for X and a master of X comes to visit, the people of that city will line up to hire him for their sons.  Socrates gives an example of equestrian knowledge in Thessaly (283e9--284b5)
\end{enumerate}

If you put this all together, it seems incredible that the Spartans don't hire Hippias for their children, but Hippias explains that it is not traditional (πάτριον 284b6) to change their laws nor to educate their sons in uncustomary ways. And it is not customary (νόμιμον 284c5) for children to receive education from non-Spartans.

Socrates, however, draws the conclusion that the well-lawed (?) Spartans are lawbreakers.  His argument is that law aims at good and benefit.  But the rule not to let foreigners and thus Hippias educate their children does harm, not benefit.  So the Spartans act contrary to law as a result of their own customs (284d1--285b7).

After this, they briefly discuss Hippias' epideictic piece in which Nestor gives advice to Neoptolemus, and then they transition to the fine.

% }}} Introduction (281a--286c2)

% {{{ Definitions of the fine (286c3--303d10)
\section{Definitions of the fine (286c3--303d10)}


% }}} Definitions of the fine (286c3--303d10)

% {{{ Conclusion (303e11--304e9)
\section{Conclusion (303e11--304e9)}


% }}} Conclusion (303e11--304e9)


\newpage
\bibliographystyle{apa}
\bibliography{plato}

\end{document}
% }}}
