% {{{ LaTeX prelude
\documentclass[11pt]{article}
\usepackage{fontspec}
\setmainfont[Ligatures={Common,TeX}]{Palatino}
\usepackage{url}
\usepackage{parskip}
\usepackage{natbib}
\bibpunct{(}{)}{;}{a}{,}{,}
\usepackage{titles}
% }}}

% {{{ LaTeX document
\begin{document}

% {{{ Title page
\begin{titlepage}
\title{Notes on Plato's \book{Hippias Major}}
\author{Peter Aronoff}
\date{July 2013}
\maketitle
\end{titlepage}
% }}}

% {{{ Characters and setting
\section{Characters and setting}

% }}} Characters and setting

% {{{ Characters
\subsection{Characters}

Hippias was a famous sophist who probably lived circa 470--395 BCE. \footnote{\citet{waterfield1987a} 213.}  His dates aren't certain, but he was younger than Socrates and Gorgias and he outlived Socrates.  He was very much a polymath: David Sider describes him as ``lecturer, historian, poet, teacher, ambassador, potter, engraver, metalworker, weaver, and cobbler".\footnote{\citet{sider1986} iv.}  In addition to the two eponymous dialogues, Hippias appears in \book{Protagoras}.  He appears at 315c, sitting among a group of people asking him questions about astronomy.\footnote{Later in the dialogue, Protagoras makes fun of Hippias, among other sophists, for dragging his students through too many subjects that they don't need or want (318d5--e4).}  He seems to be putting on an ``ask me whatever you want" type of showcase.  In addition to that sort of thing and regular teaching for money, he also gave epideictic displays.  For example, at the start of \book{Hippias Major}, he tells Socrates about a piece he wrote imagining Nestor giving moral advice to Neoptolemus (286a3--c2).

Hippias was from Elis in the northwestern Peloponnese.  The people of Elis apparently held him in high regard: They sent him on diplomatic missions all over the Greek world.  But I'm not aware of any more significant political service by Hippias.

% }}} Characters

% {{{ Setting
\subsection{Setting}

There's nothing to place the dialogue physically, but we can say something about the date.  Gorgias has already visited, so that puts the date after 427.  It is also after the death of Pheidias a sculptor (died circa 420).  It also appears to be peacetime.  David Sider guesses sometime during the Peace of Nicias (421-416), and Robin Waterfield guesses 420.\footnote{\citet{sider1986} iv; \citet{waterfield1987a} 213.}

% }}} Setting

% {{{ Introduction (281a--286c2)
\section{Introduction (281a--286c2)}

The dialogue begins with Socrates saying that it's been a long time since he's seen Hippias and Hippias telling Socrates what he's been up to.  He has been very busy acting as an ambassador for Elis, his native city---in particular, he's been busy at Sparta.  Socrates takes this opportunity to flatter Hippias: he is so capable, both as a private teacher for pay and as a citizen working for his city-state in politics.  But, Socrates wonders, why did the previous generations of wise men hold off from politics?  Hippias says that they were simply less capable than the current generation, and Socrates (ironically, one assumes) agrees that like all other craftspeople, sophists become better each generation.

They then banter a bit about money. Hippias brags about how much he earns compared with other sophists, and Socrates again draws an ironic (?) contrast between the current generation of wise men and previous ones.  The previous wise men didn't care at all about money, while the current generation look out for themselves first and they make sure to get paid.

The introduction concludes with an ironic and paradoxical argument that the Spartans, although most lawful, are lawbreakers.  It all begins when Socrates learns from Hippias that Hippias didn't earn any money in Sparta by teaching young men.  However, Hippias concedes to Socrates the following:

\begin{enumerate}
    \item Hippias improves the people who associate with him through his wisdom.  He can improves the sons of the Spartans in this same way (283c1--d2).
    \item The Spartans can afford to pay Hippias (283d2--3).
    \item The Spartans cannot educate their own sons better than Hippias (283d4--e1).
    \item The Spartans don't begrudge their children a good education.  That is, the Spartan fathers don't keep their children from Hippias as a result of resentment (283e2--8).
    \item Sparta is a city of excellent laws and customs and Hippias knows how to pass along this kind of virtue to children very well.  Normally, if a city is renowned for X and a master of X comes to visit, the people of that city will line up to hire him for their sons.  Socrates gives an example of equestrian knowledge in Thessaly (283e9--284b5)
\end{enumerate}

If you put this all together, it seems incredible that the Spartans don't hire Hippias for their children, but Hippias explains that it is not traditional (πάτριον 284b6) to change their laws nor to educate their sons in uncustomary ways. And it is not customary (νόμιμον 284c5) for children to receive education from non-Spartans.

Socrates, however, draws the conclusion that the well-lawed (?) Spartans are lawbreakers.  His argument is that law aims at good and benefit.  But the rule not to let foreigners and thus Hippias educate their children does harm, not benefit.  So the Spartans act contrary to law as a result of their own customs (284d1--285b7).

After this, they briefly discuss Hippias' epideictic piece in which Nestor gives advice to Neoptolemus, and then they transition to the fine.

% }}} Introduction (281a--286c2)

% {{{ Pre-definition preliminaries (286c3--287e1)
\section{Pre-definition preliminaries (286c3--287e1)}

% }}} Pre-definition preliminaries (286c3--287e1)

% {{{ Socrates' alter-ego (286c3--287b3)
\subsection{Socrates' alter-ego (286c3--287b3)}

Socrates tells Hippias that the question he asks ('What is the fine?') is not his own but a question someone else asked him.  Socrates says that this other man recently cast him into helplessness (\word{ἀπορία} 286c5) during a discussion.  Socrates was casually calling things \word{fine} or \word{shameful}, and the man challenged him to say what the fine is.  Socrates failed the challenge, and he now asks Hippias to help him so that he can redeem himself.  In order to help things along, Socrates frequently pretends to be the other man as he and Hippias talk.  Over the dialogue, this someone else takes on a large role.  Hippias finds this questioner infuriating since (1) he is never satisfied and (2) he is frequently rude in tone and unsophisticated in topics.

It is obvious to readers that there never was any other man: the other man is Socrates all along.  The trick allows Socrates to question Hippias insistently without directly infuriating him.  In addition, it adds another layer of irony and humor to the dialogue---which is already thick with both.  In this way, the deception is both kind and cruel.  Socrates doesn't directly, explicitly insult Hippias, but indirectly and implicitly, he is far more aggressive than in most dialogues.

Socrates also asks Hippias if it will be ok for him to imitate the other man sometimes.  Hippias agrees.  You might think that this was in order for Plato to have Socrates speak without having to say constantly ``as the other man would say", but in fact, Socrates constantly draws attention to the mask.  So I'm not quite sure what the point of this request is.  It certainly doesn't end up making the dramatic situation any more straightforward.

% }}} Socrates' alter-ego (286c3--287b3)

% {{{ Socratic metaphysics (287b4--287e1)
\subsection{Socratic metaphysics (287b4--287e1)}

Socrates assumes the voice of the other figure and runs through some preliminary metaphysics with Hippias.  Hippias has no idea what Socrates is saying, but he agrees with it all regardless.

\begin{enumerate}
    \item Just people are just by justice\footnote{The word \word{by} here indicates the common Platonic use of an instrumental/causal dative to indicate that through which someone or something possesses a quality. \citet{woodruff1982} cites \book{Euthyphro} 6b11 and \book{Phaedo} 100d7 (45).  He also argues, correctly I think, that the dative is more a matter of logical cause than true instrument or means.} (287c1--2).
    \item Justice exists, or there is such a thing as justice (287c4).
    \item In the same way the wise are wise by of wisdom and all good things are good by goodness (287c5--6).  And wisdom and goodness exist too (287c6--8).
    \item All fine things are fine by of the fine, and the fine exists (287c8-d2).
\end{enumerate}

Hippias agrees all the way through, though perhaps with very little understanding.  This becomes clear when Socrates asks Hippias the dialogue's key question:

\begin{quote}
    Εἰπὲ δή, ὦ ξένε, φήσει, τί ἐστι τοῦτο τὸ κάλον; (287d3)

    Say indeed, stranger, he will say, what is this \phrase{the fine}?
\end{quote}

Hippias is so completely confused that he responds, ``Do you mean what is beautiful?" (287d4--5).  Socrates says ``No, \emph{the} fine" (287d6), but Hippias is unable to see any difference between the two questions.  This is very important: Hippias says ``What is the difference?" (287d7).  In response to that, Socrates says only ``It doesn't seem different at all to you?" (287d9).\footnote{I'm unsure about Socrates' tone here.  Disbelief?  Confusion?}

Socrates does not challenge Hippias on his confusion here.  He simply says, ``Well you must know better" and continues.  My guess is that Socrates (and/or Plato) prefer to work Hippias up to the larger issue, rather than quarrel before they even get started.  However, it's exceedingly important for the reader to see how very confused Hippias is.  See below when I discuss Alexander Nehamas' interpretation of the first definition.

\citet{woodruff1982} offers a very different interpretation of Hippias.  He argues that Hippias is \emph{not} stupid.  Instead, Hippias knowingly agrees with Socrates, even though he has no intention of giving Socrates the kind of answer that Socrates demands.  According to Woodruff, Hippias does this for two reasons.  First, he is remarkably agreeable.  It is part of his general strategy for pleasing everyone and being all things to all people that he never disagrees.  Even when he does disagree (at some level), he starts by agreeing and then adds qualifications in a non-confrontational manner.  Second, in Woodruff's eyes, Hippias is smart enough to know that the only way to win with Socrates is not to play his game at all.  Nobody ever holds up to Socratic questioning, if they play by Socrates' rules.  So Hippias doesn't even bother to try.  I'm not sure that I buy this interpretation, but it's certainly worth thinking about.

% }}} Socratic metaphysics (287b4--287e1)

% {{{ Definitions of the fine (287e2--303d10)
\section{Definitions of the fine (287e2--303d10)}
% }}} Definitions of the fine (287e2--303d10)

% {{{ First definition (287e2-289d5)
\subsection{First definition (287e2-289d5)}

Hippias offers his first definition with great confidence:

\begin{quote}
    Μανθάνω, ὠγαθέ, καὶ ἀποκρινοῦμαί γε αὐτῷ ὅτι ἐστι τὸ καλόν, καὶ οὐ μή ποτε ἐλεγχθῶ. ἔστι γάρ, ὦ Σώκρατες, εὖ ἴσθι, εἰ δεῖ τὸ ἀληθὲς λέγειν, παρθένος καλὴ καλόν (287e2--4).

    I understand, friend, and I will tell him what the fine is, and I will not ever be refuted.  For, Socrates, know well, if it is necessary to say the truth, a fine maiden is a fine thing.
\end{quote}

Socrates asked for a completely different type of answer.  Notwithstanding his strong assertions of understanding and invincibility, Hippias appears not even to understand the question.  Notice that in the first sentence he says that he will explain \phrase{the fine}, but that in his actual answer he defines---at most---\phrase{a fine thing}.  The most natural reading of this passage, I believe, is that it demonstrates Hippias' complete ignorance of Socratic definition.  Such ignorance should not surprise us.  In both \book{Euthyprho} and \book{Laches}, for example, the interlocutors' first answers are also completely inadequate.  One reasonable view is that the early Socratic dialogues often teach their readers how to answer Socratic questions, even if they don't provide concrete results.  We learn what sort of thing would qualify as a proper answer, even if we don't ever get truly satisfactory answers.

Before I look at how Socrates responds to Hippias, I want to consider two alternative interpretations of Hippias' definition.

\citet{nehamas1975a} argues that Hippias means \phrase{Being a beautiful maiden is (what it is to be) beautiful} or \phrase{To be a beautiful maiden is (to be) beautiful} (168).  At first, this sounds impossible as a rendering of the Greek.  But given Nehamas' other examples, he seems to mean the following:

\begin{enumerate}
    \item No particular girl is at issue.  Hippias is not offering a concrete particular where Socrates asks about a universal (see page 168 and 169).  Hippias is offering ``beautiful girl" as a class, not as a specific individual.
    \item In the sentence \foreign{παρθένος καλὴ καλόν}, ``beautiful" is not a bona fide general term, but that is has a peculiar, strong sense, close to what we would mean by the expression ``is to be beautiful".
    \item Hippias does not reduce questions like ``What is the beautiful?" to ``What is beautiful?"  Instead, he does just the opposite: He interprets any statement such as ``x is beautiful" as really meaning ``To be x is to be beautiful" (see page 168, 169, 170).
\end{enumerate}

I understand the first point and agree.  Hippias has no particular girl in mind; he uses ``a beautiful girl" to stand for a class, not a specific person. I do not really understand what Nehamas intends by his second or third point, nor how he thinks the Greek supports him.  \citet{woodruff1982} takes Nehamas as meaning something like ``One way to be beautiful is to be a beautiful girl" (50).  If so, that's an interesting idea, but I'm not sure how he gets it from the Greek.\footnote{\citet{woodruff1982} 50 says ``a possible reading of the Greek", but he doesn't elaborate.}

\citet{woodruff1982} argues that Hippias understands Socrates perfectly well, but that he knowingly disregards what Socrates wants from a definition.  As he puts it ``he means to trivialize the question.  He does so by mentioning something he believes to be irrefutably fine, and something, besides, that was probably supposed to provoke nudges and titters from his audience" (50).  In Woodruff's mind, Hippias is trying to defuse the situation using humor.  He knows that he cannot win at Socrates' game, so he attempts not to play by Socrates' rules.  Unfortunately, Socrates doesn't laugh and let Hippias off the hook.  If anything, he only gets more abusive, though we'll see that later.

I believe that Hippias does not understand Socrates' requirements, but that this may not make him a complete idiot.  As I said above, we also see interlocutors who have trouble understanding what Socrates wants from them in other early Socratic dialogues.  This doesn't make any of these people stupid, however, since Socrates has very stringent demands that were probably unfamiliar to his interlocutors.  Hippias may be unusual in how insistent he is that he understands and how many times he gets things wrong, but he's not a fool because he doesn't immediately understand the rules of Socratic definitions.

Initially it looks as though Socrates will attack Hippias' first definition for not having sufficient explanatory power.  He asks Hippias whether a fine girl is what explains all the other fine things (288a7--11).\footnote{The text is corrupt at 288a10, but the overall intent of the passage seems clear enough.}  Hippias, however, misunderstands.  In response, Hippias asks whether the alter-ego will dare to try to refute this definition.  He believes that nobody would try since they would look laughable.

Perhaps because Hippias is so confused (?), Socrates changes tack at this point.\footnote{The argument that a fine girl cannot explain why \emph{all} other things are fine is left unfinished.  Readers who are familiar with other Socratic dialogues can see where Socrates would have gone, but Socrates moves on without complaint.  As he says in \book{Euthyprho}, he has to follow where the respondent leads him.}  He asks whether a fine horse is a fine thing, and Hippias agrees (288b8--c5).  Then Socrates asks about a fine lyre. Again Hippias agrees (288c6--8).  Next Socrates asks about a fine cooking pot.  Hippias becomes very annoyed at this, and it takes a moment for Socrates to bring him around.  But finally Hippias concedes that a fine cooking pot is also a fine thing (288c9--e9).

Although he agrees that a well-crafted cooking pot can be a fine thing, Hippias still insists that a cooking pot cannot compare to a fine horse or girl.  Socrates seizes on this in order to demand another definition.  He argues, referring to Heraclitus, that a fine girl cannot compare with a fine goddess, and that a wise human is no better than a monkey compared with a god in wisdom and beauty (289a1--b7).  Hippias agrees, and Socrates presses this point.  Socrates now makes the familiar complaint that Hippias' proposed \foreign{definiens} is ``no more" (οὐδὲν μᾶλλον) fine than foul (289c1--d5).  But as Socrates reminds Hippias, he didn't ask for what is fine and foul; he asked for a definition of the fine---something that needs to explain why all other things are fine.  Therefore, this definition isn't acceptable, and Hippias needs to propose something else.

% }}} First definition (287e2-289d5)

% {{{ Second definition (289d6--291c9)
\subsection{Second definition (289d6--291c9)}

Hippias next offers gold as his second definition of the fine.  Again, this answer will obviously not satisfy Socrates.  But again Hippias is responding to something that Socrates wants.  In the first definition, Hippias was trying to give something irrefutably fine to Socrates.  That way, Socrates could not possibly be challenged by his alter-ego---or so Hippias hoped.  In this second definition, Hippias responds to language that Socrates uses when challenging the first definition. At the end of the last section, Socrates uses language reminiscent of language in \book{Phaedo} about forms:

\begin{quote}
    ἔτι δὲ καὶ δοκεῖ σοι αὐτὸ τὸ καλόν, ᾧ καὶ τἆλλα πάντα κοσμεῖται καὶ καλὰ φαίνεται, ἐπειδὰν προσγένηται ἐκεῖνο τὸ εἶδος, τοῦτ᾽εἶναι παρθένος ἢ ἵππος ἢ λύρα; (289d2--5)

    And does this in fact still seem to you to be the fine itself, by means of which all the other things are adorned and seem fine whenever that form is present [in them]---this, a girl or horse or lyre?
\end{quote}

Hippias appears to take note only of the language of adornment and presence.  Astoundingly, to people who know Socratic and Platonic metaphysics better than Hippias does, he interprets this language in a physical and very literal manner.  ``You want a good looking item to pretty up anything when it's present?  Gold!"  It's as if he's imagining that the topic is fashion.  This answer is arguably grist for Woodruff's mill: Rather than believe that Hippias is \emph{this} clueless, he would argue that Hippias is deliberately yanking Socrates' chain, partly in an effort to force Socrates to let him off the hook.\footnote{Part of Woodruff's larger argument is that Hippias doesn't want to have this conversation, but he's too agreeable to simply say so.  Instead, he gives joke answers, hoping that Socrates will take a hint and change subjects.}  At the moment though, I prefer to say that Hippias isn't stupid, but he does listen to or notice only parts of what Socrates says.

Socrates attacks this second definition in a roundabout way.  His first argument is that gold is not \emph{required} for something to be fine.  As an example, he offers Pheidias' statue of Athena, which used ivory rather than gold as an adornment.

Hippias, of course, is not bothered by this at all.  His response is that ivory is fine \emph{too}.  After Socrates prods him further, Hippias adds that stone is also fine, and that all of these things (gold, ivory and stone) are fine \emph{when} they are appropriate.  Thus Hippias has now weakened his position quite a lot.

Socrates presses the point about \word{appropriate} and returns to the cooking pot.  He asks Hippias whether it is appropriate to use a golden ladle or a fig-wood ladle when cooking pea soup in the pot.  Hippias again objects to the inclusion of such low topics, but Hippias agrees that the fig-wood ladle is more appropriate for cooking soup.  Socrates claims that, by this account, fig-wood would be more fine than gold.  Hippias doesn't bother to argue the point.  Instead, he moves blithely onto his next definition.

% }}} Second definition (289d6--291c9)

% {{{ Conclusion (303e11--304e9)
\section{Conclusion (303e11--304e9)}


% }}} Conclusion (303e11--304e9)


\newpage
\bibliographystyle{apa}
\bibliography{plato}

\end{document}
% }}}
