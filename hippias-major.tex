% {{{ LaTeX prelude
\documentclass[11pt]{article}
\usepackage{fontspec}
\setmainfont[Ligatures={Common,TeX}]{Palatino}
\usepackage{url}
\usepackage{parskip}
\usepackage{natbib}
\bibpunct{(}{)}{;}{a}{,}{,}
\usepackage{titles}
% }}}

% {{{ LaTeX document
\begin{document}

% {{{ Title page
\begin{titlepage}
\title{Notes on Plato's \book{Hippias Major}}
\author{Peter Aronoff}
\date{July 2013}
\maketitle
\end{titlepage}
% }}}

% {{{ Characters and setting
\section{Characters and setting}

% }}} Characters and setting

% {{{ Characters
\subsection{Characters}

Hippias was a famous sophist who probably lived circa 470--395 BCE. \footnote{\citet{waterfield1987a} 213.}  His dates aren't certain, but he was younger than Socrates and Gorgias and he outlived Socrates.  He was very much a polymath: David Sider describes him as ``lecturer, historian, poet, teacher, ambassador, potter, engraver, metalworker, weaver, and cobbler".\footnote{\citet{sider1986} iv.}  In addition to the two eponymous dialogues, Hippias appears in \book{Protagoras}.  He appears at 315c, sitting among a group of people asking him questions about astronomy.\footnote{Later in the dialogue, Protagoras makes fun of Hippias, among other sophists, for dragging his students through too many subjects that they don't need or want (318d5--e4).}  He seems to be putting on an ``ask me whatever you want" type of showcase.  In addition to that sort of thing and regular teaching for money, he also gave epideictic displays.  For example, at the start of \book{Hippias Major}, he tells Socrates about a piece he wrote imagining Nestor giving moral advice to Neoptolemus (286a3--c2).

Hippias was from Elis in the northwestern Peloponnese.  The people of Elis apparently held him in high regard: They sent him on diplomatic missions all over the Greek world.  But I'm not aware of any more significant political service by Hippias.

% }}} Characters

% {{{ Setting
\subsection{Setting}

There's nothing to place the dialogue physically, but we can say something about the date.  Gorgias has already visited, so that puts the date after 427.  It is also after the death of Pheidias a sculptor (died circa 420).  It also appears to be peacetime.  David Sider guesses sometime during the Peace of Nicias (421-416), and Robin Waterfield guesses 420.\footnote{\citet{sider1986} iv; \citet{waterfield1987a} 213.}

% }}} Setting

% {{{ Introduction (281a--286c2)
\section{Introduction (281a--286c2)}

The dialogue begins with Socrates saying that it's been a long time since he's seen Hippias and Hippias telling Socrates what he's been up to.  He has been very busy acting as an ambassador for Elis, his native city---in particular, he's been busy at Sparta.  Socrates takes this opportunity to flatter Hippias: he is so capable, both as a private teacher for pay and as a citizen working for his city-state in politics.  But, Socrates wonders, why did the previous generations of wise men hold off from politics?  Hippias says that they were simply less capable than the current generation, and Socrates (ironically, one assumes) agrees that like all other craftspeople, sophists become better each generation.

They then banter a bit about money. Hippias brags about how much he earns compared with other sophists, and Socrates again draws an ironic (?) contrast between the current generation of wise men and previous ones.  The previous wise men didn't care at all about money, while the current generation look out for themselves first and they make sure to get paid.

The introduction concludes with an ironic and paradoxical argument that the Spartans, although most lawful, are lawbreakers.  It all begins when Socrates learns from Hippias that Hippias didn't earn any money in Sparta by teaching young men.  However, Hippias concedes to Socrates the following:

\begin{enumerate}
    \item Hippias improves the people who associate with him through his wisdom.  He can improves the sons of the Spartans in this same way (283c1--d2).
    \item The Spartans can afford to pay Hippias (283d2--3).
    \item The Spartans cannot educate their own sons better than Hippias (283d4--e1).
    \item The Spartans don't begrudge their children a good education.  That is, the Spartan fathers don't keep their children from Hippias as a result of resentment (283e2--8).
    \item Sparta is a city of excellent laws and customs and Hippias knows how to pass along this kind of virtue to children very well.  Normally, if a city is renowned for X and a master of X comes to visit, the people of that city will line up to hire him for their sons.  Socrates gives an example of equestrian knowledge in Thessaly (283e9--284b5)
\end{enumerate}

If you put this all together, it seems incredible that the Spartans don't hire Hippias for their children, but Hippias explains that it is not traditional (πάτριον 284b6) to change their laws nor to educate their sons in uncustomary ways. And it is not customary (νόμιμον 284c5) for children to receive education from non-Spartans.

Socrates, however, draws the conclusion that the well-lawed (?) Spartans are lawbreakers.  His argument is that law aims at good and benefit.  But the rule not to let foreigners and thus Hippias educate their children does harm, not benefit.  So the Spartans act contrary to law as a result of their own customs (284d1--285b7).

After this, they briefly discuss Hippias' epideictic piece in which Nestor gives advice to Neoptolemus, and then they transition to the fine.

% }}} Introduction (281a--286c2)

% {{{ Pre-definition preliminaries (286c3--287e1)
\section{Pre-definition preliminaries (286c3--287e1)}

% }}} Pre-definition preliminaries (286c3--287e1)

% {{{ Socrates' alter-ego (286c3--287b3)
\subsection{Socrates' alter-ego (286c3--287b3)}

Socrates tells Hippias that the question he asks ('What is the fine?') is not his own but a question someone else asked him.  Socrates says that this other man recently cast him into helplessness (\word{ἀπορία} 286c5) during a discussion.  Socrates was casually calling things \word{fine} or \word{shameful}, and the man challenged him to say what the fine is.  Socrates failed the challenge, and he now asks Hippias to help him so that he can redeem himself.  In order to help things along, Socrates frequently pretends to be the other man as he and Hippias talk.  Over the dialogue, this someone else takes on a large role.  Hippias finds this questioner infuriating since (1) he is never satisfied and (2) he is frequently rude in tone and unsophisticated in topics.

It is obvious to readers that there never was any other man: the other man is Socrates all along.  The trick allows Socrates to question Hippias insistently without directly infuriating him.  In addition, it adds another layer of irony and humor to the dialogue---which is already thick with both.  In this way, the deception is both kind and cruel.  Socrates doesn't directly, explicitly insult Hippias, but indirectly and implicitly, he is far more aggressive than in most dialogues.

Socrates also asks Hippias if it will be ok for him to imitate the other man sometimes.  Hippias agrees.  You might think that this was in order for Plato to have Socrates speak without having to say constantly ``as the other man would say", but in fact, Socrates constantly draws attention to the mask.  So I'm not quite sure what the point of this request is.  It certainly doesn't end up making the dramatic situation any more straightforward.

% }}} Socrates' alter-ego (286c3--287b3)

% {{{ Socratic metaphysics (287b4--287e1)
\subsection{Socratic metaphysics (287b4--287e1)}

Socrates assumes the voice of the other figure and runs through some preliminary metaphysics with Hippias.  Hippias has no idea what Socrates is saying, but he agrees with it all regardless.

\begin{enumerate}
    \item Just people are just by justice\footnote{The word \word{by} here indicates the common Platonic use of an instrumental/causal dative to indicate that through which someone or something possesses a quality. \citet{woodruff1982} cites \book{Euthyphro} 6b11 and \book{Phaedo} 100d7 (45).  He also argues, correctly I think, that the dative is more a matter of logical cause than true instrument or means.} (287c1--2).
    \item Justice exists, or there is such a thing as justice (287c4).
    \item In the same way the wise are wise by of wisdom and all good things are good by goodness (287c5--6).  And wisdom and goodness exist too (287c6--8).
    \item All fine things are fine by of the fine, and the fine exists (287c8-d2).
\end{enumerate}

Hippias agrees all the way through, though perhaps with very little understanding.  This becomes clear when Socrates asks Hippias the dialogue's key question:

\begin{quote}
    Εἰπὲ δή, ὦ ξένε, φήσει, τί ἐστι τοῦτο τὸ κάλον; (287d3)

    Say indeed, stranger, he will say, what is this \phrase{the fine}?
\end{quote}

Hippias is so completely confused that he responds, ``Do you mean what is beautiful?" (287d4--5).  Socrates says ``No, \emph{the} fine" (287d6), but Hippias is unable to see any difference between the two questions.  This is very important: Hippias says ``What is the difference?" (287d7).  In response to that, Socrates says only ``It doesn't seem different at all to you?" (287d9).\footnote{I'm unsure about Socrates' tone here.  Disbelief?  Confusion?}

Socrates does not challenge Hippias on his confusion here.  He simply says, ``Well you must know better" and continues.  My guess is that Socrates (and/or Plato) prefer to work Hippias up to the larger issue, rather than quarrel before they even get started.  However, it's exceedingly important for the reader to see how very confused Hippias is.  See below when I discuss Alexander Nehamas' interpretation of the first definition.

% }}} Socratic metaphysics (287b4--287e1)

% {{{ Definitions of the fine (287e2--303d10)
\section{Definitions of the fine (287e2--303d10)}

% }}} Definitions of the fine (287e2--303d10)


% {{{ First definition (287e2-289d5)
\subsection{First definition (287e2-289d5)}

content

% }}} First definition (287e2-289d5)

% {{{ Conclusion (303e11--304e9)
\section{Conclusion (303e11--304e9)}


% }}} Conclusion (303e11--304e9)


\newpage
\bibliographystyle{apa}
\bibliography{plato}

\end{document}
% }}}
