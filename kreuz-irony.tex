% [[- LaTeX prelude
\documentclass[12pt,letterpaper]{article}

\usepackage[no-math]{fontspec}
\setmainfont{Baskerville}

\usepackage[base]{babel} % Ugh: https://tex.stackexchange.com/a/400994/29387
\usepackage[nolocalmarks]{polyglossia}
\setdefaultlanguage{english}
\setotherlanguage[variant=classic]{latin}
\setotherlanguage[variant=ancient]{greek}
\newfontfamily\greekfont[Script=Greek,Scale=MatchLowercase]{GFS Neohellenic}

\usepackage{fnpct}

\usepackage{titlesec}
\titleformat*{\section}{\large\bfseries}
\titleformat*{\subsection}{\bfseries}
\titleformat*{\subsubsection}{\bfseries}

\usepackage{parskip}
\usepackage{csquotes}
\usepackage[style=windycity,citetracker=context,backend=biber]{biblatex}
\addbibresource{plato.bib}

\usepackage{enumitem}
\setlist{noitemsep}
\usepackage[super]{nth}

\begin{hyphenrules}{latin}
    \hyphenation{}
\end{hyphenrules}

\begin{hyphenrules}{greek}
    \hyphenation{}
\end{hyphenrules}

\usepackage{fancyhdr}
\fancypagestyle{notes}{%
    \fancyhf{}
    \renewcommand{\headrulewidth}{0pt}
    \lhead{}
    \chead{\MakeUppercase{Notes on \textit{Irony and Sarcasm}}}
    \rhead{}
    \lfoot{}
    \cfoot{\thepage}
    \rfoot{}
}
\fancypagestyle{references}{%
    \fancyhf{}
    \renewcommand{\headrulewidth}{0pt}
    \lhead{}
    \chead{\MakeUppercase{References}}
    \rhead{}
    \lfoot{}
    \cfoot{\thepage}
    \rfoot{}
}

\makeatletter

% {<file>}{<message>}{<preload>}{<postload>}{<success>}{<failure>}
\protected\long\def\blx@lbx@input@handler@simple#1#2#3#4#5#6{%
  \blx@info@noline{Trying to load #2..}%
  \IfFileExists{#1}
    {\blx@info@noline{... file '#1' found}%
     \csuse{blx@lbxfilehook@simple@preload@#1}%
     #3%
     \setbox\@tempboxa=\hbox\bgroup\@@input\@filef@und\egroup
     #4%
     \csuse{blx@lbxfilehook@simple@postload@#1}%
     #5%
     \ifcsundef{blx@file@lbx@simple@#1}
       {\listxadd\blx@list@req@stat{#1}%
        \@addtofilelist{#1}%
        \global\cslet{blx@file@lbx@simple@#1}\@empty}
       {}}
    {\blx@info@noline{... file '#1' not found}%
     \csuse{blx@lbxfilehook@simple@failure@#1}%
     #6}}

% {<file>}{<message>}{<preload>}{<postload>}{<success>}{<failure>}
\protected\long\def\blx@lbx@input@handler@once#1#2#3#4#5#6{%
  \ifcsundef{blx@file@lbx@once@#1}
    {\blx@info@noline{Trying to load #2..}%
     \IfFileExists{#1}
       {\blx@info@noline{... file '#1' found}%
        \csuse{blx@lbxfilehook@once@preload@#1}%
        #3%
        \setbox\@tempboxa=\hbox\bgroup\@@input\@filef@und\egroup
        #4%
        \csuse{blx@lbxfilehook@once@postload@#1}%
        #5%
        \ifcsundef{blx@file@lbx@simple@#1}
          {\listxadd\blx@list@req@stat{#1}%
           \@addtofilelist{#1}}
          {}}
       {\blx@info@noline{... file '#1' not found}%
        \csuse{blx@lbxfilehook@once@failure@#1}%
        #6}%
     \global\cslet{blx@file@lbx@once@#1}\@empty
     \global\cslet{blx@file@lbx@simple@#1}\@empty}
    {#5}}
\makeatother
% -]] Latex prelude

% [[- LaTeX document
\begin{document}

% [[- Title page
% \begin{titlepage}
% \title{Notes on \textit{Irony and Sarcasm}}
% \author{Peter Aronoff}
% \maketitle
% \thispagestyle{empty}
% \end{titlepage}
% -]]

\pagestyle{notes}

% [[- Introduction
\section*{Introduction}

Kreuz describes the book as ``the biography of a troublesome word,'' namely \textit{irony}.%
\footcite[xi]{kreuz-irony-and-sarcasm-2020}
Although everyone uses the word, most of us find it difficult to define precisely.
Dictionaries and studies suggest that it is not just laypeople that have this problem: the concept of irony itself defies analysis.

Kreuz also considers sarcasm---``irony's evil twin'' and another source of problems.
Sarcasm itself is difficult to define, and it is difficult to grasp the precise relationship between sarcasm and irony.
Sarcasm also appears to encompass opposed characteristics.
For example: sarcasm ruins intimacy, but people desire sarcastic partners; sarcasm is a source of cheap laughs, but a sign of intelligence.

Kreuz suggests using family resemblance, à la Wittgenstein, to understand irony and sarcasm.
He proposes ``to identify the attributes that constitute a family resemblance for the irony concept.''%
\footcite[xiv]{kreuz-irony-and-sarcasm-2020}
In addition, Kreuz promises to go through empirical studies of irony and sarcasm.
Finally, Kreuz will examine the history of irony, starting with Greek philosophy and continuing all the way through contemporary online examples.
% -]] Introduction

% [[- Chapter 1: Some Preliminaries
\section*{Chapter 1: Some Preliminaries}

% [[- Nonliteral Language
\subsection*{Nonliteral Language}

Kreuz views irony and sarcasm as nonliteral uses of language.
Although he doesn't precisely define ``literal,'' he makes clear what he means by the contrast.
Literal language means what it says, and we don't need extra information or context to understand it.
In addition, when a speaker uses language literally, we do not need to search for unspoken desires or goals.
We can describe nonliteral language by contrast.
Nonliteral language does \textit{not} mean what it says, in some sense, and we \textit{should} be on the lookout for hidden desires and purposes.

Kreuz briefly reviews several adjacent examples of nonliteral language.
He gives examples of metaphor, simile, idioms, hyperbole, litotes, rhetorical questions, and what he calls antiphrasis.
Most of these are well known, but antiphrasis is interesting.
Kreuz uses it as a catch-all for language ``which involves saying the opposite of what is literally meant.''%
\footcite[][5]{kreuz-irony-and-sarcasm-2020}
However, Kreuz does not believe that antiphrasis and irony are synonymous.
He claims that saying the opposite of what one means is neither necessary nor sufficient for irony.
I am not sure that I agree, but since he hasn't made his whole case yet, I'll wait and see.
% -]] Nonliteral Language

% [[- Guides for the Perplexed?
\subsection*{Guides for the Perplexed?}

We might hope to clear things up by checking dictionaries and other reference works.
However, Kreuz doubts this will work.
Why not?
Kreuz does not argue as explicitly as I would like, but here's what I think he means.
On the one hand, prescriptive dictionaries and reference works give a clear line.
(``This is irony, but that is not.'')
On the other hand, such prescriptions often ignore real complexities in favor of simplistic rules.
On the one hand, descriptive dictionaries and reference works give a much fuller picture.
On the other hand, they do no analysis, and they will not settle difficult questions about what is or is not irony.
Neither solution gives Kreuz what he wants, namely a thoughtful analysis of the concept of irony and its neighbors.

If my analysis of Kreuz's argument is right, then I agree with him.
Dictionaries and style guides are useless when it comes to contested concepts such as irony.
They may be able to sketch out \textit{some} of the key aspects of the concept, but we should not rely on them for a full and final accounting.
% -]] Guides for the Perplexed?

% -]] Chapter 1: Some Preliminaries

% [[- Chapter 2: The Varieties of Ironic Experience
\section*{Chapter 2: The Varieties of Ironic Experience}

Kreuz considers eight uses of the word \textit{irony} in this chapter.
Although these eight uses are diverse, Kreuz nevertheless believes that he can build a family resemblance account from them.

% [[- Socratic Irony
\subsection*{Socratic Irony}

Kreuz begins his historical survey of irony with Socratic irony.
Kreuz identifies Socratic irony as a rhetorical strategy in Plato's dialogues.
In those dialogues, Socrates claims that he doesn't know something in order to get his interlocutor to say more.
According to Kreuz, at least some of the time, Socrates says the literal truth when he claims not to know.
However, Socrates employs Socratic irony when he pretends not to know what he thinks he does know.
Many historians of ancient philosophy deny, however, that Socrates ever pretends, but Kreuz does not discuss this at first.
Like may readers, Kreuz takes it for granted that Socrates sometimes pretends or says what he does not believe.
Kreuz also seems to assume that this is a rhetorical tactic rather than a means of reaching the truth.

Kreuz also claims that ``the other characters in Plato's dialogues were not enamored with'' Socratic irony.%
\footcite[][15]{kreuz-irony-and-sarcasm-2020}
Although Kreuz says this universally, I suspect he really means that the targets of Socratic irony resented the strategy.
This better suits the evidence Kreuz provides, namely Callicles, Thrasymachus, and Alcibiades.
Many characters (e.g., Euthyphro and Hippias) do not seem to notice Socratic irony at all.
Others probably enjoy it---I think that Socrates implies this when he says that young people enjoy listening to him refute people in public.%
\footcite[][34]{worlfsdorf-eironeia-aristophanes-plato-2008}

Kreuz does not acknowledge that all of the characters who express resentment at Socratic irony are suspect for one or more reasons.
(I think this should concern us, but perhaps I am wrong?)
Both Callicles and Thrasymachus make agressive attacks on morality, and Alcibiades is drunk and deeply spoiled.
(Melissa Lane briefly refers to this trio as ``complicated and challenging figures.''%
\footcite[][237]{lane-reconsidering-socratic-irony-2011})
On the other hand, Plato appears to take the arguments of Callicles and Thrasymachus seriously in \textit{Republic}, and Alcibiades offers glowing praise and a revealing portrait of Socrates.

Kreuz also refes to Aristotle for testimony about Socratic irony.
Aristotle mentions Socrates in Book 4 of \textit{Nichomachean Ethics} while he is discussing the virtue concerning truthfulness about one's abilities.
This (unnamed) virtue, says Aristotle in \textit{NE} 4.7, is a mean between boasting and false modesty (\textgreek{ἀλαζονεία} and \textgreek{εἰρωνεία}).
Boasters claim admirable qualities that they lack or they exaggerate their share in those qualities; the \textit{eiron} disavows admirable qualities that they possess.
In this context, Aristotle says, ``The qualities that win reputation are the ones that these people especially disavow, as Socrates also used to do'' (\textit{NE} 4.7 in Irwin's translation).

Following Paul Gooch, Kreuz credits Aristotle with first associating Socrates and Socratic irony.
\footcites[][16]{kreuz-irony-and-sarcasm-2020}[][]{gooch-socratic-irony-and-arisotles-eiron-1987}
To clarify, Gooch and Kreuz do not argue that nobody before Aristotle associated Socrates with irony.
They both know that Plato demonstrates that Socrates had an association with irony long before Aristotle wrote.
However, the earlier association of \textgreek{εἰρωνεία} and Socrates was negative.
\footcite[][]{worlfsdorf-eironeia-aristophanes-plato-2008}
Therefore, Aristotle innovates by creating a more positive interpretation of Socratic irony.
For Aristotle, Socratic irony is mild and charming self-deprecation rather than malicious deception.
% TODO: Citation about Aristophanes \textit{Clouds} 449

Kreuz distinguishes rather sharply between Socratic irony as deception and Socratic irony as ``humble self-deprecation.''
\footcite[][16]{kreuz-irony-and-sarcasm-2020}
In this way, Kreuz agrees with the many scholars who see a development in the range and tone of the ancient Greek \textgreek{εἴρων}.
(Melissa Lane makes a similar argument, and Kreuz cites her with approval.
\footcite[The reference in question is][]{lane-reconsidering-socratic-irony-2011})
% -]] Socratic Irony

% -]] Chapter 2: The Varieties of Ironic Experience

% [[- Bibliography
\newpage
\pagestyle{references}
\defbibfilter{sources}{%
    ( keyword=edition or keyword=translation or keyword=commentary ) % chktex 37
}
\defbibfilter{secondary}{%
    keyword=secondary
}
\printbibliography[filter=sources,title={Ancient Sources: Editions, Translations, Commentaries}]
\printbibliography[filter=secondary,title=Secondary Literature]
% -]] Bibliography

\end{document}
% -]]
