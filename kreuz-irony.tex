% [[- LaTeX prelude
\documentclass[12pt,letterpaper]{article}

\usepackage[no-math]{fontspec}
\setmainfont{Baskerville}

\usepackage[nolocalmarks]{polyglossia}
\setdefaultlanguage{english}
\setotherlanguage[variant=classic]{latin}
\setotherlanguage[variant=ancient]{greek}
\newfontfamily\greekfont[Script=Greek,Scale=MatchLowercase]{GFS Neohellenic}

\usepackage{fnpct}

\usepackage{titlesec}
\titleformat*{\section}{\large\bfseries}
\titleformat*{\subsection}{\bfseries}
\titleformat*{\subsubsection}{\bfseries}

\usepackage{parskip}
\usepackage{csquotes}
\usepackage[style=windycity,citetracker=context,backend=biber]{biblatex}
\addbibresource{plato.bib}

\usepackage{enumitem}
\setlist{noitemsep}
\usepackage[super]{nth}

\begin{hyphenrules}{latin}
    \hyphenation{}
\end{hyphenrules}

\begin{hyphenrules}{greek}
    \hyphenation{}
\end{hyphenrules}

\usepackage{fancyhdr}
\fancypagestyle{notes}{%
    \fancyhf{}
    \renewcommand{\headrulewidth}{0pt}
    \lhead{}
    \chead{\MakeUppercase{Notes on \textit{Irony and Sarcasm}}}
    \rhead{}
    \lfoot{}
    \cfoot{\thepage}
    \rfoot{}
}
\fancypagestyle{references}{%
    \fancyhf{}
    \renewcommand{\headrulewidth}{0pt}
    \lhead{}
    \chead{\MakeUppercase{References}}
    \rhead{}
    \lfoot{}
    \cfoot{\thepage}
    \rfoot{}
}

\newcommand{\MONTH}{%
  \ifcase\the\month
  \or January% 1
  \or February% 2
  \or March% 3
  \or April% 4
  \or May% 5
  \or June% 6
  \or July% 7
  \or August% 8
  \or September% 9
  \or October% 10
  \or November% 11
  \or December% 12
  \fi}
% -]] Latex prelude

% [[- LaTeX document
\begin{document}

% [[- Title page
% \begin{titlepage}
% \title{Notes on \textit{Irony and Sarcasm}}
% \author{Peter Aronoff}
% \date{March 2020--\MONTH\ \the\year}
% \maketitle
% \thispagestyle{empty}
% \end{titlepage}
% -]]

\pagestyle{notes}

% [[- Introduction
\section*{Introduction}

Kreuz describes the book as ``the biography of a troublesome word,'' namely \textit{irony}.%
\footcite[xi]{kreuz-irony-and-sarcasm-2020}
Although everyone uses the word, most of us find it very hard to define precisely.
Dictionaries and studies suggest that it is not just laypeople that have this problem: the concept of irony itself defies analysis.

Kreuz also considers sarcasm---``irony's evil twin'' and another source of problems.
Sarcasm itself is difficult to define, and it is difficult to grasp the precise relationship between sarcasm and irony.
Sarcasm also appears to encompass opposed characteristics.
For example: sarcasm ruins intimacy, but people desire sarcastic partners; sarcasm is a source of cheap laughs, but a sign of intelligence.

Kreuz suggests using family resemblance, à la Wittgenstein, to understand irony and sarcasm.
He proposes ``to identify the attributes that constitute a family resemblance for the irony concept.''%
\footcite[xiv]{kreuz-irony-and-sarcasm-2020}
In addition, Kreuz promises to go through empirical studies of irony and sarcasm.
Finally, Kreuz will examine the history of irony, starting with Greek philosophy and continuing all the way through contemporary online examples.
% -]] Introduction

% [[- Bibliography
\newpage
\pagestyle{references}
\defbibfilter{sources}{%
    ( keyword=edition or keyword=translation or keyword=commentary )
}
\defbibfilter{secondary}{%
    keyword=secondary
}
\printbibliography[filter=sources,title={Ancient Sources: Editions, Translations, Commentaries}]
\printbibliography[filter=secondary,title=Secondary Literature]
% -]] Bibliography

\end{document}
% -]]
