% [[- LaTeX prelude
\documentclass[12pt,letterpaper]{article}

\usepackage[no-math]{fontspec}
\setmainfont{Baskerville}

\usepackage[nolocalmarks]{polyglossia}
\setdefaultlanguage{english}
\setotherlanguage[variant=ancient]{latin}
\setotherlanguage[variant=ancient]{greek}
\newfontfamily\greekfont[Script=Greek,Scale=MatchLowercase]{GFS Neohellenic}

\usepackage{fnpct}

\usepackage{titlesec}
\titleformat*{\section}{\large\bfseries}
\titleformat*{\subsection}{\bfseries}
\titleformat*{\subsubsection}{\bfseries}

\usepackage{parskip}
\usepackage{csquotes}
\usepackage[style=windycity,citetracker=context,backend=biber]{biblatex}
\addbibresource{plato.bib}

\usepackage{enumitem}
\setlist{noitemsep}
\usepackage[super]{nth}

\begin{hyphenrules}{latin}
    \hyphenation{}
\end{hyphenrules}

\begin{hyphenrules}{greek}
    \hyphenation{}
\end{hyphenrules}

\usepackage{fancyhdr}
\fancypagestyle{notes}{%
    \fancyhf{}
    \renewcommand{\headrulewidth}{0pt}
    \lhead{}
    \chead{\MakeUppercase{Notes on Love in Plato}}
    \rhead{}
    \lfoot{}
    \cfoot{\thepage}
    \rfoot{}
}
\fancypagestyle{references}{%
    \fancyhf{}
    \renewcommand{\headrulewidth}{0pt}
    \lhead{}
    \chead{\MakeUppercase{References}}
    \rhead{}
    \lfoot{}
    \cfoot{\thepage}
    \rfoot{}
}

\newcommand{\MONTH}{%
  \ifcase\the\month
  \or January% 1
  \or February% 2
  \or March% 3
  \or April% 4
  \or May% 5
  \or June% 6
  \or July% 7
  \or August% 8
  \or September% 9
  \or October% 10
  \or November% 11
  \or December% 12
  \fi}
% -]] Latex prelude

% [[- LaTeX document
\begin{document}

% [[- Title page
% \begin{titlepage}
% \title{Plato on Love}
% \author{Peter Aronoff}
% \date{March 2020--\MONTH\ \the\year}
% \maketitle
% \thispagestyle{empty}
% \end{titlepage}
% -]]

\pagestyle{notes}

% [[- Eros and Philia
\section*{Eros and Philia}

Kraut begins with \textgreek{ἔρως} and \textgreek{φιλία}.%
\footcite{kraut-plato-on-love-2019}
We can translate both roots in English as `love.'
How do they differ?
Kraut says that words from the \textgreek{ἔρως} root always involve sexuality, even if they are not limited to sexuality.
On the other hand, Kraut believes that words from the \textgreek{φιλία} root especially suit cooperative groups.
As such, \textgreek{φίλοι} can be friends or relatives.
Kraut also stresses that people who have \textgreek{φιλία} for each other may or may not also have erotic desire for one another.
That is, \textgreek{φιλία} neither implies nor rules out \textgreek{ἔρως}.

If Kraut is right, Aristotle and Plato pursued complementary investigations about \textgreek{ἔρως} and \textgreek{φιλία}.
Plato discusses \textgreek{φιλία} in \textit{Lysis}, but he does not have a full theory of \textgreek{φιλία}.
Aristotle offers a complete theory of \textgreek{φιλία}, but he does not consider \textgreek{ἔρως} in any detail.
Plato, on the other hand, considers \textgreek{ἔρως} very important, and he investigates it in both \textit{Symposium} and \textit{Phaedrus}.
Kraut implies that Plato does not have a \textit{theory} about \textgreek{ἔρως} either, but he doesn't discuss this in any detail.
% -]] Eros and Philia

% [[- Eros and Desire
\section*{Eros and Desire}

Kraut begins with the \textit{elenchus} between Socrates and Agathon.
In that conversation, Socrates argues that if someone loves (\textgreek{ἐρᾷ}), then they want something or someone.
Therefore, they need someone or something that they lack (or think they lack?).
That is, Socrates appears to believe that want implies need.
According to Kraut, Socrates handles counterexamples by redescribing them.
If you say, ``I am healthy, but I also want to be healthy,'' Socrates interprets this as ``You are healthy (now), and you want to be healthy (in the future).''
Hence, the thing that you want and the thing that you have are not the same.

Kraut sidesteps a serious criticism of Plato here.
Someone may object that ``people can want things that have nothing to do with themselves, and so desire is not the same thing as needing and wanting.''%
\footcite[554]{kraut-plato-on-love-2019}
Kraut grants that this may be a valid criticism of Plato.
Nevertheless, Kraut maintains that even if \textit{some} desires are not the same as needing and wanting, Plato may still be right about \textgreek{ἔρως}.
% -]] Eros and Desire

% [[- Bibliography
\newpage
\pagestyle{references}
\defbibfilter{sources}{%
    ( keyword=edition or keyword=translation or keyword=commentary )
}
\defbibfilter{secondary}{%
    keyword=secondary
}
\printbibliography[filter=sources,title={Ancient Sources: Editions, Translations, Commentaries}]
\printbibliography[filter=secondary,title=Secondary Literature]
% -]] Bibliography

\end{document}
% -]]
